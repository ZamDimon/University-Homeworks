\documentclass{hw_template}

\title{\huge\sffamily\bfseries Домашня Робота з Рівнянь Математичної Фізики \#3}
\author{\Large\sffamily Захаров Дмитро}
\date{\sffamily 19 квітня, 2025}

\begin{document}

\pagestyle{fancy}

\maketitle

\tableofcontents

\pagebreak

\section{Домашня Робота}

\subsection{Вправа 1 (Таблиця 1, Клітинка 8)}

\begin{problem}
    За власних функцій отримати ці функції і власні значення. Довести, що 
    вони ортогональні, обчислити норми. Крайові умови $u'(\ell,t)=0$, 
    $u'(0,t)-\gamma u(0,t)=0$.
    \begin{gather*}
        \lambda_n = \left(\frac{\mu_n}{\ell}\right)^2, \quad \mu_n \tan \mu_n = \gamma \ell, \quad n \in \mathbb{N} \\
        e_n = \left(\mu_n \cos \frac{\mu_n x}{\ell} + \gamma \ell\sin\frac{\mu_n x}{\ell}\right)\sqrt{\frac{2}{\ell} \cdot \frac{1}{\mu_n^2 + (\gamma \ell)^2 + \gamma \ell}}
    \end{gather*}
\end{problem}

\textbf{Розв'язання.} Маємо рівняння $Y_n'' + \lambda_n Y_n = 0$ з наступними
початковими умовами, згідно умові: $Y_n'(\ell)=0$ та $Y_n'(0) - \gamma Y_n(0) = 0$. 

\textcolor{blue}{\textbf{Випадок $\lambda_n=0$.}} В такому разі маємо $Y_n''=0$, звідки 
$Y_n(x) = A_nx+B_n$. Підставлямо умови: $Y_n'(\ell) = A_n = 0$, а друга умова 
має вигляд $A_n - \gamma B_n = 0$, звідки $B_n=0$ (оскільки вважаємо $\gamma \neq 0$). 
Отже, таким чином $A_n=B_n=0$ і тому $Y_n(x) \equiv 0$.

\textcolor{blue}{\textbf{Випадок $\lambda_n < 0$.}} В такому разі, розв'язки
рівняння можна записати як: 
\begin{equation*}
    Y_n(x) = A_ne^{\sqrt{-\lambda_n}x} +
B_ne^{-\sqrt{-\lambda_n}x}    
\end{equation*}
Отже, маємо наступну систему рівнянь для
знаходження $A_n$, $B_n$:
\begin{equation*}
    \begin{cases}
        A_n\sqrt{-\lambda_n}e^{\sqrt{-\lambda_n}\ell} - B_n\sqrt{-\lambda_n}e^{-\sqrt{-\lambda_n}\ell} = 0 \\
        A_n\sqrt{-\lambda_n} - B_n\sqrt{-\lambda_n} - \gamma (A_n+B_n) = 0
    \end{cases}
\end{equation*}

З першого рівняння $B_n=A_ne^{2\sqrt{-\lambda_n}\ell}$. Підставляючи у друге,
маємо $A_n\sqrt{-\lambda_n}(1 - e^{2\sqrt{-\lambda_n}\ell}) - \gamma
A_n(1+e^{2\sqrt{-\lambda_n}\ell})=0$ або $\left(\sqrt{-\lambda_n}(1 -
e^{2\sqrt{-\lambda_n}\ell}) - \gamma (1+e^{2\sqrt{-\lambda_n}\ell})\right)A_n=0$.
Зрозуміло, що звідси випливає $A_n=0$, а отже і $B_n=0$. Таким чином, $Y(x) \equiv
0$.

\textcolor{blue}{\textbf{Випадок $\lambda > 0$.}} Цей випадок найцікавіший. 
Розв'язок рівняння:
\begin{equation*}
    Y_n(x) = A_n\cos(\sqrt{\lambda_n}x) + B_n\sin(\sqrt{\lambda_n}x).
\end{equation*}

Підставляючи крайові умови, маємо:
\begin{align*}
    -A_n\sqrt{\lambda_n}\sin(\sqrt{\lambda_n}\ell) + B_n\sqrt{\lambda_n}\cos(\sqrt{\lambda_n}\ell) &= 0 \\
    B_n\sqrt{\lambda_n} - \gamma A_n &= 0
\end{align*}

Обидва рівняння за умови $\lambda_n \neq 0$ (інакше дивись випадок 1) можна записати як:
\begin{equation*}
    \frac{B_n}{A_n} = \tan(\sqrt{\lambda_n}\ell) = \frac{\gamma}{\sqrt{\lambda_n}}.
\end{equation*}

Позначимо $\lambda_n := (\mu_n/\ell)^2$. З другої рівності в такому разі маємо:
\begin{equation*}
    \tan \mu_n = \frac{\gamma \ell}{\mu_n} \implies \mu_n \tan \mu_n = \gamma\ell.
\end{equation*}

Таким чином, наш розв'язок має вигляд:
\begin{equation*}
    Y_n(x) = A_n\left(\cos \frac{\mu_n x}{\ell} + \frac{\gamma \ell}{\mu_n}\sin \frac{\mu_n x}{\ell}\right) = A_n\left(\cos \frac{\mu_n x}{\ell} + \tan \mu_n\sin \frac{\mu_n x}{\ell}\right).
\end{equation*}

Спростимо цей вираз, залишивши тільки косинус. Для цього помітимо, що:
\begin{equation*}
    \cos \frac{\mu_n x}{\ell} + \tan \mu_n\sin \frac{\mu_n x}{\ell} = \sqrt{1+\tan^2\mu_n}\left(\frac{1}{\sqrt{1+\tan^2\mu_n}}\cos \frac{\mu_n x}{\ell} + \frac{\tan\mu_n}{\sqrt{1+\tan^2\mu_n}}\sin \frac{\mu_n x}{\ell}\right).
\end{equation*}

Позначимо через $\cos\varphi = \frac{1}{\sqrt{1+\tan^2\mu_n}}$, $\sin\varphi = \frac{\tan\mu_n}{\sqrt{1+\tan^2\mu_n}}$, тоді:
\begin{equation*}
    \cos \frac{\mu_n x}{\ell} + \tan \mu_n\sin \frac{\mu_n x}{\ell} = \frac{1}{\sin\mu_n}\left(\cos\varphi\cos \frac{\mu_n x}{\ell} + \sin\varphi\sin \frac{\mu_n x}{\ell}\right) = \frac{1}{\sin\mu_n}\cos\left(\frac{\mu_n x}{\ell} - \varphi\right).
\end{equation*}

Тепер помітимо, що $\tan\varphi = \tan\mu_n$, тому $\varphi=\mu_n$, звідки
\begin{equation*}
    \textcolor{blue}{Y_n(x) = A_n\sqrt{1+\frac{\gamma^2\ell^2}{\mu_n^2}}\cos \left(\frac{\mu_n}{\ell}(x-\ell)\right)}
\end{equation*}

Щоб знайти $A_n$, потрібно скористатися умовою $\|Y_n\|^2=1$. Обраховуємо:
\begin{equation*}
    \|Y_n\|^2 = \langle Y_n, Y_n \rangle = \int_0^{\ell} Y_n^2(x)dx = A_n^2\left(1+\frac{\gamma^2\ell^2}{\mu_n^2}\right)\int_0^{\ell}\cos^2 \left(\frac{\mu_n}{\ell}(x-\ell)\right)dx
\end{equation*}

Знаходимо інтеграл $\mathcal{I}_n := \int_0^{\ell}\cos^2 \left(\frac{\mu_n}{\ell}(x-\ell)\right)dx$. Згадаємо, що 
$\cos^2\theta = \frac{1}{2}(1+\cos 2\theta)$, тому:
\begin{equation*}
    \mathcal{I}_n = \frac{1}{2}\int_0^{\ell} \left(1+\cos\left(\frac{2\mu_n}{\ell}(x-\ell)\right)\right)dx = \frac{\ell}{2} + \frac{1}{2}\int_0^{\ell}\cos\left(\frac{2\mu_n}{\ell}(x-\ell)\right)dx
\end{equation*}

Зробимо заміну $\theta := \frac{2\mu_n}{\ell}(x-\ell)$. Тоді, оскільки $dx = \frac{\ell}{2\mu_n}d\theta$, маємо:
\begin{equation*}
    \int_0^{\ell}\cos\left(\frac{2\mu_n}{\ell}(x-\ell)\right)dx = \int_{-2\mu_n}^{0}\cos\theta \cdot \frac{\ell}{2\mu_n}d\theta = \frac{\ell}{2\mu_n}\sin\theta\Big|_{\theta=-2\mu_n}^{\theta=0} = \frac{\ell}{2}\cdot\frac{\sin 2\mu_n}{\mu_n}
\end{equation*}

Таким чином, $\mathcal{I}_n = \frac{\ell}{2} + \frac{\ell}{4} \cdot \frac{\sin 2\mu_n}{\mu_n} = \left(1+\frac{\sin 2\mu_n}{2\mu_n}\right)\frac{\ell}{2}$ і тоді:
\begin{equation*}
    \frac{1}{2}\ell A_n^2\left(1+\frac{\gamma^2\ell^2}{\mu_n^2}\right)\left(1 + \frac{\sin 2\mu_n}{2\mu_n}\right) = 1
\end{equation*}

Отже, всього-навсього залишилось виразити $\sin 2\mu_n$ через $\mu_n$:
\begin{equation*}
    \sin 2\mu_n = \frac{2 \tan \mu_n}{1+\tan^2\mu_n} = \frac{2\gamma \ell/\mu_n}{1+\frac{\gamma^2\ell^2}{\mu_n^2}}
\end{equation*}

Таким чином,
\begin{equation*}
    A_n^2 = \frac{2}{\ell} \cdot \frac{1}{\left(1+\frac{\gamma^2\ell^2}{\mu_n^2}\right)\left(1+\frac{\gamma \ell}{\mu_n^2+\gamma^2\ell^2}\right)} = \frac{2}{\ell} \cdot \frac{\mu_n^2}{\mu_n^2 + \gamma^2\ell^2 + \gamma \ell}
\end{equation*}

Таким чином, остаточно маємо:
\begin{equation*}
    \textcolor{blue}{A_n = \mu_n \sqrt{\frac{2}{\ell} \cdot \frac{1}{\mu_n^2 + \gamma^2\ell^2 + \gamma \ell}}} \implies \textcolor{blue}{e_n(x) = \left(\mu_n \cos \frac{\mu_n x}{\ell} + \gamma \ell\sin\frac{\mu_n x}{\ell}\right)\sqrt{\frac{2}{\ell} \cdot \frac{1}{\mu_n^2 + (\gamma \ell)^2 + \gamma \ell}}}
\end{equation*}

\textbf{Перевірка ортогональності.} Умову $\|e_n\|=1$ ми вже перевірили,
знайшовши $A_n$. Перевіримо, що $\langle e_n, e_m \rangle = \delta_{nm}$. 
Дійсно,
\begin{equation*}
    \lambda_n\langle Y_n, Y_m \rangle = \langle \lambda_nY_n, Y_m \rangle = \langle -Y_n'', Y_m \rangle = \int_0^{\ell} -Y_n''(x)Y_m(x)dx
\end{equation*}

Зробимо інтегрування по частинам:
\begin{align*}
    \int_0^{\ell} -Y_n''(x)Y_m(x)dx &= -Y_n'(x)Y_m(x)\Big|_0^{\ell} + \int_0^{\ell} Y_n'(x)Y_m'(x)dx \\
    &= -\textcolor{red}{Y_n'(\ell)}Y_m(\ell) + Y_n'(0)Y_m(0) + \int_0^{\ell} Y_n'(x)Y_m'(x)dx \\
    & = Y_n'(0)Y_m(0) + \int_0^{\ell} Y_n'(x)Y_m'(x)dx
\end{align*}

Зробимо інтегрування частинами ще раз:
\begin{align*}
    \int_0^{\ell} Y_n'(x)Y_m'(x)dx &= Y_n(x)Y_m'(x)\Big|_0^{\ell} - \int_0^{\ell} Y_n(x)Y_m''(x)dx \\
    &= Y_n(\ell)\textcolor{red}{Y_m'(\ell)} - Y_n(0)Y_m'(0) - \int_0^{\ell} Y_n(x)Y_m''(x)dx \\
    &= - Y_n(0)Y_m'(0) + \lambda_m\langle Y_n, Y_m \rangle
\end{align*}

Компануємо усі частини разом:
\begin{align*}
    \lambda_n\langle Y_n, Y_m \rangle &= \textcolor{green!50!black}{Y_n'(0)Y_m(0) - Y_n(0)Y_m'(0)} + \lambda_m\langle Y_n, Y_m \rangle \\
    &= \textcolor{green!50!black}{\gamma Y_n(0)Y_m(0) - Y_n(0)\gamma Y_m(0)} + \lambda_m\langle Y_n, Y_m \rangle \\
    &= \lambda_m\langle Y_n, Y_m \rangle
\end{align*}

Таким чином, оскільки $\lambda_n \neq \lambda_m$ ($n \neq m$), маємо $\langle Y_n, Y_m
\rangle = 0$, звідки $\langle e_n, e_m \rangle = 0$.

\subsection{Вправа 2 (Таблиця 2, Клітинка 7)}

\begin{problem}
    Для крайових умов $u'(0,t)-\gamma u(0,t)=\mu_1(t)$ та $u(\ell,t)=\mu_2(t)$, показати, 
    що перехід до однорідних граничних умов має вигляд:
    \begin{equation*}
        v(x,t) = \frac{\gamma \mu_2(t) + \mu_1(t)}{1+\gamma \ell}x + \frac{\mu_2(t)-\ell \mu_1(t)}{1+\gamma \ell}
    \end{equation*}
\end{problem}

\textbf{Розв'язання.} Підставимо $v(x,t) := A(t)x+B(t)$, себто $v'(x,t) = A(t)$. Звідси маємо 
$A(t) - \gamma B(t) = \mu_1(t)$ та $A(t)\ell + B(t) = \mu_2(t)$. Звідси:
\begin{equation*}
    \begin{pmatrix}
        A(t) \\ B(t)
    \end{pmatrix} = \begin{pmatrix}
        1 & -\gamma \\
        \ell & 1
    \end{pmatrix}^{-1} \begin{pmatrix}
        \mu_1(t) \\ \mu_2(t)
    \end{pmatrix} = \frac{1}{1+\gamma \ell}\begin{pmatrix}
        1 & \gamma \\
        -\ell & 1
    \end{pmatrix} \begin{pmatrix}
        \mu_1(t) \\ \mu_2(t)
    \end{pmatrix} = \frac{1}{1+\gamma \ell}\begin{pmatrix}
        \mu_1(t) + \gamma \mu_2(t) \\ -\ell \mu_1(t) + \mu_2(t)
    \end{pmatrix}
\end{equation*}

Підставивши $A(t)$ та $B(t)$ у формулу для $v(x,t)$, отримаємо відповідь:
\begin{equation*}
    v(x,t) = \frac{\gamma \mu_2(t) + \mu_1(t)}{1+\gamma \ell}x + \frac{\mu_2(t)-\ell \mu_1(t)}{1+\gamma \ell}
\end{equation*}

\subsection{Вправа 15.3}

\begin{problem}
    Розв'язати рівняння:
    \begin{equation*}
        \frac{\partial u}{\partial t} = \frac{\partial^2 u}{\partial x^2} + u(x,y), \quad 0 < x < 1,\; t > 0
    \end{equation*}

    Крайові умови $u(0,x)=3$, $u(t,0)=u(t,1)=2$.
\end{problem}

\textbf{Розв'язання.} Для початку треба зробити граничні умови однорідними. 
Маємо умову виду $u(0,t) = \mu_1(t)$, $u(\ell, t) = \mu_2(t)$, для якої 
перехід до однорідних граничних умов має вигляд:
\begin{equation*}
    u(x,t) := v(x,t) + w(x,t), \quad v(x,t) := \frac{x}{\ell}(\mu_2(t)-\mu_1(t)) + \mu_1(t)
\end{equation*}

Оскільки в нашому випадку $\mu_1(t)=\mu_2(t) \equiv 2$, то достатньо лише зробити 
здвиг $u(x,t) := 2 + w(x,t)$. Таким чином, отримаємо рівняння\footnote{Тут можна 
і далі спростити заміною $q(x,t) := w(x,t)e^t$, щоб позбутися $w(x,t)$ в правій частині, але 
в даному випадку це не є обов'язковим.}:
\begin{equation*}
    \frac{\partial w}{\partial t} = \frac{\partial^2 w}{\partial x^2} + w(x,t) + 2, \quad 0 < x < 1,\; t > 0
\end{equation*}

Крайові умови при цьому $w(x,0)=1$ та $w(0,t)=w(1,t)=0$. Шукатимемо розв'язок 
у вигляді $w(x,t) = \sum_{n=1}^{\infty}w_n(t)\sin(n\pi x)$. Підставимо у диференціальне рівняння:
\begin{equation*}
    \sum_{n=1}^{\infty}\frac{\partial w_n}{\partial t}\sin(n\pi x) = \sum_{n=1}^{\infty}(1-\pi^2 n^2)w_n(t)\sin(n\pi x) + 2
\end{equation*}

Розкладемо $2$ у ряд Фур'є. Cкористаємося тим, що $1 = \frac{4}{\pi}\sum_{n=1,n \; \text{непарне}}^{\infty}\frac{\sin n \pi x}{n}$, тому:
\begin{equation*}
    \sum_{n=1}^{\infty}\frac{\partial w_n}{\partial t}\sin(n\pi x) = \sum_{n=1}^{\infty}(1-\pi^2 n^2)w_n(t)\sin(n\pi x) + \sum_{n=1, n \; \text{непарне}}^{\infty}\frac{8}{n\pi}\sin n\pi x
\end{equation*} 

Таким чином, маємо наступні диференціальні рівняння для визначення $w_n$:
\begin{align*}
    \dot{w}_n = (1-\pi^2 n^2)w_n + \frac{8}{n\pi}, & \quad n \; \text{непарне},\\
    \dot{w}_n = (1-\pi^2 n^2)w_n, & \quad n \; \text{парне}
\end{align*}

При цьому, маємо умову Коші $w(x,0)=1$, що означає, що
\begin{equation*}
    w(x,0) = \sum_{n=1}^{\infty}w_n(0)\sin(n\pi x) = 1 \implies w_n(0) = \begin{cases}
        0, & n \; \text{парне} \\
        \frac{4}{\pi n}, & n \; \text{непарне}
    \end{cases}
\end{equation*}

Отже, розглядаємо два випадки.

\textcolor{blue}{\textbf{Випадок 1. $n$ парне.}} Маємо $w_n = A_n e^{(1-\pi^2n^2)t}$. Оскільки $w_n(0)=0$, то $A_n=0$.

\textcolor{blue}{\textbf{Випадок 2. $n$ непарне.}} Маємо $w_n = A_n
e^{(1-\pi^2n^2)t} + f_n(t)$, де $f_n(t)$ --- частковий розв'язок. Можна 
показати, що наприклад $f_n(t) = \frac{8}{n\pi(1-\pi^2n^2)}(1-e^{(1-\pi^2n^2)t})$ підходить. Залишається підставити 
$w_n(0) = \frac{4}{\pi n}$ аби знайти $A_n$. Маємо $A_n = \frac{4}{\pi n}$ і тому:
\begin{equation*}
    w_n(t) = \frac{4}{\pi n}e^{(1-\pi^2n^2)t} + \frac{8}{n\pi(1-\pi^2n^2)}(1-e^{(1-\pi^2n^2)t}) 
\end{equation*}

Отже, остаточно:
\begin{equation*}
    \textcolor{blue}{u(x,t) = 2+\sum_{n=1, n \; \text{непарне}}^{\infty}\left(\frac{4}{\pi n}e^{(1-\pi^2n^2)t} + \frac{8}{n\pi(1-\pi^2n^2)}(1-e^{(1-\pi^2n^2)t})\right)\sin(n\pi x)}
\end{equation*}

 
\subsection{Вправа 15.5}

\begin{problem}
    Розв'язати рівняння:
    \begin{equation*}
        \frac{\partial u}{\partial t} = \frac{\partial^2 u}{\partial x^2} + u(x,t) - x + 2 \sin 2x \cos x, \quad 0 < x < \frac{\pi}{2},\; t > 0
    \end{equation*}

    Крайові умови $u(x,0)=x$, $u(0,t)=0$, $u'(\frac{\pi}{2},t) = 1$.
\end{problem}

\textbf{Розв'язання.} Зробимо граничні умови однорідними. Маємо умову виду 
$u(0,t)=\mu_1(t)$ та $u'(\ell,t)=\mu_2(t)$. Підстановка має вигляд:
\begin{equation*}
    u(x,t) = v(x,t) + w(x,t), \quad v(x,t) = \mu_2(t)x + \mu_1(t).
\end{equation*}

Зважаючи на те, що $\mu_1(t)=0$, то достатньо зробити просту заміну 
$u(x,t) = w(x,t) + x$. Таким чином, отримаємо:
\begin{equation*}
    \frac{\partial w}{\partial t} = \frac{\partial^2 w}{\partial x^2} + w(x,t) + 2 \sin 2x \cos x, \quad 0 < x < \frac{\pi}{2},\; t > 0,
\end{equation*}

з крайовими умовами $w(x,0)=0$, $w(0,t)=0$, $w'(\frac{\pi}{2},t) = 0$.

Зробимо заміну $w(x,t) := q(x,t)e^{t}$, щоб позбавитися від $w(x,t)$ у правій частині. Тоді:
\begin{equation*}
    q(x,t)e^t + \frac{\partial q}{\partial t}e^t = \frac{\partial^2 q}{\partial x^2}e^t + q(x,t)e^t + 2 \sin 2x \cos x, \quad 0 < x < \frac{\pi}{2},\; t > 0,
\end{equation*}

Або, якщо спростити:
\begin{equation*}
    \frac{\partial q}{\partial t} = \frac{\partial^2 q}{\partial x^2} + 2 \sin 2x \cos x e^{-t}, \quad 0 < x < \frac{\pi}{2},\; t > 0,
\end{equation*}

з крайовими умовами $q(x,0)=0$, $q(0,t)=0$, $q'(\frac{\pi}{2},t) = 0$. В такому випадку, нам 
потрібно обрати базис $e_n = \sqrt{\frac{2}{\ell}}\sin \frac{\pi (2n-1)x}{2\ell}$, де в нашому 
випадку $\ell = \frac{\pi}{2}$. Тому, нехай:
\begin{equation*}
    q(x,t) = \sum_{n=1}^{\infty}q_n(t)\sin(2n-1)x
\end{equation*}

Підставимо умову $q(x,0)=0$. Маємо:
\begin{equation*}
    q(x,0) = \sum_{n=1}^{\infty}q_n(0)\sin(2n-1)x = 0 \implies q_n(0) = 0
\end{equation*}

Нарешті, підставимо у рівняння:
\begin{equation*}
    \sum_{n=1}^{\infty}\dot{q}_n(t)\sin(2n-1)x = -\sum_{n=1}^{\infty}(2n-1)^2q_n(t) \sin (2n-1)x + 2 \sin 2x \cos x e^{-t}
\end{equation*}

Отже, залишилося розкласти праву частину у ряд Фур'є. Для цього, помітимо 
\begin{equation*}
    2 \sin 2x \cos x = \sin(2x+x) + \sin(2x-x) = \sin 3x + \sin x
\end{equation*}

Таким чином, рівняння треба розглянути окремо для $n=1$, $n=2$ та $n>2$.

\textcolor{blue}{\textbf{Випадок 1. $n=1$.}} Маємо:
\begin{equation*}
    \dot{q}_1 = -q_1 + e^{-t}, \quad q_1(0) = 0
\end{equation*}

Розв'язком цього рівняння є $q_1(t) = te^{-t}$.

\textcolor{blue}{\textbf{Випадок 2. $n=2$.}} Маємо:
\begin{equation*}
    \dot{q}_2 = -9q_2 + e^{-t}, \quad q_2(0) = 0
\end{equation*}

Розв'язком цього рівняння є $q_2(t) = \frac{1}{8}e^{-9t}(-1+e^{8t})$.

\textcolor{blue}{\textbf{Випадок 3. $n>2$.}} Маємо:
\begin{equation*}
    \dot{q}_n = -(2n-1)^2q_n, \quad q_n(0) = 0
\end{equation*}

Загальним розв'язком цього рівняння є $q_n(t) = A_n e^{-(2n-1)^2t}$.
Проте, враховуючи початкову умову $q_n(0)=0$, маємо $A_n=0$.
Отже, $q_n(t) = 0$.

Остаточно, отримуємо:
\begin{equation*}
    \textcolor{blue}{q(x,t) = \sum_{n=1}^{\infty}q_n(t)\sin(2n-1)x = te^{-t}\sin x + \frac{1}{8}e^{-9t}(-1+e^{8t})\sin 3x}
\end{equation*}

Повернемось до $u(x,t)$. Маємо
\begin{equation*}
    w(x,t) = q(x,t)e^t = t\sin x + \frac{1}{8}(1-e^{-8t})\sin 3x
\end{equation*}

Остаточна відповідь у свою чергу:
\begin{equation*}
    \textcolor{blue}{u(x,t) = w(x,t) + x = t\sin x + \frac{1}{8}(1-e^{-8t})\sin 3x + x}
\end{equation*}

\textbf{Відповідь.} $u(x,t) = t\sin x + \frac{1}{8}(1-e^{-8t})\sin 3x + x$.

\end{document}