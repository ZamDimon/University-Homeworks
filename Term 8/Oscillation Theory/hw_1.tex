\documentclass{hw_template}

\usepackage{arydshln}

\title{\bfseries Домашня Робота \#1 з Теорії Коливань}
\author{\bfseries Захаров Дмитро}
\date{15 лютого, 2025}

\begin{document}

\pagestyle{fancy}

\maketitle

\section{Задача 3}


\begin{problem}
    Частинка маси $m$ знаходиться в полі сили тяжіння і може ковзати 
    без тертя вздовж лінії (по поверхні), заданої рівнянням (вісь 
    $Oz$ спрямована вертикально догори). Знайдіть положення рівноваги
    частинки та дослідіть їхню стійкість.
    \begin{enumerate}[(A)]
        \item $z=2xy$.
        \item $z=x^4 - 2x^2y + 2y^2$
    \end{enumerate}
\end{problem}

\textbf{Розв'язання.} 

\textbf{Пункт (А).} Маємо потенціальну енергію $V(x,y) = 2mgxy$. Очевидно, що
єдине положення рівноваги відповідає точці $(0,0)$. Проте, ця точка є напівстійкою. 
Дійсно, розглянемо маленьке відхилення точки. В такому разі:
\begin{equation*}
    V(\varepsilon, \varepsilon) = 2mg\varepsilon^2 > 0, \quad V(\varepsilon, -\varepsilon) = -2mg\varepsilon^2 < 0.
\end{equation*}

Отже, при малих збуреннях, потенціальна енергія може як зростати, так і спадати,
що свідчить про напівстійкість точки $(0,0)$.

\textbf{Пункт (Б).} Маємо потенціальну енергію $V(x,y) = mg(x^4 - 2x^2y + 2y^2)$.
Тут положення рівноваги знаходиться дещо складніше. Знайдемо часткові похідні:
\begin{equation*}
    \frac{\partial V}{\partial x} = mg(4x^3-4xy), \quad \frac{\partial V}{\partial y} = mg(-2x^2+4y).
\end{equation*}

Прирівняємо їх до нуля: $x(x^2-y)=0$ та $-x^2+2y=0$. Отже, перша точка рівноваги 
відповідає $x=0$ з першого рівняння, а отже $y=0$ з другого. Друга точка рівноваги
відповідає $x^2=y$ з першого рівняння, проте підставляючи це у друге, отримуємо
так само $x=0$, $y=0$. Отже, єдине положення рівноваги відповідає точці $(0,0)$.

Тепер помітимо, що функція $V(x,y)$ може бути записана як:
\begin{equation*}
    V(x,y) = mg\left(2\left(y-\frac{x^2}{2}\right)^2 + \frac{x^4}{2}\right)
\end{equation*}

Отже, функція є додатно визначеною і є нульовою лише в точці $(0,0)$. Таким чином,
маємо $V(0,0)=0$ та $V(x,y) > 0$ для всіх інших точок. Отже, точка $(0,0)$ є
стійкою.

\section{Задача 4}

\begin{problem}
    Намистинка маси $m$ насаджена на гладке дротяне кільце з радіусом $R$ і може
ковзати по ньому без тертя. Кільце обертається зі сталою кутовою швидкістю $\Omega$
навколо свого нерухомого вертикального діаметру. Знайдіть положення відносної
рівноваги намистинки та визначте їхню стійкість.
\end{problem}

\textbf{Розв'язання.} Нехай намистинка зрушилась на кут $\phi$. Тоді її
потенціальна енергія поля тяжіння дорівнює $V_G(\phi) = mgR(1-\cos\phi)$, якщо
рахувати від найнижчої точки. Також, треба врахувати енергію кутової кінетичної
енергії намистинки, яка дорівнює $T(\phi) = -\frac{1}{2}mR^2\Omega^2\sin^2\phi$.
Таким чином, загальна функція енергії:
\begin{equation*}
    V(\phi) = T(\phi) + V_G(\phi) = -\frac{1}{2}mR^2\Omega^2\sin^2\phi + mgR(1-\cos\phi).
\end{equation*}

Знайдемо точки рівноваги. Для цього знайдемо похідну функції $V(\phi)$:
\begin{equation*}
    \frac{dV}{d\phi} = -mR^2\Omega^2\sin\phi\cos\phi + mgR\sin\phi = mR \sin\phi(g - \Omega^2 R \cos\phi)
\end{equation*}

Таким чином, маємо три можливості: дві тривіальні це $\sin\phi = 0$, що
відповідає нижній точці $\phi=0$ та $\phi=\pi$, що відповідає верхній точці.
Третя можливість це $g - \Omega^2 R \cos\phi_0 = 0$, звідки $\cos\phi_0 =
\frac{g}{\Omega^2 R}$. Таким чином, маємо ще два положення рівноваги $\phi_0 =
\pm\arccos\left(\frac{g}{\Omega^2 R}\right)$. Помітимо, що такі точки існують 
лише за умови $g < \Omega^2 R$.

Для аналізу стійкості, візьмемо другу похідну:
\begin{equation*}
    \frac{d^2V}{d\theta^2} = mR^2\Omega^2(\sin^2\phi - \cos^2\phi) + mgR \cos \phi
\end{equation*}

Підставляємо різні значення. $V''(0) = -mR^2\Omega^2 + mgR = mR(g-\Omega^2 R)$. Бачимо, 
що за умовою $g>\Omega^2 R$, нульова точка буде стійкою, а інакше --- ні. В свою чергу
$V''(\pi) = -mR^2\Omega^2 - mgR < 0$, а отже верхня точка рівноваги є завжди нестійкою.

Тепер підставимо $\phi_0$:
\begin{align*}
    V''(\phi_0) &= mR^2\Omega^2(1 - 2\cos^2\phi_0) + mgR \cos \phi_0 \\
    &= mR^2\Omega^2 - 2m\frac{g^2}{\Omega^2} + m\frac{g^2}{\Omega^2} \\
    &= mR^2\Omega^2 - m\frac{g^2}{\Omega^2} \\
    &= mR^2\left(\Omega^2 - \frac{g^2}{\Omega^2R^2}\right)
\end{align*}

Отже, за умовою $\Omega^4R^2>g^2$ або $\Omega^2R>g$, точка $\phi_0$ є стійкою. Проте,
оскільки це умова існування точки рівноваги, то це завжди буде виконуватись. Таким чином,
положення рівноваги $\phi_0$ є стійким.

\section{Задача 5}

\begin{problem}
    \textbf{Стійкість циліндрів.} Однорідний циліндр з радіусом $r$ лежить поперек
зверху на нерухомо закріпленому горизонтальному циліндрі з радіусом $R$ і
може перекочуватися ним без ковзання. Визначте умову стійкості
горизонтального положення рівноваги рухомого циліндру.
\end{problem}

\textbf{Відповідь.} Потенціальна енергія системи як функція кута $\theta$
записується як 
\begin{equation*}
    V(\theta)=mg(R\theta\sin\theta + R + r \cos\theta).    
\end{equation*}
 
Візьмемо другу похідну:
\begin{equation*}
    V''(\theta) = -R\theta\sin\theta + 2R\cos\theta - r\cos\theta
\end{equation*}

Маємо $V''(0) = 2R-r$. Ця значення додатнє за умови $r<2R$. За такою 
умовою точка $\theta=0$ є стійкою. 

\textbf{Відповідь.} За умови $r<2R$ горизонтальне положення рівноваги є стійким.

\end{document}