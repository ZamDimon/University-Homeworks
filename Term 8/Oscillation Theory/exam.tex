\documentclass{hw_template}

\usepackage{arydshln}

\title{\bfseries Залікова контрольна робота \\ з Теорії Коливань}
\author{\bfseries Захаров Дмитро}
\date{23 травня, 2025}

\begin{document}

\pagestyle{fancy}

\maketitle

\tableofcontents

\vspace{20px}

\section{Задача 1}

\begin{problem}
    Умови рівноваги голономної системи з ідеальними в'язями.
\end{problem}

\textbf{Відповідь.} Нехай маємо систему частинок $m_1,\dots,m_N$ з координатами
$\mathbf{r}_1,\dots,\mathbf{r}_N$. Положенням рівноваги такої системи будемо
називати таке положення, в якому система частинок має нульову швидкість протягом
певного інтервалу часу, тобто $\dot{\mathbf{r}}_i=0$ для всіх $i=1,\dots,N$.
Нехай усі в'язі, що ми позначаємо як $\mathbf{R}_1,\dots,\mathbf{R}_N$, є
ідеальними. Що це означає? Розглянемо другий закон Ньютона для кожної частинки:
$\mathbf{F}_i+\mathbf{R}_i=m_i\ddot{r}_i$, де $\mathbf{F}_i$ --- зовнішня сила. 
Таким чином, маємо мати:
\begin{equation*}
    \sum_{i=1}^N (\mathbf{F}_i - m_i\ddot{\mathbf{r}}_i + \mathbf{R}_i) \delta \mathbf{r}_i = 0,
\end{equation*}
де $\delta \mathbf{r}_i$ --- віртуальне переміщення частинки $i$ в околі
рівноваги. Ми вважаємо в'язі ідеальними, тому $\sum_{i=1}^N \mathbf{R}_i \delta
\mathbf{r}_i = 0$, а тому рівняння динаміки має вигляд $\sum_{i=1}^N
(\mathbf{F}_i - m_i\ddot{\mathbf{r}}_i)\delta \mathbf{r}_i = 0$. В стані
рівноваги отримуємо \textbf{рівняння статики}:
\begin{equation*}
    \sum_{i=1}^N \mathbf{F}_i \delta \mathbf{r}_i = 0.
\end{equation*}

Нехай маємо узагальнені координати $\mathbf{q}=(q_1,\dots,q_n)$ з відповідними
узагальненими силами $\mathbf{Q}_i=\sum_{j=1}^N \mathbf{F}_j \frac{\partial
\mathbf{r}_j}{\partial q_i}$. Нагадаємо, що під \textbf{голономною системою}
розуміють систему, що описується за допомогою системи рівнянь
$f_i(q_1,\dots,q_n,t)=0$ для всіх $i=1,\dots,k$. Згадаємо, що для 
таких систем:
\begin{equation*}
    \sum_{i=1}^N \mathbf{F}_i \delta \mathbf{r}_i = \sum_{j=1}^n \mathbf{Q}_j\delta \mathbf{q}_j
\end{equation*}

А отже з рівняння статики маємо $\sum_{j=1}^n \mathbf{Q}_j\delta \mathbf{q}_j =
0$. Проте, оскільки усі переміщення $\delta \mathbf{q}_j$ є незалежними, то
для того, щоб рівняння виконувалось, необхідно, щоб усі узагальнені сили
дорівнювали нулю, тобто $\mathbf{Q}_j=0$ для всіх $j=1,\dots,n$. Отже, 
умова рівноваги записується як
\begin{equation*}
    \boxed{\mathbf{Q}_j = 0, \quad j = 1,\dots,n.}
\end{equation*}

Зокрема, у випадку консервативної системи, маємо $\mathbf{Q}_j = -\frac{\partial
V}{\partial \mathbf{q}_j}(\mathbf{q})$, тому у рівновазі $\frac{\partial
V}{\partial \mathbf{q}_j}=0$. Таким чином, точка спокою консервативної 
системи це стаціонарна точка потенціальної енергії як функції від 
узагальнених координат. За теоремою Лагранжа-Діріхле, якщо в точці 
спокою потенціальна енергія має мінімум, то система буде стійкою.

\newpage 

\section{Задача 2}

\begin{problem}
    Параметричні коливання. Теорема Флоке. Параметричний резонанс.
\end{problem}

\textbf{Відповідь.} Одразу наведемо означення параметричних коливань.

\begin{definition}
\textbf{Параметричними коливаннями} називають коливання,
в яких параметри системи, що входять до рівняння руху, змінюються в часі. Такі 
зміни, у свою чергу, можуть викликати резонансні явища, які називаються
\textbf{параметричним резонансом}.
\end{definition}

Простий приклад наступний: нехай маємо математичний маятник, в якому 
довжина нитки $\ell$ є функцією часу: $\ell = \ell(t)$:
\begin{equation*}
    \ddot{\varphi} + \frac{g}{\ell(t)}\sin\varphi = 0.
\end{equation*}

Доволі широкий клас задач з параметричними коливаннями можна звести 
до \textbf{рівняння Хілла}:
\begin{equation*}
    \ddot{\varphi} + \Omega^2(t)\varphi = 0, 
\end{equation*}

де $\Omega(t)$ --- періодична функція. 

\begin{example} Малі періодичні зміни довжини нитки $\ell(t)$ описуються
рівнянням Хілла. Дійсно, можна записати $\ell(t) = \ell_0(1+\varepsilon \cos
\omega t)$, а отже
\begin{equation*}
    \ddot{\varphi} + \frac{g}{\ell_0(1+\varepsilon \cos \omega t)}\sin\varphi = 0.
\end{equation*}

За малих $\varepsilon$, маємо $\frac{1}{1+\varepsilon \cos \omega t} \approx 1 -
\varepsilon \cos \omega t$, тому рівняння зведеться до:
\begin{equation*}
    \ddot{\varphi} + \omega_0^2(1-\varepsilon \cos \omega t)\varphi = 0, \quad \omega_0^2 = \frac{g}{\ell_0},
\end{equation*}

що є рівнянням Хілла з $\Omega(t) = \omega_0\sqrt{1-\varepsilon \cos \omega t}$.
\end{example}

\textbf{Теорія Флоке.} Нехай маємо динамічну систему $\dot{\mathbf{x}} =
\mathbf{A}(t)\mathbf{x}$, де матриця $\mathbf{A}(t) \in \mathbb{R}^{n \times n}$
--- періодична матриця. Нехай її період дорівнює $T$, тобто маємо
$\mathbf{A}(t+T) = \mathbf{A}(t)$. Нас цікавить вид розв'язку такої системи.

\begin{remark} Насправді, таким рівняння можна описати доволі широкий клас
задач, зокрема і рівняння Хілла, оскільки достатньо ввести зміни $\dot{x}_1=x_2$
та $\dot{x}_2 = -\omega^2(t)x_1$, тоді для $\mathbf{x}(t) := (x_1(t),x_2(t))$
будемо мати
\begin{equation*}
    \dot{x} = \begin{pmatrix}
        \dot{x}_1 \\ \dot{x}_2
    \end{pmatrix} = 
    \begin{pmatrix}
        0 & 1 \\ -\omega^2(t) & 0
    \end{pmatrix}
    \begin{pmatrix}
        x_1 \\ x_2
    \end{pmatrix} = \mathbf{A}(t)\mathbf{x}, \quad \mathbf{A}(t) = \begin{pmatrix}
        0 & 1 \\ -\omega^2(t) & 0
    \end{pmatrix}
\end{equation*}
\end{remark}

Позначимо через $\boldsymbol{\Phi}(t,t_0)$ фундаментальну систему розв'язків
системи $\dot{\mathbf{x}} = \mathbf{A}(t)\mathbf{x}$, яка задовольняє початкову
умову $\boldsymbol{\Phi}(t_0,t_0) = \mathbf{E}_{n \times n}$, де $\mathbf{E}_{n
\times n}$ --- одинична матриця. Таким чином, розв'язок системи має вигляд
$\mathbf{x}(t) = \boldsymbol{\Phi}(t,t_0)\mathbf{x}(t_0)$. Нас цікавить 
конкретний вигляд матриці $\boldsymbol{\Phi}(t,t_0)$.

\begin{theorem}[Теорема Флоке]
    Фундаментальну систему розв'язків $\boldsymbol{\Phi}(t,0)$ системи 
    $\dot{\mathbf{x}} = \mathbf{A}(t)\mathbf{x}$ можна записати у вигляді:
    \begin{equation*}
        \boldsymbol{\Phi}(t,0) = \mathbf{P}(t)e^{\mathbf{R}t},
    \end{equation*}
    де $\mathbf{P}(t)$ --- періодична матриця $n \times n$, а $\mathbf{R} =
    \ln(\boldsymbol{\Phi}(T))/T$ є константною матрицею $n \times n$.
\end{theorem}

Залишимо цю теорему без доведення. Проте, цікавим наслідком є наступе: 
кожен момент часу $t$ ми можемо записати як $t=kT+\tau$, де $k \in \mathbb{Z}_{\geq 0}$
та $\tau \in [0,T)$. Тоді, координату $\mathbf{x}(t)$ ми можемо знайти так:
\begin{align*}
    \mathbf{x}(t) = \boldsymbol{\Phi}(kT+\tau,kT)\prod_{j=0}^{k-1}\boldsymbol{\Phi}((j+1)T,jT)\mathbf{x}(0)
\end{align*}

Через періодичність системи маємо: $\textcolor{blue}{\mathbf{x}(t) =
\boldsymbol{\Phi}(kT+\tau,kT)\boldsymbol{\Phi}(T,0)^k}$. Таким чином, стійкість
системи визначається повністю поведінкою $\boldsymbol{\Phi}(T,0)^k$, що в свою 
чергу в силу теореми Флоке визначається лише матрицею $\mathbf{R}$, а саме її 
власними числами. 

\textbf{Стійкість.} Природньо поговорити про стійкість системи. Нехай маємо
певну динамічну систему $\dot{\mathbf{y}} = \mathbf{Y}(t,\mathbf{y})$. Нехай
$\mathbf{y}(t)=\boldsymbol{f}(t)$ є частковим розв'язком --- незбурений рух.
Тоді будь-який розв'язок системи $\mathbf{y}(t)$ називають збуреним рухом і
природньо називати величину $\mathbf{x}(t) := \mathbf{y}(t)-\boldsymbol{f}(t)$
збуренням, причому $\mathbf{x}(t)$ задовільняє системи звичайних диференціальних
рівнянь, що ми називаємо рівнянням збуреного руху: $\dot{\mathbf{x}} =
\mathbf{X}(t,\mathbf{x})$, де $\mathbf{X}(t,\mathbf{x}) =
\mathbf{Y}(t,\mathbf{x}+\boldsymbol{f}(t)) - \mathbf{Y}(t,\boldsymbol{f}(t))$.
Це рівняння вочевидь має тривіальний розв'язок $\mathbf{x}(t) \equiv
\mathbf{0}$, який відповідає незбуреному руху. Ми називаємо незбурених рух
\textbf{стаціонарним} або автономною, якщо $\mathbf{X}$ не залежить від часу. 

\begin{definition}
    Незбурений рух називають \textbf{стійким за Ляпуновим} якщо
    \begin{equation*}
        (\forall \varepsilon > 0) \, (\exists \delta > 0) \, \{\mathbf{x}(t_0) \in \mathcal{U}_{\delta}(\mathbf{0}) \implies \mathbf{x}(t) \in \mathcal{U}_{\varepsilon}(\mathbf{0}), \; t > t_0\}.
    \end{equation*}
\end{definition}

Зупинимось на випадку автономної системи. Оскільки систему $\dot{\mathbf{x}} =
\mathbf{X}(\mathbf{x})$ складно розглядати в загальному вигляді, то для 
аналізу безпосередньо стійкості достатньо \textit{лінеаризувати} її 
в околі $\mathbf{0}$:
\begin{equation*}
    \dot{\mathbf{x}} = \mathbf{A}\mathbf{x} + \mathcal{O}(\|\mathbf{x}\|^2), \quad \mathbf{x} \to \mathbf{0}
\end{equation*}

Тут матриця $\mathbf{A} = \frac{\partial \mathbf{X}(0)}{\partial \mathbf{x}} \in
\mathbb{R}^{n \times n}$. Рівняння без малого (відносно $\mathbf{x} \to
\mathbf{0}$) доданку $\mathcal{O}(\|\mathbf{x}\|^2)$ називають \textit{рівняння
першого наближення}. Ляпунов показав, що стійкість незбуреного руху можна
описати за допомогою спектру матриці $\mathbf{A}$, що позначаємо як
$\sigma(\mathbf{A}) \subseteq \mathbb{C}$.

\begin{theorem}[Теорема Ляпунова]
    Якщо для всіх власних значень $\lambda_i \in \sigma(\mathbf{A})$ виконується 
    $\text{Re}(\lambda_i) < 0$, то незбурених рух є асимптотично стійким за 
    Ляпуновим. Якщо ж знайшовся хоча б один власний вектор $\lambda_i \in
    \sigma(\mathbf{A})$ з $\text{Re}(\lambda_i) > 0$, то незбурених рух є
    асимптотично нестійким за Ляпуновим. 
\end{theorem}

\newpage

\section{Задача 3}

\begin{problem}
    Скласти рівняння руху (рівняння Лагранжа) та визначити період малих коливань
однорідного диска маси $m$ і радіуса $r$, закріпленого двома пружинами
жорсткості $k$, який може котитися без проковзування по горизонтальній поверхні.
\textit{(20 балів)}
\end{problem}

\textbf{Розв'язання.} Нехай $x$ --- зміщення центра мас диска вздовж 
горизонтальної осі, $\varphi$ --- кутова координата диска. Тоді умова 
без проковзування має вигляд $x = r\varphi$, звідки $\dot{x} = r\dot{\varphi}$.

При цьому, потенціальна енергія кожної з пружин дорівнює $\frac{1}{2}kx^2$, а
отже сумарна потенціальна енергія системи дорівнює $\textcolor{blue}{V(x) =
kx^2}$.

Кінетична енергія системи дорівнює сумі обертальної кінетичної енергії диска
$\frac{1}{2}I\omega^2$ та поступальної кінетичної енергії $\frac{1}{2}mv^2$. 
Момент інерції диска відносно осі, що проходить через його центр мас дорівнює
$I=\frac{1}{2}mr^2$ і як ми вже з'ясували, $\omega = \frac{\dot{x}}{r}$, тому
\begin{equation*}
    \textcolor{blue}{K(\dot{x})} = \frac{1}{2}I\omega^2 + \frac{1}{2}mv^2 = \frac{1}{2}\cdot\frac{1}{2}mr^2\cdot\left(\frac{\dot{x}}{r}\right)^2 + \frac{1}{2}m\dot{x}^2 = \textcolor{blue}{\frac{3}{4}m\dot{x}^2}.
\end{equation*}

Отже, рівняння Лагранжа має вигляд:
\begin{equation*}
    \textcolor{blue}{\mathcal{L}(x,\dot{x}) = K(\dot{x}) - V(x) = \frac{3}{4}m\dot{x}^2 - kx^2}.
\end{equation*}

Згадаємо, що рівняння Лагранжа має вигляд $\frac{d}{dt} \frac{\partial \mathcal{L}}{\partial \dot{x}} - \frac{\partial\mathcal{L}}{\partial x} = 0$, тому
\begin{align*}
    \frac{d}{dt} \frac{\partial \mathcal{L}}{\partial \dot{x}} &= \frac{d}{dt} \left(\frac{3}{2}m\dot{x}\right) = \frac{3}{2}m\ddot{x}, \\
    \frac{\partial \mathcal{L}}{\partial x} &= -2kx.
\end{align*}

Таким чином, рівняння руху має вигляд:
\begin{equation*}
    \frac{3}{2}m\ddot{x} + 2kx = 0 \Rightarrow \boxed{\textcolor{blue}{\ddot{x} + \frac{4k}{3m}x = 0}}.
\end{equation*}

Таким чином, циклічна частота має вигляд $\omega = \sqrt{\frac{4k}{3m}} =
2\sqrt{\frac{k}{3m}}$, а період малих коливань у свою чергу тоді:
\begin{equation*}
    \textcolor{blue}{T = \frac{2\pi}{\omega} = \boxed{\pi\sqrt{\frac{3m}{k}}}}.
\end{equation*}

\end{document}