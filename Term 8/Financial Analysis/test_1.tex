\documentclass{test_template}

\usepackage{arydshln}

\title{\bfseries Контрольна робота \#1 \\з Фінансового Аналізу}
\author{\bfseries Захаров Дмитро, Варіант 3}
\date{10 квітня, 2025}

\begin{document}

\pagestyle{fancy}

\maketitle

\section{Задача}

\begin{problem}
    \textbf{Дано}: Ковариаційна матриця дохідностей ризикового активу $V$,
вектор $\mathbf{m}$ дохідностей ризикових активів, ефективність безризикового
цінного паперу $m_0$.

Скласти портфель Тобіна та обчислити ефективність та ризик складеного портфелю
\begin{enumerate}
    \item Мінімального ризику ефективності не нижче $\overline{m}+1$.
    \item Максимальної ефективності, ризик якого дорівнює заданому числу $\nu_0$.
    \item Максимальної ефективності, ризик якого не перевищує числа $\nu_0-1$.
    \item Мінімального ризику заданої ефективності $\overline{m}-1$ при
    умові заборони на короткий продаж 1-го цінного паперу.
\end{enumerate}

\begin{equation*}
    V = \begin{pmatrix}
        3 & -1 & 0 \\
        -1 & 2 & -1 \\
        0 & -1 & 3
    \end{pmatrix}, \quad \mathbf{m} = \begin{pmatrix}
        7 \\ 1 \\ 4
    \end{pmatrix}, \quad \overline{m} = 6, m_0 = 3, \nu_0 = 3
\end{equation*}
\end{problem}

\textbf{Розв'язання.} 

\textbf{Пункт 1.} Маємо наступну постановку задачі:
\begin{equation*}
    \begin{cases}
        \mathbf{x}^{\top}V\mathbf{x} \to \min & \textcolor{gray}{\text{Мінімізація ризику}} \\
        x_0 + \mathbf{1}_n^{\top}\mathbf{x} = 1 & \textcolor{gray}{\text{Умова нормування}} \\
        m_0x_0 + \mathbf{m}^{\top}\mathbf{x} \geq \overline{m}+1 & \textcolor{gray}{\text{Ефективність не нижче за $\overline{m}+1$}}
    \end{cases}
\end{equation*}

Для початку, потрібно позбутись $x_0$. Для цього введемо $\mathbf{M} :=
\mathbf{m}-m_0 = (7-3,1-3,4-3)=(4,-2,1)$ та $\hat{m} := \overline{m} + 1 - m_0 =
4$. Тоді задача записується дещо простіше:
\begin{equation*}
    \begin{cases}
        \mathbf{x}^{\top}V\mathbf{x} \to \min \\
        \hat{m} - \mathbf{M}^{\top}\mathbf{x} \leq 0
    \end{cases}
\end{equation*}

\begin{theorem}
    Якщо $\hat{m}>0$, $\mathbf{M} \neq 0$, то розв'язок початкової задачі має вигляд:
    \begin{equation*}
        \mathbf{x} = \frac{\hat{m}V^{-1}\mathbf{M}}{\mathbf{M}^{\top}V^{-1}\mathbf{M}}, \quad x_0 = 1 - \mathbf{1}_n^{\top}\mathbf{x}.
    \end{equation*}
\end{theorem}

\textbf{Доведення.} Складаємо функцію Лагранжа
$\mathcal{L}(\mathbf{x},\lambda)=\mathbf{x}^{\top}V\mathbf{x} + \lambda(\hat{m}
- \mathbf{M}^{\top}\mathbf{x})$. Розв'язок задачі Тобіна в такому разі 
знаходять з системи:
\begin{equation*}
    \begin{cases}
        \partial \mathcal{L}(\mathbf{x},\lambda)/\partial\mathbf{x} = \mathbf{0}_n, \\
        \lambda(\hat{m} - \mathbf{M}^{\top}\mathbf{x}) = 0, \; \lambda \geq 0 \\
        \hat{m} - \mathbf{M}^{\top}\mathbf{x} \leq 0
    \end{cases}
\end{equation*}

Оскільки $\partial\mathcal{L}/\partial\mathbf{x} = 2V\mathbf{x} -
\lambda\mathbf{M}$, то ми образу знаходимо $\mathbf{x} = \frac{1}{2}\lambda
V^{-1}\mathbf{M}$. З другого рівняння якщо $\lambda=0$, то $\mathbf{x} =
\mathbf{0}$ та $x_0=1$ --- тобто вкладаємо все в безризиковий актив. Проте, цей
випадок можливий лише за умови $\hat{m} \geq 0$, що за умовою не виконується.

Отже, $\hat{m} = \mathbf{M}^{\top}\mathbf{x}$. Тому, якщо підставити вираз 
для $\mathbf{x}$, отримуємо
\begin{equation*}
    \lambda = \frac{2\hat{m}}{\mathbf{M}^{\top}V^{-1}\mathbf{M}} \geq 0
\end{equation*}

Отже, підставляючи $\lambda$ у вираз для $\mathbf{x}$, отримаємо
\begin{equation*}
    \textcolor{blue}{\mathbf{x} = \frac{\hat{m}V^{-1}\mathbf{M}}{\mathbf{M}^{\top}V^{-1}\mathbf{M}}}
\end{equation*}

Вираз для $x_0$ знаходиться з умови нормування. \hfill $\blacksquare$

\vspace{10px}

\textcolor{purple}{\textbf{Чисельний розв'язок.}} Обернена матриця
до $V$ має вигляд:
\begin{equation*}
    V^{-1} = \begin{pmatrix}
        \frac{5}{12} & \frac{1}{4} & \frac{1}{12} \\
        \frac{1}{4} & \frac{3}{4} & \frac{1}{4} \\
        \frac{1}{12} & \frac{1}{4} & \frac{5}{12}
    \end{pmatrix}
\end{equation*}

Таким чином, підставляючи:
\begin{equation*}
    \hat{m}V^{-1}\mathbf{M} = 4\begin{pmatrix}
        \frac{5}{12} & \frac{1}{4} & \frac{1}{12} \\
        \frac{1}{4} & \frac{3}{4} & \frac{1}{4} \\
        \frac{1}{12} & \frac{1}{4} & \frac{5}{12}
    \end{pmatrix}\begin{pmatrix}
        4 \\ -2 \\ 1
    \end{pmatrix} = \begin{pmatrix}
        5 \\ -1 \\ 1
    \end{pmatrix}, \quad \mathbf{M}^{\top}V^{-1}\mathbf{M} = \frac{23}{4}
\end{equation*}

Таким чином, остаточно:
\begin{equation*}
    \textcolor{blue}{\mathbf{x} = \begin{pmatrix}
        \frac{20}{23} \\ -\frac{4}{23} \\ \frac{4}{23}
    \end{pmatrix} \implies x_0 = 1-\frac{20}{23} = \frac{3}{23}}.
\end{equation*}

\textbf{Відповідь.} $x_0=\frac{3}{23}$, $x_1=\frac{20}{23}$,
$x_2=-\frac{4}{23}$, $x_3=\frac{4}{23}$.

\vspace{10px}

\textbf{Пункт 2.} Постановка задачі виглядає наступним чином (після позбавлення від $x_0$):
\begin{equation*}
    \begin{cases}
        \mathbf{M}^{\top}\mathbf{x} \to \max & \textcolor{gray}{\text{Максимізація ефективності}} \\
        \mathbf{x}^{\top}V\mathbf{x} = \nu_0^2, & \textcolor{gray}{\text{Ризик дорівнює $\nu_0$}} \\
    \end{cases}
\end{equation*}

Наведемо наступну теорему з лекції.
\begin{theorem}
    Якщо $\mathbf{M} \neq 0$, то розв'язок задачі має вигляд:
    \begin{equation*}
        \mathbf{x} = \frac{\nu_0V^{-1}\mathbf{M}}{\sqrt{\mathbf{M}^{\top}V^{-1}\mathbf{M}}}, \quad x_0 = 1 - \mathbf{1}_n^{\top}\mathbf{x}
    \end{equation*}
\end{theorem}

\textbf{Доведення.} Складаємо функцію Лагранжа:
$\mathcal{L}(\mathbf{x},\lambda)=\mathbf{M}^{\top}\mathbf{x} +
\lambda(\mathbf{x}^{\top}V\mathbf{x}-\nu_0^2)$. Розв'язок задачі Тобіна в такому разі
знаходять з системи:
\begin{equation*}
    \begin{cases}
        \partial\mathcal{L}(\mathbf{x},\lambda)/\partial\mathbf{x} = \mathbf{0}_n, \\
        \partial\mathcal{L}(\mathbf{x},\lambda)/\partial\lambda = 0
    \end{cases}
\end{equation*}

В нашому випадку це еквівалентно системі:
\begin{equation*}
    \begin{cases}
        \mathbf{M} + 2\lambda V\mathbf{x} = 0, \\
        \mathbf{x}^{\top}V\mathbf{x} = \nu_0^2
    \end{cases}
\end{equation*}

З першого рівняння отримуємо $\mathbf{x} = -\frac{1}{2\lambda}V^{-1}\mathbf{M}$.
В підставлені у друге рівняння, отримаємо
\begin{equation*}
    \lambda^2 = \frac{\mathbf{M}^{\top}V^{-1}\mathbf{M}}{4\nu_0^2} \implies 
    \lambda = \pm \frac{\sqrt{\mathbf{M}^{\top}V^{-1}\mathbf{M}}}{2\nu_0}
\end{equation*}

Оскільки ми хочемо отримати максимум, то $\partial^2\mathcal{L}/\partial\mathbf{x}^2 = 2 \lambda V < 0$, 
тому потрібно обрати від'ємний корінь. Отже, $\lambda = -\frac{\sqrt{\mathbf{M}^{\top}V^{-1}\mathbf{M}}}{2\nu_0}$ і тому
\begin{equation*}
    \textcolor{blue}{\mathbf{x} = \frac{\nu_0V^{-1}\mathbf{M}}{\sqrt{\mathbf{M}^{\top}V^{-1}\mathbf{M}}}, \quad x_0 = 1 - \mathbf{1}_n^{\top}\mathbf{x}}. \quad\quad \blacksquare
\end{equation*}

\textcolor{purple}{\textbf{Чисельний розв'язок.}} Підставляємо числа (більшість
ми вже обраховували вище):
\begin{equation*}
    \nu_0V^{-1}\mathbf{M} = 3\begin{pmatrix}
        \frac{5}{12} & \frac{1}{4} & \frac{1}{12} \\
        \frac{1}{4} & \frac{3}{4} & \frac{1}{4} \\
        \frac{1}{12} & \frac{1}{4} & \frac{5}{12}
    \end{pmatrix}\begin{pmatrix}
        4 \\ -2 \\ 1
    \end{pmatrix} = \begin{pmatrix}
        \frac{15}{4} \\ -\frac{3}{4} \\ \frac{3}{4}
    \end{pmatrix}, \quad \sqrt{\mathbf{M}^{\top}V^{-1}\mathbf{M}} = \sqrt{\frac{23}{4}} = \frac{\sqrt{23}}{2}
\end{equation*}

Таким чином, остаточно:
\begin{equation*}
    \textcolor{blue}{\mathbf{x} = \begin{pmatrix}
        \frac{15}{2\sqrt{23}} \\ -\frac{3}{2\sqrt{23}} \\ \frac{3}{2\sqrt{23}}
    \end{pmatrix} \implies x_0 = 1 - \frac{15}{2\sqrt{23}}}
\end{equation*}

\textbf{Відповідь.} $x_0 = 1 - \frac{15}{2\sqrt{23}}$, $x_1 = \frac{15}{2\sqrt{23}}$,
$x_2 = -\frac{3}{2\sqrt{23}}$, $x_3 = \frac{3}{2\sqrt{23}}$.

\vspace{10px}

\textbf{Пункт 3.} Формально маємо наступну постановку задачі:
\begin{equation*}
    \begin{cases}
        \mathbf{M}^{\top}\mathbf{x} \to \max, & \textcolor{gray}{\text{Максимізація ефективності}} \\
        \mathbf{x}^{\top}V\mathbf{x} \leq (\nu_0-1)^2. & \textcolor{gray}{\text{Ризик не перевищує $\nu_0-1$}} \\
    \end{cases}
\end{equation*}

Наведемо наступну теорему з лекції.
\begin{theorem}
    Якщо $\mathbf{M} \neq 0$, то розв'язок задачі має вигляд (для $\widetilde{\nu}_0:=\nu_0-1$):
    \begin{equation*}
        \mathbf{x} = \frac{\widetilde{\nu}_0V^{-1}\mathbf{M}}{\sqrt{\mathbf{M}^{\top}V^{-1}\mathbf{M}}}, \quad x_0 = 1 - \mathbf{1}_n^{\top}\mathbf{x}
    \end{equation*}
\end{theorem}

\textbf{Доведення.} Введемо функцію Лагранжа $\mathcal{L}(\mathbf{x},\lambda) =
-\mathbf{M}^{\top}\mathbf{x} +
\lambda(\mathbf{x}^{\top}V\mathbf{x}-\widetilde{\nu}_0^2)$ (мінус в першому
члені стоїть через те, що нам потрібно мінімізувати вираз
$\mathbf{M}^{\top}\mathbf{x}$). Тоді згідно теореми Куна-Таккера:
\begin{equation*}
    \begin{cases}
        \textcolor{gray}{\partial \mathcal{L}/\partial \mathbf{x}} = \mathbf{0}_n, \\
        \lambda(\mathbf{x}^{\top}V\mathbf{x}-\widetilde{\nu}_0^2) = 0, \; \lambda \geq 0 \\
        \mathbf{x}^{\top}V\mathbf{x} - \widetilde{\nu}_0^2 \leq 0, 
    \end{cases} \iff \begin{cases}
        \textcolor{gray}{2\lambda V\mathbf{x} - \mathbf{M}} = \mathbf{0}_n, \\
        \lambda(\mathbf{x}^{\top}V\mathbf{x}-\widetilde{\nu}_0^2) = 0, \; \lambda \geq 0 \\
        \mathbf{x}^{\top}V\mathbf{x} - \widetilde{\nu}_0^2 \leq 0, 
    \end{cases}
\end{equation*}

Якщо $\lambda=0$, то $\mathbf{M}=\mathbf{0}_n$, що протирічить умові задачі.
Тому $\lambda \neq 0$ і тому $\mathbf{x} = \frac{1}{2\lambda}V^{-1}\mathbf{M}$.
Якщо підставити це у рівняння
$\mathbf{x}^{\top}V\mathbf{x}=\widetilde{\nu}_0^2$, то отримуємо
$\lambda^2=\mathbf{M}^{\top}V\mathbf{M}/4\widetilde{\nu}_0^2$. При взятті
кореня, відкидаємо від'ємний корінь (оскільки $\lambda \geq 0$), тому
$\lambda=\sqrt{\mathbf{M}^{\top}V\mathbf{M}}/2\widetilde{\nu}_0$ і остаточно:
\begin{equation*}
    \textcolor{blue}{\mathbf{x} = \frac{\widetilde{\nu}_0V^{-1}\mathbf{M}}{\sqrt{\mathbf{M}^{\top}V^{-1}\mathbf{M}}}, \quad x_0 = 1 - \mathbf{1}_n^{\top}\mathbf{x}}. \quad \quad \blacksquare
\end{equation*}

\textcolor{purple}{\textbf{Чисельний розв'язок.}} Маємо $\widetilde{\nu}_0=2$, 
тому:
\begin{equation*}
    \widetilde{\nu}_0V^{-1}\mathbf{M} = 2\begin{pmatrix}
        \frac{5}{12} & \frac{1}{4} & \frac{1}{12} \\
        \frac{1}{4} & \frac{3}{4} & \frac{1}{4} \\
        \frac{1}{12} & \frac{1}{4} & \frac{5}{12}
    \end{pmatrix}\begin{pmatrix}
        4 \\ -2 \\ 1
    \end{pmatrix} = \begin{pmatrix}
        \frac{5}{2} \\ -\frac{1}{2} \\ \frac{1}{2}
    \end{pmatrix}, \quad \sqrt{\mathbf{M}^{\top}V^{-1}\mathbf{M}} = \sqrt{\frac{23}{4}} = \frac{\sqrt{23}}{2}
\end{equation*}

Згідно формулі вище, отримуємо розв'язок задачі:
\begin{equation*}
    \textcolor{blue}{\mathbf{x} = \begin{pmatrix}
        \frac{5}{\sqrt{23}} \\ -\frac{1}{\sqrt{23}} \\ \frac{1}{\sqrt{23}}
    \end{pmatrix} \implies x_0 = 1-\frac{5}{\sqrt{23}}}.
\end{equation*}

\textbf{Відповідь.} $x_0 = 1 - \frac{5}{\sqrt{23}}$, $x_1 = \frac{5}{\sqrt{23}}$,
$x_2 = -\frac{1}{\sqrt{23}}$, $x_3 = \frac{1}{\sqrt{23}}$.

\vspace{10px}

\textbf{Пункт 4.} Формальна постановка задачі наступна:
\begin{equation*}
    \begin{cases}
        \mathbf{x}^{\top}V\mathbf{x} \to \min & \textcolor{gray}{\text{Мінімізація ризику}} \\
        m_0x_0 + \mathbf{m}^{\top}\mathbf{x} = \overline{m} - 1 & \textcolor{gray}{\text{Ефективність дорівнює $\overline{m}-1$}} \\
        x_0 + \mathbf{1}_n^{\top}\mathbf{x} = 1 & \textcolor{gray}{\text{Умова нормування}} \\
        x_1 \geq 0 & \textcolor{gray}{\text{Заборона короткого продажу 1-го активу}} \\
    \end{cases}
\end{equation*}

Виключимо $x_0$ з задачі, як і в класичній задачі Тобіна. Вводимо $\hat{m} = \overline{m}-1-m_0=2$, 
вектор $\mathbf{M} = (4,-2,1)$ і тоді задача має вигляд:
\begin{equation*}
    \begin{cases}
        \mathbf{x}^{\top}V\mathbf{x} \to \min \\
        \mathbf{M}^{\top}\mathbf{x} = \hat{m} \\
        x_1 \geq 0
    \end{cases}
\end{equation*}

Застосуємо теорему Куна-Таккера. Введемо функцію Лагранжа (тут в позначеннях лекції 
$h_1(x) = \hat{m}- \mathbf{M}^{\top}\mathbf{x}$ та $f_1(x) = -x_1$):
\begin{equation*}
    \mathcal{L}(\mathbf{x},\lambda,\mu) = \mathbf{x}^{\top}V\mathbf{x} + \lambda(\hat{m} - \mathbf{M}^{\top}\mathbf{x}) - \mu x_1
\end{equation*}

В такому разі, має виконуватись система рівнянь:
\begin{equation*}
    \begin{cases}
        \textcolor{gray}{\partial \mathcal{L}(\mathbf{x},\lambda,\mu)/\partial\mathbf{x}} = \mathbf{0}_n, \\
        \hat{m}- \mathbf{M}^{\top}\mathbf{x} = 0, \\
        x_1 \geq 0, \\
        \mu x_1 = 0, \; \mu \geq 0
    \end{cases} \iff \begin{cases}
        \textcolor{gray}{2V\mathbf{x} - \lambda\mathbf{M} - \mu\mathbf{e}_1 = 0}, \\
        \hat{m}- \mathbf{M}^{\top}\mathbf{x} = 0, \\
        x_1 \geq 0, \\
        \mu x_1 = 0, \; \mu \geq 0
    \end{cases}
\end{equation*}

Виразимо $\mathbf{x}$ з першого рівняння:
\begin{equation*}
    \mathbf{x} = \frac{1}{2}V^{-1}\left(\lambda \mathbf{M} + \mu \mathbf{e}_1\right)
\end{equation*}

Тепер розглянемо випадки. \textcolor{purple}{\textbf{Випадок 1.}} Нехай $\mu=0$. В такому разі, $\mathbf{x} = \frac{\lambda}{2}V^{-1}\mathbf{M}$
і тоді розв'язок задачі збігається з пунктом 1:
\begin{equation*}
    \mathbf{x} = \frac{\hat{m}V^{-1}\mathbf{M}}{\mathbf{M}^{\top}V^{-1}\mathbf{M}}, \quad x_0 = 1 - \mathbf{1}_n^{\top}\mathbf{x}
\end{equation*}

Якщо підставити конкретні значення:
\begin{equation*}
    \textcolor{blue}{\mathbf{x} = \begin{pmatrix}
        \frac{10}{23} \\ -\frac{2}{23} \\ \frac{2}{23}
    \end{pmatrix} \implies x_0 = \frac{13}{23}}.
\end{equation*}

Оскільки $x_1 \geq 0$, то розв'язок підходить.

\textcolor{purple}{\textbf{Випадок 2.}} Нехай $\mu>0$. В такому разі, $x_1=0$ і маємо 
систему рівнянь:
\begin{equation*}
    \begin{cases}
        2V\mathbf{x} - \lambda\mathbf{M} - \mu\mathbf{e}_1 = 0, \\
        \hat{m}- \mathbf{M}^{\top}\mathbf{x} = 0, \\
        x_1 = 0
    \end{cases}
\end{equation*}

Маємо $n-1$ невідомих в $\mathbf{x}$ та дві невідомі $(\lambda,\mu)$, тобто в сумі $n+1$.
Перша строчка містить $n$ рівнянь, друга --- $1$ рівняння, тобто рівнянь нам достатньо. 
Рівняння лінійні і тому можемо їх спокійно розв'язати. Підставимо значення:
\begin{equation*}
    2V\mathbf{x} - \lambda\mathbf{M} - \mu\mathbf{e}_1 = 0 \implies
    2\begin{pmatrix}
        3 & -1 & 0 \\
        -1 & 2 & -1 \\
        0 & -1 & 3
    \end{pmatrix}\begin{pmatrix}
        0 \\ x_2 \\ x_3
    \end{pmatrix} - \lambda\begin{pmatrix}
        4 \\ -2 \\ 1
    \end{pmatrix} - \mu\begin{pmatrix}
        1 \\ 0 \\ 0
    \end{pmatrix} = 0
\end{equation*}

Звідси система рівнянь:
\begin{equation*}
    \begin{cases}
        -2x_2 - 4\lambda - \mu = 0 \\
        4x_2 - x_3 + 2\lambda = 0 \\
        -x_2 + 3x_3 - \lambda = 0 \\
    \end{cases}
\end{equation*}

Додамо другу строчку: $\mathbf{M}^{\top}\mathbf{x} = -2x_2 + x_3 = 2$. Тому остаточно:
\begin{equation*}
    \begin{cases}
        -2x_2 - 4\lambda - \mu = 0 \\
        4x_2 - 2x_3 + 2\lambda = 0 \\
        -2x_2 + 6x_3 - \lambda = 0 \\
        -2x_2 + x_3 = 2
    \end{cases}
\end{equation*}

Розв'язок цього рівняння: $x_2=-1$, $x_3=0$, $\lambda=2$, $\mu=-6$. Який би 
розв'язок не був красивий, проте він не підходить: $\mu<0$, що суперечить 
накладеній умові $\mu \geq 0$. 

\textbf{Відповідь.} \textcolor{blue}{$x_1 = \frac{10}{23}$, $x_2 = -\frac{2}{23}$, $x_3 = \frac{2}{23}$,
$x_0 = \frac{13}{23}$}.

\end{document}