\documentclass{hw_template}

\usepackage{arydshln}

\title{\bfseries Домашня Робота \#1 \\з Фінансового Аналізу}
\author{\bfseries Захаров Дмитро}
\date{18 лютого, 2025}

\begin{document}

\pagestyle{fancy}

\maketitle

\section{Задача 1}

\begin{problem}
    Дано матрицю наслідків 
    \begin{equation*}
        Q = \begin{pmatrix}
            5 & -2 & 6 & -5 & 9 & 4 \\
            7 & 5 & 5 & -3 & 8 & 1 \\
            1 & 3 & -1 & 10 & 5 & 2 \\
            9 & -6 & 7 & 1 & 3 & -4 
        \end{pmatrix}
    \end{equation*}     
    Побудувати матрицю ризиків $R$, а також здійснити вибір рішення за
    правилами Вальда, Севіджа и Гурвіца (з параметром $\alpha=0.65$).
\end{problem}

\textbf{Розв'язання.} Спочатку знайдемо максимальні доходи:
\begin{equation*}
    \hat{q}_1 = 9, \hat{q}_2 = 5, \hat{q}_3 = 7, \hat{q}_4 = 10, \hat{q}_5 = 9, \hat{q}_6 = 4.
\end{equation*}

Будуємо матрицю ризиків:
\begin{equation*}
    R = \begin{pmatrix}
        4 & 7 & 1 & 15 & 0 & 0\\
        2 & 0 & 2 & 13 & 1 & 3\\
        8 & 2 & 8 & 0 & 4 & 2\\
        0 & 11 & 0 & 9 & 6 & 8
    \end{pmatrix}
\end{equation*}

\textbf{Правило Вальда.} Вибираємо $a_{i_0} = \max_i\min_j q_{ij}$. Позначимо 
$a_i := \min_j q_{ij}$, тоді $a_1=-5$, $a_2=-3$, $a_3=-1$, $a_4=-6$. Видно, 
що максимум відповідає значенню $a_3=-1$, тому обираємо рішення $i_0=3$.

\textbf{Правило Гурвіца.} Вибираємо
\begin{equation*}
    a_{i_0}(\alpha) = \max_i \left\{ \alpha \max_j q_{ij} + (1-\alpha)\min_j q_{ij} \right\}.
\end{equation*}

Мінімуми ми вже знайшли: $a_1=-5$, $a_2=-3$, $a_3=-1$, $a_4=-6$. Тепер знайдемо
максимуми $b_i := \max_j q_{ij}$: $b_1=9$, $b_2=8$, $b_3=10$, $b_4=9$. Позначимо 
$c_i := \alpha b_i + (1-\alpha)a_i$. Отримаємо:
\begin{align*}
    c_1 &= 0.65 \cdot 9 + 0.35 \cdot (-5) = 4.1, \\
    c_2 &= 0.65 \cdot 8 + 0.35 \cdot (-3) = 4.15, \\
    c_3 &= 0.65 \cdot 10 + 0.35 \cdot (-1) = 6.15, \\
    c_4 &= 0.65 \cdot 9 + 0.35 \cdot (-6) = 3.75.
\end{align*}

Отже маємо $a_{i_0}(\alpha) = 6.15$, тому обираємо рішення $i_0=3$.

\textbf{Правило Севіджа.} Знаходимо $c_i := \max_j r_{ij}$:
\begin{equation*}
    c_1 = 15, \quad c_2 = 13, \quad c_3 = 8, \quad c_4 = 11
\end{equation*}

Маємо знайти мінімум з цих значень: $a_{i_0} = 8$, тому обираємо рішення $i_0=3$.

\section{Задача 2}

\begin{problem}
    Розглянемо фінансову операцію, яку пов'язано з випадковим доходом $\xi_1$:
    \begin{gather*}
        \text{Pr}[\xi_1=3.2] = 0.1, \text{Pr}[\xi_1=4.5] = 0.3, \text{Pr}[\xi_1=6.2] = 0.3, \\ \text{Pr}[\xi_1=8.0] = 0.2, \text{Pr}[\xi_1=10.5] = 0.1
    \end{gather*}

    та фінансову операцію, що пов'язано із доходом $\xi_2$:
    \begin{gather*}
        \text{Pr}[\xi_2=4.5] = 0.2, \text{Pr}[\xi_2=5.2] = 0.2, \text{Pr}[\xi_2=8.5] = 0.2, \\ \text{Pr}[\xi_2=10.3] = 0.2, \text{Pr}[\xi_1=11.7] = 0.2
    \end{gather*}

    \begin{itemize}
        \item Знайти ефективність та ризик обох фінансових операцій.
        \item Чи можна надати перевагу однієї з цих фінансових операцій лише за
        ефективністю і ризиком?
        \item Припустимо, що $\xi_1$ та $\xi_2$ --- незалежні випадкові доходи. Знайти
        ефективність та ризик суми $\xi_1 + \xi_2$.
        \item Розглянемо фінансову операцію, випадковий дохід від якої
        описується величиною $\xi_1$. Проведіть диверсифікацію ризику цієї
        операції так, щоб ризик знизився втричі, розглянувши кілька незалежних
        випадкових величин, що мають тий самий розподіл, що і $\xi_1$.
        \item Розглянемо фінансову операцію, випадковий дохід від якої
        описується величиною $\xi_2$. Провести, якщо це можливо, геджування
        ризику у вигляді додавання величини $\xi_3=a\xi_2+b$, лінійно залежної
        від $\xi_2$.
    \end{itemize}
\end{problem}

\textbf{Розв'язання.}

\textbf{Пункт (а).} Маємо наступні ефективності:
\begin{gather*}
    \mathbb{E}[\xi_1] = \sum_i \text{Pr}[\xi_1=x_i]x_i = 6.18, \quad \mathbb{E}[\xi_2] = \sum_i \text{Pr}[\xi_2=x_i]x_i = 8.04
\end{gather*}

Для ризику рахуємо математичні сподівання квадратів:
\begin{gather*}
    \mathbb{E}[\xi_1^2] = \sum_i \text{Pr}[\xi_1=x_i]x_i^2 = 42.456, \quad \mathbb{E}[\xi_2^2] = \sum_i \text{Pr}[\xi_2=x_i]x_i^2 = 72.504
\end{gather*}

Ризики:
\begin{gather*}
    \sigma[\xi_1] = \sqrt{\mathbb{E}[\xi_1^2] - \mathbb{E}[\xi_1]^2} = \sqrt{4.2636} \approx 2.06, \\
    \sigma[\xi_2] = \sqrt{\mathbb{E}[\xi_2^2] - \mathbb{E}[\xi_2]^2} = \sqrt{7.8624} \approx 2.80
\end{gather*}

\textbf{Пункт (б).} Не можна, оскільки хоч друга операція має більше ефективність,
але має більший ризик.

\textbf{Пункт (в).} Математичне сподівання суми можна знайти без використання
умови на незалежність:
\begin{equation*}
    \mathbb{E}[\xi_1 + \xi_2] = \mathbb{E}[\xi_1] + \mathbb{E}[\xi_2] = 14.22
\end{equation*}

Для ризику вже потрібно використовувати умову на незалежність:
\begin{equation*}
    \sigma[\xi_1 + \xi_2] = \sqrt{\sigma[\xi_1]^2 + \sigma[\xi_2]^2} = \sqrt{12.126} \approx 3.48
\end{equation*}

\textbf{Пункт (г).} Нехай $\eta_1,\dots,\eta_n$ --- незалежні випадкові величини,
що мають той самий розподіл, що і $\xi_1$. Введемо у розгляд випадкову величину 
$\zeta := \frac{1}{n}\sum_{j=1}^n\eta_j$. В такому разі будемо мати таке саме 
математичне сподівання $\mathbb{E}[\zeta] = \mathbb{E}[\xi_1] = 6.18$, але ризик
буде зменшуватися: $\sigma[\zeta] = \sigma[\xi_1]/\sqrt{n}$. Щоб він зменшився 
втричі, потрібно взяти $n=9$ випадкові величини.

\textbf{Пункт (д).} Маємо $\xi_3 = a\xi_2 + b$. Для геджування ризику потрібно
накласти умову $\mathbb{E}[\xi_3]=0$, себто $a\mathbb{E}[\xi_2]+b=0$, звідки 
$b=-a\mathbb{E}[\xi_2]$. Дисперсія, у свою чергу:
\begin{equation*}
    \text{Var}[\xi_2+\xi_3] = \text{Var}[(1+a)\xi_2 + b] = (1+a)^2\text{Var}[\xi_2]
\end{equation*}

Потрібно, аби $\text{Var}[\xi_2+\xi_3] < \text{Var}[\xi_2]$, себто $(1+a)^2<1$.
Тому, достатньо обрати будь-який $a \in (-2, 0)$. Оберемо $a=-1$. 
В такому разі $b=\mathbb{E}[\xi_2]=8.04$ і тоді $\xi_3=-\xi_2+8.04$. 

\end{document}