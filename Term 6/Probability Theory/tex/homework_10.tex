%! TEX program = pdflatex

\documentclass[oneside,solution]{karazin-prob-theory-assign}

\usepackage[utf8]{inputenc}
\usepackage[english,ukrainian]{babel}

\usepackage{dsfont}
\usepackage{float}

\title{Домашня робота}
\author{Захаров Дмитро}
\studentID{МП-31}
\instructor{Півень О.Л.}
\date{\today}
\duedate{18:00 1 травня, 2024}
\assignno{10}
\semester{Весняний семестр 2024}
\mainproblem{Закон великих чисел}

\begin{document}

\maketitle

% \startsolution[print]

\problem{Завдання 1} 

\hspace{20px}\textbf{Умова.} Дана послідовність незалежних випадкових величин $\{\xi_n\}_{n \in \mathbb{N}}$. Випадкова величина $\xi_n, n \geq 3$ може приймати тільки три значення $-\sqrt{n},0,\sqrt{n}$ з ймовірностями, що дорівнюють, відповідно $\frac{1}{n},1-\frac{2}{n},\frac{1}{n}$, величини $\xi_1,\xi_2$ мають дисперсію. Чи можна застосувати до цієї послідовності закон великих чисел?

\textbf{Розв'язання.} Маємо $\mathbb{E}[\xi_n]=0$ та $\text{Var}[\xi_n] = 2$, також нехай маємо дисперсії $\sigma_i^2 = \text{Var}[\xi_i],\; i\in\{1,2\}$. Тоді справедливо
\begin{equation}
    \forall n \in \mathbb{N}: \text{Var}[\xi_n] \leq \max\{\sigma_1^2, \sigma_2^2, 2\}
\end{equation}

Таким чином, за теоремою Чебишева, закон великих чисел застосовний.

\problem{Завдання 2} 

\hspace{20px}\textbf{Умова.} Дана послідовність незалежних випадкових величин $\{\xi_n\}_{n \in \mathbb{N}}$. Випадкова величина $\xi_n$ може приймати тільки три значення $-\alpha n, 0, \alpha n$ ($\alpha>0$) з ймовірностями, що дорівнюють, відповідно (a) $\frac{1}{2n^2},1-\frac{1}{n^2},\frac{1}{2n^2}$, (б) $\frac{1}{2^n},1-\frac{1}{2^{n-1}},\frac{1}{2^n}$. Чи можна застосувати до цієї послідовності закон великих чисел?

\textbf{Розв'язання.} 

\textit{Пункт а.} $\mathbb{E}[\xi_n]=0$, тому дисперсія:
\begin{equation}
    \text{Var}[\xi_n] = (-\alpha n)^2 \cdot \frac{1}{2n^2} + (\alpha n)^2 \cdot \frac{1}{2n^2} = \alpha^2
\end{equation}

Таким чином, за теоремою Чебишева, закон великих чисел застосовний.

\textit{Пункт б.} Знову $\mathbb{E}[\xi_n]=0$, тому дисперсія:
\begin{equation}
    \text{Var}[\xi_n] = \frac{2 \cdot (\alpha n)^2}{2^n} = \frac{\alpha^2 n^2}{2^{n-1}}
\end{equation}

Легко бачити, що $\lim_{n \to \infty} \text{Var}[\xi_n] = 0$, тому послідовність $\{\text{Var}[\xi_n]\}_{n \in \mathbb{N}}$ обмежена, а отже за теоремою Чебишева, закон великих чисел застосовний.

\end{document}
