%! TEX program = pdflatex

\documentclass[oneside,solution]{karazin-prob-theory-assign}

\usepackage[utf8]{inputenc}
\usepackage[english,ukrainian]{babel}

\title{Домашня робота}
\author{Захаров Дмитро}
\studentID{МП-31}
\instructor{Півень О.Л.}
\date{\today}
\duedate{15:00 21 лютого, 2024}
\assignno{3}
\semester{Весняний семестр 2024}
\mainproblem{Геометрична Ймовірність}

\begin{document}

\maketitle

% \startsolution[print]

\problem{Номер 8.1 (п. 3, 6), Турчин} 

\hspace{20px}\textbf{Умова.} У квадрат $[0,1]\times[0,1]$ навмання кидають точку. Обчислити ймовірність того, що для її координат $(x,y)$ справджуються співвідношення:
\begin{enumerate}
    \item $\min\{y-x^2,x-y^2\}\geq 0$.
    \item $x^2+y^2<\frac{1}{4}$.
\end{enumerate}

\textbf{Розв'язання.} 

\textit{Пункт 1.} Ця умова еквівалентна тому, що одночасно $y-x^2\geq 0, x-y^2 \geq 0$. Геометрично, перша умова еквівалентна тому, що точка лежить вище $y=x^2$, а друга умова -- лежить нижче $y=\sqrt{x}$. Отже, шукана ймовірність:
\begin{equation}
    \int_0^1 (\sqrt{x}-x^2)dx = \boxed{\frac{1}{3}}
\end{equation}

\textit{Пункт 2.} Ця умова означає, що точка лежить у колі радіусу $\frac{1}{2}$ з центром у початку координат. Площа кругу $\frac{\pi}{4}$, проте лише одна чверть лежить у квадраті $[0,1]^2$, тому шукана ймовірність $\boxed{\frac{\pi}{16}}$.

\problem{Номер 8.3 (п. 1), Турчин} 

\hspace{20px}\textbf{Умова.} На відрізок $[0,1]$ навмання кидають пару точок $x,y$. Знайти ймовірність $\max\{x,y\}<\frac{1}{2}$.

\textbf{Розв'язання.} Подія $\max\{x,y\}<\frac{1}{2}$ означає, що як $x$, так і $y$ одночасно меньше $\frac{1}{2}$. Тому $\mathbb{P}(\max\{x,y\}<\frac{1}{2})=\mathbb{P}(x<\frac{1}{2})\mathbb{P}(y<\frac{1}{2}) = \boxed{\frac{1}{4}}$.

\problem{Номер 8.7, Турчин} 

\hspace{20px}\textbf{Умова.} Два судна мають підійти до одного причалу. Моменти підходу суден до причалу -- незалежні випадкові події, рівноможливі протягом доби (позначимо $\tau=24 \; \text{год}$). Знайти ймовірність того, що одному із суден доведеться чекати звільнення причалу, якщо час стоянки першого судна -- $\Delta_1=1 \; \text{год}$, а другого $\Delta_2=2 \; \text{год}$.

\textbf{Розв'язання.} Зафіксуємо час стоянки першого судна $T_1 \in [0,\tau]$. Тоді, чекати доведеться другому судну якщо $T_2$ буде у проміжку між $T_1$ до $\min\{\tau,T_1+\Delta_1\}$, тобто ймовірність такого дорівнює:
\begin{equation}
   f(T_1)=\frac{1}{\tau}\left(\min\{\tau,T_1+\Delta_1\}- T_1\right)
\end{equation}
Далі будемо пробігати від $0$ до $\tau$, щоб знайти шукану ймовірність:
\begin{gather}
    p_1 = \frac{1}{\tau^2}\int_0^{\tau}\left(\min\{\tau,T_1+\Delta_1\}- T_1\right)dT_1 \nonumber \\
    = \frac{1}{\tau^2}\int_0^{\tau-\Delta_1}\Delta_1 dT_1 + \frac{1}{\tau^2}\int_{\tau-\Delta_1}^{\tau}(\tau-T_1)dT_1 \\
    = \frac{\Delta_1(\tau-\Delta_1)}{\tau^2} + \frac{\Delta_1^2}{2\tau^2} = \frac{2\Delta_1\tau - \Delta_1^2}{2\tau^2} = \frac{\Delta_1}{\tau} - \frac{\Delta_1^2}{2\tau^2}
\end{gather}
Аналогічно, якщо фіксувати друге судно, то ймовірність очікування $\frac{\Delta_2}{\tau} - \frac{\Delta_2^2}{2\tau^2}$. Таким чином, відповідь:
\begin{equation}
    \frac{\Delta_1+\Delta_2}{\tau} - \frac{\Delta_1^2+\Delta_2^2}{2\tau^2} = \frac{139}{1152} \approx \boxed{0.12065}
\end{equation}

\problem{Номер 8.8, Турчин} 

\hspace{20px}\textbf{Умова.} на паркетну підлогу навмання кидають монету діаметра $d$. Паркет має форму квадратів зі стороною $a$ ($a > d$). Яка ймовірність того, що монета не перетне жодної сторони квадратів паркету?

\textbf{Розв'язання.} Положення монети задамо координатою центра $(x,y)$, причому обидві координати вибираються рівномірно з відрізку $[-a/2,a/2]$ (будемо вважати, що центр квадратного паркету знаходиться у $(0,0)$). Тоді подія $E$, що відповідає тому, що монета не перетнула паркету, можна записати як:
\begin{equation}
    E = \left\{(x,y): x-\frac{d}{2}>-\frac{a}{2} \wedge y - \frac{d}{2} > -\frac{a}{2} \wedge x + \frac{d}{2} < \frac{a}{2} \wedge y + \frac{d}{2} < \frac{a}{2}\right\}
\end{equation}
Простіше це можна описати як $E=\{(x,y): \left|x\right| < \frac{a-d}{2} \wedge |y|<\frac{a-d}{2}\}$, що відповідає квадрату зі стороною $a-d$. Тому, за геометричною ймовірністю, ймовірність такої події:
\begin{equation}
    \mathbb{P}(E) = \frac{\lambda(E)}{a^2} = \frac{(a-d)^2}{a^2} = \boxed{\left(1 - \frac{d}{a}\right)^2}
\end{equation}

\problem{Номер 8.16, Турчин} 

\hspace{20px}\textbf{Умова.} Дві особи, які вирішили зустрітися протягом години, домовилися, що кожна незалежно від іншої приходить на місце зустрічі у навмання обранний момент зазначенної години.

\begin{enumerate}
    \item Якщо особи домовилися, що кожна чекатиму іншу впродовж $t$ годин, після чого піде з місця зустрічі, то яка ймовірність, що зустріч відбудеться?
    \item Яка ймовірність того, що дана особа прийде на місце зустрічі (a) раніше від іншої, (б) раніше від іншої на час, не менший ніж $q$ годин.
\end{enumerate}


\textbf{Розв'язок.} 

\textit{Пункт 1.} Нехай дві особи прийшли у моменти $\xi,\eta \sim \mathcal{U}[0,1]$. Нам потрібно знайти ймовірність того, що $|\xi-\eta|<t$, тобто наша ймовірність:
\begin{gather}
    \mathbb{P}_{\xi,\eta \sim \mathcal{U}[0,1]} = \frac{\mu\left\{\langle x,y \rangle \in [0,1]^2: |x-y|<t\right\}}{\mu [0,1]^2}\\ = \mu\left\{\langle x,y \rangle \in [0,1]^2: |x-y|<t\right\}
\end{gather}
Легше знайти протилежну подію -- для цього достатньо знайти площи двох прямокутник трикутників зі стороною $1-t$. Така площа дорівнює $(1-t)^2$, тому шукана ймовірність $1-(1-t)^2=\boxed{t(2-t)}$. 

\textit{Пункт 2.} (а) Зафіксуємо $\eta \in [0,1]$, тоді ймовірність того, що $\xi \sim \mathcal{U}[0,1]$ буде меньше за $\eta$ дорівнює $\eta$. Якщо тепер проінтегрувати $\eta$ по всім можливим значенням, то отримаємо $\int_0^1\eta d\eta = \boxed{\frac{1}{2}}$. 

(б) Знову фіксуємо $\eta \in [0,1]$. Ймовірність того, що $\xi \sim \mathcal{U}[0,1]$ буде меньше $\eta$ не меньше ніж на $q$ дорівнює $\max\{0,\eta-q\}$. Тоді загальна ймовірність:
\begin{equation}
    \int_0^1 \max\{0,\eta-q\}d\eta = \int_q^1 (\eta-q)d\eta = \boxed{\frac{(1-q)^2}{2}}
\end{equation}

\problem{Номер 8.18, Турчин} 

\hspace{20px}\textbf{Умова.} Яка ймовірність того, що з трьох навмання взятих відрізків, довжина яких не перевищує $1$, можна побудувати трикутник?

\textbf{Розв'язок.} Нехай ми обрали $a,b,c \sim \mathcal{U}[0,1]$. Щоб з них можна було побудувати трикутник, нам потрібно мати $a+b>c, a+c>b, b+c>a$. Тобто, ми шукаємо ймовірність:
\begin{equation}
    \mathbb{P}_{a,b,c \sim \mathcal{U}[0,1]}\left[a+b>c \wedge a+c>b \wedge b+c>a\right]
\end{equation}
Усі можливі $a,b,c$ на $[0,1]$ описують одиничний куб із центром у $\left(\frac{1}{2},\frac{1}{2},\frac{1}{2}\right)$. Тоді, наша ймовірність може бути описана як:
\begin{gather}
    \mathbb{P}_{a,b,c \sim \mathcal{U}[0,1]}\left[a+b>c \wedge a+c>b \wedge b+c>a\right] \nonumber \\
    = \mu\left(\left\{ \langle x,y,z\rangle \in [0,1]^3: x+y>z \wedge x+z>y \wedge y+z>x \right\}\right)
\end{gather}

Можна обрахувати такий об'єм, а можна діяти наступним чином: зафіксуємо найбільшу сторону $c$. Тоді ймовірність сформувати трикутник з двох інших сторін $a,b$ дорівнює мірі $\mu\left(\left\{(a,b):0 \leq a\leq c, 0 \leq b \leq c, a+b \geq c\right\}\right)$. Така площа є просто площою прямокутного трикутиника з двома сторонами $c$, тобто $\frac{c^2}{2}$. Оскільки $c\in[0,1]$, то загальна ймовірність $\int_0^1 \frac{c^2dc}{2} = \frac{1}{6}$. Оскільки ми зафіксували лише одну сторону з трьох, то аналогічна ймовірність буде, якщо фіксувати $a,b$. Тоді загальна ймовірність дорівнює $\boxed{\frac{1}{2}}$.

\problem{Номер 8.19, Турчин}

\hspace{20px}\textbf{Умова.} На відрізку $[-1,1]$ навмання вибирають дві точки. Нехай $p,q$ -- координати цих точок. Знайти ймовірність того, що квадратне $x^2+px+q=0$:
\begin{enumerate}
    \item має дійсні корені;
    \item не має дійсних коренів.
\end{enumerate}

\textbf{Розв'язок.} Щоб квадратне рівняння $x^2+px+q=0$ мало дійсні корені, дискримінант $p^2-4q$ має бути невід'ємним. Отже, задача полягає у знаходженні
\begin{equation}
    \mathbb{P}_{p,q \sim \mathcal{U}[-1,1]}\left[p^2 \geq 4q\right]
\end{equation}

Помітимо, що геометрично, набір $(p,q)$, де кожна компонента обирається з відрізку $[-1,1]$, задає квадрат з центром у $(0,0)$ зі стороною довжини $2$.

Нехай ми відкладаємо $p$ по $x$, а $q$ по $y$. Тоді подія $p^2 \geq 4q$ геометрично задає частину площини на $\mathbb{R}^2$ під кривою $y = \frac{x^2}{4}$. Таким чином, наша шукана ймовірність:
\begin{equation}
    \mathbb{P}_{p,q \sim \mathcal{U}[-1,1]}\left[p^2 \geq 4q\right] = \frac{\mu(\{\langle x, y \rangle \in [-1,1]^2: y \leq \frac{1}{4}x^2\})}{\mu([-1,1]^2)},
\end{equation}
де $\mu$ -- міра Лебега на $\mathbb{R}^2$. Порахувати $\mu([-1,1]^2)$ дуже легко -- це $4$. А ось з виразом у чисельнику трошки складніше. 

$y=\frac{1}{4}x^2$ задає параболу з вершиною у $(0,0)$, що проходить через точки $(\pm 1, 1/4)$ -- точки на лівій і правій сторін квадрату. Тому таку міру можна обрахувати як:
\begin{equation}
    \mu(\{\langle x, y \rangle \in [-1,1]^2: y \leq \frac{1}{4}x^2\}) = 2 + \int_{-1}^1 \frac{1}{4}x^2dx = \frac{13}{6},
\end{equation}
тут ми додаємо $2$, оскільки треба додати прямутник зі сторонами $2 \times 1$.

Таким чином, остаточна ймовірність дорівнює $\boxed{\frac{13}{24}}$. Відповідно, ймовірність другого пункту $1-\frac{13}{24}=\boxed{\frac{11}{24}}$. 

\problem{Номер 8.21, Турчин}

\hspace{20px}\textbf{Умова.} У круг вписано квадрат. Точку навмання кидають у круг. Знайти умовірність того, що вона потрапить у квадрат.

\textbf{Розв'язок.} Нехай радіус кругу $r$. Тоді, оскільки діагональ квадрату є діаметром кругу, то діагональ квадрата дорівнює $2r$. Це означає, що сторона квадрата $\sqrt{2}r$, а тому площа $2r^2$.

У свою чергу, площа кругу $\pi r^2$. Тому ймовірність потрапляння у квадрат -- відношення площі квадрата до площі круга $\frac{2r^2}{\pi r^2} = \boxed{\frac{2}{\pi}}$.

\problem{Стор. 10, В, №5(а)}

\hspace{20px}\textbf{Умова.} Довести, що $\mathbb{P}(A \Delta B)=\mathbb{P}(A)+\mathbb{P}(B)-2\mathbb{P}(A\cap B)$.

\textbf{Розв'язання.} 
\begin{gather}
    \mathbb{P}(A \Delta B) = \mathbb{P}((A \setminus B) \cup (B \setminus A)) = \mathbb{P}(A \setminus B) + \mathbb{P}(B \setminus A) \nonumber \\
    = \mathbb{P}(A) - \mathbb{P}(A \cap B) + \mathbb{P}(B) - \mathbb{P}(B \cap A) = \mathbb{P}(A)+\mathbb{P}(B)-2\mathbb{P}(A \cap B)
\end{gather}

Тут ми скористалися тим, що події $A \setminus B$ та $B \setminus A$ мають нульовий перетин, а також те, що $\mathbb{P}(A \setminus B) = \mathbb{P}(A) - \mathbb{P}(A \cap B)$ (було доведено на попередній парі). \qed 

\problem{Стор. 10, В, №5(б)}

\hspace{20px}\textbf{Умова.} Довести нерівність трикутника:
\begin{equation}
    \mathbb{P}(A \Delta B) \leq \mathbb{P}(A \Delta C)+\mathbb{P}(C \Delta B)
\end{equation}

\textbf{Розв'язання.} Помітимо наступний факт:
\begin{equation}
    A \Delta B \subset (A \Delta C) \cup (C \Delta B)
\end{equation}
Дійсно,
\begin{gather}
    A \Delta B = (A \Delta B) \Delta (B \Delta C)\nonumber \\ = ((A \Delta C) \cup (C \Delta B)) \setminus ((A \Delta C) \cap (C \Delta B)) \nonumber \\ \subset (A \Delta C) \cup (C \Delta B)
\end{gather}
Отже,
\begin{equation}
    \mathbb{P}(A \Delta B) \leq \mathbb{P}\left((A \Delta C) \cup (C \Delta B)\right) \leq \mathbb{P}(A \Delta C) + \mathbb{P}(C \Delta B)\qed
\end{equation}

\problem{Стор. 10, В, №7}

\hspace{20px}\textbf{Умова.} Нехай $A_k \in \mathcal{F}$, $\mathbb{P}(A_k)=1,k \geq 1$. Довести, щоб $\mathbb{P}(\bigcap_{k \in \mathbb{N}}A_k)=1$.

\textbf{Розв'язок.} 
\begin{gather}
    1 \geq \mathbb{P}\left(\bigcap_{k \in \mathbb{N}}A_k\right) = 1 - \mathbb{P}\left(\overline{\bigcap_{k \in \mathbb{N}}A_k}\right) \nonumber \\
    = 1 - \mathbb{P}\left(\bigcup_{k \in \mathbb{N}}\overline{A}_k\right) \geq 1 - \sum_{k \in \mathbb{N}}\underbrace{\mathbb{P}(\overline{A}_k)}_{=0} = 1
\end{gather}
Отже, ми отримали $1 \geq \mathbb{P}\left(\bigcap_{k \in \mathbb{N}}A_k\right) \geq 1$, тому $\mathbb{P}\left(\bigcap_{k \in \mathbb{N}}A_k\right)=1$. \qed

\problem{Стор. 11, №2(a)}

\hspace{20px}\textbf{Умова.} Довести нерівність Буля:
\begin{equation}
    \mathbb{P}(A \cap B) \geq 1 - \mathbb{P}(\overline{A}) - \mathbb{P}(\overline{B})
\end{equation}
\textbf{Розв'язання.} 
\begin{equation}
    \mathbb{P}(A \cap B) = \mathbb{P}(\Omega \setminus \overline{A \cap B}) = 1 - \mathbb{P}(\overline{A} \cup \overline{B}) \geq 1-\mathbb{P}(\overline{A}) - \mathbb{P}(\overline{B})\qed
\end{equation}

\problem{Стор. 11, №4}

\hspace{20px}\textbf{Умова.} Подія $C$ вдвічі більш ймовірна, ніж $A$, а подія $B$ настільки ж ймовірна, як події $A$ і $C$ разом. Ці події несумісні, і їх об'єднання співпадає з усім простором елементарних подій. Знайти ймовірності $A,B,C$.

\textbf{Розв'язання.} 

\begin{enumerate}
    \item Подія $C$ вдвічі більш ймовірна, ніж $A$ означає, що $\mathbb{P}(C) = 2\mathbb{P}(A)$.
    \item Подія $B$ настільки ж ймовірна, як події $A$ і $C$ разом -- $\mathbb{P}(B) = \mathbb{P}(A \cup C)$.
    \item Ці події несумісні -- тобто попарно і разом є неперетинними.
    \item Об'єднання збігається з простором елементарних подій -- $A \cup B \cup C = \Omega$.
\end{enumerate}

З третього та четвертого пунктів випливає $\mathbb{P}(A) + \mathbb{P}(B)+\mathbb{P}(C)=1$, а другу умову можна записати як $\mathbb{P}(B)=\mathbb{P}(A)+\mathbb{P}(C)$. Якщо позначимо $\mathbb{P}(A)=p_A,\mathbb{P}(B)=p_B,\mathbb{P}(C)=p_C$, то отримаємо
\begin{equation}
    \begin{cases}
        p_C = 2p_A \\
        p_B = p_A+p_C \\
        p_A+p_B+p_C=1
    \end{cases}
\end{equation}
Підставимо перше рівняння у наступні дві:
\begin{equation}
    \begin{cases}
    p_B = 3p_A \\
    3p_A+p_B = 1
    \end{cases}
\end{equation}
Звідси одразу $p_B=\frac{1}{2}$, $p_A=\frac{1}{6}$. Тому $p_C=\frac{1}{3}$.

\textbf{Відповідь.} $p_A=\frac{1}{6},p_B=\frac{1}{2},p_C=\frac{1}{3}$.

\problem{Стор. 11, №9}

\hspace{20px}\textbf{Умова.} Нехай $\mathbb{P}(A) \geq 0.8,\mathbb{P}(B) \geq 0.8$. Довести, що $\mathbb{P}(A\cap B) \geq 0.6$.

\textbf{Розв'язання.} 
\begin{equation}
    \mathbb{P}(A \cup B) = \mathbb{P}(A) + \mathbb{P}(B) - \mathbb{P}(A \cap B) = 1.6 - \mathbb{P}(A \cap B) \leq 1
\end{equation}
Звідси отримуємо $\mathbb{P}(A \cap B) \geq 0.6$. \qed

\end{document}
