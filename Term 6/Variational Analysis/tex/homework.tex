%! TEX program = pdflatex

\documentclass[oneside,solution]{template}

\usepackage[utf8]{inputenc}
\usepackage[english,ukrainian]{babel}

\title{Домашня робота}
\author{Захаров Дмитро}
\studentID{МП-31}
\instructor{Бебія М.О.}
\date{\today}
\duedate{11:20 25 травня}
\assignno{1}
\semester{Весняний семестр 2024}
\mainproblem{Умова Веєрштрасса-Ердмана}

\begin{document}

\maketitle

\section{Завдання}

\textbf{Умова.}
\begin{equation*}
    \begin{cases}
        J(y) = \int_0^1 ((y')^4 - 2(y')^2+3)dx \to \inf \\
        y(0) = y(1) = 0
    \end{cases}
\end{equation*}

\textbf{Розв'язок.} Знайдемо точну нижню грань:
\begin{gather*}
    J(y) = \int_0^1 ((y')^4 - 2(y')^2 + 3)dx = \int_0^1 ((y'^2-1)^2 + 2)dx \\
    = \underbrace{\int_0^1 (y'^2-1)^2dx}_{\geq 0} + \underbrace{2\int_0^1 dx}_{=2} \geq 2
\end{gather*}
Отже, $2$ -- точна нижня грань, причому досягається вона при
\begin{equation*}
    \int_0^1 (y'^2 - 1)^2dx = 0 \implies y'^2 = 1 \implies y' = \pm 1
\end{equation*}
Отже, $J(y^*)=2$ -- точна нижня грань для будь-якої функції
\begin{equation*}
    y^*(x)' = \pm 1, \; y^*(0) = y^*(1) = 0
\end{equation*}

Тепер знайдемо екстремаль з однією точкою зламу. Запишемо рівняння Ейлера:
\begin{equation*}
    \frac{\partial F}{\partial y} - \frac{d}{d x} \frac{\partial F}{\partial y'} = 0, \; F(x,y,y') = y'^4 - 2y'^2+3
\end{equation*}
Підставляючи, маємо:
\begin{equation*}
    0 - \frac{d}{dx}\left(4y'^3 - 4y'\right) = 0 \implies y'(y'^2-1) = \text{const} \implies y' = \text{const}
\end{equation*}
Отже, якщо $y'$ є константою, то $y$ має вигляд $y=\alpha x + \beta$ на кожному проміжку гладкості.

Якщо перед нами одна точку зламу, то функцію можна записати як:
\begin{equation*}
    y = \begin{cases}
        \alpha_- x + \beta_-, \; x \in [0,\xi] \\
        \alpha_+ x + \beta_+, \; x \in [\xi,1]
    \end{cases}
\end{equation*}

Підставимо умову $y(0)=y(1)=0$:
\begin{gather*}
    y(0) = 0 \implies \beta_- = 0 \\
    y(1) = 0 \implies \alpha_+ + \beta_+ = 0 \implies \beta_+ = - \alpha_+
\end{gather*}
Тому, функцію можна дещо спростити:
\begin{equation*}
    y = \begin{cases}
        \alpha_- x, & x \in [0,\xi] \\
        \alpha_+(x-1), & x \in [\xi,1]
    \end{cases}, \;\;\; y' = \begin{cases}
        \alpha_-, & x \in [0,\xi) \\
        \alpha_+, & x \in (\xi, 1]
    \end{cases}
\end{equation*}

Далі треба впевнитись, що $y$ -- неперервна. Для цього має виконуватись:
\begin{equation*}
    \lim_{x\to\xi^-}y(x) = \lim_{x \to \xi^+}y(x) \implies \alpha_-\xi = \alpha_+(\xi-1)
\end{equation*}

Звідси зручно виразити $\xi$:
\begin{equation*}
    \xi = \frac{\alpha_+}{\alpha_+-\alpha_-}
\end{equation*}

Отже, лишається лише дві невідомі: $(\alpha_+,\alpha_-)$. 

Для їх знаходження скористаємося умовою Веєрштрасса-Ердмана:
\begin{gather*}
    \frac{\partial F}{\partial y'}\Big|_{x=\xi^-} = \frac{\partial F}{\partial y'}\Big|_{x=\xi^+} \\
    \left(F-y'\frac{\partial F}{\partial y'}\right)\Big|_{x=\xi^-} = \left(F-y'\frac{\partial F}{\partial y'}\right)\Big|_{x=\xi^+}
\end{gather*}

Перед тим, як розписати ці умови, знайдемо $F-y'F_{y'}$:
\begin{equation*}
    F-y'\frac{\partial F}{\partial y'} = y'^4 - 2y'^2 + 3 - y'(4y'^3-4y') = -3y'^4+2y'^2+3
\end{equation*}

Отже, маємо наступну систему рівнянь:
\begin{equation*}
    \begin{cases}
        \alpha_-^3-\alpha_- = \alpha_+^3 - \alpha_+ \\
        -3\alpha_-^4 + 2\alpha_-^2 = -3\alpha_+^4 + 2\alpha_+^2
    \end{cases}
\end{equation*}

Далі залишається розв'язати. Запишемо:
\begin{equation*}
    \begin{cases}
        \alpha_+^3 - \alpha_-^3 = \alpha_+-\alpha_- \\
        3(\alpha_+^4 - \alpha_-^4) = 2(\alpha_+^2 - \alpha_-^2)
    \end{cases} \implies \begin{cases}
        (\alpha_+ - \alpha_-)(\alpha_+^2+\alpha_+\alpha_- + \alpha_-^2) = \alpha_+-\alpha_- \\
        3(\alpha_+^2+\alpha_-^2)(\alpha_+^2-\alpha_-^2) = 2(\alpha_+^2 - \alpha_-^2)
    \end{cases}
\end{equation*}

Далі користуємось тим, що $\alpha_+ \neq \alpha_-$, бо інакше при $x=\xi$ не було б точки зламу. Тому, в першому рівнянні можемо скоротити на $\alpha_+-\alpha_-$.

З другим складніше: на $\alpha_+^2-\alpha_-^2$ скоротити не можемо, бо може статися (і воно станеться), що $\alpha_-=-\alpha_+$. Перевіримо цей випадок, підставивши у перше рівняння:
\begin{equation*}
    2\alpha_+^2 - \alpha_+^2 = 1 \implies \alpha_+^2 = 1
\end{equation*}
Тому звідси або $(\alpha_+,\alpha_-)=(1,-1)$, або $(\alpha_+,\alpha_-)=(-1,1)$ -- підходить. При цьому, $\xi=\frac{1}{2}$ у обох випадках.

Якщо ж $\alpha_+^2 \neq \alpha_-^2$, то тоді маємо систему:
\begin{equation*}
    \begin{cases}
        \alpha_+^2 + \alpha_+\alpha_- + \alpha_-^2 = 1 \\
        3\alpha_+^2 + 3\alpha_-^2 = 2
    \end{cases}
\end{equation*}
Розв'язавши, маємо або $\alpha_+=\alpha_-=\pm\frac{1}{\sqrt{3}}$ -- немає точки зламу, тоді це не підходить. 

\textbf{Відповідь.} $J(y^*) = 2$ -- точна нижня грань при 
\begin{equation*}
    y^* = \begin{cases}
        x, & x \in [0,\frac{1}{2}] \\
        1-x, & x \in [\frac{1}{2},1]
    \end{cases} \vee y^* = \begin{cases}
        -x, & x \in [0,\frac{1}{2}] \\
        x-1, & x \in [\frac{1}{2},1]
    \end{cases}
\end{equation*}
\end{document}
