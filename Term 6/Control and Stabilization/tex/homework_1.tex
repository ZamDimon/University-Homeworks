%! TEX program = pdflatex

\documentclass[oneside,solution]{karazin-control-assign}

\usepackage[utf8]{inputenc}
\usepackage[english,ukrainian]{babel}

\title{Домашня Робота}
\author{Захаров Дмитро}
\studentID{МП-31}
\instructor{Сморцова Т.І.}
\date{\today}
\duedate{10:10 8 березня, 2024}
\assignno{1}
\semester{Весняний семестр 2024}
\mainproblem{Керованість лінійних систем}

\begin{document}

\maketitle

% \startsolution[print]

\problem{Завдання 1}

\hspace{20px}\textit{Умова.} Нехай задано систему
\begin{equation*}
    \begin{cases}
        \dot{x}_1 = u \\
        \dot{x}_2 = x_1
    \end{cases}
\end{equation*}
Навесті кусково-неперервне керування з однією точкою розриву, що зі стану $[-2,2]^{\top}$ за проміжок $[0,2]$ переведе систему у точку $[0,0]^{\top}$, тобто
\begin{equation}
    u(t) = \begin{cases}
        u_1(t), \; 0 \leq t < \tau \\
        u_2(t), \; \tau \leq t \leq 2
    \end{cases}
\end{equation}

\textit{Розв'язання.} Спробуємо наступну функцію:
\begin{equation}
    u(t) = \begin{cases}
        \alpha t + \beta, \; 0 \leq t < \tau \\
        \gamma t + \delta, \; \tau \leq t < 2
    \end{cases}
\end{equation}

Якщо зафіксувати $\tau=1$, то маємо безліч значень відносно $(\alpha,\gamma)$. 

\problem{Завдання 2}

\hspace{20px}\textit{Умова.} Розглянемо систему
\begin{equation*}
    \begin{cases}
        \dot{x}_1 = -x_1+2x_2 \\
        \dot{x}_2 = -2x_1+3x_2+u
    \end{cases}
\end{equation*}
Чи можна перевести систему з $[0,0]^{\top}$ за проміжок $[0,1]$ у точку $[x_1^0,x_2^0]^{\top}$.

\textit{Розв'язання.} Запишемо систему у матричному вигляді, позначивши $\mathbf{x} := [x_1,x_2]^{\top}$, тоді
\begin{equation}
    \dot{\mathbf{x}} = \mathbf{A}\mathbf{x} + \mathbf{B}u,
\end{equation}
де в нашому випадку
\begin{equation}
    \mathbf{A} = \begin{bmatrix}
        -1 & 2 \\ -2 & 3
    \end{bmatrix}, \; \mathbf{B} = \begin{bmatrix}
        0 \\ 1
    \end{bmatrix}
\end{equation}

Побудуємо матрицю Калмана:
\begin{equation}
    \mathbf{Q} = \begin{bmatrix}
        \mathbf{B} & \mathbf{AB}
    \end{bmatrix} = \begin{bmatrix}
        0 & 2 \\ 1 & 3
    \end{bmatrix}
\end{equation}

Бачимо, що $\text{rang}\,\mathbf{Q} = 2$, тому система повністю керована на $[0,1]$. Знайдемо функцію керування за формулою:
\begin{equation}
    u(t) = \mathbf{B}^*e^{-\mathbf{A}^*t}\cdot\mathbf{N}^{-1}(0,1)\cdot e^{-\mathbf{A}}\mathbf{x}_1
\end{equation}

Отже, залишається лише підставити усе в цю формулу. Спочатку знайдемо експоненти:
\begin{equation}
    e^{-\mathbf{A}} = \exp \begin{bmatrix}
        -1 & 2 \\ -2 & 3
    \end{bmatrix} = \frac{1}{e}\begin{bmatrix}
        3 & -2 \\ 2 & -1
    \end{bmatrix}, \; e^{-\mathbf{A}^*t} = e^{-t}\begin{bmatrix}
        1+2t & 2t \\ -2t & 1-2t
    \end{bmatrix}
\end{equation}

Нарешті, матриця $\mathbf{N}(0,1)$:
\begin{gather}
    \mathbf{N}(0,1) = \int_0^1 e^{-\mathbf{A} t}\mathbf{B}\mathbf{B}^*e^{-\mathbf{A}^* t}dt \\
    = \int_0^1 \begin{bmatrix}
        -2te^{-t} \\ e^{-t}(1-2t)
    \end{bmatrix}\begin{bmatrix}
        -2te^{-t} & e^{-t}(1-2t)
    \end{bmatrix}dt \\
    = \int_0^1\begin{bmatrix}
        4e^{-2t}t^2 & -2e^{-2t}(1-2t)t \\
        -2e^{-2t}(1-2t)t & e^{-2t}(1-2t)^2
    \end{bmatrix}dt \\
    = \begin{bmatrix}
        1 - \frac{5}{e^2} & \frac{1}{2} - \frac{7}{2e^2} \\
        \frac{1}{2}-\frac{7}{2e^2} & \frac{1}{2} - \frac{5}{2e^2}
    \end{bmatrix}
\end{gather}
В такому разі обернена матриця:
\begin{equation}
    \mathbf{N}^{-1}(0,1) = \frac{2e^2}{1-6e^2+e^4}\begin{bmatrix}
        -5+e^2 & 7-e^2 \\
        7-e^2 & 2(-5+e^2)
    \end{bmatrix}
\end{equation}
В такому разі керування:
\begin{equation*}
    u(t) = \begin{bmatrix}
        -2e^{-t}t & e^{-t}(1-2t)
    \end{bmatrix} \cdot \frac{2e^2}{1-6e^2+e^4}\begin{bmatrix}
        -5+e^2 & 7-e^2 \\
        7-e^2 & 2(-5+e^2)
    \end{bmatrix} \cdot \frac{1}{e}\begin{bmatrix}
        3 & -2 \\ 2 & -1
    \end{bmatrix}\begin{bmatrix}
        x_1^0 \\ x_2^0
    \end{bmatrix}
\end{equation*}
Якщо спростити:
\begin{equation}
    \boxed{u(t) = -\frac{2e^{1-t}((-1+e^2(4t-1)))x_1^0 - 2(-2+t+e^2t)x_2^0}{1-6e^2+e^4}}
\end{equation}

\end{document}
