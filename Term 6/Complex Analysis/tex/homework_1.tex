%! TEX program = pdflatex

\documentclass[oneside,solution]{karazin-complan-assign}

\usepackage[utf8]{inputenc}
\usepackage[english,ukrainian]{babel}

\title{Домашня робота}
\author{Захаров Дмитро}
\studentID{МП-31}
\instructor{Гиря Н.П.}
\date{\today}
\duedate{23:59 3 березня, 2024}
\assignno{1}
\semester{Весняний семестр 2024}
\mainproblem{Обчислення лишків. Варіант 5}

\begin{document}

\maketitle

% \startsolution[print]

\problem{}

\textbf{Умова.} Обчислити лишки в усіх ізольованих особливих точках, в тому числі при $z=\infty$:
\begin{equation*}
    f(z) = \frac{z+4}{z-6}
\end{equation*}

\textbf{Розв'язання.} Тут маємо лише дві особливі точки: $z=6$ -- полюс першого порядку, а також $z=\infty$ -- усувна особливість, оскільки
\begin{equation}
    f(\infty)=\lim_{z \to \infty}f(z) = \lim_{z \to \infty} \frac{1+\frac{4}{z}}{1-\frac{6}{z}} = 1
\end{equation}
Отже, щоб обчислити лишок у $z=\infty$, використовуємо формулу:
\begin{gather}
    \text{Res}_{z=\infty}f(z) = \lim_{z \to \infty}\left(z(f(\infty)-f(z))\right) = \lim_{z \to \infty}\left(z\left(1-\frac{z+4}{z-6}\right)\right) \nonumber \\
    = \lim_{z \to \infty} \frac{-10z}{z-6} = -10\lim_{z \to \infty} \frac{1}{1-\frac{6}{z}} = -10
\end{gather}
Оскільки в нас всього одна особлива точка $z=6$ окрім $\infty$, то лишок в ній можна обчислити, використовши формулу
\begin{equation}
    \text{Res}_{z=\infty}f(z) + \text{Res}_{z=6}f(z) =.0 \implies \text{Res}_{z=6}f(z) = -\text{Res}_{z=\infty}f(z) = 10
\end{equation}
Або, для самоперевірки, обрахуємо її наступним чином:
\begin{equation}
    \text{Res}_{z=6}f(z) = \lim_{z \to 6}(z-6)f(z)=\lim_{z \to 6}(z+4) = 10
\end{equation}
\textbf{Відповідь.} $\text{Res}_{z=6}f(z)=10,\;\text{Res}_{z=\infty}f(z)=-10$.
\pagebreak

\problem{}

\textbf{Умова.} Обчислити лишки в усіх ізольованих особливих точках, в тому числі при $z=\infty$:
\begin{equation*}
    f(z) = \frac{1}{z^5(z-3)(z+2i)^2}
\end{equation*}

\textbf{Розв'язання.} Перерахуємо усі особливі точки:
\begin{itemize}
    \item $z=0$ -- полюс $5$ порядку.
    \item $z=3$ -- полюс $1$ порядку.
    \item $z=-2i$ -- полюс $2$ порядку.
    \item $z=\infty$ -- усувна особливість (легко бачити, що $\lim_{z \to \infty}f(z)=0$).
\end{itemize}
Окреслимо нашу ``стратегію'': найлегше обрахувати лишки у $z=\infty$, $z=3$ та $z=-2i$, а вже лишок $z=0$ п'ятого порядку знайдемо з рівності $\sum_{k}\text{Res}_{z=a_k}f(z)+\text{Res}_{z=\infty}f(z)=0$.

По-перше бачимо, що $zf(z) \xrightarrow[z \to \infty]{}0$, тому $\text{Res}_{z=\infty}f(z)=0$. 

Для $z=3$ скористаємося тим, що
\begin{gather}
    \text{Res}_{z=3}f(z) = \lim_{z \to 3}(z-3)f(z) = \lim_{z \to 3}\frac{1}{z^5(z+2i)^2} \nonumber \\ 
    = \frac{1}{3^5 \cdot (3+2i)^2} = \frac{1}{3^5} \cdot \frac{1}{5+12i} = \frac{5-12i}{13^2 \cdot 3^5} \nonumber \\ = \frac{5}{13^2 \cdot 3^5} - \frac{4i}{3^4 \cdot 13^2} = \frac{5}{41067} - \frac{4i}{13689}
\end{gather}
А для $z=-2i$ скористаємось схожою формулою:
\begin{gather}
    \text{Res}_{z=-2i}f(z) = \lim_{z \to -2i}((z+2i)^2f(z))' = \lim_{z \to -2i}\left(\frac{1}{z^5(z-3)}\right)' \nonumber \\
    = \lim_{z \to -2i} \frac{6z^5-15z^4}{z^{10}(z-3)^2} = \lim_{z \to -2i} \frac{6z-15}{z^6(z-3)^2} = \frac{-12i-15}{(-2i)^6(-2i-3)^2} \nonumber \\
    = \frac{15+12i}{2^6(5+12i)} = -\frac{219}{10816} + \frac{15i}{1352}
\end{gather}
Нарешті, для обрахунку $\text{Res}_{z=0}f(z)$ знаходимо:
\begin{gather}
    \text{Res}_{z=0}f(z) = -\text{Res}_{z=3}f(z) - \text{Res}_{z=-2i}f(z) = \frac{313}{15552} - \frac{7i}{648}
\end{gather}

\textbf{Відповідь.} $\text{Res}_{z=0}f(z)=\frac{313}{15552} - \frac{7i}{648},\;\text{Res}_{z=-2i}f(z)=-\frac{219}{10816} + \frac{15i}{1352},\text{Res}_{z=3}f(z)=\frac{5}{41067} - \frac{4i}{13689},\;\text{Res}_{z=\infty}f(z)=0$.

\problem{}

\textbf{Умова.} Обчислити лишки в усіх ізольованих особливих точках, в тому числі при $z=\infty$:
\begin{equation*}
    f(z) = \exp\left(\frac{1}{z+7i}\right)
\end{equation*}

\textbf{Розв'язання.} $z=\infty$ є усувною особливістю, оскільки гранично маємо $f(z) \xrightarrow[z \to \infty]{}e^0 = 1=f(\infty)$. В такому разі лишок:
\begin{gather}
    \text{Res}_{z=\infty}f(z) = \lim_{z \to \infty}\left(z(f(\infty)-f(z))\right) = \lim_{z \to \infty}z\left(1-\exp\left(\frac{1}{z+7i}\right)\right)
\end{gather}

На нескінченності асимптотично $1-\exp\left(\frac{1}{z+7i}\right) \sim_{z \to \infty} -\frac{1}{z+7i}$, тому
\begin{equation}
    \text{Res}_{z=\infty}f(z) = -\lim_{z \to \infty} z \cdot \frac{1}{z+7i} = -\lim_{z \to \infty} \frac{1}{1+\frac{7i}{z}} = -1
\end{equation}
Друга особлива точка -- це $z=-7i$, що є істотною особливістю (оскільки маємо нескінченне число доданків в головній частині ряда Лорана). Розглянемо ряд Лорана:
\begin{equation}
    f(z) = \sum_{k=0}^{\infty}\frac{1}{k!}\left(\frac{1}{z+7i}\right)^k = 1 + \frac{1}{z+7i} + \frac{1}{2}\left(\frac{1}{z+7i}\right)^2 + \dots
\end{equation}
Лишок -- це коефіцієнт $c_{-1}$ в розкладі Лорана перед $\frac{1}{z+7i}$, в цьому випадку -- $1$. Тому, $\text{Res}_{z=-7i}f(z)=1$.

\textbf{Відповідь.} $\text{Res}_{z=\infty}f(z)=-1,\;\text{Res}_{z=-7i}f(z)=1$. 
\end{document}
