\chapter*{Анотації}
\markboth{Анотації}{Анотації}
\addcontentsline{toc}{chapter}{Анотації}

\subsubsection{Захаров Д.О. Математичні основи штучних нейронних мереж.}

В сучасному світі важко уявити сферу, де б не використовувались штучні нейронні
мережі. Проте, незважаючи на таку кількість різноманітних досліджень, більшість
з них не обгрунтовують чому саме нейронні мережі так добре апроксимують данні та
видають гарну точність. Ця робота призначена опису фундаменту нейронних мереж
та, в певній мірі, формалізації процесу навчання: що саме розв'язують нейронні
мережі, чому вони (теоретично) здатні апроксимувати важливі для нас залежності і
як на практиці реалізувати процес підбору параметрів моделі.

\subsubsection{Zakharov D.O. Mathematical foundations of artificial neural networks.}

In the modern world, it is almost impossible to imagine a technical field 
where artificial neural networks are not used. However, despite the abundance
of research of their applications, most of them do not explain why neural networks
are so good at approximating data and achieving high accuracy. This work is
dedicated to describing the fundamentals of neural networks and, to some extent,
formalizing the learning process: what neural networks solve, why they are
(theoretically) capable of approximating important dependencies, and how to
practically implement the process of selecting model parameters.
