\chapter*{Вступ}
\markboth{Вступ}{Вступ}
\addcontentsline{toc}{chapter}{Вступ}

Без всяких сумнівів, нейронні мережі є одними з найбільш популярних інструментів
машинного навчання для пошуку складних залежностей. Вони використовуються в
безлічі різних областей, таких як комп'ютерний зір \cite{cv-survey}, обробка
природних мов \cite{nlp-survey} та біометричних данних \cite{biometrics-survey},
розробка рекомендаційних систем \cite{recommendation-systems-survey}, генерації
зображень \cite{gan-survey} тощо. Наші нещодавні дослідження
\cite{our-biometrics-1,our-biometrics-2,our-biometrics-3} додатково підтвердили
високу ефективність нейронних мереж у задачах кібербезпеки та систем захисту
біометричних даних. Що уж там, мабуть кожна людина чула або використовувала
новітні розробки OpenAI --- архітектуру \textit{GPT-3} (Generative Pre-trained
Transformer) \cite{chatgpt} або \textit{Open AI o1}, що вже навіть здатна
розв'язувати задачі з міжнародної олімпіади з математики або аналізувати складні
наукові тексти. 

Проте, незважаючи на таку кількість різноманітних досліджень, більшість з них 
зводиться до доволі типічного алгоритму (звичайно, з варіаціями в залежності від
конкретної задачі):
\begin{enumerate}
    \item Визначення типу задачі (класифікація, регресія, сегментація, тощо).
    \item Підбір набору даних (далі, скорочено --- датасет).
    \item Вибір архітектури моделі, функції втрати та метрик якості.
    \item Тренування та корегування параметрів моделі для максимізації метрики якості.
    \item Аналіз результатів.
\end{enumerate}

Проте, протягом цього процесу, ми зазвичай пропускаємо одне доволі
фундаментальне питання: а чому, взагалі кажучи, обрані архітектури нейронних
мереж здатні вирішувати такі задачі? Звичайно, що для практичних задач це
питання часто не принципове: якщо воно працює і працює добре, то цього більш,
ніж достатньо\footnote{Тим не менш, в сучасних роботах іноді трапляється спроба
пояснити, чому описана методологія може теоретично дати, скажімо, мінімум для
певної метрики, як це було зроблено в оригінальному описі генеративних
адверсальних мереж (Generative Adversarial Networks) \cite{gan}}.

Саме тому ми присвятили цю роботу опису фундаменту нейронних
мереж та, в певній мірі, формалізації процесу навчання: що саме 
розв'язують нейронні мережі, чому вони (теоретично) здатні
апроксимувати важливі для нас залежності і як на практиці
реалізувати процес підбору параметрів моделі.

