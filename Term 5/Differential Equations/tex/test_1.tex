\documentclass[14pt]{extarticle}
\usepackage[english,ukrainian]{babel}
\usepackage[utf8]{inputenc}
\usepackage{amsmath,amssymb}
\usepackage{parskip}
\usepackage{graphicx}
\usepackage{xcolor}
\usepackage{tcolorbox}
\tcbuselibrary{skins}
\usepackage[framemethod=tikz]{mdframed}
\usepackage{chngcntr}
\usepackage{enumitem}
\usepackage{hyperref}
\usepackage{float}
\usepackage{subfig}
\usepackage{chngcntr}
\usepackage{esint}
\usepackage{subfig}
\usepackage[top=2.5cm, left=3cm, right=3cm, bottom=4.0cm]{geometry}
\usepackage[table]{xcolor}
\usepackage{algorithm}
\usepackage{algpseudocode}
\usepackage{listings}

\title{ІДЗ \#1 з курсу ``Диференціальні рівняння''}
\author{Студента 3 курсу групи МП-31 Захарова Дмитра}
\date{\today}

\begin{document}

\maketitle

\textbf{Варіант 5}

\section*{Завдання 1.} 

\textbf{Умова.} Розв'язати ЛОС $\dot{\textbf{x}} = \boldsymbol{A}\textbf{x}$ із заданою матрицею методом невизначених коефіцієнтів:
\[
\boldsymbol{A} = \begin{bmatrix}
    1 & 3 & -1 \\
    -3 & 5 & 0 \\
    -4 & 4 & 2
\end{bmatrix}
\]

\textbf{Розв'язок.} Спочатку знаходемо характеристичний поліном:
\[
\chi_A(\lambda) = \det (\boldsymbol{A} - \lambda \boldsymbol{E}) = -\lambda^3 + 8\lambda^2 - 22\lambda + 20
\]

Перший корінь вгадати легко: $\lambda_1 = 2$. Два інші корені знаходяться з рівняння:
\[
\frac{\chi_A(\lambda)}{\lambda - 2} = 0 \implies \lambda^2 - 6\lambda+10 = 0
\]

Звідси $\lambda_{2,3} = 3 \pm i$. Знаходимо відповідні власні вектори. Для $\lambda_1=2$ маємо $\textbf{v}_1 = [1,1,2]^{\top}$, для $\lambda_2=3+i$ маємо $\textbf{v}_2=[2-i,3,4]^{\top}$, а для $\lambda_3=3-i$ отримуємо $\textbf{v}_3=[2+i,3,4]^{\top}$.

Знаходимо дійсні і комплексні частини виразу $\mathbf{v}_2e^{\lambda_2 t} = [2-i,3,4]^{\top}e^{(3+i)t}$:
\[
\text{Re} \, \mathbf{v}_2e^{\lambda_2 t} = \begin{bmatrix}
2e^{3t}\cos t + e^{3t}\sin t \\
3e^{3t}\cos t \\ 4e^{3t}\cos t
\end{bmatrix}, \; \text{Im} \, \mathbf{v}_2e^{\lambda_2 t} = \begin{bmatrix}
   2e^{3t}\sin t - e^{3t} \cos t \\ 3e^{3t}\sin t \\ 4e^{3t}\sin t
\end{bmatrix}
\]

Отже, остаточна відповідь:
\[
\mathbf{x}(t) = C_1 \cdot \mathbf{v}_1e^{\lambda_1 t} + C_2\cdot\text{Re}\,\mathbf{v}_2e^{\lambda_2t} + C_3\cdot\text{Im}\,\mathbf{v}_2e^{\lambda_2t}
\]

Або:
\[
\mathbf{x}(t) = C_1e^{2t}\begin{bmatrix}
    1 \\ 1 \\ 2
\end{bmatrix} + C_2e^{3t}\begin{bmatrix}
    2\cos t + \sin t \\ 3 \cos t \\ 4 \cos t
\end{bmatrix} + C_3e^{3t}\begin{bmatrix}
    2\sin t - \cos t \\ 3\sin t \\ 4 \sin t
\end{bmatrix}
\]

\section*{Завдання 2.} 

\textbf{Умова.} Розв’язати ЛОС $\dot{\textbf{x}}=\boldsymbol{A}\textbf{x}$ за допомогою функції від матриці
\[
\boldsymbol{A} = \begin{bmatrix}
    1 & 2 & -1 \\
    0 & -3 & 2 \\
    0 & -6 & 4
\end{bmatrix}
\]

\textbf{Розв'язок.} Характеристичний поліном має вигляд:
\[
\chi_A(\lambda) = -\lambda^3+2\lambda^2-\lambda = \lambda(\lambda-1)^2
\]

Знаходимо мінімальний многочлен $m(\lambda)$. Помічаємо, що:
\[
\boldsymbol{A}(\boldsymbol{A} - \boldsymbol{E}) = \boldsymbol{O}_{3 \times 3}
\]

Отже, $m(\lambda)=\lambda(\lambda-1)$. Тому інтерполяційний поліном:
\[
p(\lambda) = e^{0\cdot t} \cdot \frac{\lambda - 1}{0 - 1} + e^{t}\cdot \frac{\lambda-0}{1-0} = \lambda e^t + (1-\lambda)
\]

Таким чином, фундаментальна матриця розв'язків
\[
e^{\boldsymbol{A}t} = p(\boldsymbol{A}) = \begin{bmatrix}
    0 & -1 & 2 \\ 1 & 4 & -1 \\ 1 & 7 & -3
\end{bmatrix} + \begin{bmatrix}
    1 & 2 & -1 \\ 0 & -3 & 2 \\ 0 & -6 & 4
\end{bmatrix}e^{t}
\]

А загальний розв'язок можемо записати у вигляді:
\[
\textbf{x}(t) = \begin{bmatrix}
    e^t & -1+2e^t & 2-e^t \\ 1 & 4 - 3e^t & -1+2e^t \\ 1 & 7-6e^t & -3+4e^t
\end{bmatrix}\mathbf{u}, \; \mathbf{u} \in \mathbb{R}^3
\]

Обмеження на константи в такому випадку: $u_1=-3u_3,u_2=0$.

\section*{Завдання 3.} 

\textbf{Умова.} Розв’язати ЛНС методом невизначених коефіцієнтів:
\[
\begin{cases}
    \dot{x} = 3x - y \\
    \dot{y} = 4x - y - 4(t-1)e^{-t}
\end{cases}
\]

\textbf{Розв'язок.} Спочатку розв'язуємо ЛОС. Матриця:
\[
\boldsymbol{A} = \begin{bmatrix}
    3 & -1 \\ 4 & -1
\end{bmatrix}
\]
Характеристичний поліном:
\[
\chi_A(\lambda) = \lambda^2 - 2\lambda + 1 = (\lambda-1)^2
\]
Отже, маємо одне власне число кратності $2$: $\lambda=1$. Відповідний власний вектор $\mathbf{v}=[1,2]^{\top}$. 

Пропущенний розв'язок шукаємо у вигляді $\textbf{x}=e^{\lambda_1t}(\textbf{v}_1t + \textbf{u})$ де $\textbf{u}=[u_1,u_2]^{\top}$. Інакшими словами, $\textbf{x} = \begin{bmatrix}
    e^t(t+u_1) \\ e^t(2t+u_2)
\end{bmatrix}$. Підставляючи у $\dot{\textbf{x}}=\boldsymbol{A}\textbf{x}$, отримуємо
\[
\begin{cases}
e^t + te^t + u_1e^t = 3e^t(t+u_1) - e^t(2t+u_2) \\
2e^t+2te^t+u_2e^t = 4e^t(t+u_1)-e^t(2t+u_2)
\end{cases}
\]
Спростивши, маємо
\[
\begin{cases}
    1+u_1 = 3u_1 - u_2 \\ 
    2+u_2 = 4u_1-u_2
\end{cases} \implies u_2 = 2u_1-1
\]

Підставимо $u_1=0$, тоді $u_2=-1$. Таким чином, розв'язок ЛОС має вид:
\[
\textbf{x}_{0}(t) = \begin{bmatrix}
    C_1e^t + C_2te^t \\ 2C_1e^t + C_2e^t(2t-1)
\end{bmatrix}
\]

Тепер знайдемо частковий розв'язок. Оскільки нелінійна частина має вигляд $e^{\gamma t}P_m(t)$ (в нашому випадку $\gamma=-1,P_m(t)=4(1-t)$), то розв'язок шукаємо у вигляді $e^{\gamma t}Q_{m+r}(t)$ де $r$ кратність $\gamma$ як кореня мінімального многочлена. Тоді шукаємо розв'язок у вигляді:
\[
\textbf{x}(t) = \begin{bmatrix}
    e^{-t}(\alpha_1 t^3+\beta_1 t^2 + \gamma_1 t + \delta_1) \\ e^{-t}(\alpha_2 t^3 + \beta_2 t^2 + \gamma_2 t + \delta_2)
\end{bmatrix}
\]

Підставивши у наше рівняння, отримуємо:
\[
\begin{cases}
    (\alpha_2-4\alpha_1)t^3 + (3\alpha_1-4\beta_1+\beta_2)t^2 + (2\beta_1-4\gamma_1+\gamma_2)t + (\gamma_1-4\delta_1+\delta_2) = 0 \\
    -4\alpha_1t^3 + (3\alpha_2-4\beta_1)t^2 + (2\beta_2-4\gamma_1+4)t + (-4+\gamma_2-4\delta_1) = 0
\end{cases}
\]

Одразу видно $\alpha_1=\alpha_2=0$. З квадратичного доданку другого рівняння $\alpha_2 = \frac{4}{3}\beta_1$. Покладемо $\beta_1=0$, тоді $\alpha_2=0$. В такому разі $\beta_2=-3\alpha_1$. Зручно покласти $\alpha_1=0$, тоді $\beta_2=0$. В такому разі $\gamma_1=1,\gamma_2=4,\delta_1=0,\delta_2=-1$. Отже, маємо частковий розв'язок:
\[
\textbf{x}_1(t) = \begin{bmatrix}
    e^{-t}t \\ e^{-t}(4t-1)
\end{bmatrix}
\]

Отже, остаточно маємо розв'язок:
\[
\textbf{x}(t) = \begin{bmatrix}
    C_1e^t + C_2te^t + e^{-t}t \\
    2C_1e^t + C_2e^t(2t-1) + e^{-t}(4t-1)
\end{bmatrix}
\]

\section*{Завдання 4.} 

\textbf{Умова.} Розв’язати ЛНС методом Лагранжа та методом Коші
\[
\begin{cases}
    \dot{x} = x - y + \sin 2t \\
    \dot{y} = 5x-y + \sin 2t + 2 \cos 2t
\end{cases}
\]

\textbf{Розв'язок.} \textbf{Метод Коші} полягає у знаходженні:
\[
\boldsymbol{K}(t,\tau) = e^{(t-\tau)\boldsymbol{A}}, \; \textbf{y} = \int_{t_0}^t \boldsymbol{K}(t,\tau)\boldsymbol{\beta}(\tau)d\tau
\]

Отже, знайдемо $e^{(t-\tau)\boldsymbol{A}}$. Діагоналізувавши матрицю ЛОС, отримаємо:
\[
\boldsymbol{A} = \boldsymbol{V}\begin{bmatrix}
    2i & 0 \\ 0 & -2i
\end{bmatrix}\boldsymbol{V}^{-1}, \; \boldsymbol{V} = \begin{bmatrix}
    1+2i & 1-2i \\ 5 & 5
\end{bmatrix}
\]

Як ми знаємо,
\[
\exp \left(\text{diag} \{\lambda_1,\dots,\lambda_m\}\right) =\text{diag} \{e^{\lambda_1},\dots,e^{\lambda_m}\}
\]

В такому разі:
\[
e^{\boldsymbol{A}t} = \boldsymbol{V} \begin{bmatrix}
    e^{2it} & 0 \\ 0 & e^{-2it}
\end{bmatrix} \boldsymbol{V}^{-1} = \begin{bmatrix}
    \frac{(2-i)e^{2it} + (2+i)e^{-2it}}{4} & -\frac{\sin 2t}{2} \\ \frac{5\sin 2t}{2} & \frac{(2+i)e^{2it} + (2-i)e^{-2it}}{4}
\end{bmatrix}
\]

Якщо взяти тільки дійсну частину:
\[
\text{Re} \, e^{\boldsymbol{A}(t-\tau)} = \begin{bmatrix}
    \cos 2(t-\tau) + \frac{1}{2}\sin 2(t-\tau) & -\frac{1}{2}\sin 2(t-\tau) \\ 
    \frac{5}{2}\sin 2(t-\tau) & \cos 2(t-\tau) - \frac{1}{2}\sin 2(t-\tau)
\end{bmatrix}
\]

Отже,
\[
\boldsymbol{K}(t,\tau) = \begin{bmatrix}
    4 \cos 2(t-\tau) + 2\sin 2(t-\tau) & -\frac{1}{2}\sin 2(t-\tau) \\ \frac{5}{2}\sin 2(t-\tau) & 4\cos 2(t-\tau) - 2\sin 2(t-\tau)
\end{bmatrix}
\]

В такому разі:
\[
\textbf{y} = \int_{t_0}^t \boldsymbol{K}(t,\tau)\begin{bmatrix}
    \sin 2t \\ \sin 2t + 2\cos 2t
\end{bmatrix}d\tau 
\]

Якщо розписати:
\[
\textbf{y} = \int_{t_0}^t \begin{bmatrix}-\sin (2t-4\tau) \\ 2\cos(2t-4\tau)-\sin (2t-4\tau)\end{bmatrix}d\tau
\]

Зручно взяти $t_0=0$. Тоді інтеграл зведеться до:
\[
\textbf{y}(t) = \begin{bmatrix}
    0 \\ \sin 2t
\end{bmatrix}
\]

Отже, повний розв'язок:
\[
\textbf{x}(t) = \text{Re}\, e^{\boldsymbol{A}t} \cdot \textbf{u} + \textbf{y} = \begin{bmatrix}
    \cos 2t + \sin 2t/2 & -\sin 2t/2 \\ 5\sin 2t/2 & \cos 2t - \sin 2t/2
\end{bmatrix}\begin{bmatrix}
    u_1 \\ u_2
\end{bmatrix} + \begin{bmatrix}
    0 \\ \sin 2t
\end{bmatrix}
\]

Тепер використаємо \textbf{метод варіації} (метод Лагранжа). Знайдемо розв'язок ЛОС.

Маємо власне число $\lambda=2i$ з власним вектором $\textbf{v}=[1+2i,5]^{\top}$. Отже, знайдемо дійсну і уявну частину $\textbf{v}e^{\lambda t}$:
\[
\text{Re}\,\textbf{v}e^{\lambda t} = \begin{bmatrix}
    \cos 2t - 2\sin 2t \\ 5\cos 2t
\end{bmatrix}, \; \text{Im}\,\textbf{v}e^{\lambda t} = \begin{bmatrix}
    \sin 2t + 2\cos 2t \\ 5\sin 2t
\end{bmatrix}
\]

Таким чином, маємо розв'язок ЛОС:
\[
\textbf{w}(t) = \begin{bmatrix}
    C_1(\cos 2t - 2\sin 2t) + C_2(\sin 2t + 2\cos 2t) \\ 5C_1 \cos 2t + 5 C_2 \sin 2t
\end{bmatrix} = \begin{bmatrix}
    (C_1+2C_2)\cos 2t + (-2C_1+C_2)\sin 2t \\ 
    5C_1 \cos 2t + 5C_2 \sin 2t
\end{bmatrix}
\]

Отже тепер нехай $C_1=C_1(t),C_2=C_2(t)$. Тоді:
\[
\dot{\textbf{w}} = \begin{bmatrix}
    \dot{C}_1\cos 2t - 2C_1\sin 2t + 2\dot{C}_2 \cos 2t - 4C_2 \sin 2t - 2\dot{C}_1 \sin 2t - 4C_1\cos 2t + \dot{C}_2 \sin 2t + 2C_2 \cos 2t \\ 5\dot{C}_1 \cos 2t - 10C_1 \sin 2t + 5\dot{C}_2 \sin 2t + 10 C_2 \cos 2t
\end{bmatrix} 
\]

З цього виразу підуть усі члени, що містять $C_1,C_2$, оскільки вони є розв'язками ЛОС. Отже, отримуємо систему:
\[
\begin{cases}
    \dot{C}_1 (\cos 2t - 2\sin 2t) + \dot{C}_2(\sin 2t + 2\cos 2t) = \sin 2t \\ 5\dot{C}_1 \cos 2t + 5\dot{C}_2 \sin 2t = \sin 2t + 2\cos 2t 
\end{cases}
\]

Звідси:
\[
\dot{C}_1(t) = \frac{2\cos 4t + \sin 4t}{5},\;\dot{C}_2(t) = \frac{2\sin 4t-\cos 4t}{5}  
\]

Якщо проінтегрувати:
\[
C_1(t) = -\frac{1}{20}\cos 4t + \frac{1}{10}\sin 4t + \delta_1, \; C_2(t)=-\frac{1}{10}\cos 4t - \frac{1}{20}\sin 4t + \delta_2
\]

Далі можна це підставити у вираз зверху і отримати остаточну відповідь. Оскільки одну (по методу Коші) ми отримали, то повторюватись не будемо :)

\end{document}

