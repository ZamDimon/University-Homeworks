\documentclass[14pt]{extarticle}
\usepackage[english,ukrainian]{babel}
\usepackage[utf8]{inputenc}
\usepackage{amsmath,amssymb}
\usepackage{parskip}
\usepackage{graphicx}
\usepackage{xcolor}
\usepackage{tcolorbox}
\tcbuselibrary{skins}
\usepackage[framemethod=tikz]{mdframed}
\usepackage{chngcntr}
\usepackage{enumitem}
\usepackage{hyperref}
\usepackage{float}
\usepackage{subfig}
\usepackage{esint}
\usepackage[top=2.5cm, left=3cm, right=3cm, bottom=4.0cm]{geometry}
\usepackage[table]{xcolor}
\usepackage{algorithm}
\usepackage{algpseudocode}
\usepackage{listings}

\title{Домашня робота з курсу ``Теорія Ймовірності''}
\author{Студента 3 курсу групи МП-31 Захарова Дмитра}
\date{\today}

\begin{document}

\maketitle

\section*{Завдання 1.} 

\textbf{Умова.} Пивний завод відправив у магазин $n=400$ ящиків пива. Імовірність того, що ящик буде розбитий при транспортуванні в даних умовах, що дорівнює $\theta=0.005$. Після приїзду в магазин експедитор, який перевозив вантаж, заявив, що $m=7$ ящиків із пивом були розбиті під час транспортування. Розмірковуючи, чи можна довіряти експедитору, директор магазину хоче знайти можливість розбити $m$ ящиків, ймовірність розбити не менше $m$ ящиків, математичне сподівання, дисперсію та середнє квадратичне відхилення кількості ящиків $\xi$, розбитих під час транспортування, щоб оцінити можливість втрат, заявлених експедитором. Знайти вказані величини.

\textbf{Розв'язок.} Кількість розбитих ящиків розглядаємо як випадкову величину $\xi$. Ця величина має біноміальний розподіл, тобто $\xi \sim \text{Bin}(n,\theta)$ (маємо спостереження, в яких з ймовірністю $\theta$ відбудеться деяка подія, а з ймовірністю $1-\theta$ -- ні). Згідно означенню біноміального розподілу, маємо:
\[
p(\xi = k) = C_n^k \theta^k (1-\theta)^{n-k}
\]
Таким чином, можемо відповісти на питання з якою ймовірністю було розбито $m=7$ ящиків:
\[
p(\xi = 7) = C_{400}^7 \times (0.005)^7 \times (1-0.005)^{400-7} \approx 0.00336
\]
Ймовірність розбити не менше $m$ ящиків це $p(\xi \geq m)$. З іншого боку:
\[
p(\xi \geq m) = 1 - p(\xi < m) = 1 - \sum_{k=0}^{m-1}p(\xi=k)
\]
Отже, підставляємо у формулу:
\[
p(\xi \geq 7) = 1 - \sum_{k=0}^{6}p(\xi=k) \approx 0.00441
\]
Нарешті, математичне сподівання, дисперсію та середнє квадратичне відхилення знайдемо за формулами біноміального розподілу:
\[
\mathbb{E}[\xi] = n\theta = 2
\]
\[
\text{Var}[\xi] = n\theta(1-\theta) = 1.99, \; \sigma[\xi] = \sqrt{\text{Var}[\xi]} \approx 1.41
\]
Отже, бачимо, що $7$ ящиків наврядше могло бути розбито.

\section*{Завдання 2.}
\textbf{Умова.} До банку надійшло $n=4000$ пакетів грошових знаків. Імовірність того, що пакет містить недостатню чи надмірну кількість грошових знаків, що
дорівнює $\theta=0.0001$. Знайти: 
\begin{enumerate}
    \item ймовірність того, що під час перевірки буде
виявлено хоча б один помилково укомплектований пакет;      \item ймовірність того, що під час перевірки буде виявлено не більше трьох помилково укомплектованих пакетів; 
    \item математичне сподівання та дисперсію числа
помилково укомплектованих пакетів.
\end{enumerate}

\textbf{Розв'язок.} Нехай $\xi$ -- число помилково укомплектованих пакетів. Як і в минулій задачі, $\xi \sim \text{Bin}(n,\theta)$, тобто
\[
p(\xi = k) = C_n^k\theta^k(1-\theta)^{n-k}
\]

\textit{Пункт 1.} Потрібно знайти $p(\xi \geq 1)$. Легше записати це значення як
\[
p(\xi \geq 1) = 1 - p(\xi=0) = 1 - (1-\theta)^n \approx 0.33
\]

\textit{Пункт 2.} Потрібно знайти $p(\xi \leq 3)$. Легше це записати як
\[
p(\xi \leq 3) = \sum_{k=0}^3 p(\xi=k) \approx 0.9992
\]

\textit{Пункт 3.} Як і в минулій задачі:
\[
\mathbb{E}[\xi] = n\theta = 0.4, \; \text{Var}[\xi] = n\theta(1-\theta) = 0.39996
\]

\section*{Завдання 3.}
\textbf{Умова.} Для просування своєї продукції ринку фірма розкладає по поштовим скринькам рекламні листки. Колишній досвід роботи фірми показує, що приблизно в одному випадку з $m=2000$ приходить замовлення. Знайти ймовірність того, що при розміщенні $n=10000$ рекламних листків надійде хоча б одне замовлення, середню кількість замовлень, що надійшли, і дисперсію числа замовлень, що надійшли.

\textbf{Розв'язок.} Розглядаємо випадкову величину $\xi_N$ -- кількість замовлень, якщо дана кількість рекламних листків $N$. 

Розглянемо наступну математичну модель: нехай з деякою ймовірністю $\theta$ людина, побачивши рекламний листок, зробить замовлення. Тоді $\xi$ буде відповідати біноміальному розподілу: $\xi \sim \text{Bin}(N,\theta)$. 

Те, що приблизно в одному випадку з $m=2000$ приходить замовлення, можна інтерпретувати так: математичне сподівання $\xi$ при $m=2000$ дорівнює $1$. Оскільки математичне сподівання можна виразити як $\mathbb{E}[\xi_N] = \theta N$, то з рівняння $\mathbb{E}[\xi_{m}] = \theta \cdot m$ отримуємо $\theta = \frac{1}{m}$. 

Отже, якщо розглянемо $n=10000$ рекламних листків, то маємо розподіл:
\[
p(\xi_n = k) = C_n^k\theta^k(1-\theta)^{n-k}
\]

За умовою треба знайти $p(\xi_n \geq 1)$, $\mathbb{E}[\xi_n]$ та $\text{Var}[\xi_n]$. Отже:
\[
p(\xi_n \geq 1) = 1 - p(\xi_n = 0) = 1 - (1-1/m)^{n} \approx 0.9933
\]
\[
\mathbb{E}[\xi_n] = n\theta = \frac{n}{m} = 5
\]
\[
\text{Var}[\xi_n] = n\theta(1-\theta) = \frac{n}{m}\left(1 - \frac{1}{m}\right) \approx 4.9975
\]

\section*{Завдання 4.}
\textbf{Умова.} Гральний кубик підкидують до тих пір, поки не випаде шістка. Знайти математичне сподівання та дисперсію числа підкидань.

\textbf{Розв'язок.} Розглядаємо випадкову величину $\xi$ -- число підкидань, поки не випала шістка. Перед нами геометричний розподіл $\xi \sim G(\theta)$ де $\theta=\frac{1}{6}$, який має вигляд:
\[
p(\xi = k) = (1-\theta)^{k-1}\theta
\]

Математичне сподівання геометричного розподілу:
\[
\mathbb{E}[\xi] = \frac{1}{\theta} = 6
\]

Дисперсія:
\[
\text{Var}[\xi] = \frac{1-\theta}{\theta^2} = 30
\]

\section*{Завдання 5.}
\textbf{Умова.} Кинуто 3 розрізні гральні кубики. Знайти таблицю розподілу випадкової величини -- числа кубиків, на яких випаде шістка. Знайти математичне сподівання та дисперсію цієї випадкової величини.

\textbf{Розв'язок.} Маємо біномінальний розподіл $\xi \sim \text{Bin}\left(3,\frac{1}{6}\right)$, тобто:
\[
p(\xi = k) = C_3^k \times \left(\frac{1}{6}\right)^k \times \left(\frac{5}{6}\right)^{3-k}
\]
Конкретні значення:
\[
p(\xi=0) \approx 0.58, \; p(\xi=1)\approx 0.35
\]
\[
p(\xi=2) \approx 0.07, \; p(\xi=3)\approx 0.005
\]
Математичне очікування:
\[
\mathbb{E}[\xi] = 3 \cdot \frac{1}{6} = 0.5
\]
Дисперсія:
\[
\text{Var}[\xi] = 3 \cdot \frac{1}{6} \cdot \frac{5}{6} = \frac{5}{12} \approx 0.417
\]

\section*{Завдання 6.}
\textbf{Умова.} В урні 4 білих та 4 чорних кулі. З неї навмання витягують 3 кулі. Побудувати таблицю розподілу випадкової величини $\xi$ – числа білих куль, які витягнуто з урни. Знайти математичне сподівання та дисперсію цієї випадкової величини.

\textbf{Розв'язок.} Вибрати $3$ кулі з $8$ можна $C_8^3$ способами. Вибрати $0$ білих куль -- це кількість способів вибрати $3$ кулі з $4$ чорних, тобто $C_{4}^3$. Ітого:
\[
p(\xi = 0) = \frac{C_{4}^3}{C_{8}^3}
\]
Вибрати 1 білу кулю означає знайти кількість способів взяти $1$ білу кулю з $4$, тобто $C_{4}^1$, помножити на кількість способів вибрати $2$ чорні кулі з $4$, тобто $C_{4}^2$. Ітого:
\[
p(\xi = 1) = \frac{C_{4}^2C_{4}^1}{C_8^3}
\]
По аналогії можемо отримати:
\[
p(\xi = k) = \frac{C_{4}^{3-k}C_4^k}{C_8^3}, \; k \in \{0,1,2,3\}
\]
Математичне сподівання цієї величини:
\[
\mathbb{E}[\xi] = \frac{1}{C_{8}^3}\sum_{k=0}^3 kC_{4}^{k}C_4^{3-k}= 1.5
\]
Математичне сподівання квадрату величини:
\[
\mathbb{E}[\xi^2] = \frac{1}{C_{8}^3}\sum_{k=0}^3 k^2C_{4}^{k}C_4^{3-k}\approx 2.79
\]
Дисперсія:
\[
\text{Var}[\xi] = \mathbb{E}[\xi^2] - \mathbb{E}[\xi]^2 \approx 0.536
\]

\section*{Завдання 7.}
\textbf{Умова.} Навести приклад дискретної випадкової величини, яка має математичне сподівання та не має дисперсію.

\textbf{Розв'язок.} Оскільки дисперсія може бути знайдена за допомогою формули
\[
\text{Var}[\xi] \triangleq \mathbb{E}[\xi^2] - \mathbb{E}[\xi]^2
\]
і $\mathbb{E}[\xi]$ існує, то достатньо підібрати таку випадкову величину, щоб $\mathbb{E}[\xi^2]$ не існувало. 

Окрім того, $\xi(\Omega)$ має бути зліченною, оскільки для скінченних $\xi(\Omega)$ математичне сподівання $\xi^2$ завжди існує. 

Отже, треба знайти, коли:
\begin{align*}
\exists\, \mathbb{E}[\xi] \triangleq \sum_{k=1}^{\infty} p(\xi = x_k)x_k \; \text{та} \\ \not \exists\, \mathbb{E}[\xi^2] \triangleq \sum_{k=1}^{\infty} p(\xi = x_k)x_k^2 \\ \text{таке, що} \; \sum_{k=1}^{\infty} p(\xi = x_k) = 1
\end{align*}
де $\xi(\Omega) := \{x_k\}_{k=1}^{\infty}$ -- можливі значення випадкової величини. 

Покажемо, що при розподілі $p(\xi = x_k) = 2^{-k}$ з можливими значеннями випадкової величини $x_k=(3/2)^k$ виконується умова.

Дійсно, $\sum_{k=1}^{\infty}p(\xi=x_k) = \sum_{k=1}^{\infty}2^{-k} = 1$. Знаходимо математичні сподівання:
\begin{align*}
\mathbb{E}[\xi] = \sum_{k=1}^{\infty} \left(\frac{3}{2}\right)^k \cdot 2^{-k} = \sum_{k=1}^{\infty} 3^k \cdot 2^{-2k} = \sum_{k=1}^{\infty} e^{k \ln 3} \cdot e^{-2k \ln 2} = \\
\sum_{k=1}^{\infty} e^{k(\ln 3 - 2 \ln 2)} = \sum_{k=1}^{\infty} \left(\frac{3}{4}\right)^k = 3 < +\infty
\end{align*}

Що стосується математичного сподівання квадрату:
\[
\mathbb{E}[\xi^2] = \sum_{k=1}^{\infty} \left(\frac{9}{4}\right)^k \cdot 2^{-k} = \sum_{k=1}^{\infty}\left(\frac{9}{8}\right)^k
\]
Цей ряд не збігається, оскільки це нескінченна геометрична прогресія зі знаменником більшим на $1$. 

Отже, розподіл визначено коректно, $\mathbb{E}[\xi]$ визначено, а $\mathbb{E}[\xi^2]$ не визначено, звідки випливає і невизначенність $\text{Var}[\xi]$. 

\end{document}

