\documentclass[14pt]{extarticle}
\usepackage[english,ukrainian]{babel}
\usepackage[utf8]{inputenc}
\usepackage{amsmath,amssymb}
\usepackage{parskip}
\usepackage{graphicx}
\usepackage{xcolor}
\usepackage{tcolorbox}
\tcbuselibrary{skins}
\usepackage[framemethod=tikz]{mdframed}
\usepackage{chngcntr}
\usepackage{enumitem}
\usepackage{hyperref}
\usepackage{float}
\usepackage{subfig}
\usepackage{esint}
\usepackage[top=2.5cm, left=3cm, right=3cm, bottom=4.0cm]{geometry}
\usepackage[table]{xcolor}
\usepackage{algorithm}
\usepackage{algpseudocode}
\usepackage{listings}


\title{Домашня робота з курсу ``Теорія Ймовірності''}
\author{Студента 3 курсу групи МП-31 Захарова Дмитра}
\date{\today}

\begin{document}

\maketitle

\section*{Завдання 1.}

\textbf{Умова.} Дано таблицю розподілу двовимірного випадкового вектору $[\xi,\eta]^{\top}$. Перевірити, чи є незалежними $\xi$ та $\eta$.

\begin{center}
\begin{tabular}{ |c|c|c|c| } 
 \hline
 -- & $\eta=0$ & $\eta=1$ & $\eta=2$ \\ 
 \hline
 $\xi=0$ & $0.1$ & $0.1$ & $0.2$ \\ 
 \hline
 $\xi=1$ & $0.1$ & $0.1$ & $0.1$ \\ 
 \hline
 $\xi=2$ & $0.2$ & $0.1$ & $0.0$ \\
 \hline
\end{tabular}
\end{center}

\textbf{Розв'зок.} Покажемо, що $p(\xi=0,\eta=0) \neq p(\xi=0)p(\eta=0)$. Помітимо, що:
\[
p(\xi=0) = p(\xi=0,\eta=0) + p(\xi=0,\eta=1) + p(\xi=0,\eta=2) = 0.1 + 0.1 + 0.2 = 0.4
\]
\[
p(\eta=0) = p(\xi=0,\eta=0) + p(\xi=1,\eta=0) + (\xi=2,\eta=0) = 0.1 + 0.1 + 0.2 = 0.4
\]

Проте, добре видно, що $p(\xi=0,\eta=0)=0.1 \neq 0.4 \times 0.4 = 0.16$. Отже, $\eta$ та $\xi$ не є незалежними.

\pagebreak
\section*{Завдання 2.}

\textbf{Умова.} Випадкові величини $\xi,\eta$ незалежні, причому випадкова величина $\xi$ приймає кожне значення $0,1,2$ з ймовірностями $0.2,0.1,0.7$, а випадкова величина $\eta$ приймає значення $-1,0,1$ з ймовірностями $0.3,0.3,0.4$, відповідно. Побудувати
таблицю розподілу випадкового вектор $[\xi,\eta]^{\top}$. Знайти розподіл суми $\xi+\eta$. 

\textbf{Розв'язок.} Оскільки випадкові величини незалежні, $p(\xi=\alpha,\eta=\beta)=p(\xi=\alpha)p(\eta=\beta)$ за означенням. Таким чином, розподіл $[\xi,\eta]^{\top}$ має наступний вигляд:

\begin{center}
\begin{tabular}{ |c|c|c|c| } 
 \hline
 -- & $\eta=-1$ & $\eta=0$ & $\eta=1$ \\ 
 \hline
 $\xi=0$ & $0.06$ & $0.06$ & $0.08$ \\ 
 \hline
 $\xi=1$ & $0.03$ & $0.03$ & $0.04$ \\ 
 \hline
 $\xi=2$ & $0.21$ & $0.21$ & $0.28$ \\
 \hline
\end{tabular}
\end{center}

Знайдемо розподіл $\xi+\eta$. Якщо перебрати усі суми, то множина можливих значень $\{-1,0,1,2,3\}$. Скористаємось тим фактом, що
\[
p(\xi+\eta=\zeta) = \sum_{(\alpha,\beta):\alpha+\beta=\zeta} p(\xi=\alpha,\eta=\beta)
\]

Таким чином,
\[
p(\xi+\eta=-1)=p(\xi=0,\eta=-1)=0.06
\]
\[
p(\xi+\eta=0)=p(\xi=0,\eta=0)+p(\xi=1,\eta=-1)=0.09
\]
\[
p(\xi+\eta=1)=p(\xi=0,\eta=1)+p(\xi=1,\eta=0)+p(\xi=2,\eta=-1)=0.32
\]
\[
p(\xi+\eta=2)=p(\xi=1,\eta=1)+p(\xi=2,\eta=0)=0.25
\]
\[
p(\xi+\eta=3)=p(\xi=2,\eta=1)=0.28
\]

Таким чином, маємо наступний розподіл:

\begin{center}
\begin{tabular}{ |c|c|c|c|c|c| } 
 \hline
 $\zeta$ & $-1$ & $0$ & $1$ & $2$ & $3$ \\ 
 \hline
 $p(\xi+\eta=\zeta)$ & $0.06$ & $0.09$ & $0.32$ & $0.25$ & $0.28$\\ 
 \hline
\end{tabular}
\end{center}
\pagebreak

\section*{Завдання 3.}

\textbf{Умова.} Двічі кинуто монету. Нехай $\xi$ -- число гербів, які випали при першому кидку,
$\eta$ –- число гербів, які випали при двох кидках. Чи є незалежними випадкові величини $\xi,\eta$?

\textbf{Розв'язок.} Якщо $\xi,\eta$ незалежні, то має місце $p(\xi=0,\eta=0) = p(\xi=0)p(\eta=0)$. Проте, $p(\eta=0)=p(\xi=0,\eta=0)$, оскільки якщо б $\xi=1$, то і $\eta>0$. Звідси випливає $p(\xi=0)=1$ або $p(\eta=0)=0$ -- протиріччя. Отже $\xi,\eta$ є залежними. 

\section*{Завдання 4.}

\textbf{Умова.} З колоди гральних карт вилучають дві карти. Нехай $X$ -- число тузів, $Y$ -- число
карт червоного кольору серед вилучених карт. Чи є незалежними випадкові величини $X$ та $Y$?

\textbf{Розв'язок.} Всього дві карти можна вибрати $C_{36}^2$ способами. Розглянемо три ймовірності: $p(X=0,Y=0), p(X=0),p(Y=0)$. 

$p(X=0)$ відповідає ймовірності не отримати жодного туза. Кількість способів вибрати $2$ ``не туза'' це $C_{32}^2$, тому ймовірність $p(X=0)=\frac{C_{32}^2}{C_{36}^2}$.

$p(Y=0)$ відповідає ймовірності не отримати жодної карти червого кольору. Аналогічно, $p(Y=0) = \frac{C_{18}^2}{C_{36}^2}$.

$p(X=0,Y=0)$ відповідає ймовірності не отримати жодного туза і жодної карти червого кольору. Всього тузів $4$, а червоних карт $18$, проте оскільки два тузи є червоними, то кількість карт, що є ані тузами, ані картами червого кольору, дорівнює $16$. Тому $p(X=0,Y=0) = \frac{C_{16}^2}{C_{36}^2}$.

Оскільки $C_{32}^2C_{18}^2 \neq C_{16}^2$, то і $p(X=0,Y=0)\neq p(X=0)p(Y=0)$, що означає, що випадкові величини $X,Y$ є залежними. 

\pagebreak

\section*{Завдання 5.}

\textbf{Умова.} $2$ білі та $3$ чорні кулі навмання розкладають по двом ящикам. Нехай $\xi$ --
число білих куль у першому ящику, $\eta$ -- номер ящика, в якому лежить більшість чорних куль. Чи є незалежними випадкові величини $\xi$ та $\eta$?

\textbf{Розв'язок.} Розглянемо $p(\xi=n,\eta=m), \; n \in \{0,1,2\}, m \in \{1,2\}$, тобто ймовірність, що у першому ящику буде $n$ білих куль, а у $m$ому ящику буде більшість чорних куль.

Достатньо легко бачити, що $p(\xi=n,\eta=m)=p(\xi=n)p(\eta=m)$. Дійсно, події ``у першому ящику $n$ білих куль'' (що відповідає $\xi=n$) та ``у $m$ому ящику більшість чорних куль'' ($\eta=m$) є незалежними, оскільки на те, в якому ящику більшість чорних куль не впливає кількість білих куль у першому ящику і навпаки. Тому випадкові величини також є незалежними.

\end{document}

