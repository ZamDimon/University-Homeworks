\documentclass[12pt]{extarticle}
\usepackage[english,ukrainian]{babel}
\usepackage[utf8]{inputenc}
\usepackage{amsmath,amssymb}
\usepackage{parskip}
\usepackage{graphicx}
\usepackage{xcolor}
\usepackage{tcolorbox}
\tcbuselibrary{skins}
\usepackage[framemethod=tikz]{mdframed}
\usepackage{chngcntr}
\usepackage{enumitem}
\usepackage{hyperref}
\usepackage{float}
\usepackage{subfig}
\usepackage{esint}
\usepackage[top=2.5cm, left=3cm, right=3cm, bottom=4.0cm]{geometry}
\usepackage[table]{xcolor}
\usepackage{algorithm}
\usepackage{algpseudocode}
\usepackage{listings}

\title{Контрольна робота (частина 1) з курсу ``Дискретна теорія ймовірності''}
\author{Студента групи МП-31 Захарова Дмитра Олеговича}
\date{\today}

\begin{document}

\maketitle

\textbf{Варіант 3.}

\section*{Завдання 1.} 

\textbf{Умова.} Знайти ймовiрнiсть того, що в навмання обраному п’ятизначному числi хоча б двi цифри
однаковi.

\textbf{Розв'язок.} Всього існує $9 \times 10^4$ п'ятизначних чисел. Знайдемо кількість різних п'ятизначних чисел, де хоча б дві цифри однакові і позначимо як $n$. В такому разі, відповіддю буде $p=n/(9 \times 10^4)$. 

Для цього легше знайти кількість п'ятизначних чисел, де всі цифри різні. На перше місце ми можемо поставити лише $9$ цифр, оскільки нуль поставити не можемо. На друге місце $9$ цифр (оскільки одне з $10$ ми вже використали). На третє, аналогічно, $8$, а далі $7$ та $6$. Отже, остаточно кількість способів $9 \times 9 \times 8 \times 7 \times 6$. Таким чином, $n = 9 \times 10^4 - 9^2 \times 8 \times 7 \times 6$. 

Тому остаточно
\[
p = \frac{9 \times 10^4 - 9^2 \times 8 \times 7 \times 6}{9 \times 10^4} = 1 - \frac{9 \times 8 \times 7 \times 6}{10^4} = 0.6976
\]
\textbf{Відповідь.} $1 - \frac{9\times 8 \times 7 \times 6}{10^4} = 0.6976$. 

\pagebreak

\section*{Завдання 2.}

\textbf{Умова.} При заповненнi документа перший бухгалтер помиляється з ймовiрнiстю $p_1=0.05$, а другий
iз ймовiрнiстю $p_2=0.01$. За певний час перший бухгалтер заповнив $n_1=80$ таких документiв,
а другий -- $n_2=120$. Усi документи були складенi в папку та перемiшанi. Навмання взятий
iз папки документ виявився з помилкою. Яка ймовiрнiсть, що його заповнював другий бухгалтер?

\textbf{Розв'язок.} Наведемо інтуїтивний розв'язок. В середньому (якщо бути точним, то ми кажемо про математичне очікування) кількість неправильних документів першого бухгалтера дорівнює $p_1n_1$, а у другого $p_2n_2$. Ймовірність того, що помилка була саме другого бухгалтера, це його частка у цій кількості, тобто $\frac{p_2n_2}{p_1n_1+p_2n_2} = \frac{1.2}{1.2+4}= \frac{3}{13}\approx 0.23$. 

Отримаємо цю відповідь більш строго. Нехай подія $A_i \; (i=1,2)$ відповідає, що обраний документ належить $i$ому бухгалтеру, $B_1$ означає, що документ правильний, а $B_2$ -- неправильний. 

Нам потрібно знайти $p(A_2 \mid B_2)$. За формулою Баєса:
\[
p(A_2 \mid B_2) = \frac{p(B_2 \mid A_2) p(A_2)}{p(B_2)}
\]
За умовою $p(B_2 \mid A_2) = p_2$. Ймовірність, що документ належить другому бухгалтеру, дорівнює $p(A_2)=\frac{n_2}{n_1+n_2}$.

Отже, залишилось лише знайти $p(B_2)$. Помітимо, що за формулою повної ймовірності
\[
p(B_2)=p(B_2 \mid A_1)p(A_1) + p(B_2 \mid A_2)p(A_2)
\]
За умовою $p(B_2 \mid A_i)=p_i$. Як ми вже виводили раніше, $p(A_i) = \frac{n_i}{n_1+n_2}$. Отже
\[
p(B_2) = \frac{p_1n_1 + p_2n_2}{n_1+n_2} 
\]

Таким чином:
\[
p(A_2 \mid B_2) = \frac{p_2 \cdot \frac{n_2}{n_1+n_2}}{\frac{p_1n_1+p_2n_2}{n_1+n_2}} = \frac{p_2n_2}{p_1n_1+p_2n_2}
\]

\textbf{Відповідь.} $\frac{3}{13} \approx 0.23$.

\pagebreak

\section*{Завдання 3.}

\textbf{Умова.} Скiльки разiв потрiбно пiдкидати гральний кубик, щоб з ймовiрнiстю не меншою $q=0.95$ отримати хоча б одне випадання п’ятiрки?

\textbf{Розв'язок.} Розглядаємо класичну схему Бернуллі. Нехай подія $A$ -- п'ятірка \textbf{не} випала, а також нехай ми зробили $n$ підкидувань. Ймовірність випадіння п'ятірки з одного ``спостереження'' дорівнює $\frac{1}{6}$, отже ймовірність невипадіння дорівнює $p=\frac{5}{6}$. 

Схема Бернуллі знаходить ймовірність події $B_k$, котра полягає в тому, що $A$ відбулося рівно $k$ разів:
\[
p(B_k) \triangleq C_n^k p^k (1-p)^{n-k}
\]
Розглянемо подію $C_n$, котра полягає в тому, що не випало жодної п'ятірки. Вона дорівнює $B_n$ (тобто не випала п'ятірка рівно $n$ разів з $n$ підкидувань). Нас цікавить подія $D_n=\overline{C}_n$ -- випала хоча б одна п'ятірка. Таким чином:
\[
p(D_n) = 1 - p(B_n) = 1 - C_n^np^n(1-p)^{n-n} = 1 - p^n
\]
Нас цікавить $N=\min \{n \in \mathbb{N}: p(D_n) \geq q\}$. Для цього розв'язуємо нерівність:
\[
1 - p^n \geq q \implies p^n \leq 1-q \implies n \geq \frac{\log(1-q)}{\log p} \approx 16.43
\]

Отже $N = 17$.

\textbf{Відповідь.} Хоча б $17$ разів.
\pagebreak
\section*{Завдання 4.}

\textbf{Умова.} З колоди в 36 карт навмання вибирають 7. Яка ймовiрнiсть, що серед них є туз, якщо
вiдомо, що витягнутi карти мають однакову масть?

\textbf{Розв'язок.} 

\textbf{Спосіб I.} Нехай подія $A$ -- з $7$ карт випав хоча б один туз, а подія $B$ -- витягнуті $7$ карт мають однакову масть. Отже, нас цікавить ймовірність:
\[
p(A \mid B) \triangleq \frac{p(A \cap B)}{p(B)}
\]
Знайдемо $p(A \cap B)$, тобто ймовірність того, що карти будуть однакової масті та серед них буде туз. Всього $7$ карт з $36$ можемо обрати $C_{36}^7$ способами. Оскільки є туз, то зафіксуємо його. Окрім туза, маємо $8$ позицій (оскільки в одній масті $9$ карт, а туз ми вже взяли), а на інші $6$ позицій можемо поставити карти $C_8^6$ способами. Також враховуємо, що масті $4$. Таким чином, $p(A\cap B) = \frac{4C_8^6}{C_{36}^7}$

Знайдемо $p(B)$, тобто ймовірність витягнути одну масть. Всього способів вибрати $7$ карт з $36$ дорівнює $C_{36}^7$. Кількість способів вибрати з $9$ карт якоїсь однієї масті (з чотирьох можливих) $7$ карт дорівнює $C_9^7$, а оскільки масті $4$, то загальна кількість $4C_9^7$. Отже $p(B) = \frac{4C_9^7}{C_{36}^7}$. Отже:
\[
p(A \mid B) = \frac{4C_8^6}{C_{36}^7} \cdot \frac{C_{36}^7}{4C_9^7} = \frac{C_8^6}{C_9^7}
\]
\textbf{Спосіб II.} Якщо карти мають однакову масть, то серед $9$ карт однієї масті нам потрібно знайти ймовірність взяти $7$ карт з тузом. Фіксуємо туз, а на інші $8$ позицій можемо поставити туз $C_8^6$ способами. Всього з $9$ карт можемо взяти сім $C_9^7$ способами. Так само отримуємо $\frac{C_8^6}{C_9^7}$. 

Цей вираз можна порахувати точніше. $C_8^6 = \frac{8!}{6!\times 2!}, \; C_9^7 = \frac{9!}{7!\times 2!}$, тому:
\[
\frac{C_8^6}{C_9^7} = \frac{8! \times 7!}{9!\times 6!} = \frac{7}{9}
\]
\textbf{Відповідь.} $\frac{7}{9}$.

\end{document}

