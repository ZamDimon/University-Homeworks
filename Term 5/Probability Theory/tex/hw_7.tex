\documentclass[14pt]{extarticle}
\usepackage[english,ukrainian]{babel}
\usepackage[utf8]{inputenc}
\usepackage{amsmath,amssymb}
\usepackage{parskip}
\usepackage{graphicx}
\usepackage{xcolor}
\usepackage{tcolorbox}
\tcbuselibrary{skins}
\usepackage[framemethod=tikz]{mdframed}
\usepackage{chngcntr}
\usepackage{enumitem}
\usepackage{hyperref}
\usepackage{float}
\usepackage{subfig}
\usepackage{esint}
\usepackage[top=2.5cm, left=3cm, right=3cm, bottom=4.0cm]{geometry}
\usepackage[table]{xcolor}
\usepackage{algorithm}
\usepackage{algpseudocode}
\usepackage{listings}

\title{Домашня робота з курсу ``Теорія Ймовірності''}
\author{Студента 3 курсу групи МП-31 Захарова Дмитра}
\date{\today}

\begin{document}

\maketitle
\section*{Вправа 1.}

\textbf{Теорема.} Нехай на дискретному ймовірністному просторі $(\Omega,p)$ задано випадковий вектор $\boldsymbol{\xi}: \Omega \to \mathbb{R}^n$. Нехай $f:\boldsymbol{\xi}(\Omega) \to \mathbb{R}$. Якщо $|\boldsymbol{\xi}(\Omega)| \in \mathbb{N}$, то завжди існує математичне сподівання:
\[
\mathbb{E}[f(\boldsymbol{\xi})] = \sum_{\boldsymbol{x} \in \boldsymbol{\xi}(\Omega)} f(\boldsymbol{x})p(\boldsymbol{\xi} = \boldsymbol{x})
\]
Якщо ж $\boldsymbol{\xi}(\Omega)$ зліченна, то це математичне очікування існує, якщо ряд $\sum_{\boldsymbol{x} \in \boldsymbol{\xi}(\Omega)} f(\boldsymbol{x})p(\boldsymbol{\xi}=\boldsymbol{x})$ збігається абсолютно.

\textbf{Доведення.} Розглянемо випадок, коли множина $\boldsymbol{\xi}(\Omega)$ скінченна. Нехай $f(\boldsymbol{\xi}(\Omega)) = \{y_1,\dots,y_m\}$ і позначимо $\mathcal{X}_i := \{\boldsymbol{x} \in \boldsymbol{\xi}(\Omega): f(\boldsymbol{x}) = y_i\}$. Тоді, за означенням,
\[
\mathbb{E}[f(\boldsymbol{\xi})] \triangleq \sum_{i=1}^m y_i p(f(\boldsymbol{\xi}) = y_i)
\]
Проте, цей вираз можна записати так:
\[
\mathbb{E}[f(\boldsymbol{\xi})] = \sum_{i=1}^m y_i p (\boldsymbol{\xi} \in \mathcal{X}_i) = \sum_{i=1}^m y_i \sum_{\boldsymbol{x} \in \mathcal{X}_i}p(\boldsymbol{\xi} = \boldsymbol{x}) = \sum_{i=1}^m \sum_{\boldsymbol{x} \in \mathcal{X}_i} y_i p(\boldsymbol{\xi}=\boldsymbol{x})
\]

Оскільки $\forall \boldsymbol{x} \in \mathcal{X}_i: f(\boldsymbol{x}) = y_i$, то
\[
\mathbb{E}[f(\boldsymbol{\xi})] = \sum_{i=1}^m \sum_{\boldsymbol{x} \in \mathcal{X}_i} f(\boldsymbol{x})p(\boldsymbol{\xi}=\boldsymbol{x})
\]
Нарешті помічаємо, що $\sum_{i=1}^m\sum_{\boldsymbol{x} \in \mathcal{X}_i} = \sum_{\boldsymbol{x} \in \boldsymbol{\xi}(\Omega)}$ (оскільки $\bigcap_{i=1}^m \mathcal{X}_i = \emptyset$ і при цьому $\bigcup_{i=1}^m \mathcal{X}_i=\boldsymbol{\xi}(\Omega)$), тому остаточно
\[
\mathbb{E}[f(\boldsymbol{\xi})] = \sum_{\boldsymbol{x} \in \boldsymbol{\xi}(\Omega)} f(\boldsymbol{x})p(\boldsymbol{\xi}=\boldsymbol{x})
\]
Як бачимо, у разі скінченності $\boldsymbol{\xi}(\Omega)$ усі переходи справедливі. Якщо ж $\boldsymbol{\xi}(\Omega)$ зліченна, то для останнього переходу має виконуватись абсолютна збіжність $\sum_{\boldsymbol{x} \in \boldsymbol{\xi}(\Omega)} f(\boldsymbol{x})p(\boldsymbol{\xi}=\boldsymbol{x})$.

\pagebreak
\section*{Вправа 2.}

\textbf{Умова.} Нехай на дискретному ймовiрнiсному просторi $(\Omega,p)$ задано дискретнi випадковi величини $\xi$, $\eta$, причому iснують $\mathbb{E}[\xi],\mathbb{E}[\eta]$. Тодi для будь-яких $a,b \in \mathbb{R}$ iснує
\[
\mathbb{E}[a\xi + b\eta] = a\mathbb{E}[\xi] + b\mathbb{E}[\eta]
\]

\textbf{Розв'язок.} Якщо існують $\mathbb{E}[\xi],\mathbb{E}[\eta]$, то існують і $\mathbb{E}[a\xi]$ та $\mathbb{E}[b\eta]$. Таким чином, якщо розглянути випадкові величини $\widetilde{\xi}=a\xi,\widetilde{\eta}=b\eta$, то маємо:
\[
\mathbb{E}[a\xi+b\eta] = \mathbb{E}[\widetilde{\xi}+\widetilde{\eta}]
\]
Використаємо лінійність математичного очікування ($\mathbb{E}[\widetilde{\xi}]$ та $\mathbb{E}[\widetilde{\eta}]$ існують, як показано вище):
\[
\mathbb{E}[\widetilde{\xi} + \widetilde{\eta}] = \mathbb{E}[\widetilde{\xi}] + \mathbb{E}[\widetilde{\eta}]
\]
Таким чином,
\[
\mathbb{E}[a\xi + b\eta] = \mathbb{E}[a\xi] + \mathbb{E}[b\eta] = a\mathbb{E}[\xi] + b \mathbb{E}[\eta]
\]
В останьому переході ми використали $\mathbb{E}[a\xi]=a\mathbb{E}[\xi]$ та $\mathbb{E}[b\eta]=b\mathbb{E}[\eta]$ оскільки за умовою $\mathbb{E}[\xi]$ та $\mathbb{E}[\eta]$ існують.
\pagebreak
\section*{Вправа 3.}

Довести, що
\[
m_n \triangleq \mathbb{E}[\xi^n]\;\;\text{існує} \iff \mu_n \triangleq \mathbb{E}[(\xi-\mathbb{E}[\xi])^n] \;\; \text{існує}
\]

\textbf{Розв'язок.} Доведемо теорему у бік $\rightarrow$. Оскільки існує $m_n$, то існує і $m_{n-1},\dots,m_0$ (це було доведено в лекції). Таким чином:
\begin{align*}
\mu_n \triangleq \mathbb{E}[(\xi - \mathbb{E}[\xi])^n] = \mathbb{E}\left[\sum_{k=0}^n (-1)^k C_n^k \cdot \mathbb{E}[\xi]^{n-k}\xi^k\right]  \\
=\sum_{k=0}^n (-1)^k C_n^k \cdot \mathbb{E}[\xi^{k}]\mathbb{E}[\xi]^{n-k} = \sum_{k=0}^n (-1)^k C_n^k \cdot m_km_1^{n-k}
\end{align*}

Оскільки всі $m_j, j \in \{0,\dots,n\}$ існують, то існує і вираз $\sum_{k=0}^n (-1)^k C_n^k m_{n-k}m_1^k$, отже існує і $\mu_n$.

Доводимо у бік $\leftarrow$ за індукцією. Нехай виконується
\[
\mu_n \; \; \text{існує} \implies m_n \; \; \text{існує}
\]
Доведемо, що це справедливо і для $n+1$. Отже:
\begin{align*}
\mu_{n+1} = \mathbb{E}[(\xi-\mathbb{E}[\xi])^{n+1}] = \mathbb{E}\left[\sum_{k=0}^{n+1} (-1)^k C_{n+1}^k \cdot \mathbb{E}[\xi]^{n+1-k}\xi^{k}\right] \\
= \sum_{k=0}^{n+1} (-1)^k C_{n+1}^k \cdot m_km_1^{n+1-k} = \underbrace{\sum_{k=0}^n (-1)^kC_{n+1}^k \cdot m_k m_1^{n+1-k}}_{=: R_n} + (-1)^{n+1} \cdot m_{n+1} \\
= R_n + (-1)^{n+1}m_{n+1}
\end{align*}
Звідси, можемо виразити $m_{n+1}$:
\[
m_{n+1} = (-1)^{n+1}(\mu_{n+1}-R_n)
\]
Ми знаємо, що $R_n:=\sum_{k=0}^n (-1)^kC_{n+1}^k \cdot m_k m_1^{n+1-k}$ існує, оскільки воно містить лише $m_0,m_1,\dots,m_n$, а вони існують, оскільки $m_n$ існує. Також, за припущенням індукції, існує $\mu_{n+1}$, тому і $\mu_{n+1}-R_n$ існує, звідки випливає існування $m_{n+1}$. Твердження доведено.
\pagebreak
\section*{Вправа 4.}

\textbf{Умова.} Довести наступні властивості середнього квадратичного відхилення:
\begin{enumerate}
    \item $\sigma(\xi) \geq 0$
    \item $\sigma(\xi) = 0 \iff \exists c \in \mathbb{R}: p(\xi=c)=1$
    \item $\sigma(\xi+c)=\sigma(\xi)$
    \item $\sigma(c\xi) = |c|\sigma(\xi)$
    \item $\sigma(\xi) = \sqrt{\mathbb{E}[\xi^2] - \mathbb{E}^2[\xi]}$
\end{enumerate}

\textbf{Розв'язок.} 

\textit{Властивість 1.} За означенням $\sigma(\xi) \triangleq \sqrt{\text{Var}[\xi]}$. За властивістю дисперсії, $\text{Var}[\xi] \geq 0$, тому і $\sqrt{\text{Var}[\xi]}$ визначено і більше за $0$.

\textit{Властивість 2.} Оскільки $\sigma(\xi) \triangleq \sqrt{\text{Var}[\xi]}=0$, то
\[
\sigma(\xi) = 0 \iff \text{Var}[\xi] = 0
\]

Тоді з першої властивості дисперсії, а саме
\[
\text{Var}[\xi] = 0 \iff \exists c \in \mathbb{R}: p(\xi=c)=1
\]
випливає така сама еквівалентність для $\sigma(\xi)$.

\textit{Пункт 3.} Маємо:
\[
\sigma(\xi+c) \triangleq \sqrt{\text{Var}[\xi+c]} = \sqrt{\text{Var}[\xi]} = \sigma(\xi)
\]

\textit{Пункт 4.} Маємо:
\[
\sigma(c\xi) \triangleq \sqrt{\text{Var}[c\xi]} = \sqrt{c^2\text{Var}[\xi]} = |c|\sqrt{\text{Var}[\xi]} = |c|\sigma(\xi)
\]

\textit{Пункт 5.} Оскільки $\text{Var}[\xi] = \mathbb{E}[\xi^2] - \mathbb{E}^2[\xi]$, а $\sigma(\xi) \triangleq \sqrt{\text{Var}[\xi]}$, маємо $\sigma(\xi) = \sqrt{\mathbb{E}[\xi^2] - \mathbb{E}^2[\xi]}$.

\end{document}

