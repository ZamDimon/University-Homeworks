\documentclass[12pt]{extarticle}
\usepackage[english,ukrainian]{babel}
\usepackage[utf8]{inputenc}
\usepackage{amsmath,amssymb}
\usepackage{parskip}
\usepackage{graphicx}
\usepackage{xcolor}
\usepackage{tcolorbox}
\tcbuselibrary{skins}
\usepackage[framemethod=tikz]{mdframed}
\usepackage{chngcntr}
\usepackage{enumitem}
\usepackage{hyperref}
\usepackage{float}
\usepackage{subfig}
\usepackage{esint}
\usepackage[top=2.5cm, left=3cm, right=3cm, bottom=4.0cm]{geometry}
\usepackage[table]{xcolor}
\usepackage{algorithm}
\usepackage{algpseudocode}
\usepackage{listings}

\title{Домашня робота з курсу ``Теорія Ймовірності''}
\author{Студента 3 курсу групи МП-31 Захарова Дмитра}
\date{\today}

\begin{document}

\maketitle

\textbf{Завдання 26.1.} 

Відповідь на $i$те запитання позначатимемо $E_i$.

1. $E_1 = \bigcap_{i=1}^n A_i$ 

2. $E_2 = \left(\bigcap_{i=1}^n A_i\right) \cup \left(\bigcap_{i=1}^n \overline{A}_i\right)$

3. $E_3 = \bigcup_{i=1}^n A_i$

4. $E_4 = \bigcup_{i=1}^n\left(A_i \cap \bigcap_{j=1,j\neq i}^n \overline{A}_j\right)$

5. Або всі кулі чорні, тобто $\bigcap_{i=1}^n \overline{A}_i$, або тільки одна куля біла, тобто $\bigcup_{i=1}^n\left(A_i \cap \bigcap_{j\neq i}^n \overline{A}_j\right)$, або рівно 2 кулі білі, тобто $\bigcup_{i,j: i \neq j}^n \left(A_i \cap A_j \cap \bigcap_{k \neq i,j}^n \overline{A}_k\right)$. Ітого:
\[
E_5 = \bigcap_{i=1}^n \overline{A}_i \cup \bigcup_{i=1}^n\left(A_i \cap \bigcap_{j\neq i}^n \overline{A}_j\right) \cup \bigcup_{i,j: i \neq j}^n \left(A_i \cap A_j \cap \bigcap_{k \neq i,j}^n \overline{A}_k\right)
\]

6. $E_6 = \overline{E}_5$

7. $E_7 = \bigcup_{i,j: i \neq j}^n \left(A_i \cap A_j \cap \bigcap_{k \neq i,j}^n \overline{A}_k\right)$

\textbf{Завдання 26.2.} 

1. Потрібно об'єднати усі події, коли для $A_{1i} A_{2j}$, справедливо $j>i$. Тобто:
\[
E_1 = \bigcup_{i,j:j > i}^n A_{1i} \cap A_{2j} 
\]

2. Потрібно обрати усі $A_{1j}$ для яких $j \leq k$. Тобто:
\[
E_2 = \bigcup_{i=1}^k A_{1j} 
\]

\textbf{Завдання 27.6.}


Якщо поставити туру у будь-яку клітинку, то ``зона ураження'' скаладається з $2(n-1)$ клітин. Тому ймовірність того, що дві тури в цьому випадку поб'ються, дорівнює $\frac{2(n-1)}{n^2 - 1} = \frac{2}{n+1}$. Щоб побиття було більш ймовірним сценарієм, потрібно виконання $\frac{2}{n+1} > 0.5$, тобто $n<3$. При $n=3$ отримуємо строгу рівність, тобто обидва випадки рівноймовірні. 

\textbf{Відповідь.} При $n = 2$ ймовірніше тури поб'ються, при $n=3$ однакова ймовірність на обидва випадки, $n>3$ ймовірність того, що вони не поб'ються, більша. 

\textbf{Завдання 27.8}

Позначимо відповідь на питання $j$ як $p_j$.

1. Всього шестизначних чисел $9 \cdot 10^5$. На кожну з позицій можемо поставити $8$ довільних цифр, отже всього таких чисел $8^6$. Тому ймовірність $p_1 = 8^6/(9 \cdot 10^5)$

2. Маємо $6$ способів поставити $9$, а на інші можемо поставити $9^5$ способами. Тому $p_2 = (6 \cdot 9^5) / (9 \cdot 10^5)$

3. Є $5$ способів поставити $0$ на позиції, $8$ способів поставити цифри на першу позицію та $9^4$ на інші. Тому $p_3=(5 \cdot 8 \cdot 9^4)/(9 \cdot 10^5)$

4. Протилежне твердженню ``є і 0, і 9'' це ``є 0, але немає 9 або є 9, але немає 0''. Отже, $p_4 = 1 - (p_2+p_3)$.

5. Це твердження можна розбити як: ``є 9, але немає 0'' або ``є 0, але немає 9'' або ``немає 9 та 0''. Отже, $p_5 = p_1 + p_2 + p_3$. 

\textbf{Завдання 27.9}. Не розумію, що таке ``однакові'' та ``різні'' кубики. 

\textbf{Завдання 27.1}

Легше порахувати ймовірність, що червоних кульок було рівно 0. Кількість таких подій $C_9^3$. Всього 3 кульки можна витягнути $C_{14}^3$ способами, отже ймовірність $C_9^3/C_{14}^3$. 

Отже, ймовірність мати хоча б одну червону кульку $1 - C_9^3/C_{14}^3$. 


\textbf{Завдання 27.13.}

Всього варіантів дней народжень $12^{12}$. Варіантів перестановок $12$ місяців існує $12!$, тому ймовірність $12!/12^{12}$.

\textbf{Завдання 27.14}

1. $1/8^3$. 

2. $1/8^2$.

3. $A_8^3 / 8^3$

4. $A_8^2 / 8^3$

\textbf{Завдання 27.16}

1. Шанс, що не випаде жодна одиниця у першого гравця, дорівнює $(5/6)^{6}$. Тому шанс випадіння хоча б однієї дорівнює $p_1=1-(5/6)^{6}$. 

У другого шанс випадіння жодної одиниці $(5/6)^{12}$, а шанс випадіння рівно однієї дорівнює $(12\cdot 5^{11})/6^{12}$, тому загальний шанс $p_2=1-(5/6)^{12}-2\cdot (5/6)^{11}$. 

Якщо порівняти, виходить $p_1>p_2$. 

2. Шанс випадіння рівно однієї для першого дорівнює $p_1=(6 \cdot 5^5)/6^6 \approx 0.402$, а для другого $p_2=(12 \cdot 5^{11})/6^{12} \approx 0.269$. Отже знову $p_1>p_2$.  

\textbf{Завдання 27.17}

В першому випадку шанс дорівнює $p_1=1-(5/6)^4 \approx 0.517$. 

В другому випадку шанс не отримати 2 одиниці за один кидок двох кубиків дорівнює $1 - (1/6)^2=35/36$. Отже шанс дорівнює $p_2=1-(35/36)^{24} \approx 0.49$. Отже в першому випадку шанс більший.

\textbf{Завдання 27.18}

Всього варіантів обрати $K$ куль дорівнює $C_{a+b}^K$. Вважаємо, що в умові мається на увазі ``хоча б $k$ білих куль''. Тоді існує $C_{a}^k$ способів обрати $k$ білих куль та $C_{a+b-k}^{K-k}$ інші. Отже, відповідь $\frac{C_a^kC_{a+b-k}^{K-k}}{C_{a+b}^K}$.

Якщо мається на увазі ``рівно $k$ білих куль'', то $\frac{C_a^kC_b^{K-k}}{C_{a+b}^K}$

\textbf{Завдання 27.20}

Маємо $C_{2n}^n$ варіантів розподілу на команди та $C_{2n-2}^{n-2}$ розподілу 2 гравців в одну команду. Отже, $1-C_{2n-2}^{n-2}/C_{2n}^n$ 

\textbf{Завдання 27.26}

Або 3 числа парні, або 2 числа непарні, а інше парне. Всього парних чисел $k$, тому ймовірність першого:
\[
p_1 = \frac{k(k-1)(k-2)}{(2k)^3} = \frac{(k-1)(k-2)}{8k^2}
\]

Якщо 2 числа непарне, а інше парне, то ймовірність цього:
\[
p_2 = \frac{k^2(k-1)}{8k^3} = \frac{k-1}{8k} 
\]

Отже, загальна ймовірність
\[
p = p_1 + p_2 = \frac{(k-1)(k-2) + k^2 - k}{8k^2} = \frac{k^2-3k+2+k^2-k}{8k^2} = \frac{(k-1)^2}{4k^2}
\]

\end{document}

