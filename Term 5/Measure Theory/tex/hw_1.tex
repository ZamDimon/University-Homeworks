\documentclass[12pt]{extarticle}
\usepackage[english,ukrainian]{babel}
\usepackage[utf8]{inputenc}
\usepackage{amsmath,amssymb}
\usepackage{parskip}
\usepackage{graphicx}
\usepackage{tcolorbox}
\tcbuselibrary{skins}
\usepackage[framemethod=tikz]{mdframed}
\usepackage{chngcntr}
\usepackage{enumitem}
\usepackage{hyperref}
\usepackage{float}
\usepackage{subfig}
\usepackage{esint}
\usepackage[top=2.5cm, left=3cm, right=3cm, bottom=4.0cm]{geometry}
\usepackage[table]{xcolor}
\usepackage{algorithm}
\usepackage{algpseudocode}
\usepackage{listings}

\title{Домашня робота з курсу ``Теорія міри''}
\author{Студента 3 курсу групи МП-31 Захарова Дмитра}
\date{\today}

\begin{document}

\maketitle

\section*{Завдання О3}

\textbf{Умова.} 

Довести, що

\begin{enumerate}
    \item кільце є замкненим відносно операцій $\cap$ та $\Delta$;
    \item об'єднання та перетин скінченної сукупності елементів кільця належать до кільця.
\end{enumerate}

\textbf{Розв'язок.} 

1. Нехай маємо кільце $\mathcal{H}$. Тоді, згідно означенню, справедливо:
\begin{enumerate}
    \item $\forall A, B \in \mathcal{H}: A \cup B \in \mathcal{H}$
    \item $\forall A, B \in \mathcal{H}: A \setminus B \in \mathcal{H}$
\end{enumerate}

Кільце є замкненим відносно $\cap$ якщо $\forall A, B \in \mathcal{H}$ буде справедливо $A \cap B \in \mathcal{H}$. Цю властивість було доведено на лекції наступним чином: запишемо
\[
A \cap B = A \setminus (A \setminus B)
\]

Згідно означенню $1$ різниця множин буде належати $\mathcal{H}$, а отже після двічі застосування різниці знову опиняємось у $\mathcal{H}$.

Доведемо тепер, що $\forall A, B \in \mathcal{H}: A \,\Delta\, B \in \mathcal{H}$. Згідно означенню:
\[
A \, \Delta \, B = (B \setminus A) \cup (A \setminus B)
\]

Згідно властивості 2, маємо $B \setminus A \in \mathcal{H}, A \setminus B \in \mathcal{H}$. Згідно властивості 1, об'єднання елементів з кільця дасть елемент кільця, а отже весь вираз $A \Delta B$ знаходиться в кільці. 

2. Нехай маємо $\{H_k\}_{k=1}^n \subset \mathcal{H}, n>1$. Потрібно довести $\bigcup_{k=1}^n H_k, \bigcap_{k=1}^n H_k \in \mathcal{H}$. 

Випадок $n=2$ доведений з поперднього пункту (для операції $\cap$) та з означення (для операції $\cup$). 

Якщо $n>2$, то можна довести, наприклад, за індукцією. База в нас вже є. Отже, нехай твердження справедливе для $m>2$, тобто $\bigcup_{k=1}^m H_k =: S_m \in \mathcal{H}$. Тоді це справедливо і для $m+1$, оскільки $\bigcup_{k=1}^{m+1} H_k = S_m \cup H_{m+1} \in \mathcal{H}$, що випливає з означення кільця. Аналогічно можна довести і для $\cap$.

\section*{Завдання С3}

\textbf{Умова.} Довести, що сукупність усіх обмежених підмножин прямої $\mathbb{R}$ утворює кільце, але не є ані $\sigma$-кільцем, ані $\sigma$-алгеброю. 

\textbf{Розв'язок.} Нехай маємо сукупність обмежених підмножин $\mathcal{H}$ прямої $\mathbb{R}$. Тоді
\[
\forall H \in \mathcal{H} \; \exists \rho > 0 \; \forall x, y \in H: d(x,y) < \rho 
\]

Доведемо, що $\mathcal{H}$ є кільцем. Спочатку доведемо, що $\forall A, B \in \mathcal{H}: A \cup B \in \mathcal{H}$. Тобто, нехай ми знаємо, що
\[
\exists \rho_A > 0 \; \forall x,y \in A: d(x,y) < \rho_A
\]
\[
\exists \rho_B > 0 \; \forall x,y \in B: d(x,y) < \rho_B
\]

Нам потрібно знайти таке $\rho_{A \cup B} > 0$, що $\forall x,y \in A \cup B: d(x,y) < \rho_{A \cup B}$. Для цього покладемо $\rho_{A \cup B} := \rho_A + \rho_B$. Тоді, який елемент б ми не взяли, будь це з $A$ або $B$, все одно відстань між ними буде менша за $\rho_A + \rho_B$. 

Тепер покажемо, що $A \setminus B \in \mathcal{H}$. Тобто знайдемо таке $\rho_{A\setminus B} > 0$, що $\forall x,y \in A \setminus B: d(x,y)<\rho_{A \setminus B}$. Для цього достатньо покласти $\rho_{A \setminus B} := \rho_A$, оскільки віднімання від $A$ якоїсь частини не збільшує ``радіус'' множини. 

Доведемо, що $\mathcal{H}$ не є $\sigma$-кільцем. Згідно означенню, має виконуватись:
\begin{enumerate}
    \item $\forall \{H_k\}_{k=1}^{\infty} \subset \mathcal{H}: \bigcup_{k=1}^{\infty}H_k \in \mathcal{H}$
    \item $\forall A,B \in \mathcal{H}: A \setminus B \in \mathcal{H}$
\end{enumerate}

Друга властивість, як ми довели вище, виконується. Доведемо, що
\[
\exists \{H_k\}_{k=1}^{\infty} \subset \mathcal{H}: \bigcup_{k=1}^{\infty} H_k \notin \mathcal{H}
\]

Дійсно, візьмем $H_k := [k,k+1]$. Тоді якщо позначити $S_n := \bigcup_{k=1}^{n}H_k$, то $S_n = [1,n+1]$. В такому разі $\lim_{n\to\infty}S_n = [1,+\infty)$, що звичайно не є обмеженою множиною, тобто вона не належить $\mathcal{H}$. 

Оскільки $\mathcal{H}$ не є $\sigma$-кільцем, то вона і не є $\sigma$-алгеброю. Окрім цього, $\mathbb{R} \notin \mathcal{H}$, оскільки $\mathbb{R}$ не є обмеженою. 

\section*{Завдання Д1}

\textbf{Умова.} Довести, що клас множин є кільцем, якщо він замкнений відносно 1. $(\cup, \Delta)$ та 2. $(\cap, \Delta)$. 

\textbf{Розв'язок.} 

1. Нехай маємо $\mathcal{H}$, що є замкненою відносно $(\cup,\Delta)$. Нам потрібно довести замкнення відносно $(\cup,\setminus)$, тобто лише відносно $\setminus$. Візьмемо дві множини $A, B \in \mathcal{H}$. Тоді помітимо, що:
\[
A \setminus B = (A \cup B) \Delta B
\]

Дійсно,
\[
(A \cup B) \Delta B = ((A \cup B) \setminus B) \cup (B \setminus (A \cup B)) = (A \setminus B) \cup \emptyset = A \setminus B
\]

Оскільки $A \cup B \in \mathcal{H}$, то і $A\setminus B = (A \cup B) \Delta B \in \mathcal{H}$.

2. Нехай $\mathcal{H}$ замкнена відносно $(\cap, \Delta)$. Доведемо замкнення відносно $\setminus$. Маємо
\[
A \setminus B = (A \Delta B) \cap A
\]

Знову, ми виразили все через операції $(\Delta,\cap)$, тому звідси випливає замкненість через $\setminus$. 

Об'єднання можемо записати як:
\[
A \cup B = \underbrace{(A \Delta B)}_{\in \mathcal{H}} \Delta \underbrace{(A \cap B)}_{\in \mathcal{H}} \in \mathcal{H}
\]

Отже, ми довели замкненість по $(\cup, \setminus)$, що означає, що $\mathcal{H}$ є кільцем.

\end{document}

