\documentclass[14pt]{extarticle}
\usepackage[english,ukrainian]{babel}
\usepackage[utf8]{inputenc}
\usepackage{amsmath,amssymb}
\usepackage{parskip}
\usepackage{graphicx}
\usepackage{tcolorbox}
\tcbuselibrary{skins}
\usepackage[framemethod=tikz]{mdframed}
\usepackage{chngcntr}
\usepackage{enumitem}
\usepackage{hyperref}
\usepackage{float}
\usepackage{subfig}
\usepackage{esint}
\usepackage[top=2.5cm, left=3cm, right=3cm, bottom=4.0cm]{geometry}
\usepackage[table]{xcolor}
\usepackage{algorithm}
\usepackage{algpseudocode}
\usepackage{listings}
\usepackage{dsfont}

\title{Самостійна робота з курсу ``Теорія міри''}
\author{Студента 3 курсу групи МП-31 Захарова Дмитра}
\date{\today}

\begin{document}

\maketitle

\section*{Завдання}
\textbf{Умова.} Нехай
\[
F(x) = \begin{cases}
    1, & x < -3 \\
    3, & x \geq -3
\end{cases}, x \in \mathbb{R}
\]
$\lambda_F: \mathcal{P}_1 \to [0,+\infty),\lambda_F((a,b])=F(b)-F(a)$. Знайти
\begin{enumerate}
    \item $\lambda_F^*(A), A \in 2^{\mathbb{R}}$.
    \item $\sigma$-алгебру $\mathcal{S}_F$ $\lambda_F^*$-вимірних підмножин $\mathbb{R}$.
\end{enumerate}

\textbf{Розв'язок.} 

Для наступних пунктів зручно помітити, що
\[
\lambda_F((a,b]) = \begin{cases}
    2, & -3 \in (a,b]\\
    0, & -3 \not \in (a,b]
\end{cases} = 2 \cdot \mathds{1}_{(a,b]}(-3)
\]
\textit{Пункт 1.}  Візьмемо $X \subset \mathbb{R}$. Розглянемо два випадки: $-3$ належить та не належить $X$.

\textit{Випадок 1.} $-3 \not \in X$. За означенням маємо для $\forall X \subset \mathbb{R} \setminus \{-3\}$:
\[
\lambda^*_F(X) = 
    \inf\left\{\sum_{n \in \mathbb{N}}\lambda_F(A_n): X \subset \bigcup_{n \in \mathbb{N}}A_n \wedge \{A_n\}_{n \in \mathbb{N}} \subset \mathcal{P}_1\right\}
\]
Помітимо, що $\mathbb{R} \setminus \{-3\}$ можемо записати як
\[
(-\infty,-3) \cup (-3,+\infty) = \bigcup_{n \in \mathbb{N}} (-n, -3-\frac{1}{n}\Big] \cup \bigcup_{n \in \mathbb{N}} (-3,n]
\]
В такому разі:
\begin{gather*}
0 \leq \lambda_F^*(\mathbb{R} \setminus \{3\}) \leq \lambda_F\left(\bigcup_{n \in \mathbb{N}} (-n, -3-\frac{1}{n}\Big]\right) + \lambda_F\left(\bigcup_{n \in \mathbb{N}}(-3,n]\right) \\
\leq \sum_{n \in \mathbb{N}}\lambda_F\left((-n,-3-\frac{1}{n}\Big]\right) + \sum_{n \in \mathbb{N}} \lambda_F((-3,n]) \\
= 2\sum_{n \in \mathbb{N}}\mathds{1}_{(n,-3-1/n]}(-3) + 2\sum_{n \in \mathbb{N}}\mathds{1}_{(-3,n]}(-3) = 0
\end{gather*}
Звідси, $\lambda_F^*(\mathbb{R} \setminus \{3\}) = 0$. Оскільки $\lambda_F^*$ є моноотонною, то $\forall X \subset \mathbb{R} \setminus \{-3\}: \lambda_F^*(X)=0$. 

\textit{Випадок 2.} $-3 \in X$. Помітимо, що інфінум відповідає такій побудові $\{A_n\}_{n \in \mathbb{N}}$:
\[
\exists m \in \mathbb{N}: \{-3 \in A_m\} \wedge \{-3 \not \in A_n \; \forall n \neq m\} 
\]
В такому разі:
\[
\lambda_F^*(X) = \sum_{n \in \mathbb{N}} \lambda_F(A_n)=\lambda_F(A_m) + \underbrace{\sum_{n \in \mathbb{N}: n \neq m}\lambda_F(A_n)}_{=0} = \lambda_F(A_m) = 2
\]
Відповімо, чому це відповідає інфінуму. Дійсно, нехай ще якийсь номер $p \in \mathbb{N}: -3 \in A_p$. Тоді за аналогією, $\sum_{n \in \mathbb{N}}\lambda_F(A_n) = \lambda_F(A_m)+\lambda_F(A_p)=4 > 2$. Якщо ж не знайдеться номера $m$ з властивістю вище, то тоді $-3 \not\in X$, що суперечить припущенню цього випадку. 

Отже, маємо:
\[
\lambda_F^*(X) = 2 \cdot \mathds{1}_X(-3)
\]

\textit{Пункт 2.} З'ясуємо, коли $X \subset \mathbb{R}$ є $\lambda_F^*$-вимірною множиною. За означенням:
\[
\forall E \subset \mathbb{R}: \lambda_F^*(E) = \lambda_F^*(E \cap X) + \lambda_F^*(E \cap \overline{X})
\]
Ця умова еквівалентна:
\[
\mathds{1}_E(-3) = \mathds{1}_{E \cap X}(-3) + \mathds{1}_{E \cap \overline{X}}(-3)
\]
Нехай $-3 \in E$. В такому разі, якщо $-3 \in X$, то $-3 \in E \cap X$, проте $-3 \not \in E \cap \overline{X}$. Аналогічно, якщо $-3 \not\in X$, то $-3 \not \in E \cap X$, але $-3 \in E \cap \overline{X}$. Отже, як ліва, так і права частина буде дорівнювати $2$, що каже про те, що для $-3 \in E$ ця умова виконується для будь-яких $X$.

Тепер нехай $-3 \not\in E$. В такому разі, як $-3 \not\in E \cap X$, так і $-3 \not\in E \cap \overline{X}$. В цьому випадку обидві частини дорівнюють $0$, звідки випливає справедливість цього твердження для будь-яких $-3 \not\in E$ та $X$. 

Отже, це твердження справедливе для будь-якої пари $X,E \subset \mathbb{R}$, а тому будь-яка множина $X \subset \mathbb{R}$ є $\lambda_F^*$-вимірною. Звідси $\mathcal{S}_F = 2^{\mathbb{R}}$.

\textbf{Відповідь.} 1. $\lambda_F^*(X) = 2 \cdot \mathds{1}_X(-3)$. 2. $\mathcal{S}_F=2^{\mathbb{R}}$.

\end{document}

