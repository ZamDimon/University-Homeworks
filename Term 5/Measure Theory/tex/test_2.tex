\documentclass[12pt]{extarticle}
\usepackage[english,ukrainian]{babel}
\usepackage[utf8]{inputenc}
\usepackage{amsmath,amssymb}
\usepackage{parskip}
\usepackage{graphicx}
\usepackage{tcolorbox}
\tcbuselibrary{skins}
\usepackage[framemethod=tikz]{mdframed}
\usepackage{chngcntr}
\usepackage{enumitem}
\usepackage{hyperref}
\usepackage{float}
\usepackage{subfig}
\usepackage{esint}
\usepackage[top=2.5cm, left=3cm, right=3cm, bottom=4.0cm]{geometry}
\usepackage[table]{xcolor}
\usepackage{algorithm}
\usepackage{algpseudocode}
\usepackage{listings}

\title{Самостійна робота з курсу ``Теорія міри''}
\author{Студента 3 курсу групи МП-31 Захарова Дмитра}
\date{\today}

\begin{document}

\maketitle

\section*{Завдання 1}

\textbf{Умова.} Нехай $X = \mathbb{N}$, а також
\[
\mathcal{H} = \{ \{2k\}: k \in \mathbb{N}\} \cup \{\emptyset\}
\]

Чи є $\mathcal{H}$ кільцем? Якщо ні, то знайдіть $k(\mathcal{H})$.

\textbf{Розв'язок.} За означенням кільця, має виконуватись наступні дві умови:
\begin{enumerate}
    \item $\forall A,B \in \mathcal{H}: A \cup B \in \mathcal{H}$
    \item $\forall A,B \in \mathcal{H}: A \setminus B \in \mathcal{H}$
\end{enumerate}

Одразу бачимо, що перша умова не виконується. Наприклад, нехай $A = \{2\} \in \mathcal{H}, B = \{10\} \in \mathcal{H}$. В такому разі $A \cup B = \{2,10\} \not\in \mathcal{H}$. 

Знайдемо кільце, породжене класом $\mathcal{H}$. За означенням:
\[
k(\mathcal{H}) \triangleq \bigcap_{\mathcal{H} \subset \mathcal{K}_{\alpha},\; \mathcal{K}_{\alpha} \, \text{є кільцем}} \mathcal{K}_{\alpha}
\]

$k(\mathcal{H})$ є кільцем і окрім того, має містити $\mathcal{H}$. Значить, нам потрібно якимось чином мінімально доповнити $\mathcal{H}$ до кільця. 

Наприклад, таким доповненням може бути просто множина усіх скінченних множин з парних чисел. Дійсно, $\mathcal{H}$ треба доповнити, додавши усі можливі об'єднання елементів з $\mathcal{H}$ (включно з тими, що ми ``потенціально'' додамо). Якщо в нас будуть усі скінченні множини з парних чисел, то як би ми не об'єднували або віднімали дві скінченні множини з парних чисел, ми все ще будемо отримувати якусь іншу скінченну множину парних чисел. Так само очевидно, що сам $\mathcal{H}$ міститься в $k(\mathcal{H})$, оскільки $\mathcal{H}$ це просто множина з множин, що містять одне парне число, що теж є скінченною множиною парних чисел (просто лише одного).

\textbf{Відповідь.} $\mathcal{H}$ не є кільцем. $k(\mathcal{H}) = \{A \in 2^{2\mathbb{N}}: A \text{ -- скінченна}\}$


\end{document}

