\documentclass[14pt]{extarticle}
\usepackage[english,ukrainian]{babel}
\usepackage[utf8]{inputenc}
\usepackage{amsmath,amssymb}
\usepackage{parskip}
\usepackage{graphicx}
\usepackage{tcolorbox}
\tcbuselibrary{skins}
\usepackage[framemethod=tikz]{mdframed}
\usepackage{chngcntr}
\usepackage{enumitem}
\usepackage{hyperref}
\usepackage{float}
\usepackage{subfig}
\usepackage{esint}
\usepackage[top=2.5cm, left=3cm, right=3cm, bottom=4.0cm]{geometry}
\usepackage[table]{xcolor}
\usepackage{algorithm}
\usepackage{algpseudocode}
\usepackage{listings}
\usepackage{dsfont}

\title{Самостійна робота з курсу ``Теорія міри''}
\author{Студента 3 курсу групи МП-31 Захарова Дмитра}
\date{\today}

\begin{document}

\maketitle

\section*{Завдання}
\textbf{Умова.} Довести, що множина $A$ є борельовою і обчислити $\lambda_1(A)$ для
\[
A = \bigcup_{n \in \mathbb{Z}^+}\left(e^n -\frac{1}{\cosh n + \sinh n},e^n + \frac{1}{\cosh n + \sinh n}\right]
\]

\textbf{Розв'язок.} З теоретичного матеріалу відомо, що будь-який відрізок $(\alpha,\beta] \in \mathcal{B}(\mathbb{R})$. Отже, нескінченне об'єднання борельових множин теж є борельовою множиною, тому $A \in \mathcal{B}(\mathbb{R})$. 

Обчислимо міру Лебега:
\begin{gather*}
\lambda_1(A) = \sum_{n=0}^{\infty}\left(e^n + \frac{1}{\cosh n + \sinh n} - \left(e^n - \frac{1}{\cosh n + \sinh n}\right)\right) \\ = 2\sum_{n=0}^{\infty} \frac{1}{\cosh n + \sinh n} = 2\sum_{n=0}^{\infty} \frac{1}{\frac{e^n+e^{-n}}{2} + \frac{e^n - e^{-n}}{2}} = 2\sum_{n=0}^{\infty}e^{-n} = \frac{2e}{e-1}
\end{gather*}

\textbf{Відповідь.} $\frac{2e}{e-1}$.

\end{document}

