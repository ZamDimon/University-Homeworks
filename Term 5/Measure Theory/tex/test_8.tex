\documentclass[14pt]{extarticle}
\usepackage[english,ukrainian]{babel}
\usepackage[utf8]{inputenc}
\usepackage{amsmath,amssymb}
\usepackage{parskip}
\usepackage{graphicx}
\usepackage{tcolorbox}
\tcbuselibrary{skins}
\usepackage[framemethod=tikz]{mdframed}
\usepackage{chngcntr}
\usepackage{enumitem}
\usepackage{hyperref}
\usepackage{float}
\usepackage{subfig}
\usepackage{esint}
\usepackage[top=2.5cm, left=3cm, right=3cm, bottom=4.0cm]{geometry}
\usepackage[table]{xcolor}
\usepackage{algorithm}
\usepackage{algpseudocode}
\usepackage{listings}
\usepackage{dsfont}

\title{Самостійна робота з курсу ``Теорія міри''}
\author{Студента 3 курсу групи МП-31 Захарова Дмитра}
\date{\today}

\begin{document}

\maketitle

\section*{Завдання}
\textbf{Умова.} Довести, що множина $A$ є борельовою та знайти її мiру Лебега $\lambda_2(A)$ для
\[
A = ([-4,8] \times (2,4)) \setminus (\mathbb{Q} \times \mathbb{R})
\]

\textbf{Розв'язок.} 

\textit{Підпункт 1.} Доведемо, що $A \in \mathcal{B}(\mathbb{R}^2)$. Помітимо, що
\[
\mathbb{Q} \times \mathbb{R} = \bigcup_{q \in \mathbb{Q}} \{q\} \times \mathbb{R}
\]
Помітимо, що $\{q\} \times \mathbb{R} \in \mathcal{B}(\mathbb{R}^2)$, тому і нескінченне об'єднання буде борельовою множиною. Отже, ми маємо, що $\mathbb{Q} \times \mathbb{R} \in \mathcal{B}(\mathbb{R}^2)$.

Далі розберемося з множиною $[-4,8] \times (2,4)$. Запишемо цей добуток наступним чином:
\begin{gather*}
[-4,8] \times (2,4) = \bigcup_{n \in \mathbb{N}} \left(\{-4\} \cup (-4,8]\right) \times \left( 2, 4 - \frac{1}{n} \right] \\
= \bigcup_{n \in \mathbb{N}} \left( \{-4\} \times \left(2,4-\frac{1}{n}\right]\right) \cup \left((-4,8] \times \left(2,4-\frac{1}{n}\right]\right)
\end{gather*}
Проаналізуємо тепер цей вираз. $\{-4\}\times (2,4-\frac{1}{n}] \in \mathcal{B}(\mathbb{R}^2)$, а отже і нескінченне об'єднання також буде належати $\mathcal{B}(\mathbb{R}^2)$. 

Для правого виразу $(4,8] \times \left(2,4-\frac{1}{n}\right]$ помічаємо, що він належить $\mathcal{P}_2$, а отже належить і $\mathcal{B}(\mathbb{R}^2)$. 

Отже, маємо об'єднання двох множин $\mathcal{B}(\mathbb{R}^2)$, що теж є борельовою множиною. Таким чином, $[-4,8] \times (2,4) \in \mathcal{B}(\mathbb{R}^2)$.

Отже, $A$ є різницею двох борельових множин, а оскільки $\sigma$-алгебра є замкненою відносно $\setminus$, то і $A \in \mathcal{B}(\mathbb{R}^2)$.

\textit{Підпункт 2.} Знайдемо $\lambda_2(A)$. Помітимо, що з іншого боку
\[
A = ([-4,8) \setminus \mathbb{Q}) \times (2,4) = \bigcup_{n \in \mathbb{N}} ([-4,8] \setminus \mathbb{Q}) \times \left(2,4-\frac{1}{n}\right]
\]
Отже, якщо позначити $A_n := \left([-4,8] \setminus \mathbb{Q}\right) \times \left(2,4-\frac{1}{n}\right]$, то
\begin{gather*}
\lambda_2(A) = \lambda_2(\lim_{n \to \infty}A_n) = \lim_{n \to \infty}\lambda_2(A_n) = \lim_{n \to \infty}\left((8+4) \cdot \left(2 - \frac{1}{n}\right)\right) \\
= 12 \lim_{n \to \infty}\left(2 - \frac{1}{n}\right) = 24
\end{gather*}

\textbf{Відповідь.} $\lambda_2(A) = 24$.

\end{document}

