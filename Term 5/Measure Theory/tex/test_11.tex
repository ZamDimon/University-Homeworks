\documentclass[14pt]{extarticle}
\usepackage[english,ukrainian]{babel}
\usepackage[utf8]{inputenc}
\usepackage{amsmath,amssymb}
\usepackage{parskip}
\usepackage{graphicx}
\usepackage{tcolorbox}
\tcbuselibrary{skins}
\usepackage[framemethod=tikz]{mdframed}
\usepackage{chngcntr}
\usepackage{enumitem}
\usepackage{hyperref}
\usepackage{float}
\usepackage{subfig}
\usepackage{esint}
\usepackage[top=2.5cm, left=3cm, right=3cm, bottom=4.0cm]{geometry}
\usepackage[table]{xcolor}
\usepackage{algorithm}
\usepackage{algpseudocode}
\usepackage{listings}
\usepackage{dsfont}

\title{Самостійна робота з курсу ``Теорія міри''}
\author{Студента 3 курсу групи МП-31 Захарова Дмитра}
\date{\today}

\begin{document}

\maketitle

\section*{Завдання}
\textbf{Умова.} Обчислити інтеграл
\[
\mathcal{I} = \int_{[0,100)} \frac{d\lambda_1(x)}{[5x+2][5x+4]}
\]

\textbf{Розв'язок.} Помітимо, що $[0,100) = \bigcup_{k=0}^{499} \left[ \frac{k}{5}, \frac{k+1}{5} \right)$. В такому разі скористаємося теоремою про $\sigma$-адитивність інтеграла Лебега:
\[
\mathcal{I} = \sum_{k=0}^{499} \int_{[\frac{k}{5},\frac{k+1}{5})} \frac{d\lambda_1(x)}{[5x+2][5x+4]}
\]

Помітимо, що на інтервалі $\left[ \frac{k}{5}, \frac{k+1}{5} \right)$ значення $5x+2$ лежать між $k+2$ до $k+3$ не включно, тому $[5x+2]\Big|_{x \in \left[ \frac{k}{5}, \frac{k+1}{5} \right)} = k+2$. Аналогічно отримуємо $[5x+4]\Big|_{x \in \left[ \frac{k}{5}, \frac{k+1}{5} \right)} = k+4$. Таким чином:
\[
\mathcal{I} = \sum_{k=0}^{499}\int_{[\frac{k}{5},\frac{k+1}{5})} \frac{d\lambda_1(x)}{(k+2)(k+4)} = \sum_{k=0}^{499} \frac{1}{(k+2)(k+4)} \cdot \int_{[\frac{k}{5},\frac{k+1}{5})} d\lambda_1(x)
\]
Користуючись тим фактом, що $\int_{[\alpha,\beta)} d\lambda_1(x) = \beta - \alpha$, отримуємо:
\[
\mathcal{I} = \frac{1}{5} \cdot \sum_{k=0}^{499} \frac{1}{(k+2)(k+4)}
\]
Далі розкладаємо $\frac{1}{(k+2)(k+4)}$ на прості дроби:
\[
\frac{1}{(k+2)(k+4)} = \frac{\alpha}{k+2} + \frac{\beta}{k+4} \implies (\alpha+\beta)k + (4\alpha+2\beta) \equiv 1
\]
Звідси $\beta=-\alpha$, тоді $\alpha=\frac{1}{2}$, а отже $\beta=-\frac{1}{2}$. Остаточно:
\[
\mathcal{I} = \frac{1}{10} \cdot \sum_{k=0}^{499} \left(\frac{1}{k+2} - \frac{1}{k+4}\right)
\]
Якщо розписати суму
\begin{gather*}
\sum_{k=0}^{499}\left(\frac{1}{k+2} - \frac{1}{k+4}\right) = \sum_{k=0}^{499} \frac{1}{k+2} - \sum_{k=0}^{499} \frac{1}{k+4} = \sum_{k=2}^{501} \frac{1}{k} - \sum_{k=4}^{503} \frac{1}{k} \\
= \frac{1}{2} + \frac{1}{3} - \frac{1}{502} - \frac{1}{503} = \frac{314125}{378759}
\end{gather*}
Таким чином:
\[
\boxed{\mathcal{I} = \frac{62825}{757518}}
\]

\textbf{Відповідь.} $\frac{62825}{757518} \approx 0.083$.

\end{document}

