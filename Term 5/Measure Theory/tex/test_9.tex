\documentclass[14pt]{extarticle}
\usepackage[english,ukrainian]{babel}
\usepackage[utf8]{inputenc}
\usepackage{amsmath,amssymb}
\usepackage{parskip}
\usepackage{graphicx}
\usepackage{tcolorbox}
\tcbuselibrary{skins}
\usepackage[framemethod=tikz]{mdframed}
\usepackage{chngcntr}
\usepackage{enumitem}
\usepackage{hyperref}
\usepackage{float}
\usepackage{subfig}
\usepackage{esint}
\usepackage[top=2.5cm, left=3cm, right=3cm, bottom=4.0cm]{geometry}
\usepackage[table]{xcolor}
\usepackage{algorithm}
\usepackage{algpseudocode}
\usepackage{listings}
\usepackage{dsfont}

\title{Самостійна робота з курсу ``Теорія міри''}
\author{Студента 3 курсу групи МП-31 Захарова Дмитра}
\date{\today}

\begin{document}

\maketitle

\section*{Завдання}
\textbf{Умова.} Довести, що функція $f: \mathbb{R}^2 \to \mathbb{R}$ є борельовою:
\[
f(x,y) = \frac{\text{sign}\left( \sin (xy)\right)}{1+y^2}, \; (x,y) \in \mathbb{R}^2
\]

\textbf{Розв'язок.} Помітимо, що $f = \frac{g}{h}$ де
\[
g(x,y) = \text{sign}\left(\sin(xy)\right), \; h(y) = 1+y^2
\]

Доведемо, що як $g$, так і $h$ є борельовою, звідки буде випливати те, що $f$ теж борельова. 

Отже, почнемо з $g$. Ми можемо записати $g$ як композицію $g_1 \circ g_2 \circ g_3$, де
\[
g_1(x) = \text{sign}(x), \; g_2(x) = \sin x, \; g_3(x,y) = xy
\]
Покажемо, що $g_i$ є борельовими. 

\begin{itemize}
    \item  $g_1$ є монотонною, тому вона є борельовою.
    \item $g_2$ є неперервною, тому теж є борельовою.
    \item $g_3$ також є неперервною на $\mathbb{R}^2$, тому є борельовою. 
\end{itemize}

Отже, користуючись відповідною теоремою, композиція борельових множин є також борельовою. 

Також легко бачити, що $h$ є борельовою, оскільки є неперервною. 

Згідно теоремі про властивості вимірних функцій, $\frac{g}{h}\mathds{1}_{\{(x,y) \in \mathbb{R}^2: h(y) \neq 0\}}$ є борельовою, проте оскільки $\{(x,y) \in \mathbb{R}^2: h(y)=1+y^2 \neq 0\} = \mathbb{R}^2$, то $\mathds{1}_{\{\dots\}} \equiv 1$ і тому $f=\frac{g}{h}$ теж є борельовою. 

\end{document}

