\documentclass[12pt]{extarticle}
\usepackage[english,ukrainian]{babel}
\usepackage[utf8]{inputenc}
\usepackage{amsmath,amssymb}
\usepackage{parskip}
\usepackage{graphicx}
\usepackage{tcolorbox}
\tcbuselibrary{skins}
\usepackage[framemethod=tikz]{mdframed}
\usepackage{chngcntr}
\usepackage{enumitem}
\usepackage{hyperref}
\usepackage{float}
\usepackage{subfig}
\usepackage{esint}
\usepackage[top=2.5cm, left=3cm, right=3cm, bottom=4.0cm]{geometry}
\usepackage[table]{xcolor}
\usepackage{algorithm}
\usepackage{algpseudocode}
\usepackage{listings}

\title{Самостійна робота з курсу ``Теорія міри''}
\author{Студента 3 курсу групи МП-31 Захарова Дмитра}
\date{\today}

\begin{document}

\maketitle

\section*{Завдання 1}

\textbf{Умова.} Знайти верхню і нижню границі послідовності $\{A_n\}_{n=0}^{\infty}$, якщо:
\[
A_n = \begin{cases}
    [1,n^3+3), & n = 2k, \\
    \left(1 - \frac{1}{n},\ln^2 n\right], & n = 2k+1
\end{cases}, \; n \in \mathbb{Z}^+
\]

\textit{Коментар.} Тут і далі позначатимемо $\mathbb{Z}^+ \triangleq \mathbb{N} \cup \{0\}$.

\textbf{Розв'язок.} Як було доведено на практиці,
\[
\underset{n \to \infty}{\overline{\lim}} A_n = \bigcap_{n=0}^{\infty}\bigcup_{k=n}^{\infty}A_k, \; \underset{n \to \infty}{\underline{\lim}}A_n = \bigcup_{n=0}^{\infty}\bigcap_{k=n}^{\infty}A_k
\]

Отже, акуратно застосуємо ці формули. Розіб'ємо розв'язок на дві частини: обрахунок верхньої та нижньої границі послідовності.

\textbf{Знаходження верхньої границі.} Для початку, знайдемо $\textcolor{red}{W_n} \triangleq \bigcup_{k=n}^{\infty}A_k$. 

Помітимо, що
\[
\textcolor{red}{W_n} \triangleq \bigcup_{k=n}^{\infty}A_k = \underbrace{\left(\bigcup_{k=0}^{\infty}A_{n+2k}\right)}_{\textcolor{orange}{U_n}} \cup \underbrace{\left(\bigcup_{k=0}^{\infty}A_{n+2k+1}\right)}_{\textcolor{blue}{V_n}}
\]

Таким чином, ми розбили одне ціле об'єднання на два окремих, для яких індекси мають однакову парність. Нехай для конкретності $n=2m$. В такому разі маємо \textcolor{orange}{парну} суму, позначену $\textcolor{orange}{U_{2m}}$ та \textcolor{blue}{непарну}, позначену $\textcolor{blue}{V_{2m}}$.

В такому разі:
\[
\textcolor{orange}{U_{2m}} \triangleq \bigcup_{k=0}^{\infty}A_{2m+2k} = \bigcup_{k=0}^{\infty} [1,8(m+k)^3+3)
\]

Маємо об'єднання відрізків виду $[1, x_k)$ де $x_k \triangleq 8(m+k)^3+3$. Ця послідовність необмежена зверху, бо $\lim_{k \to \infty}x_k = +\infty$. Окрім цього, монотонно зростає, починаючи з $x_0 = 8m^3+3$. Отже, якщо взяти об'єднання усіх таких відрізків, то отримаємо 
\[
\textcolor{orange}{U_{2m}} = [1, +\infty).
\]

Що стосується другого доданку,
\[
\textcolor{blue}{V_{2m}} \triangleq \bigcup_{k=0}^{\infty} A_{2m+2k+1} = \bigcup_{k=0}^{\infty} \left(1 - \frac{1}{2m+2k+1}, \ln^2\left(2m+2k+1\right)\right].
\]

Тут проаналізуємо акуратніше. Позначимо $\ell_k \triangleq 1 - \frac{1}{2m+2k+1}$ та $r_k \triangleq \ln^2(2m+2k+1)$. 

Ліва межа відрізка починається з $\ell_0 = 1-\frac{1}{2m+1}$ та монотонно збільшується, наближаючись до $1$ ($\lim_{k \to \infty}\ell_k = 1$). Права ж межа починається з $r_0 = \ln^2(2m+1)$ та монотонно необмежено збільшується ($\lim_{k \to \infty}r_k = +\infty$). 

Отже, об'єднання зліва буде починатися з $1-\frac{1}{2m+1}$ не включно, оскільки це ``найлішіва'' точка, а праворуч буде $+\infty$. Таким чином:
\[
\textcolor{blue}{V_{2m}} = \left(1-\frac{1}{2m+1}, +\infty\right)
\]

Якщо $n=2m+1$, то вирази не змінюються. Дійсно, в такому разі $\textcolor{orange}{U_{2m+1}}=\textcolor{blue}{V_{2m}}=\left(1-\frac{1}{2m+1},+\infty\right)$, а $\textcolor{blue}{V_{2m+1}} = \textcolor{orange}{U_{2m+2}} = [1,+\infty)$. 

Отже, узагальнити об'єднання можна таким чином:
\[
\textcolor{orange}{U_n} \cup \textcolor{blue}{V_n} = [1,+\infty) \cup \left(1 - \frac{1}{2\left\lfloor\frac{n}{2}\right\rfloor + 1},+\infty\right) = \left(1 - \frac{1}{2\left\lfloor\frac{n}{2}\right\rfloor + 1},+\infty\right)
\]
Таким чином, $\textcolor{red}{W_n} = \left(1 - \frac{1}{2\left\lfloor\frac{n}{2}\right\rfloor + 1},+\infty\right)$. Тому
\[
\underset{n \to \infty}{\overline{\lim}} A_n = \bigcap_{n=0}^{\infty} \left(1 - \frac{1}{2\left\lfloor\frac{n}{2}\right\rfloor + 1},+\infty\right) = \bigcap_{k=0}^{\infty} \left(1 - \frac{1}{2k+1}, +\infty\right)
\]

Для обрахунку, потрібно знайти $\sup_{k \in \mathbb{Z}^+} \left(1 - \frac{1}{2k+1}\right)$. Послідовність монотонно зростає від $0$ і наближається до $1$, тобто $\sup_{k \in \mathbb{Z}^+} \left(1 - \frac{1}{2k+1}\right) = 1$. Супремум варто включити, оскільки $1$ належить будь-якому відрізку виду $(1-\frac{1}{2k+1},+\infty)$, а отже і перетину усіх таких відрізків. Остаточно:
\[
\underset{n \to \infty}{\overline{\lim}} A_n = [1, +\infty)
\]

\textbf{Знаходження нижньої границі.} Спочатку знаходимо $\textcolor{red}{W_n} \triangleq \bigcap_{k=n}^{\infty}A_k$:
\[
\textcolor{red}{W_n} \triangleq\bigcap_{k=n}^{\infty}A_k = \underbrace{\left(\bigcap_{k=0}^{\infty}A_{n+2k}\right)}_{\textcolor{orange}{U_n}} \cap \underbrace{\left(\bigcap_{k=0}^{\infty}A_{n+2k+1}\right)}_{\textcolor{blue}{V_n}}
\]

Нехай знову для конкретики $n=2m$. Тоді:
\[
\textcolor{orange}{U_{2m}} = \bigcap_{k=0}^{\infty}A_{2m+2k} = \bigcap_{k=0}^{\infty}[1,8(m+k)^3+3) = [1,8m^3+3)
\]

Це випливає з того, що перший відрізок $[1,8m^3+3] \subset [1,8(m+k)^3+3]$ для будь-якого $k \in \mathbb{Z}^+$, тому перетин усіх відрізків дасть лише перший. 

Друга границя:
\[
\textcolor{blue}{V_{2m}} = \bigcap_{k=0}^{\infty}A_{2m+2k+1} = \bigcap_{k=0}^{\infty}\left(1 - \frac{1}{2m+2k+1}, \ln^2\left(2m+2k+1\right)\right]
\]

Тут знову проаналізуємо акуратніше. Нам, по суті, потрібно знайти супремум лівої границі та інфінум правої. Позначимо наші відрізки $\{(\ell_k, r_k]\}_{k \in \mathbb{Z}^+}$. Ліва границя відрізка, як ми казали раніше, починається з $1 - \frac{1}{2m+1}$ і монотонно зростає до $1$ не включно. Права границя необмеженно зростає від $\ln^2(2m+1)$ до $+\infty$. 

Отже, супремум лівої частини дорівнює $1$, а інфінум лівої $\ln^2(2m+1)$. Таким чином:
\[
\textcolor{blue}{V_{2m}} = [1, \ln^2(2m+1)]
\]
Якщо $n=2m+1$, то перший доданок $\textcolor{orange}{U_{2m+1}} = \textcolor{blue}{V_{2m}} = [1,\ln^2(2m+1)]$, а другий $\textcolor{blue}{V_{2m+1}} = \textcolor{orange}{U_{2m+2}} = [1,8(m+1)^3+3)$. 

Таким чином, можемо узагальнити перетин:
\[
\textcolor{orange}{U_n} \cap \textcolor{blue}{V_n} = \left[1,\ln^2\left(2\left\lfloor\frac{n}{2}\right\rfloor+1\right)\right] \cap \left[1,8\left\lfloor\frac{n+1}{2}\right\rfloor^3+3\right)
\]

Проте, $\forall n \in \mathbb{Z}^+: 8\left\lfloor\frac{n+1}{2}\right\rfloor^3+3 > \ln^2\left(2\left\lfloor\frac{n}{2}\right\rfloor+1\right)$, тому
\[
\textcolor{red}{W_n} = \textcolor{orange}{U_n} \cap \textcolor{blue}{V_n} = \left[1,\ln^2\left(2\left[\frac{n}{2}\right]+1\right)\right]
\]

Отже,
\[
\underset{n \to \infty}{\underline{\lim}}A_n = \bigcup_{n=0}^{\infty}\textcolor{red}{W_n} = \bigcup_{n=0}^{\infty}\left[1,\ln^2\left(2\left[\frac{n}{2}\right]+1\right)\right] = \bigcup_{k=0}^{\infty} [1,\ln^2(2k+1)]
\]

Оскільки права границя відрізків прямує на $+\infty$, аналогічно першій границі, отримуємо $\underset{n \to \infty}{\underline{\lim}}A_n = [1,+\infty)$.

\textbf{Відповідь.} $\underset{n \to \infty}{\overline{\lim}} = \underset{n \to \infty}{\underline{\lim}}A_n=[1,+\infty)$.

\end{document}

