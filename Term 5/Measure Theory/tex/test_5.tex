\documentclass[14pt]{extarticle}
\usepackage[english,ukrainian]{babel}
\usepackage[utf8]{inputenc}
\usepackage{amsmath,amssymb}
\usepackage{parskip}
\usepackage{graphicx}
\usepackage{tcolorbox}
\tcbuselibrary{skins}
\usepackage[framemethod=tikz]{mdframed}
\usepackage{chngcntr}
\usepackage{enumitem}
\usepackage{hyperref}
\usepackage{float}
\usepackage{subfig}
\usepackage{esint}
\usepackage[top=2.5cm, left=3cm, right=3cm, bottom=4.0cm]{geometry}
\usepackage[table]{xcolor}
\usepackage{algorithm}
\usepackage{algpseudocode}
\usepackage{listings}

\title{Самостійна робота з курсу ``Теорія міри''}
\author{Студента 3 курсу групи МП-31 Захарова Дмитра}
\date{\today}

\begin{document}

\maketitle

\section*{Завдання}
\textbf{Умова.} Нехай
\[
X = \{A,B,C\}, \; \mathcal{P} = \{\emptyset, \{A\}, \{B\}\},
\]
\[
\lambda: \mathcal{P} \to [0,+\infty), \; \lambda(\emptyset) = 0, \; \lambda(\{A\}) = 1, \; \lambda(\{B\}) = 1.
\]
\begin{enumerate}
    \item Перевірити, що $\lambda$ є мірою на півкільці $\mathcal{P}$.
    \item Побудувати зовнішню міру $\lambda^*$ на $2^X$.
    \item Знайти клас $\mathcal{S}$ вимірних за Каратеодорі відносно міри $\lambda^*$ множин.
\end{enumerate}

\textbf{Розв'язок.} 

\textit{Пункт 1.} Згідно означенню \textit{міри}, функція $\lambda$ має бути невід'ємною $\sigma$-адитивною функцією множин, заданих на півкільці.

Зрозуміло, що $\mathcal{P}$ є півкільцем. $\lambda$ також є невід'ємною, оскільки за умовою $\lambda(\{A\}) = \lambda(\{B\}) = 1 > 0$ і $\lambda(\emptyset) = 0$. 

Залишилося перевірити $\sigma$-адитивність. За означенням $\lambda$ є $\sigma$-адитивною, якщо
\begin{gather*}
\forall \{A_n\}_{n \in \mathbb{N}} \subset \mathcal{P}: \left\{\{A_n\}_{n \in \mathbb{N}} \; \text{неперетинні} \wedge \bigcup_{n \in \mathbb{N}}A_n \in \mathcal{P}\right\} \\ \implies \lambda\left(\bigcup_{n \in \mathbb{N}}A_n\right) = \sum_{n \in \mathbb{N}}\lambda(A_n)
\end{gather*}
Оскільки $\bigcup_{n \in \mathbb{N}}A_n \in \mathcal{P}$ і всі елементи є неперетинними, то $\{A_n\}_{n \in \mathbb{N}}$ складається з одного або жодного $\{A\}$, $\{B\}$ і нескінченної кількості $\emptyset$, тобто
\begin{gather}
\{A_n\}_{n \in \mathbb{N}} = \{\{A\},\emptyset,\dots\} \; \text{або} \\
\{A_n\}_{n \in \mathbb{N}} = \{\{B\},\emptyset,\dots\} \; \text{або} \\
A_n \equiv \emptyset
\end{gather}
Випадок (3) тривільний, тому розглянемо випадок (1), що є аналогічним (2). В такому разі, $\bigcup_{n \in \mathbb{N}}A_n = \{A\}$, тому $\lambda(\bigcup_{n \in \mathbb{N}}A_n)=\lambda(\{A\})$. З іншого боку, $\sum_{n \in \mathbb{N}}\lambda(A_n) = \lambda(\{A\}) + \lambda(\emptyset) + \lambda(\emptyset) + \dots = \lambda(\{A\})$. Отже, рівність виконується.

\textit{Пункт 2.} Побудуємо зовнішню міру за допомогою продовження $\lambda$ на $2^X$. Маємо для $\forall H \in 2^X$:
\[
\lambda^*(H) = \begin{cases}
    \inf\left\{\sum_{n \in \mathbb{N}}\lambda(A_n): H \subset \bigcup_{n \in \mathbb{N}}A_n \wedge \{A_n\}_{n \in \mathbb{N}} \subset \mathcal{P}\right\}, \\ \;\;\;\;\;\;\;\;\text{якщо} \; \exists \{A_n\}_{n \in \mathbb{N}} \subset \mathcal{P}: H \subset \bigcup_{n \in\mathbb{N}}A_n \\ +\infty, \; \text{в іншому випадку}
\end{cases}
\]

Отже, почнемо розглядати усі елементи з $2^X$. По-перше, $\lambda^*(\emptyset) = 0$ з означення зовнішньої міри. 

Далі $\lambda(\{A\}) = \lambda(\{B\}) = 1$. Що стосується $\lambda(\{C\})$, то тут ми не можемо знайти $\{A_n\}_{n \in \mathbb{N}} \subset \mathcal{P}$ так, щоб $\{C\} \subset \bigcup_{n \in \mathbb{N}}A_n$, оскільки $\mathcal{P}$ не містить $\{C\}$ або інші елементи, що мають в собі $\{C\}$. Тому $\lambda(\{C\})=+\infty$. 

Аналогічно для усіх множин з $2^X$, що містять $C$. Тому залишилося розглянути $\lambda^*(\{A,B\})$. Беремо мінімально $\{A_n\}_{n \in \mathbb{N}} := \{\{A\},\{B\},\emptyset,\dots\}$, тоді $\sum_{n \in \mathbb{N}}\lambda(A_n) = \lambda(\{A\}) + \lambda(\{B\}) = 2$. 

Отже,
\[
\forall H \in 2^X: \lambda^*(H) = \begin{cases}
    |H|, & C \not \in H \\
    +\infty, & C \in H
\end{cases}
\]

\textit{Пункт 3.} Скориставшись означенням вимірності за Каратеодорі, нам потрібно знайти
\[
\mathcal{S} = \{H \subset X: \forall E \subset X \; \; \lambda^*(E) = \lambda^*(E \cap H) + \lambda^*(E \cap \overline{H})\}
\]
По-перше, $\emptyset \in \mathcal{S}$, оскільки
\[
\lambda^*(E) = \lambda^*(E \cap \emptyset) + \lambda^*(E \cap X) = \lambda^*(\emptyset) + \lambda^*(E) = \lambda^*(E)
\]
Розглянемо $H \subset X: C \not\in H$. Тоді
\[
\forall E \subset X: \lambda^*(E) = \lambda^*(E \cap H) + \lambda^*(E \cap \overline{H}) \iff \forall E: |E| = |E \cap H| + |E \cap \overline{H}|
\]
Беремо $H = \{A\}$. Якщо перебрати усі варіанти, то властивість вище виконується. Так само для $H=\{B\}$ і $H=\{A,B\}$. 

Тепер беремо ті множини $H \subset X: C \in H$. Можна переконатись, що тут результат буде такий самий. Отже, $\mathcal{S} = 2^X$.

\textbf{Відповідь.} 1. Див. розв'язок. 2. $\lambda^*(H) = \begin{cases}
    |H|, & C \not\in H \\
    +\infty, & C \in H
\end{cases}$. 3. $\mathcal{S} = 2^X$.

\end{document}

