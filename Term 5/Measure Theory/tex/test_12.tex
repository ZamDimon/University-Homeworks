\documentclass[14pt]{extarticle}
\usepackage[english,ukrainian]{babel}
\usepackage[utf8]{inputenc}
\usepackage{amsmath,amssymb}
\usepackage{parskip}
\usepackage{graphicx}
\usepackage{tcolorbox}
\tcbuselibrary{skins}
\usepackage[framemethod=tikz]{mdframed}
\usepackage{chngcntr}
\usepackage{enumitem}
\usepackage{hyperref}
\usepackage{float}
\usepackage{subfig}
\usepackage{esint}
\usepackage[top=2.5cm, left=3cm, right=3cm, bottom=4.0cm]{geometry}
\usepackage[table]{xcolor}
\usepackage{algorithm}
\usepackage{algpseudocode}
\usepackage{listings}
\usepackage{dsfont}

\title{Самостійна робота з курсу ``Теорія міри''}
\author{Студента 3 курсу групи МП-31 Захарова Дмитра}
\date{\today}

\begin{document}

\maketitle

\section*{Завдання}
\textbf{Умова.} Обчислити інтеграл
\[
\mathcal{I} = \int_{\mathbb{R}^+}8^{-[3x]}d\lambda_1(x)
\]

\textbf{Розв'язок.} Помітимо, що
\[
\mathbb{R}^+ = \bigcup_{n \in \mathbb{N}}\left[\frac{n-1}{3},\frac{n}{3}\right),
\]
і позначимо $A_n=\left[\frac{n-1}{3},\frac{n}{3}\right)$. Елементи послідовності $\{A_n\}_{n \in \mathbb{N}}$ є попарно неперетинними, а тому, використовуючи $\sigma$-адитивність інтеграла Лебега, маємо
\[
\mathcal{I} = \sum_{n \in \mathbb{N}}\int_{A_n}8^{-[3x]}d\lambda_1(x)
\]
Розглянемо тепер значення підінтегральної функції на множині $A_n$. Значення $3x$ в такому разі лежать між $[n-1,n)$, а тому $[3x]=n-1$. 

Тому маємо:
\[
\mathcal{I} = \sum_{n \in \mathbb{N}} \int_{\left[\frac{n-1}{3},\frac{n}{3}\right)}8^{-(n-1)}d\lambda_1(x) = \sum_{n \in \mathbb{N}} 8^{1-n} \int_{\left[\frac{n-1}{3},\frac{n}{3}\right)}d\lambda_1(x)
\]
Далі скористаємося тим фактом, що $\int_{[\alpha,\beta)}d\lambda_1(x) = \beta-\alpha$. Звідси
\[
\mathcal{I} = \sum_{n \in \mathbb{N}} 8^{1-n} \cdot \frac{n-(n-1)}{3} = \frac{1}{3}\sum_{n=0}^{\infty}8^{-n}
\]
Отже, маємо суму геометричної прогресії з першим членом $1$ і знаменником $\frac{1}{8}$, тому
\[
\mathcal{I} = \frac{1}{3} \cdot \frac{1}{1-\frac{1}{8}} = \frac{1}{3} \cdot \frac{8}{7} = \frac{8}{21}
\]

\textbf{Відповідь.} $\mathcal{I} = \frac{8}{21}$.

\end{document}

