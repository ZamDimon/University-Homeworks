\documentclass[14pt]{extarticle}
\usepackage[english,ukrainian]{babel}
\usepackage[utf8]{inputenc}
\usepackage{amsmath,amssymb}
\usepackage{parskip}
\usepackage{graphicx}
\usepackage{tcolorbox}
\tcbuselibrary{skins}
\usepackage[framemethod=tikz]{mdframed}
\usepackage{chngcntr}
\usepackage{enumitem}
\usepackage{hyperref}
\usepackage{float}
\usepackage{subfig}
\usepackage{esint}
\usepackage[top=2.5cm, left=3cm, right=3cm, bottom=4.0cm]{geometry}
\usepackage[table]{xcolor}
\usepackage{algorithm}
\usepackage{algpseudocode}
\usepackage{listings}

\title{Самостійна робота з курсу ``Теорія міри''}
\author{Студента 3 курсу групи МП-31 Захарова Дмитра}
\date{\today}

\begin{document}

\maketitle

\textbf{Умова.} Нехай $\mathcal{A}$ є алгеброю пiдмножин $X$, $\varphi: \mathcal{A} \to [0,+\infty)$ є адитивною. Довести формулу
\[
\varphi(\overline{A}\Delta \overline{B}) = \varphi(A) + \varphi(B) - 2\varphi(A \cap B)
\]
Формули для $\varphi(A \cup B)$, $\varphi(A \setminus B)$ i $\varphi(A \Delta B)$ можна використовувати без доведення.

\textbf{Розв'язок.} Спочатку скористуємось тим фактом, що $\varphi(A \Delta B) = \varphi(A)+\varphi(B) - 2\varphi(A \cap B)$, тобто
\[
\varphi(\overline{A} \Delta \overline{B}) = \varphi(\overline{A}) + \varphi(\overline{B}) - 2\varphi(\overline{A} \cap \overline{B})
\]
Оскільки $\overline{A} = X \setminus A$, то $\varphi(\overline{A}) = \varphi(X) - \varphi(A)$. Аналогічно, $\varphi(\overline{B}) = \varphi(X) - \varphi(B)$. 

Нарешті, $\overline{A} \cap \overline{B} = \overline{A \cup B}$, тому
\[
\varphi(\overline{A} \cap \overline{B}) = \varphi(X) - \varphi(A \cup B) = \varphi(X) - \varphi(A) - \varphi(B) + \varphi(A \cap B)
\]
Таким чином,
\begin{gather*}
\varphi(\overline{A} \Delta \overline{B}) = \varphi(X) - \varphi(A) + \varphi(X) - \varphi(B) - 2(\varphi(X) - \varphi(A) - \varphi(B) + \varphi(A \cap B)) \\
= \varphi(A) + \varphi(B) - 2\varphi(A \cap B),
\end{gather*}
що і потрібно було довести.

\end{document}

