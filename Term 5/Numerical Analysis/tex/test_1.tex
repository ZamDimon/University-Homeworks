\documentclass[12pt]{extarticle}
\usepackage[english,ukrainian]{babel}
\usepackage[utf8]{inputenc}
\usepackage{amsmath,amssymb}
\usepackage{parskip}
\usepackage{graphicx}
\usepackage{xcolor}
\usepackage{tcolorbox}
\tcbuselibrary{skins}
\usepackage[framemethod=tikz]{mdframed}
\usepackage{chngcntr}
\usepackage{enumitem}
\usepackage{hyperref}
\usepackage{float}
\usepackage{subfig}
\usepackage{esint}
\usepackage[top=2.5cm, left=3cm, right=3cm, bottom=4.0cm]{geometry}
\usepackage[table]{xcolor}
\usepackage{algorithm}
\usepackage{algpseudocode}

\title{Самостійна робота з курсу ``Чисельний аналіз''}
\author{Студента 3 курсу групи МП-31 Захарова Дмитра}
\date{\today}

\begin{document}

\maketitle

\textbf{Умова.} 
\begin{enumerate}
    \item Побудувати таблицю кінцевих різниць і скласти інтерполяційний поліном Ньютона назад на вузлах $\{(-3,5),((-2.5,1),(0,6),(1,7)\}$
    \item Визначити значення полінома Ньютона для $\widetilde{x}_1 = -1, \widetilde{x}_2 = 0.5$.
\end{enumerate}

\textbf{Розв'язок.} Поліном Ньютона назад має вигляд:
\[
N_n^-(x) = y_n + \sum_{k=1}^n y_{n,\dots,n-k}\prod_{i=0}^{k-1}(x-x_{n-i}), \; y_{i,\dots,j} \triangleq \frac{y_{i+1,\dots,j}-y_{i,\dots,j-1}}{x_j-x_i}
\]

В нашому випадку:
\[
N_n^{-}(x) = y_3 + y_{3,2}(x-x_3) + y_{3,2,1}(x-x_3)(x-x_2) + y_{3,2,1,0}(x-x_3)(x-x_2)(x-x_1) 
\]

З умови маємо $y_3=7$, а роздільні різниці мають значення: 
\[
y_{3,2} = \frac{y_2-y_3}{x_2-x_3} = \frac{6-7}{0-1} = 1
\]
\[
y_{3,2,1} = \frac{y_{2,1} - y_{3,2}}{x_1 - x_3} = \frac{\frac{y_1-y_2}{x_1-x_2}-y_{3,2}}{x_1-x_3} = \frac{\frac{1-6}{-2.5-0}-1}{-2.5-1} = \frac{2-1}{-3.5} = \frac{1}{-7/2} = -\frac{2}{7}
\]

Отже, нам залишилося лише знайти $y_{3,2,1,0}$:
\[
y_{3,2,1,0} = \frac{y_{2,1,0}-y_{3,2,1}}{-3-1}
\]

$y_{3,2,1}$ ми вже знаємо: він дорівнює $-\frac{2}{7}$. Знаходимо $y_{2,1,0}$:
\[
y_{2,1,0} = \frac{y_{1,0}-y_{2,1}}{x_0-x_2} = \frac{\frac{5-1}{-3+2.5} - 2}{-3-0} = \frac{-8-2}{-3} = \frac{10}{3}
\]

Отже:
\[
y_{3,2,1,0} = \frac{\frac{10}{3}+\frac{2}{7}}{-4} = \frac{76}{-4 \cdot 21} = -\frac{19}{21}
\]

Отже остаточно поліном має вигляд:
\[
N_n^{-}(x) = 7 + (x-1) - \frac{2}{7}(x-1)x - \frac{19}{21}(x-1)x(x+2.5)
\]

Або:
\[
N_n^{-}(x) = 6+x - \frac{2x}{7}(x-1) - \frac{19x}{21}(x-1)(x+2.5)
\]

Обрахуємо значення у точках:
\[
N_n^{-}(-1) = \frac{12}{7}, \; N_n^-(0.5) = \frac{29}{4}
\]

\textbf{Відповідь.} 

\textit{Пункт 1.} $N_n^-(x) = 6+x - \frac{2x}{7}(x-1) - \frac{19x}{21}(x-1)(x+2.5)$.

\textit{Пункт 2.} $N_n^{-}(\widetilde{x}_1=-1)=\frac{12}{7}, \; N_n^-(\widetilde{x}_2=0.5)=\frac{29}{4}$.


\end{document}

