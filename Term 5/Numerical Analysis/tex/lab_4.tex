\documentclass[12pt]{extarticle}
\usepackage[english,ukrainian]{babel}
\usepackage[utf8]{inputenc}
\usepackage{amsmath,amssymb}
\usepackage{parskip}
\usepackage{graphicx}
\usepackage{xcolor}
\usepackage{tcolorbox}
\tcbuselibrary{skins}
\usepackage[framemethod=tikz]{mdframed}
\usepackage{chngcntr}
\usepackage{enumitem}
\usepackage{hyperref}
\usepackage{float}
\usepackage{subfig}
\usepackage{chngcntr}
\usepackage{esint}

\usepackage[top=2.5cm, left=3cm, right=3cm, bottom=4.0cm]{geometry}
\usepackage[table]{xcolor}

\usepackage{algorithm}
\usepackage{algpseudocode}
\usepackage{listings}
\usepackage{xcolor}
\usepackage{dsfont}

\definecolor{codegreen}{rgb}{0,0.6,0}
\definecolor{codegray}{rgb}{0.5,0.5,0.5}
\definecolor{codepurple}{rgb}{0.58,0,0.82}
\definecolor{backcolour}{rgb}{0.95,0.95,0.92}

\lstdefinestyle{mystyle}{
    backgroundcolor=\color{backcolour},   
    commentstyle=\color{codegreen},
    keywordstyle=\color{magenta},
    numberstyle=\tiny\color{codegray},
    stringstyle=\color{codepurple},
    basicstyle=\ttfamily\footnotesize,
    breakatwhitespace=false,         
    breaklines=true,                 
    captionpos=b,                    
    keepspaces=true,                 
    numbers=left,                    
    numbersep=5pt,                  
    showspaces=false,                
    showstringspaces=false,
    showtabs=false,                  
    tabsize=2
}
\lstset{style=mystyle}

\usepackage{ragged2e}
\begin{document}

\begin{titlepage}
	\centering
	\includegraphics[width=0.15\textwidth]{images/lab_1/logo.png}\par\vspace{0.3cm}
	{\textbf{Міністерство освіти і науки України}\par
 Харківський національний університет імені В.Н. Каразіна\par}
    \vspace{1cm}
	{\Large \textsc{Лабораторна робота \#4}\par
    \textbf{Обчислення інтегралів за складеними квадратурними формулами трапецій і парабол}\par}
	\vfill
 \begin{FlushRight}
	\textbf{Виконав:}\par Захаров Дмитро Олегович \par Група МП-31
\end{FlushRight}
	\vfill

% Bottom of the page
	{\large Харків -- 2023\par}
\end{titlepage}

\tableofcontents
\pagebreak

\section{Постановка задачі}

Обчислити заданий інтеграл з точністю $\varepsilon=10^{-6}$ за квадратурними формулами трапецій та парабол, використовуючи двійний перерахунок і оцінку за Рунге. 

На друк вивести наближене значення інтегралу.

\begin{center}
\textbf{Варіант 5.} 
\end{center}

\[
\int_{[0,1]} \frac{1 - \cos x}{x}dx
\]

\pagebreak
\section{Опис методу}

\subsection{Складова квадратурна формула трапеції}\label{section:quadratic_formula}

Нехай треба знайти значення інтегралу
\[
\mathcal{I} = \int_{[\alpha,\beta]}f(x)dx
\]
Інтеграл обчислюємо за допомогою наступного алгоритму:
\begin{algorithm}
\caption{Складова квадратурна формула трапеції, спосіб \#1}\label{alg:cap}
\begin{algorithmic}
\State $h \gets \frac{\beta-\alpha}{2}$
\State $\mathcal{I}_h \gets \frac{f(\alpha)+f(\beta)}{2}\cdot h$
\State $\rho \gets \infty, k \gets 1$
\For{$\left|\rho\right| > \varepsilon$}
    \State $n \gets 2^k$
    \State $h_{2n} \gets \frac{h}{2n}$
    \State $\mathcal{I}_{2n} \gets \frac{1}{2}\mathcal{I}_h + h_{2n}\sum_{i=1}^n f\left(\alpha+(2i-1)h_{2n}\right)$
    \State $k \gets k+1$
    \State $\rho \gets \frac{\mathcal{I}_h-\mathcal{I}_{2n}}{3}$
\EndFor

\Return $\mathcal{I}_{2n} + \rho$
\end{algorithmic}
\end{algorithm}

Також, як альтернатива, ми використаємо наступний алгоритм:

\begin{algorithm}
\caption{Складова квадратурна формула трапеції, спосіб \#2}\label{alg:cap}
\begin{algorithmic}
\State $h \gets \beta - \alpha,\; k \gets 1,\; \delta \gets +\infty,\; \mathcal{I} \gets 0$
\For{$|\delta| > \varepsilon$}
    \State $n \gets 2^k$ 
    \Comment Розбиваємо відрізок на $2^k$ частин
    \State $h_n \gets \frac{h}{n}$
    \Comment Знаходимо розмір інтервалів
    \State $\{x_j\}_{j=0}^n \gets \{\alpha + jh_n\}_{j}$
    \Comment Задаємо вузли
    \State $\mathcal{I}_n \gets \frac{h_n}{2}\sum_{j=1}^n \left\{f(x_{k-1})+f(x_k)\right\}$
    \Comment Значення інтегралу для $2^k$ розбиттів
    \State $\delta = \mathcal{I}_n - \mathcal{I}$
    \Comment Знаходимо різницю між сусідніми апроксимаціями
    \State $\mathcal{I} \gets \mathcal{I}_n$
\EndFor

\Return $\mathcal{I}$
\end{algorithmic}
\end{algorithm}

\subsection{Складова квадратурна формула парабол}\label{section:simpson_formula}
Тут ми застосовуємо наступний алгоритм:
\begin{algorithm}[H]
\caption{Складова квадратурна формула парабол, спосіб \#1}\label{alg:cap}
\begin{algorithmic}
\State $h \gets \frac{\beta-\alpha}{2}, s_0 \gets f(\alpha)+f(\beta), s_1 \gets f(\alpha+h), s_2 \gets 0$
\State $\mathcal{I}_h \gets (s_0+4s_1+2s_2)\frac{h}{3}$
\State $\rho \gets \infty, k \gets 1$
\For{$\left|\rho\right| > \varepsilon$}
    \State $n \gets 2^k$
    \State $h_{2n} \gets \frac{h}{2n}$
    \State $\mathcal{I}_{2n} \gets \frac{h_{2n}}{3}\left(s_0+4\sum_{i=1}^n f\left(\alpha+(2i-1)h_{2n}\right) + 2\sum_{i=1}^{n-1}f\left(\alpha+2ih_{2n}\right)\right)$
    \State $k \gets k+1$
    \State $\rho \gets \frac{\mathcal{I}_h-\mathcal{I}_{2n}}{15}$
\EndFor

\Return $\mathcal{I}_{2n} + \rho$
\end{algorithmic}
\end{algorithm}

Також, як альтернатива, ми використаємо наступний алгоритм:

\begin{algorithm}[H]
\caption{Складова квадратурна формула парабол, спосіб \#2}\label{alg:cap}
\begin{algorithmic}
\State $h \gets \beta - \alpha,\; k \gets 1,\; \delta \gets +\infty,\; \mathcal{I} \gets 0$
\For{$|\delta| > \varepsilon$}
    \State $n \gets 2^k$ 
    \Comment Розбиваємо відрізок на $2^k$ частин
    \State $h_n \gets \frac{h}{n}$
    \Comment Знаходимо розмір інтервалів
    \State $\{x_j\}_{j=0}^n \gets \{\alpha + jh_n\}_{j}$
    \Comment Задаємо вузли
    \State $\mathcal{I}_n \gets \frac{h_n}{3}\left[f(x_0)+4\sum_{j=1}^{n/2}f(x_{2j-1}) + 2\sum_{j=1}^{n/2-1}f(x_{2j})+f(x_n)\right]$
    \State $\delta = \mathcal{I}_n - \mathcal{I}$
    \Comment Знаходимо різницю між сусідніми апроксимаціями
    \State $\mathcal{I} \gets \mathcal{I}_n$
\EndFor

\Return $\mathcal{I}$
\end{algorithmic}
\end{algorithm}

\pagebreak
\section{Текст програми}

Повний текст програми можна знайти за \href{https://github.com/ZamDimon/University-Homeworks/tree/main/Term%205/Numerical%20Analysis/code/lab_4}{цим посиланням} ($\leftarrow$ напис клікабельний) на \textit{GitHub} сторінку.


\subsection{Складова квадратурна формула трапеції}

Створимо файл \texttt{integrators.py} та реалізуємо алгоритм з розділу \ref{section:quadratic_formula} у класі \texttt{TrapezoidalIntegrator}. При ініціалізації ми будемо вказувати функцію, що будемо інтегрувати, а також додамо метод \texttt{evaluate}, що приймає інтервал для інтегрування, а також бажану точність $\varepsilon$:

\begin{lstlisting}[language=Python, caption=Реалізація складової квадратурної формули трапеції]
from typing import Callable, Tuple, TypeAlias
from abc import ABC, abstractmethod

# Define type alias for an interval type
Interval: TypeAlias = Tuple[float, float]

class Integrator(ABC):
    """
    Abstract class for classes that implement function integration
    """
    
    def __init__(self, fn: Callable[[float], float]) -> None:
        """ Initialize the integrator with a function to integrate

        Args:
            fn (Callable[[float], float]): Function to integrate
        """
        self._fn = fn
        pass
    
    @abstractmethod
    def evaluate(self, interval: Interval, accuracy: float = 1e-6) -> float:
        """
        Evaluate the integral of the function over the interval
        
        Args:
            interval (Interval): Interval to integrate over
            accuracy (float, optional): Desired accuracy of the integral. Defaults to `1e-6`.
        """
        pass
    
class TrapezoidalIntegrator(Integrator):
    """
    Class for trapezoidal integration
    """
    
    # Maximum number of iterations
    MAX_ITERATIONS: int = 20
    
    def __init__(self, fn: Callable[[float], float]) -> None:
        super().__init__(fn)
    
    def evaluate(self, interval: Interval, accuracy: float = 1e-6) -> float:
        MAX_VALUE: float = 1e12
        interval_size: float = interval[1] - interval[0]
        estimate: float = MAX_VALUE 
        error: float = MAX_VALUE
        k: int = 1 # Number of iterations
        
        while abs(error) > accuracy:
            intervals_number: int = 2**k
            step_size: float = interval_size / intervals_number
            k = k + 1
            
            nodes = [interval[0] + i * step_size for i in range(0, intervals_number+1)]
            current_esimate = (step_size / 2) * sum([(self._fn(nodes[i]) + self._fn(nodes[i+1])) for i in range(0, intervals_number)])
            
            error = current_esimate - estimate
            estimate = current_esimate
            if k > TrapezoidalIntegrator.MAX_ITERATIONS:
                raise RuntimeError("Maximum number of iterations reached")
            
        return estimate
\end{lstlisting}

\subsection{Складова квадратурна формула парабол}

Ідея така сама, як в попередньому пункті.
\begin{lstlisting}[language=Python, caption=Реалізація складової квадратурної формули парабол]
class SimpsonIntegrator(Integrator):
    """
    Class for Simpson integration
    """
    
    # Maximum number of iterations
    MAX_ITERATIONS: int = 20
    
    def __init__(self, fn: Callable[[float], float]) -> None:
        super().__init__(fn)
    
    def evaluate(self, interval: Interval, accuracy: float = 1e-6) -> float:
        MAX_VALUE: float = 1e12
        interval_size: float = interval[1] - interval[0]
        estimate: float = MAX_VALUE 
        error: float = MAX_VALUE
        k: int = 1 # Number of iterations
        
        while abs(error) > accuracy:
            intervals_number: int = 2**k
            step_size: float = interval_size / intervals_number
            k = k + 1
            
            nodes = [interval[0] + i * step_size for i in range(0, intervals_number+1)]
            current_esimate = (step_size / 3) * (
                self._fn(nodes[0]) + 
                4*sum([self._fn(nodes[i]) for i in range(1, intervals_number, 2)]) +
                2*sum([self._fn(nodes[i]) for i in range(2, intervals_number, 2)]) +
                self._fn(nodes[-1])
            )
            
            error = current_esimate - estimate
            estimate = current_esimate
            if k > SimpsonIntegrator.MAX_ITERATIONS:
                raise RuntimeError("Maximum number of iterations reached")
            
        return estimate
\end{lstlisting}

\subsection{Програма запуску}

Програма для запуску зовсім проста: створюємо \texttt{TrapezoidalIntegrator} та \texttt{SimpsonIntegrator}, а потім проганяємо нашу функцію $f(x) = \frac{1-\cos x}{x}$ через неї:

\begin{lstlisting}[language=Python, caption=Перевірка результатів]
from typing import Callable
from rich import print
from math import cos
from integrators import Interval, TrapezoidalIntegrator, SimpsonIntegrator

if __name__ == '__main__':
    fn: Callable[[float], float] = lambda x: (1 - cos(x)) / x
    # We define the left endpoint of the interval to be EPSILON instead of 0 to avoid division by zero
    EPSILON: float = 1e-12
    interval: Interval = (EPSILON, 1.0) 
    
    trapezoidal_integrator: TrapezoidalIntegrator = TrapezoidalIntegrator(fn)
    trapezoidal_integral_estimate = trapezoidal_integrator.evaluate(interval, accuracy=1e-6)
    
    simpson_integrator: SimpsonIntegrator = SimpsonIntegrator(fn)
    simpson_integrator_estimate = simpson_integrator.evaluate(interval, accuracy=1e-6)
    
    print(f"Trapezoidal integral estimate: {trapezoidal_integral_estimate}")
    print(f"Simpson integral estimate: {simpson_integrator_estimate}")
\end{lstlisting}

\pagebreak

\section{Результати}

Після запуску програми, отримуємо два майже однакових значення, що приблизно дорівнюють:
\[
\int_{[0,1]} \frac{1-\cos x}{x}dx \approx 0.239812
\]
Різниця у значеннях починається з $7$ знака після коми. Більш конкретні значення наводимо нижче:
\begin{center}
\begin{tabular}{ |c|c| } 
 \hline
 \textbf{Формула трапеції} & \texttt{\textcolor{olive}{0.239811}\textcolor{gray}{59166750359857}} \\
 \hline
 \textbf{Формула парабол} & \texttt{\textcolor{olive}{0.239811}\textcolor{green}{74}\textcolor{gray}{863719770252}} \\
 \hline
 \textbf{Wolfram Mathematica} & \texttt{\textcolor{olive}{0.239811}\textcolor{green}{74}\textcolor{gray}{200056472594}} \\
 \hline
\end{tabular}
\end{center}

У \textit{Wolfram Mathematica} ми використали наступну команду для приблизного обчислення:
\begin{lstlisting}[language=wolfram, caption=Обрахунок інтегралу в \textit{Wolfram Mathematica}]
NIntegrate[(1 - Cos[x])/x, {x, 0, 1}, AccuracyGoal -> 30, WorkingPrecision -> 20]
\end{lstlisting}

\pagebreak
\section{Висновки}

В цій лабораторній роботі ми:
\begin{itemize}
\item навчилися чисельно інтегрувати функцію на заданому відрізку $\int_{[\alpha,\beta]}f(x)dx$.
\item навчилися писати комп'ютерну програму (на прикладі мови \texttt{Python}), що приймає на вхід функцію $f(x)$ та інтервал $[\alpha,\beta]$, і інтегрує її.
\item Порівняли отримані результати з математичним пакетом \textit{Wolfram Mathematica}.
\end{itemize}

При заданій точності, інтеграли збігаються дуже швидко ($1-2$ кроки), оскільки ми проходимось по ступеням двійки. При цьому, точність досяглася більше, використовуючи формулу парабол, якщо порівнювати з результатом в пакеті \textit{Wolfram Mathematica}. Отже, ми спробували два точних і відносно швидких методи для знаходження приблизних значень визначених інтегралів.

\end{document}

