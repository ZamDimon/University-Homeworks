\documentclass[14pt]{extarticle}
\usepackage[english,ukrainian]{babel}
\usepackage[utf8]{inputenc}
\usepackage{amsmath,amssymb}
\usepackage{parskip}
\usepackage{graphicx}
\usepackage{xcolor}
\usepackage{tcolorbox}
\tcbuselibrary{skins}
\usepackage[framemethod=tikz]{mdframed}
\usepackage{chngcntr}
\usepackage{enumitem}
\usepackage{hyperref}
\usepackage{float}
\usepackage{subfig}
\usepackage{esint}
\usepackage[top=2.5cm, left=3cm, right=3cm, bottom=4.0cm]{geometry}
\usepackage[table]{xcolor}
\usepackage{algorithm}
\usepackage{algpseudocode}
\usepackage{listings}
\usepackage{xcolor}

\title{Домашня робота \#3 з курсу ``Комплексний аналіз'' (частина друга)}
\author{Студента 3 курсу групи МП-31 Захарова Дмитра}
\date{\today}

\begin{document}

\maketitle

\begin{center}
\textbf{Варіант 5}
\end{center}

\section*{Завдання.} 

\textbf{Умова.} Знайти всі особливі точки та класифікувати їх
\begin{enumerate}
    \item $f(z) = \frac{1}{(z+3)(z-12)}$;
    \item $f(z) = \frac{\sin 3z}{(z+1)^3}$;
    \item $f(z) = \frac{z+\frac{5}{2}}{\cos \pi z}$.
\end{enumerate}

\textbf{Розв'язок.} 

\textbf{Пункт 1.} Тут $z=-3,z=12$ -- полюси першого порядку. Для цього можемо скористатися критерієм полюса: $f(z)$ має полюс порядку $m$ у точці $z_0$ тоді і тільки тоді, коли $\frac{1}{f(z)}$ має корінь $z_0$ кратності $m$. Оскільки $\frac{1}{f(z)} = (z+3)(z-12)$, то маємо два полюси першого порядку $z=-3,z=12$.

Також $z=\infty$ є усувною, оскільки $\lim_{z \to \infty}f(z) = 0$. 

\textbf{Пункт 2.} Нуль знаменника $z=-1$ кратності $3$, а нулі чисельника $3z_k=\pi k \to z_k = \frac{\pi k}{3}$ для $k \in \mathbb{Z}$. 

Отже, $z=-1$ є полюсом третього порядку, оскільки є коренем кратності $3$ виразу $\frac{1}{f(z)} = \frac{(z+1)^3}{\sin 3z}$. При цьому корені чисельника і знаменника не збігаються, оскільки рівняння $\frac{\pi k}{3} = -1$ немає розв'язків для $k \in \mathbb{Z}$. 

Точка $z=\infty$ є усувною, оскільки $\lim_{z \to \infty} \frac{\sin 3z}{(z+1)^3}=0$. Дійсно,
\[
\left|\frac{\sin 3z}{(z+1)^3}\right| < \frac{1}{(z+1)^3} \xrightarrow[\text{$z \to \infty$}]{} 0
\]

\textbf{Пункт 3.} Нуль чисельника $z=-\frac{5}{2}$, а у знаменника $\cos\pi z_k = 0 \implies \pi z_k = \frac{\pi}{2}+\pi k$, тобто $z_k = \frac{1}{2}+k$.

Бачимо, що при $k=-3$, корені чисельника та знаменника збігаються. Тому проаналізуємо границю:
\[
\lim_{z \to -\frac{5}{2}} \frac{z+\frac{5}{2}}{\cos \pi z} = \lim_{w \to 0} \frac{w}{\cos\left(\pi\left(w-\frac{5}{2}\right)\right)} = \lim_{w \to 0} \frac{w}{\cos \left(\pi w - \frac{5 \pi}{2}\right)} = \lim_{w \to 0} \frac{w}{\sin \pi w} = \frac{1}{\pi}
\]

Отже, точка $z=-\frac{5}{2}$ є усувною. Всі точки $z_k = \frac{1}{2}+k, k \in \mathbb{Z} \setminus \{-3\}$ є полюсами першого порядку. 

$z = +\infty$ є істотною особливістю, оскільки $\lim_{z \to \infty} f(z)$ не визначено. 

\vspace{10px}
\begin{center}
\textit{Ітогові відповіді на наступній сторінці}
\end{center}

\pagebreak
\textbf{Відповідь.}

\textbf{Пункт 1.} 
\begin{enumerate}
\item $z=-3,z=12$ -- полюси (першого порядку);
\item $z=\infty$ -- усувна особливість.
\end{enumerate}

\textbf{Пункт 2.} 
\begin{enumerate}
\item $z=-1$ -- полюс (третього порядку);
\item $z=\infty$ -- усувна особливість.
\end{enumerate}

\textbf{Пункт 3.} 
\begin{enumerate}
\item $z=-\frac{5}{2}$ -- усувна особливість;
\item $z_k=\frac{1}{2}+k, \; k \in \mathbb{Z} \setminus \{-3\}$ -- полюси (першого порядку);
\item $z=\infty$ -- істотна особливість.
\end{enumerate}

\end{document}

