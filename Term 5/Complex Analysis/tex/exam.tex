\documentclass[14pt]{extarticle}
\usepackage[english,ukrainian]{babel}
\usepackage[utf8]{inputenc}
\usepackage{amsmath,amssymb}
\usepackage{parskip}
\usepackage{graphicx}
\usepackage{xcolor}
\usepackage{tcolorbox}
\tcbuselibrary{skins}
\usepackage[framemethod=tikz]{mdframed}
\usepackage{chngcntr}
\usepackage{enumitem}
\usepackage{hyperref}
\usepackage{float}
\usepackage{subfig}
\usepackage{esint}
\usepackage[top=2.5cm, left=3cm, right=3cm, bottom=4.0cm]{geometry}
\usepackage[table]{xcolor}
\usepackage{algorithm}
\usepackage{algpseudocode}
\usepackage{listings}

\tcbuselibrary{theorems}

\newtcbtheorem[number within=section]{lemma}{Лема}%
{colback=blue!5,colframe=blue!35!black,fonttitle=\bfseries}{th}

\newtcbtheorem[number within=section]{theorem}{Теорема}%
{colback=green!5,colframe=green!35!black,fonttitle=\bfseries}{th}

\newtcbtheorem[number within=section]{def}{Визначення}%
{colback=red!5,colframe=red!35!black,fonttitle=\bfseries}{th}

\newtcbtheorem[number within=section]{coll}{Наслідок}%
{colback=yellow!5,colframe=yellow!35!black,fonttitle=\bfseries}{th}

\title{Колоквіум з предмету ``Комплексний аналіз''}
\author{Студента 3 курсу групи МП-31 Захарова Дмитра}
\date{\today}

\begin{document}

\maketitle

\section*{Питання 1.}

Наслiдки теореми про розклад в степеневий ряд. Нерiвнiсть Кошi. Теорема Лiувiлля.

\textbf{Відповідь.}

Тут і далі будемо позначати $\mathcal{H}(\cdot)$ -- множину голоморфних функцій на заданній множині.

Отже, сформулюємо теорему про розклад в степеневий ряд.

\begin{theorem*}{Розклад в степеневий ряд.}
    Нехай $f(z) \in \mathcal{H}(D)$ і $B(a,r) \subset D$. Тоді
\begin{enumerate}
\item $
\exists \{c_n\}_{n=0}^{\infty}: f(z) = \sum_{n=0}^{\infty} c_n(z-a)^n \; \text{в $B(a,r)$}
$
\item $c_n = \frac{f^{(n)}(a)}{n!}$
\item $r_{\text{збіжності}} > r$
\end{enumerate}
\end{theorem*}

Цю теорему ми залишаємо без доведення і запишемо наслідки цієї теореми. 

\begin{theorem*}{Нерівність Коші}
Нехай $f(z) \in \mathcal{H}(B(a,r))$, причому $|f(z)| \leq M$. Тоді:
\[
|c_n| \leq \frac{M}{r^n},
\]    
де $\{c_n\}_{n=0}^{\infty}$ -- це коефіцієнти розкладання $f(z)$ у ряд.
\end{theorem*}

\textbf{Доведення.} Застосовуємо інтегральну формулу Коші:
\[
c_n = \frac{1}{2\pi i}\oint_{\partial B(a,\rho)} \frac{f(\zeta)d\zeta}{(\zeta-a)^{n+1}}
\]
Оцінюємо:
\[
|c_n| \leq \frac{1}{2\pi} \cdot 2\pi \rho \cdot \frac{\max|f(z)|}{\rho^{n+1}} \leq \frac{M}{\rho^n}
\]
Якщо спрямуємо праву частину до $r$, то отримаємо те, що повинні були довести. $\blacksquare$

\begin{theorem*}{Теорема Ліувілля}
    Нехай $f(z) \in \mathcal{H}(\mathbb{C})$ і $|f(z)| \leq M$. Тоді $f \equiv \text{const}$.
\end{theorem*}

\textbf{Доведення.} Розкладаємо $f(z)$:
\[
f(z) = \sum_{n=0}^{\infty} c_nz^n
\]
З нерівності Коші, $|c_n| \leq \frac{M}{\rho^{n}}$. Спрямовуємо $\rho \to \infty$. Для $n \neq 0$ тоді маємо
\[
|c_n| \leq \frac{M}{\rho^n} \xrightarrow[\rho \to \infty]{} 0 \implies c_n \equiv 0 \; \forall n > 0
\]
Отже $f(z) = c_0 = \text{const}$. 

Також, маємо наступні два наслідки, котрі ідейно доводяться так само, як дві теореми вище:

\begin{coll*}{Розкладу в степеневий ряд}
    1. $\underline{\lim}_{r \to \infty} \max_{|z|=r}|f(z)| < \infty \implies f(z) \equiv \text{const}$.  

2. $\exists \alpha: \underline{\lim}_{r \to \infty}\max_{|z|=r}|f(z)| \cdot \frac{1}{z^{\alpha}}<\infty \implies f(z) \; \text{поліном} \; P(z)$, $\text{deg}\, P(z) \leq [\alpha]$. 
\end{coll*}

\section*{Питання 2.}

Чи може лишок у $\infty$ дорівнювати $\infty$?

\textbf{Відповідь.}

Треба знайти таку функцію $f(z)$, щоб
\[
\text{Res}_{z=\infty}f(z)=\lim_{z \to \infty} z(f(\infty)-f(z)) = \infty
\]
Для цього асимптотично треба мати якусь функцію $f(z) \sim \frac{1}{z^{\alpha}}$, де $\alpha \in (0,1)$. Тоді $f(\infty)=0$ і окрім цього
\[
\lim_{z \to \infty} z(f(\infty)-f(z)) = -\lim_{z \to \infty} z^{1-\alpha} = \infty
\]
Проте чи існує така функція, я не знаю. Головна проблема в тому, що вирази виду $z^{\alpha}$ погано обумовлені коли мова йде про комплексний простір (як мінімум, це не функції, а їх сімейство).

Тоді скористаємось тим, що щоб лишок був визначеним, $f(z)$ має бути мераморфною. А якщо функція є мераморфною, то вона розкладається в ряд Лорана. За означенням, лишку буде відповідати коефіцієнт $c_{-1}$, котрий не може бути нескінченним, якщо функція розкладається у ряд. Отже, він є скінчениим.

\section*{Питання 3.}

Наслiдки теореми про максимум
модуля. Лема Шварца.

\textbf{Відповідь.}

Знову ж таки, запишемо теорему про максимум модуля і наведемо наслідки до нього.

\begin{theorem*}{Про максимум модуля}
    1. Якщо $u(z)$ гармонічна в $D$ та $u(z) \not\equiv \text{const}$, тоді
    \[
    \inf_{\zeta \in D} u(\zeta) < u(z) < \sup_{\zeta \in D} u(\zeta) \; \forall z \in D
    \]
    
    2. Якщо $f(z) \in \mathcal{H}(D)$, причому $f \not\equiv \text{const}$. Тоді
    \[
    |f(z)| < \sup_{\zeta \in D}|f(\zeta)| \; \forall z \in D
    \]
\end{theorem*}

Тепер, наводимо наслідки.

\begin{theorem*}{Головний наслідок}
    Нехай маємо функцію $f(z) \in \mathcal{H}(D) \cap \mathcal{C}(\overline{D})$, $D$ -- обмежена область. 
    \[
    |f(z)| \leq \max_{\zeta \in \partial D}|f(\zeta)| \; \forall z
    \]
\end{theorem*}

\textbf{Доведення.} Розглядаємо $|f(z)|$ -- неперервна функція на $\overline{D}$. Оскільки вона неперервна, то вона десь досягає максимума. Скажімо, вона це робить в точці $z_0$. Маємо, що $z_0 \in \partial D$, оскільки вона не може належати $D$, бо це б суперечило теоремі. Тоді, отримуємо результат, котрий треба довести.

\begin{theorem*}{Теорема Вейерштрасса в іншій формі.} 
Нехай $f_n(z) \in \mathcal{H}(D) \cap \mathcal{C}(\overline{D})$, $D$ обмежена, $f_n(z) \rightrightarrows \varphi(z)$ на $\partial D$. Тоді 
\[
\exists f(z) \in \mathcal{H}(D) \cap \overline{C}(\overline{D}): f_n(z) \rightrightarrows f(z) \; \text{в $\overline{D}$}
\]
\end{theorem*}

\begin{lemma*}{Лема Шварца.} 

Нехай $f(z) \in \mathcal{H}(\{z \in \mathbb{C}: |z|<1\})$, $|f(z)| < 1$ та $f(0)=0$. Тоді справедливо
\[
|f(z)| \leq |z|,
\]

причому якщо $\exists z_0 \in \mathbb{C}: f(z_0)=z_0$, то $f(z) =e^{i\gamma}z, \; \gamma \in \mathbb{R}$
\end{lemma*}

\textbf{Відповідь.} Розглядаємо допоміжну функцію
\[
g(z) := \frac{f(z)}{z}
\]
Щоб проаналізувати характер поведінки у нуля, розглядаємо розкладання $f(z)$ в ряд Тейлора в околі $0$:
\[
f(z) = c_0 + c_1 z + c_2z^2 + \dots 
\]
Помічаємо, що $c_0=f(0)=0$, тому
\[
g(z) = \frac{c_1z + c_2z^2+\dots}{z} = c_1+c_2z + \dots 
\]
Отже, в нас немає ніяких проблем з $z=0$ для нашої функції $g(z)$.

Розглядаємо тепер множину $|z| \leq r < 1$. Застосовуємо головний наслідок теореми про максимум модуля:
\[
|g(z)| \leq \max_{|z|=r}\frac{|f(z)|}{|z|} \leq \frac{1}{r}
\]
Гранично переходячи до $r \to 1$, маємо
\[
|g(z)| < 1 \implies \frac{|f(z)|}{|z|} < 1 \implies |f(z)| < |z|.
\]
Якщо ж знайшовся $z_0$ такий, що $|g(z_0)|=1$, то $|g(z_0)|\equiv \text{const}$, то $g(z)=ze^{i\gamma},\gamma \in \mathbb{R}$. $\blacksquare$

\end{document}

