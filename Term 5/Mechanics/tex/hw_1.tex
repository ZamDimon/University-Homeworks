\documentclass[12pt]{extarticle}
\usepackage[english,ukrainian]{babel}
\usepackage[utf8]{inputenc}
\usepackage{amsmath,amssymb}
\usepackage{parskip}
\usepackage{graphicx}
\usepackage{xcolor}
\usepackage{tcolorbox}
\tcbuselibrary{skins}
\usepackage[framemethod=tikz]{mdframed}
\usepackage{chngcntr}
\usepackage{enumitem}
\usepackage{hyperref}
\usepackage{float}
\usepackage{subfig}
\usepackage{esint}
\usepackage[top=2.5cm, left=3cm, right=3cm, bottom=4.0cm]{geometry}
\usepackage[table]{xcolor}
\usepackage{algorithm}
\usepackage{algpseudocode}
\usepackage{listings}

\title{Домашня робота з курсу ``Теоретична механіка''}
\author{Студента 3 курсу групи МП-31 Захарова Дмитра}
\date{\today}

\begin{document}

\maketitle

\textbf{Завдання 1.} Нехай $v_x=20 \, \text{м}/\text{с}$, $\omega=20 \, \text{рад}/\text{с}$, $R=1 \, \text{м}$. Тоді 
\[
x(t) = v_xt - R \sin \omega t, y(t) = R(1 - \cos \omega t)
\]

Швидкості окремо по вісям:
\[
\dot{x}(t) = v_x - \omega R \cos \omega t, \; \dot{y}(t) = \omega R \sin \omega t
\]

Враховуючи, що $\omega = v_x/R$, тобто колесо рухається без ковзання, можемо дещо спростити:
\[
\dot{x}(t) = v_x(1-\cos \omega t), \dot{y}(t) = v_x \sin \omega t
\]

Вектор швидкості $\boldsymbol{v} = [\dot{x},\dot{y}]^{\top} = v_x[1 - \cos \omega t, \sin \omega t]^{\top}$, а модуль:
\[
v(t) = v_x \sqrt{\sin^2 \omega t + (1-\cos \omega t)^2} = v_x \sqrt{2 - 2 \cos \omega t} = 2v_x \sin \frac{\omega t}{2}
\]

Або чисельно $v(t)=40 \sin 10 t$. Прискорення має вигляд:
\[
\boldsymbol{a} = \dot{\boldsymbol{v}} = \omega^2 R[\sin \omega t, \cos \omega t]^{\top}
\]

Чисельно $\boldsymbol{a}(t) = 400[\sin 20t, \cos 20t]^{\top}, a(t) \equiv 400$. 

Доведемо, що радіус кривини $\rho$ дорівнює $2MA$. Знаходимо тангенсальне прискорення:
\[
a_{\tau} = \frac{dv}{dt} = \omega v_x \cos \frac{\omega t}{2} = \omega^2 R \cos \frac{\omega t}{2}
\]

Тоді нормальне прискорення $a_n = \sqrt{a^2-a_{\tau}^2} = \sqrt{1 - \cos^2 \frac{\omega t}{2}}\omega^2R = \omega^2 R \sin \frac{\omega t}{2}$. 

Тому радіус кривини дорівнює:
\[
\rho(t) = \frac{v^2}{a_n} = \frac{4v_x^2 \sin^2 \omega t/2}{\omega^2 R \sin \frac{\omega t}{2}} = 4 \sin \frac{\omega t}{2} R
\]

Знайдемо довжину відрізка $MA$. $A$ має координату $\boldsymbol{r}_A(t)=[v_xt,0]^{\top}$, а радіус-вектор $M$ вже заданий за умовою. Отже:
\[
MA = \|\boldsymbol{r}_A(t)-\boldsymbol{r}_M(t)\| =  \sqrt{\sin^2 \omega t + (1-\cos\omega t)^2}R = 2R \sin \frac{\omega t}{2}
\]

Нескладно бачити, що $\rho \equiv 2MA$

\textbf{Завдання 2.} Позначимо кутову швидкість $\omega=4\pi \, \frac{\text{рад}}{\text{с}}$, довжину шарнирів $l=60 \, \text{см}$ та довжину $MB=\mu l, \mu=1/3$. 

Оскільки перед нами рівнобічний трикутник, то координату точки $M$ можна знайти як:
\[
x(t) = 2l \cos\varphi - \mu l \cos\varphi = l\cos\varphi(2 - \mu)
\]
\[
y(t) = \mu l \sin \varphi
\]

Тобто радіус-вектор траєкторії:
\[
\boldsymbol{r}(t) = l[(2-\mu)\cos \omega t, \mu \sin \omega t]^{\top}
\]

Це еліпс з півосями $(2-\mu)l$ та $\mu l$. При $\mu=1$ отримуємо коло, що цілком логічно, оскільки в такому разі $M=A$. При $\mu=1/3$, тобто за умовою, півосі будуть дорівнювати $5l/3, l/3$. 

Функція швидкості:
\[
\boldsymbol{v}(t) = \dot{\boldsymbol{r}}(t) = \omega l[(\mu-2)\sin \omega t,\mu \cos \omega t]^{\top}
\]

Отже $\boldsymbol{v}(0) = \omega l [0, \mu]^{\top}$. Тому модуль $v(0) = \mu \omega l$. Якщо підставимо, отримуємо $80\pi \, \frac{\text{см}}{\text{с}}$, тобто приблизно $2.5 \, \text{м}/\text{с}$. 

Прискорення:
\[
\boldsymbol{a}(t) = -\omega^2 l [(2-\mu)\cos \omega t, \mu \sin \omega t]^{\top}
\]

В нулі отримуємо $\boldsymbol{a}(0) = (\mu - 2)\omega^2 l \hat{x}$. Для $\mu=1$ цілком логічно отримуємо $\boldsymbol{a}(0)=-\omega^2 l \hat{x}$ -- доцентрове прискорення при русі по колу. Для нашого конкретного випадку маємо $a(0)=5\omega^2 l/3 \approx 157.9 \frac{\text{м}}{\text{с}^2}$. 

Знайдемо радіус кривини. Дотичне прискорення $a_{\tau} = \frac{dv}{dt}$. Модуль швидкості:
\[
v(t) = \omega l \sqrt{\mu^2 \cos^2 \omega t + (2-\mu)^2 \sin^2 \omega t}
\]

Похідна:
\begin{align*}
a_{\tau} = \frac{dv}{dt} = \frac{\omega^2 l^2}{2v} \left(2\mu^2 \cos \omega t (-\omega) \sin \omega t + 2(2-\mu)^2 \sin \omega t\cdot \omega \cos \omega t \right) = \\
\frac{\sin 2\omega t ((2-\mu)^2-\mu^2)}{2 \sqrt{\mu^2 \cos^2 \omega t + (2-\mu)^2 \sin^2 \omega t}} \cdot \omega^2 l = \frac{2(1-\mu)\sin2\omega t}{\sqrt{\mu^2 \cos^2 \omega t + (2-\mu)^2 \sin^2 \omega t}} \cdot \omega^2 l
\end{align*}

Для $t=0$ отримуємо $a_{\tau}(0) = 0$. Отже $a_n(0) = a(0) = (2-\mu)\omega^2 l$. Тоді радіус кривини:
\[
\rho(0) = \frac{v^2}{a_n}\Big|_{t=0} = \frac{\mu^2\omega^2l^2}{(2-\mu)\omega^2 l} = \frac{\mu^2}{2-\mu} l
\]

Для $\mu=1/3$ маємо $\rho(0) = \frac{1/9}{5/3}l = l/15 = 4 \, \text{см}$. 

\end{document}

