\documentclass[14pt]{extarticle}
\usepackage[english,ukrainian]{babel}
\usepackage[utf8]{inputenc}
\usepackage{amsmath,amssymb}
\usepackage{parskip}
\usepackage{graphicx}
\usepackage{xcolor}
\usepackage{tcolorbox}
\tcbuselibrary{skins}
\usepackage[framemethod=tikz]{mdframed}
\usepackage{chngcntr}
\usepackage{enumitem}
\usepackage{hyperref}
\usepackage{float}
\usepackage{subfig}
\usepackage{esint}
\usepackage[top=2.5cm, left=3cm, right=3cm, bottom=4.0cm]{geometry}
\usepackage[table]{xcolor}
\usepackage{algorithm}
\usepackage{algpseudocode}
\usepackage{listings}

\title{Залікова робота з курсу ``Теоретична механіка''}
\author{Студента 3 курсу групи МП-31 Захарова Дмитра}
\date{\today}

\begin{document}

\maketitle

\begin{center}
\textbf{Варіант 6}
\end{center}

\section*{Завдання 1} 

\textbf{Умова.} Динаміка частинки в неінерціальній системі відліку. Сили інерції. Вага тіла.

\textbf{Відповідь.} Нехай маємо деяку інерціальну систему відліку $Oxyz$ (з базисами $\hat{x},\hat{y},\hat{z}$) -- тобто система відліку, що рухається без прискорення.

Нехай маємо частку, на котру діє сумарна сила $\mathbf{F}$. Тоді, другий закон Ньютона запишеться як $m\mathbf{a}=\mathbf{F}$. Тут $\mathbf{a}=\ddot{x}\hat{x}+\ddot{y}\hat{y}+\ddot{z}\hat{z}$.

Проте, іноді в задачах зручніше розглядати рух частинки відносно \textbf{не}інерціальної системи відліку. Нехай неінерціальна система відліку $O'x'y'z'$ з базисами $\hat{x}',\hat{y}',\hat{z}'$ рухається відносно інерціальної. Тоді абсолютне прискорення у неінерціальній системи можна подати у вигляді
\[
\mathbf{a} = \mathbf{a}_{\text{rel}} + \mathbf{a}_{\text{tr}} + \mathbf{a}_{K},
\]
де $\mathbf{a}_{\text{rel}}$ -- відносне прискорення, $\mathbf{a}_{\text{tr}}$ -- переносне, а $\mathbf{a}_K$ -- прискорення Коріоліса. Згадаємо вигляд цих прискорень:
\[
\mathbf{a}_{\text{rel}} = \ddot{x}'\hat{x}'+\ddot{y}'\hat{y}'+\ddot{z}'\hat{z}'
\]
тобто відносне прискорення це вираз прискорення в неінерціальній системі, тільки записане у нових координатах.

Переносне прискорення має вигляд:
\[
\mathbf{a}_{\text{tr}} = \mathbf{a}_{O'} + [\boldsymbol{\varepsilon} \times \mathbf{r}'] + [\boldsymbol{\omega} \times [\boldsymbol{\omega} \times \mathbf{r}']], 
\]
де $\mathbf{a}_{O'}$ -- це прискорення точки відліку $O'$, $\mathbf{r}'$ -- радіус вектор частинки від $O'$, $\boldsymbol{\omega},\boldsymbol{\varepsilon}$ -- кутова швидкість та кутове прискорення системи $O'x'y'z'$.

Нарешті, прискорення Коріоліса:
\[
\mathbf{a}_K = 2[\boldsymbol{\omega} \times \mathbf{v}_{\text{rel}}],
\]
де $\mathbf{v}_{\text{rel}}$ -- це швидкість тіла відносно рухомої системи відліку, тобто $\mathbf{v}_{\text{rel}}=\dot{x}'\hat{x}'+\dot{y}'\hat{y}'+\dot{z}'\hat{z}'$. 

Повертаємось до динаміки. Підставляємо наше прискорення у другий закон Ньютона:
\[
m(\mathbf{a}_{\text{rel}}+\mathbf{a}_{\text{tr}}+\mathbf{a}_K) = \mathbf{F} \implies m\mathbf{a}_{\text{rel}} = \mathbf{F} - m\mathbf{a}_{\text{tr}} - m\mathbf{a}_K
\]
Трошки розтлумачимо результат. Нехай ми спостерігач, що сидить у рухомій системі відліку (наприклад, у салоні автомобіля) і ми спостерігаємо рух частинки. В такому разі, ми спостерігаємо саме величину $\mathbf{a}_{\text{rel}}$. І, як бачимо, в такому разі, окрім зовнішньої сили $\mathbf{F}$, додаються ще дві фіктивні сили, котрі називаються \textbf{силами інерції}:
\[
\boxed{\mathbf{F}_{\text{tr}} = m\left(\mathbf{a}_{O'} + [\boldsymbol{\varepsilon} \times \mathbf{r}'] + [\boldsymbol{\omega} \times [\boldsymbol{\omega} \times \mathbf{r}']]\right), \; \mathbf{F}_K = 2m[\boldsymbol{\omega} \times \mathbf{v}_{\text{rel}}]}
\]
Перший доданок безпосередньо відноситься до відцентрової сили. Нехай $\boldsymbol{\varepsilon} = \mathbf{a}_{O'} = \mathbf{0}$. Тоді якщо $\mathbf{r'} \perp \boldsymbol{\omega}$, то модуль сили можна записати як $F_{\text{tr}}=m\omega^2r'$ -- це достатньо відомий вираз для відцентрової сили. Наприклад, якщо кататися на каруселі, то при достатньо високих кутових швидкостях, предмет/людину на ній починає ``відштовхувати'' від вісі пропорційно квадрату кутової швидкості. Другий відповідає співвідношенню між лінійною швидкістю тіла та кутовою швидкістю. Тобто, якщо ми почнемо якимось чином рухатись по каруселі, то виникала б ще одна сила по модулю $2m\omega v_{\text{rel}}$, де $v_{\text{rel}}$ -- наша швидкість по каруселі. 

Розглянемо питання ваги. Нехай кутова швидкість обертання Землі $\boldsymbol{\Omega}$ відносно вісі $NS$ (північ-південь), радіус Землі $R$. Нехай базис вздовж $NS$ дорівнює $\hat{z}$, а в площині екватору $\hat{x},\hat{y}$ -- це наша інерціальна вісь. Розглянемо також тіло $m$ на нитці над поверхнею і розглядаємо його координати у базисі $O'x'y'z'$, де $\hat{x}',\hat{y}'$ знаходяться у дотичній площині, а $\hat{z}'$ -- до Землі в землю (у випадку ідеальної сфери, цей базис б дивився до її центру). 

Нехай гравітаційна сила $\mathbf{F}_G$. Тоді, другий закон Ньютона запишеться як:
\[
m\mathbf{a} = \mathbf{F}_G - m\mathbf{a}_{O'} - m\mathbf{F}_K + \mathbf{T}.
\]
Тут ми позначили через $\mathbf{T}$ -- силу натягу нитки. Видно, що $\mathbf{F}_K=0$, а $\mathbf{a}_{O'} = -\Omega^2\mathbf{h}$, де $\mathbf{h}$ -- вектор від вісі обертання $NS$ до тіла. Тут $\mathbf{a}_{O'}$ виступає також у ролі відцентрового прискорення. Звідси:
\[
\mathbf{T} = -\left(\mathbf{F}_G-m\Omega^2\mathbf{h}\right)
\]
\textbf{Вагою} називають силу, з котрою тіло діє на підвіс або опору. Тобто в нашому конкретно випадку вона дорівнює $-\mathbf{T}=\mathbf{F}_G-m\Omega^2\mathbf{h}$. 

Інший простий приклад -- нехай людина знаходиться в ліфті, що рухається вниз з постійним прискоренням $\frac{1}{3}g$. В якості інерціальної системи беремо Землю, а в якості неінерціальної -- ліфт. Тоді, якщо запишемо другий закон Ньютона:
\[
m\mathbf{a} = \mathbf{F}_G - m\mathbf{a}_{O'} + \mathbf{N},
\]
де $\mathbf{N}$ -- сила нормальної реакції опори. Оскільки людина в ліфті не рухається прискоренно (ну, щонайменше сподіваємося на це), то $\mathbf{a}=\boldsymbol{0}$ і тоді
\[
\mathbf{N} = m\mathbf{a}_{O'} - \mathbf{F}_G
\]
Тоді вага це $\mathbf{F}_G-m\mathbf{a}_{O'}$. Якщо врахувати, що $\mathbf{F}_G=m\mathbf{g},\mathbf{a}_{O'}=\frac{1}{3}\mathbf{g}$ за умовою, то отримуємо, що вага дорівнює $\frac{2}{3}m\mathbf{g}$. Дійсно, коли ліфт рухається вниз, то вага сприймається меньше, ніж якщо б ліфт рухався вгору.

\pagebreak

\section*{Завдання 2} 

\textbf{Умова.} Диференціальне рівняння обертання твердого тіла навколо нерухомої осі. Фізичний маятник.

\textbf{Відповідь.} Для записання диференціального рівняння обертання, спочатку введемо поняття \textit{моменту інерції}.

Нехай маємо систему часток $\{m_{\nu}\}_{\nu=1}^N$. Введемо відстані від кожної з точок до деякої вісі $\ell$ як $\{h_{\nu}(\ell)\}_{\nu=1}^N$. Розглянемо наступну суму:
\[
I_{\ell} \triangleq \sum_{\nu=1}^N m_{\nu}h_{\nu}^2(\ell).
\]
Така сума називається осьовим \textbf{моментом інерції} системи відносно вісі $\ell$. У випадку, коли маємо тіло з розподілом густини $\rho(\mathbf{r})$:
\[
I_{\ell} \triangleq \int_V \rho(\mathbf{r})h^2(\mathbf{r},\ell)d^3\mathbf{r},
\]
де $h(\mathbf{r},\ell)$ -- відстань між точкою $\mathbf{r}$ та віссю $\ell$. 

\textbf{Приклад.} Нехай треба знайти момент інерції однородного суцільного диску маси $m$ та радіусу $R$ навколо нерухомої вісі, що проходить через його центр. Розіб'ємо диск на багато кілець малої товщини $dr$ радіусу $r$. Тоді маса такого кільця $2\sigma\pi r$, де $\sigma=\frac{m}{\pi R^2}$ -- поверхнева густина диску. Тоді:
\[
I = \int_{[0,R]} 2\pi r\sigma dr \cdot r^2 =2\pi\sigma \cdot \frac{R^4}{4} = \frac{\pi R^4}{2} \cdot \frac{m}{\pi R^2} = \frac{mR^2}{2}
\]

\textbf{Диференціальне рівняння обертання твердого тіла.} Перейдемо до безпосередньо диференціального рівняння обертання тіла навколо вісі. Цей закон записується наступним чином:
\[
I\frac{d\omega}{dt} = \tau,
\]
де $I$ -- момент інерції, $\omega$ -- кутова швидкість тіла навколо заданої вісі, а $\tau$ -- момент сил, відносно вісі. 

Взагалі кажучи, момент сил є величиною векторною. Якщо на тіло діють набір сил $\{\mathbf{F}_{\mu}\}_{\mu=1}^M$, то цей момент сил можна знайти як:
\[
\boldsymbol{\tau} = \sum_{\mu=1}^M [\mathbf{r}_{\mu} \times \mathbf{F}_{\mu}] 
\]
і далі знайти його проекцію на вісь обертання $\ell$. 

\textbf{Фізичний маятник}. Фізичним маятником називають тверде тіло довільної форми, яке під дією сили тяжіння здійснює коливання навколо вертикальної вісі, що проходить через центр мас. Нехай ми закріпили тіло в точці $P$. Частоту коливань такого маятника тоді можна знайти за формулою:
\[
f = \frac{1}{2\pi}\sqrt{\frac{mg\ell_P}{I_P}},
\]
де $I_P$ -- момент інерції навколо вісі закріплення, $\ell_P$ -- відстань між точкою закріплення $P$ та центром мас.

\textbf{Виведення.} Відхилемо маятник на малий кут $\theta$ від положення рівноваги. В такому разі, момент сили тяжіння можна записати як:
\[
\tau = -mg\ell_P \sin\theta
\]
Тоді, диференціальне рівняння обертання твердого тіла запишеться як:
\[
I_P \ddot{\theta} = -mg\ell_P \sin \theta \implies \boxed{\ddot{\theta} + \frac{mg\ell_P}{I_P}\sin\theta = 0}
\]
Якщо вважати коливання малими, то $\sin\theta \approx \theta$ і тоді маємо гармонічні коливання з циклічною частотою $\omega=\sqrt{\frac{mg\ell_P}{I_P}}$. Звідси частота $f=\frac{\omega}{2\pi}$, що і треба було показати.

\pagebreak

\section*{Завдання 3.}

\textbf{Умова.} Складіть рівняння Лагранжа для візка маси $m_1$, колеса якого є двома суцільними дисками маси $m_2$ і радіуса $r$. Візок закріплений пружиною жорсткості $k$.

\textbf{Розв'язок.} Для початку визначимось, що вважати за координату. Нехай $x$ -- це відхилення довжини пружини від початкової $\ell_0$. 

Отже, нам потрібно записати вирази для кінетичної енергії $T(x,\dot{x},t)$ та потенціальної $V(x,t)$. Тоді, Лагранжиан запишеться як:
\[
\mathcal{L}(x,\dot{x},t) \triangleq T(x,\dot{x},t) - V(x,t)
\]

Отже, почнемо з потенціальної. Вона складається лише з потенціальної енергії пружини $V(x,t) = \frac{kx^2}{2}$. 

Більш цікаве питання -- яка кінетична енергія нашої системи. По-перше, швидкість візка дорівнює $\dot{x}$, тому його кінетична енергія $T_1 = \frac{m_1\dot{x}^2}{2}$. Далі знаходимо потенціальну енергію дисків. Центр колеса, оскільки він є частиною візка, рухається також зі швидкістю $\dot{x}$ (інакше б візок деформувався). Тоді, кутова швидкість колеса $\omega = \frac{\dot{x}}{r}$. В такому разі кінетична енергія колеса:
\[
\widetilde{T}_2(\dot{x}) = \frac{I\omega^2}{2} + \frac{m_2\dot{x}^2}{2} = \frac{m_2r^2}{2} \cdot \frac{1}{2} \left(\frac{\dot{x}}{r}\right)^2 + \frac{m_2\dot{x}^2}{2} = \frac{3m_2\dot{x}^2}{4}
\]
Оскільки маємо два колеса, загальна кінетична енергія колес $T_2(\dot{x})=2\widetilde{T}_2(\dot{x})$. Таким чином, повна кінетична енергія системи:
\[
T(\dot{x}) = T_1 + T_2 = \frac{m_1\dot{x}^2}{2} + \frac{3m_2\dot{x}^2}{2}
\]
Отже, лагранжиан системи:
\[
\mathcal{L}(x,\dot{x}) = \frac{(m_1+3m_2)\dot{x}^2}{2} - \frac{kx^2}{2}
\]
Тепер, запишемо рівняння Лагранжа:
\[
\frac{d}{dt}\left(\frac{\partial\mathcal{L}}{\partial\dot{x}}\right) - \frac{\partial\mathcal{L}}{\partial x} = 0
\]
Підставляємо:
\[
\frac{\partial\mathcal{L}}{\partial\dot{x}} = m_1\dot{x} + 3m_2\dot{x} = (m_1+3m_2)\dot{x}, \; \frac{d}{dt}\left(\frac{\partial\mathcal{L}}{\partial\dot{x}}\right) = (m_1+3m_2)\ddot{x}
\]
\[
\frac{\partial\mathcal{L}}{\partial x} = -kx
\]
Отже, остаточно рівняння:
\[
(m_1+3m_2)\ddot{x} + kx = 0 \implies \boxed{\ddot{x} + \frac{k}{m_1+3m_2}x = 0}
\]
Це є гармонійними коливаннями з циклічною частотою $\omega=\sqrt{\frac{k}{m_1+3m_2}}$. Бачимо, що якщо $m_2=0$, це б відповідало звичайним гармонічними коливанням грузика на пружині і частота б відповідала $\omega\Big|_{m_2=0}=\sqrt{\frac{k}{m_1}}$. 

\textbf{Відповідь.} $\ddot{x} + \frac{k}{m_1+3m_2}x = 0$. Це є гармонічними коливаннями з циклічною частотою $\sqrt{\frac{k}{m_1+3m_2}}$.

\end{document}

