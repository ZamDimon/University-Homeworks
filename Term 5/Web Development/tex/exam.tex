\documentclass[12pt]{extarticle}
\usepackage[english,ukrainian]{babel}
\usepackage[utf8]{inputenc}
\usepackage{amsmath,amssymb}
\usepackage{parskip}
\usepackage{graphicx}
\usepackage{xcolor}
\usepackage{tcolorbox}
\tcbuselibrary{skins}
\usepackage[framemethod=tikz]{mdframed}
\usepackage{chngcntr}
\usepackage{enumitem}
\usepackage{hyperref}
\usepackage{float}
\usepackage{subfig}
\usepackage{chngcntr}
\usepackage{esint}

\usepackage[top=2.5cm, left=3cm, right=3cm, bottom=4.0cm]{geometry}
\usepackage[table]{xcolor}

\usepackage{algorithm}
\usepackage{algpseudocode}
\usepackage{listings}
\usepackage{xcolor}
\usepackage{dsfont}

\definecolor{codegreen}{rgb}{0,0.6,0}
\definecolor{codegray}{rgb}{0.5,0.5,0.5}
\definecolor{codepurple}{rgb}{0.58,0,0.82}
\definecolor{backcolour}{rgb}{0.95,0.95,0.92}

\lstdefinestyle{mystyle}{
    backgroundcolor=\color{backcolour},   
    commentstyle=\color{codegreen},
    keywordstyle=\color{magenta},
    numberstyle=\tiny\color{codegray},
    stringstyle=\color{codepurple},
    basicstyle=\ttfamily\footnotesize,
    breakatwhitespace=false,         
    breaklines=true,                 
    captionpos=b,                    
    keepspaces=true,                 
    numbers=left,                    
    numbersep=5pt,                  
    showspaces=false,                
    showstringspaces=false,
    showtabs=false,                  
    tabsize=2
}
\lstset{style=mystyle}

\usepackage{ragged2e}

\usepackage{color}
\definecolor{lightgray}{rgb}{.9,.9,.9}
\definecolor{darkgray}{rgb}{.4,.4,.4}
\definecolor{purple}{rgb}{0.65, 0.12, 0.82}

\lstdefinelanguage{JavaScript}{
  keywords={typeof, new, true, false, catch, function, return, null, catch, switch, var, if, in, while, do, else, case, break, async, for, const, await},
  keywordstyle=\color{blue}\bfseries,
  ndkeywords={class, export, boolean, throw, implements, import, this, console},
  ndkeywordstyle=\color{darkgray}\bfseries,
  identifierstyle=\color{black},
  sensitive=false,
  comment=[l]{//},
  morecomment=[s]{/*}{*/},
  commentstyle=\color{purple}\ttfamily,
  stringstyle=\color{red}\ttfamily,
  morestring=[b]',
  morestring=[b]"
}

\lstset{
   language=JavaScript,
   backgroundcolor=\color{lightgray},
   extendedchars=true,
   basicstyle=\footnotesize\ttfamily,
   showstringspaces=false,
   showspaces=false,
   numbers=left,
   numberstyle=\footnotesize,
   numbersep=9pt,
   tabsize=2,
   breaklines=true,
   showtabs=false,
   captionpos=b
}

\title{Залікова робота з курсу ``Основи \textit{web}-програмування''}
\author{Студента 3 курсу групи МП-31 Захарова Дмитра Олеговича}
\date{\today}

\begin{document}

\maketitle

\begin{center}
\textbf{Білет \#5}
\end{center}

\section*{Питання 1} 

\textbf{Умова.} Що таке асинхронність у \textit{Javascript}? Які методи використовуються для управління асинхронним кодом?

\textbf{Відповідь.} Зазвичай, базові програми працюють таким чином, що є деяка послідовність операцій, котрі виконуються послідовно. Для застосунків це не завжди зручно і, головне, не завжди швидко. \textbf{Асинхронність} дозволяє виконувати деякі операції \textit{паралельно}. 

Як базовий приклад -- нехай нам потрібно зробити API для вебсайту, де, наприклад, ми приймаємо великий \textit{.pdf} файл, і нам потрібно якимось чином його обробити. Обробка -- операція дорога і ми не можемо дозволити одразу прийняти запит і його обробити (інакше після 5-10 запитів наша API повністю зависне). Тому, замість цього, ми можемо створити чергу і запустити декілька процесів для обробки кожного. Таким чином, ми зможемо використати асинхронність.

Для управління асинхронністю в \textit{Javascript} використовуються два ключових слова: \texttt{async} та \texttt{await}. \texttt{async} використовується для позначення методу або функції, що буде використовуватися у асинхронному режимі. Функція в такому разі не буде повертати значення безпосередньо, а так званий \texttt{Promise} -- тобто обгортку навколо значення. 

Наприклад, давайте створимо наступну функцію:

\begin{lstlisting}[language=javascript, caption=Створення асинхронної функції]
const NUMBERS_TO_DISPLAY = 10
const MILISECONDS_TO_WAIT = 1000

async function fn(a) {
    await new Promise(r => setTimeout(r, MILISECONDS_TO_WAIT * (NUMBERS_TO_DISPLAY - a)))
    return a
}

for (var i = 0; i < NUMBERS_TO_DISPLAY; i++) {
    fn(i).then(number => console.log(number))
}
\end{lstlisting}

\textit{Коротке пояснення:} ми проходимось по числам від $0$ до \texttt{NUMBERS\_TO\_DISPLAY-1}. Далі викликаємо асинхронну функцію \texttt{fn} для кожного числа \texttt{k}, на початку котрої чекаємо \texttt{NUMBERS\_TO\_DISPLAY-k} секунд (тобто, для числа $0$ ми чекаємо \texttt{NUMBER\_TO\_DISPLAY} секунд, для наступного числа на одну секунду меньше тощо). Для очікування вже бачимо ключове клово \texttt{await}, але поки на нього не звертатимемо увагу.

Якщо б функція не була асинхронною, то ми б спочатку зачекали $10$ секунд і вивели б $0$, потім ще $9$ секунд і вивели б $1$ і так далі до $9$. Тобто сумарно ми б чекали $55$ секунд.

Якщо ж ми використовуємо асинхронність, то ми запускаємо усі процеси і усі числа будуть виведені через $10$ секунд (час найбільшого очікування з усіх процесів). Причому, не у порядку зростання, як б було при послідовному виконанні, а навпаки -- спочатку найбільше (оскільки треба чекати лише $1$ секунду) і до $0$. Для виведення ми використали метод \texttt{.then}, в котрий ми можемо передати анонімну функцію, що виконається при завершенні відповідного процесу.

Розглянемо тепер ключове слово \texttt{await}. Трошки модифікуємо нашу програму наступним чином:
\begin{lstlisting}[language=javascript, caption=Використання ключового слову \texttt{await}]
const NUMBERS_TO_DISPLAY = 10
const MILISECONDS_TO_WAIT = 1000

async function fn(a) {
    await new Promise(r => setTimeout(r, MILISECONDS_TO_WAIT * (NUMBERS_TO_DISPLAY - a)))
    return a
}

async function main() {
    for (var i = 0; i < NUMBERS_TO_DISPLAY; i++) {
        let result = await fn(i)
        console.log(result)
    }
}

main()
\end{lstlisting}

Тепер, для кожної цифри ми будемо очікувати результат з функції \texttt{fn}. Це, по своїй суті, буде відповідати послідовному виконанню операцій і ми будемо чекати $55$ секунд. Часто це використовується тоді, коли ми не можемо розпаралелити процеси і нам потрібно дочекати конкретний результат. 

Також, іноді асинхронні функції можуть повертати помилки і їх треба оброблювати. Приклад можна побачити знизу:
\begin{lstlisting}[language=javascript, caption=Обробка помилок]
const NUMBERS_TO_DISPLAY = 10
const MILISECONDS_TO_WAIT = 1000

async function fn(a) {
    throw new Error('Error in fn')
    return a // This will never be returned
}

async function main() {
    for (var i = 0; i < NUMBERS_TO_DISPLAY; i++) {
        fn(i).then(result => {
            console.log(result)
        }).catch(error => {
            console.log('Hey, I know how to handle errors in asynchronous js!')
            console.log(error)
        })
    }
}

main()
\end{lstlisting}

На виході отримаємо виведення помилок у консоль разом із повідомленням ``\textit{Hey, I know how to handle errors in asynchronous js!}''.  

\section*{Питання 2}

\textbf{Умова.} Медіа запити в \textit{CSS}.

\textbf{Відповідь.} Найбільш широке використання медіа запитів -- це додавання адаптивності до веб-сайтів. Тобто, ми можемо задавати умови на розмір екрану, і відповідно до цього, задавати унікальні стилі (зменшити розмір шрифта або зробити інше вирівнювання на сторінці тощо). Синтаксис команди має наступний вигляд:
\[
\texttt{@media <media-query-list> \{ <rule-list> \}}
\]
Для демонстрації надамо наступну програму:
\begin{lstlisting}[language=html, caption=Повна \texttt{html} сторінка з використанням \texttt{@media} запитів]
<!doctype html>

<html lang="en">
<head>
    <meta charset="utf-8">
    <title>Programming exam</title>
</head>
<style>
    body { background-color: rgb(230.0, 230.0, 230.0); }
    h1   { color: rgb(20.0, 20.0, 20.0); }

    @media screen and (max-width: 900px) {
        .small-screen {
            color: red;
        }
    }
    @media screen and (min-width: 900px) {
        .huge-screen {
            color: red;
        }
    }
</style>
<body>
    <!-- Will be displayed based on the screen's size -->
    <h1 class="huge-screen"> Huge screen here! </h1>
    <h1 class="small-screen"> Small screen here! </h1>

    <!-- This launches a .js script for the first exam question -->
    <script src="main.js"></script>
</body>
</html>
\end{lstlisting}

Вебсайт складається з двох чорних заголовків: ``\textit{Huge screen here!}'', ``\textit{Small screen here!}''. Якщо екран маленький, то червоним помітиться напис ``\textit{Small screen here!}'', а якщо великий -- ``\textit{Huge screen here!}''. Це досягається за допомогою медіазапиту:
\begin{lstlisting}[language=css, caption=Використання \texttt{@media} запитів]
@media screen and (max-width: 900px) {
    .small-screen {
        color: red;
    }
}
@media screen and (min-width: 900px) {
    .huge-screen {
        color: red;
    }
}
\end{lstlisting}

За допомогою умови \texttt{@media screen and (max-width: 900px) \{ ... \}} ми кажемо \textit{CSS}'у, що те, що в дужках, має застосовуватись лише за умови, що ширина екрану має значення до 900 пікселів. Відповідно, умова на \texttt{min-width} позначає, що зміни відбуваються \textbf{від} ширини екрану в 900 пікселів. 

\end{document}

