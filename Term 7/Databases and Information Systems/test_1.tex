\documentclass{hw_template}

\title{\huge\sffamily\bfseries Контрольна Робота з Бази даних та інформаційних систем}
\author{\Large\sffamily Захаров Дмитро}
\date{\sffamily 14 листопада, 2024}

\begin{document}

\pagestyle{fancy}

\maketitle

\begin{center}
    \textbf{Варіант 5}
\end{center}

\tableofcontents

\pagebreak

\section{Завдання 1: Мережва Модель Даних}

\begin{problem}
    Мережева модель даних.
\end{problem}

\textbf{Відповідь.} Мережева модель даних була дуже поширеною на певному етапі
розвитку баз даних і досі використовується в деяких системах. Вона подібна до
ієрархічної моделі, оскільки також складається з набору записів, які можуть бути
власниками або членами групи, але головна відмінність полягає в тому, що один
запис у мережевій моделі може бути частиною кількох груп одночасно. Кожна група
має свою назву, а типи групових відносин відрізняються від їхніх конкретних
екземплярів. Тип визначає спільні характеристики для всіх екземплярів, а сам
екземпляр є записом-власником із підлеглими елементами. При цьому обмеження
таке: один запис не може одночасно бути частиною двох груп одного типу.

У цій моделі будь-які елементи можуть бути пов’язані один з одним, а підлеглий
запис може мати більше одного власника. Це дозволяє підтримувати зв’язки типу
``багато-до-багатьох''. Щоб спростити мережеву структуру, можна додати надмірні
дані, що дозволяє створити дерево через дублювання деяких елементів.

Іноді така надмірність є прийнятною, але в деяких випадках вона може бути
значною. Методи, що добре працюють для деревовидних структур, часто не підходять
для мережевих моделей, через що програми для обробки дерев не можуть працювати з
мережею.

Ситуацію ускладнює те, що методи, які застосовуються до одного виду мережевої
структури, можуть бути неефективними для іншого виду. Суттєвим мінусом мережевої
моделі є необхідність суворого дотримання фізичної структури даних, що ускладнює
організацію запитів. Однак її ключова перевага -- можливість створювати складні
моделі, які краще відповідають реальним процесам. Щоб зменшити складність
ієрархічної та мережевої моделей, було розроблено реляційну модель.

Сильні сторони мережевої моделі — це її ефективність у використанні пам'яті та
швидкість обробки. У порівнянні з ієрархічною, мережева модель точніше
відображає предметну область завдяки додатковим зв’язкам між елементами. Проте
серед її мінусів — складність у використанні через жорстку схему. Контроль
цілісності даних у мережевій моделі послаблений через можливість довільних
зв’язків, і при зміні даних часто потрібно змінювати програмне забезпечення. 

\newpage

\section{Завдання 2: Псевдоніми}

\begin{problem}
    Псевдоніми в SQL
\end{problem}

\textbf{Відповідь.} У SQL псевдоніми (aliases) використовуються для тимчасового
перейменування таблиць або стовпців у запитах. Це дозволяє зробити код більш
зрозумілим і зручним для читання, особливо при роботі зі складними виразами чи
кількома таблицями.

Псевдонім для стовпця дозволяє змінити назву виведеного результату на більш
зрозумілу для користувача. Для цього використовується ключове слово \texttt{AS}, хоча
його можна опустити. Наприклад,
\begin{lstlisting}[language=SQL]
SELECT 
    AVG(math_score) AS avg_math_score,
    AVG(age) AS avg_age
FROM students;
\end{lstlisting}

вибере середній бал з математики та вік учнів з таблиці \texttt{students} та
перейменує їх на \texttt{avg\_math\_score} та \texttt{avg\_age} відповідно.

Альтернативно, псевдоніми можна використовувати для усієї таблиці. Наприклад,
якщо ми хочемо взяти бали з математики та вік з таблиці \texttt{students}, 
то можна використати псевдоніми для усієї таблиці:
\begin{lstlisting}[language=SQL]
SELECT s.math_score, s.age
FROM students s;
\end{lstlisting}

Для наступної таблиці, ця команда цілком працює:
\begin{lstlisting}[language=SQL]
-- Create the tables
CREATE TABLE students (
    student_id INT PRIMARY KEY,
        name       VARCHAR(50),
    math_score INT,
    age        INT
);

-- Insert data into students table
INSERT INTO students (student_id, name, math_score, age) VALUES
(1, 'Dmytro', 95, 15),
(2, 'Olena', 100, 14),
(3, 'Mykhailo', 70, 15),
(4, 'John', 40, 14);
\end{lstlisting}

Помітимо, що тут \texttt{AS} є необов'язковим, але він може полегшити читання
запиту.

Окрім цього, псевдоніми можна створювати з декількох слів через пробіл.
Наприклад,
\begin{lstlisting}[language=SQL]
SELECT name AS [Super Guys]
FROM students; 
\end{lstlisting}

Альтернативно, можна було використати
\begin{lstlisting}[language=SQL]
SELECT name AS "Super Guys"
FROM students;
\end{lstlisting}

Ба більше, в якості псевдоніму можна конкатенувати рядки:
\begin{lstlisting}[language=SQL]
SELECT 
    student_id as "Student ID", 
    CONCAT(name, ', ', math_score, ', ', age) AS "Student Info"
FROM students;
\end{lstlisting}

Результатом буде таблиця:
\begin{lstlisting}
Student ID 	Student Info
1 	        Dmytro, 95, 15
2 	        Olena, 100, 14
3 	        Mykhailo, 70, 15
4 	        John, 40, 14
\end{lstlisting}

\textit{Замітка.} Команда \texttt{CONCAT} використовується для конкатенації
рядків, проте, наскільки мені відомо, вона підтримується не усюди (наприклад,
підтримується у \texttt{MySQL}).

\newpage

\section{Завдання 3: Середні значення}

\begin{problem}
При вибірці з таблиці employees створіть нові псевдоніми \texttt{avg\_sal},
\texttt{avg\_age} в яких будуть лежати середні арифметичні зарплати
(\texttt{salary}) і віку (\texttt{age}). Підрахуйте для усієї таблиці.
\end{problem}

\textbf{Відповідь.} Для початку, давайте вкажемо міграцію для створення таблиці
та заповнення її тестовими даними:
\begin{lstlisting}[language=SQL]
-- Create the employees table
CREATE TABLE employees (
    employee_id INT PRIMARY KEY,
    name VARCHAR(50),
    salary DECIMAL(10, 2),
    age INT
);

-- Insert data into employees table
INSERT INTO employees (employee_id, name, salary, age) VALUES
(1, 'John Doe', 50000.00, 30),
(2, 'Jane Smith', 60000.00, 40),
(3, 'Alice Johnson', 55000.00, 35),
(4, 'Bob Brown', 48000.00, 28);
\end{lstlisting}

Тепер, можемо використати псевдоніми для підрахунку середніх значень зарплати та
віку:
\begin{lstlisting}[language=SQL]
-- Query to calculate average salary and age
-- Query to select average salary and age with aliases
SELECT 
    AVG(salary) AS avg_sal,
    AVG(age) AS avg_age
FROM employees;
\end{lstlisting}

Виходом буде:
\begin{lstlisting}
       avg_sal   | avg_age
    53250.000000 | 33.2500
\end{lstlisting}

\newpage

\section{Завдання 4: Вкладені запити}

\begin{problem}
    Задані таблиці
    \begin{align*}
        \texttt{table1}: && \texttt{user\_id}, \texttt{username}, \texttt{dpt\_name} \\
        \texttt{table2}: && \texttt{phone\_id}, \texttt{dpt\_name}, \texttt{phone\_number} \\
        \texttt{table3}: && \texttt{room\_id}, \texttt{phone\_number}, \texttt{room\_name}
    \end{align*}

    Необхідно вибрати номери кімнат в яких сидить користувачі з ім'ям
    \texttt{Nick} Не вживати \texttt{JOIN}. 
\end{problem}

\textbf{Відповідь.} Для початку, давайте вкажемо міграцію для створення таблиць
та заповнення їх даними:
\begin{lstlisting}[language=SQL]
-- Create the tables
CREATE TABLE table1 (
    user_id INT PRIMARY KEY,
    username VARCHAR(50),
    dpt_name VARCHAR(50)
);

CREATE TABLE table2 (
    phone_id INT PRIMARY KEY,
    dpt_name VARCHAR(50),
    phone_number VARCHAR(20)
);

CREATE TABLE table3 (
    room_id INT PRIMARY KEY,
    phone_number VARCHAR(20),
    room_name VARCHAR(50)
);

-- Insert data into table1
INSERT INTO table1 (user_id, username, dpt_name) VALUES
(1, 'Nick', 'HR'),
(2, 'Alice', 'IT'),
(3, 'Nick', 'Finance'),
(4, 'John', 'HR');

-- Insert data into table2
INSERT INTO table2 (phone_id, dpt_name, phone_number) VALUES
(1, 'HR', '12345'),
(2, 'IT', '67890'),
(3, 'Finance', '11223'),
(4, 'Marketing', '44556');

-- Insert data into table3
INSERT INTO table3 (room_id, phone_number, room_name) VALUES
(1, '12345', 'Room A'),
(2, '67890', 'Room B'),
(3, '11223', 'Room C'),
(4, '44556', 'Room D');
\end{lstlisting}

Далі, скористаємось функцією вкладених запитів для вирішення поставленої задачі:
\begin{lstlisting}[language=SQL]
-- Query to select room names where users named 'Nick' sit
SELECT room_name
FROM table3
WHERE phone_number IN (
    SELECT phone_number
    FROM table2
    WHERE dpt_name IN (
        SELECT dpt_name
        FROM table1
        WHERE username = 'Nick'
    )
);
\end{lstlisting}

Виходом буде:
\begin{lstlisting}
    room_name 
      Room A 
      Room C 
\end{lstlisting}

Тут, ми спочатку вибираємо назви відділів, в яких працює користувач з ім'ям
\texttt{Nick}. Потім, вибираємо номери телефонів, які належать цим відділам. Нарешті,
вибираємо назви кімнат, в яких знаходяться телефони з вибраними номерами.

\textit{Відмітка.} Ця команда працює лише за умови, що телефон з таблиці 2 
відповідає саме цілому відділу, а не окремому користувачеві, бо інакше 
гарантувати зв'язок між користувачем та його телефоном буде неможливо, які 
б команди ми не використовували.

\end{document}