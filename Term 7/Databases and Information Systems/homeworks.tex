\documentclass{hw_template}

\title{\huge\sffamily\bfseries Домашні Роботи з Бази даних та інформаційних систем}
\author{\Large\sffamily Захаров Дмитро}
\date{\sffamily 28 листопада, 2024}

\begin{document}

\pagestyle{fancy}

\maketitle

\tableofcontents

\pagebreak

\section{Домашня Робота 1: Підготовка}

\begin{problem}

Потрібно:
\begin{enumerate}
    \item Встановити обладнання за моєю інструкцією з практики 1.
    \item Ознайомитися з елементами інтерфейсу.
\end{enumerate}

\end{problem}

\textbf{Відповідь.} Це завдання суто по установці, тому відповідь на неї не наводжу
(проте, обладнання було встановлене!).

\newpage

\section{Домашня Робота 2: Вибірка з умовами, сортування, группування}

\begin{problem}
    Є таблиця \texttt{employee}
    
    \begin{table}[H]
        \centering
        \begin{tabular}{|c|c|c|c|c|c|c|c|}
            \hline
            \texttt{l\_name} & \texttt{f\_name} & \texttt{age} & \texttt{sex} & \texttt{status} & \texttt{dpt} & \texttt{position} & \texttt{salary} \\
            \hline
            Mashko & Oleg & 22 & m & single & IT & Admin & 20000 \\
            Petrenko & Zoya & 24 & m & married & IT & Programmer & 10000 \\
            Vovk & Zahar & 32 & m & single & Sales & Admin & 8000 \\
            Kononov & Oleg & 26 & f & single & Sales & Sales & 8500 \\
            Galenda & Anna & 26 & m & single & IT & Programmer & 25000 \\
            Osokor & Galina & 18 & f & married & IT & QA & 18000 \\
            \hline
        \end{tabular}
    \end{table}

    \begin{enumerate}
    \item Вибрати прізвища та посади всіх, у кого ім'я \texttt{Oleg}, впорядкувати за прізвищами за
    алфавітом
    \item Вибрати імена, прізвища, стать (\texttt{sex}) та вік усіх, хто не є програмістами,
    впорядкувати за віком в порядку зменшення.
    \item Підрахувати та вивести середню зарплату усіх співробітників, яким не 26 років.
    \item Вибрати прізвища, відділи (\texttt{dpt}) та посади всіх співробітників із зарплатами понад
    9000, впорядкувати за прізвищами.
    \item Вибрати всіх, хто не має сім'ї (\texttt{single}) і не адмін.
    \item Вибрати відділи (\texttt{dpt}), у якому працюють адміни.
    \item Написати, що виведе наступний запит:
    \texttt{SELECT f\_name, l\_name, age FROM employee WHERE status='single' ORDER BY 2
    DESC}
    \item Написати, що виведе наступний запит:
    \texttt{SELECT l\_name, sex, status FROM employee ORDER BY status, sex}
    \item Підрахувати та вивести максимальній вік адмінів.
    \item Вивести список співробітників, що мають зарплату, вищу за середню
    \end{enumerate}
\end{problem}

\textbf{Відповідь.} Перед виконанням усіх завдань, створимо задану таблицю:
\begin{lstlisting}[language=SQL]
-- Specifying the table schema
CREATE TABLE employee (
    l_name VARCHAR(50),
    f_name VARCHAR(50),
    age INT,
    sex CHAR(1),
    status VARCHAR(20),
    dpt VARCHAR(20),
    position VARCHAR(50),
    salary INT
);

-- Inserting the data
INSERT INTO employee (l_name, f_name, age, sex, status, dpt, position, salary)
VALUES
('Mashko', 'Oleg', 22, 'm', 'single', 'IT', 'Admin', 20000),
('Petrenko', 'Zoya', 24, 'm', 'married', 'IT', 'Programmer', 10000),
('Vovk', 'Zahar', 32, 'm', 'single', 'Sales', 'Admin', 8000),
('Kononov', 'Oleg', 26, 'f', 'singe', 'Sales', 'Sales', 8500),
('Galenda', 'Anna', 26, 'm', 'single', 'IT', 'Programmer', 25000),
('Osokor', 'Galina', 18, 'f', 'married', 'IT', 'QA', 18000);
\end{lstlisting}

\textbf{Питання 1.} Використовуємо наступний запит:
\begin{lstlisting}[language=SQL]
SELECT l_name, position
FROM employee
WHERE f_name = 'Oleg'
ORDER BY l_name;
\end{lstlisting}

Як і просить завдання, ми вибираємо значення колонок \texttt{l\_name} та \texttt{position} з таблиці \texttt{employee},
де значення колонки \texttt{f\_name} дорівнює \texttt{Oleg}. Для сортування за прізвищами використовуємо \texttt{ORDER BY l\_name}.
Результат виконання запиту:
\begin{table}[H]
    \centering
    \begin{tabular}{|c|c|}
        \hline
        \texttt{l\_name} & \texttt{position} \\
        \hline
        Kononov & Sales \\
        Mashko & Admin \\
        \hline
    \end{tabular}
\end{table}

\textbf{Питання 2.} Використовуємо наступний запит:
\begin{lstlisting}[language=SQL]
SELECT f_name, l_name, sex, age
FROM employee
WHERE position != 'Programmer'
ORDER BY age DESC;
\end{lstlisting}

Як результат, отримуємо:
\begin{table}[H]
    \centering
    \begin{tabular}{|c|c|c|c|}
        \hline
        \texttt{f\_name} & \texttt{l\_name
        } & 
        \texttt{sex} & \texttt{age} \\
        \hline
        Zahar   & Vovk       & m   & 32 \\
        Oleg     & Kononov    & f   & 26 \\
        Oleg     & Mashko     & m   & 22 \\
        Galina   & Osokor     & f   & 18 \\
        \hline
    \end{tabular}
\end{table}

\textbf{Питання 3.} Використовуємо наступний запит:
\begin{lstlisting}[language=SQL]
SELECT AVG(salary) AS average_salary
FROM employee
WHERE age != 26;
\end{lstlisting}

Зарплати співробітників, яким не 26 років: 20000, 10000, 8000, 18000, тому
середня зарплата буде дорівнювати $14000$. 

\textbf{Питання 4.} Використовуємо наступний запит:
\begin{lstlisting}[language=SQL]
SELECT l_name, dpt, position
FROM employee
WHERE salary > 9000
ORDER BY l_name;
\end{lstlisting}

Результат виконання запиту:
\begin{table}[H]
    \centering
    \begin{tabular}{|c|c|c|}
        \hline
        \texttt{l\_name} & \texttt{dpt} & \texttt{position} \\
        \hline
        Galenda & IT & Programmer \\
        Mashko  & IT & Admin \\
        Osokor  & IT & QA \\
        Petrenko & IT & Programmer \\
        \hline
    \end{tabular}
\end{table}

\textbf{Питання 5.} Використовуємо наступний запит:
\begin{lstlisting}{language=SQL}
SELECT *
FROM employee
WHERE status = 'single' AND position != 'Admin';
\end{lstlisting}

Цей запис поверне усі колонки робітників Petrenko Zoya та Galenda Anna.

\textbf{Питання 6.} Використовуємо наступний запит:
\begin{lstlisting}[language=SQL]
SELECT DISTINCT dpt
FROM employee
WHERE position = 'Admin';
\end{lstlisting}

Результатом буде одна колонка з рядками \texttt{IT} та \texttt{Sales}.

\textbf{Питання 7.} \texttt{SELECT f\_name, l\_name, age FROM employee WHERE
status='single' ORDER BY 2 DESC} вибере імена, прізвища та вік усіх
співробітників, які не одружені, впорядковані за прізвищами (друга колонка) у
зворотньому порядку. Виведе наступне:
\begin{table}[H]
    \centering
    \begin{tabular}{|c|c|c|}
        \hline
        \texttt{f\_name} & \texttt{l\_name} & \texttt{age} \\
        \hline
        Zahar    & Vovk       & 32 \\
        Oleg     & Mashko     & 22 \\
        Oleg     & Kononov    & 26 \\
        Anna     & Galenda    & 26 \\
        \hline
    \end{tabular}
\end{table}

\textbf{Питання 8.} \texttt{SELECT l\_name, sex, status FROM employee ORDER BY
status, sex} вибере прізвища, стать та статус усіх співробітників, впорядкованих
спочатку за статусом, а потім за статтю. Результат:

\begin{table}[H]
    \centering
    \begin{tabular}{|c|c|c|}
        \hline
        \texttt{l\_name} & \texttt{sex} & \texttt{status} \\
        \hline
        Osokor & f &  married  \\
        Petrenko &  m &  married  \\            
        Kononov & f & single \\       
        Galenda & m & single \\           
        Vovk & m & single \\           
        Mashko & m & single \\
        \hline
    \end{tabular}
\end{table}

\textbf{Питання 9.} Використовуємо наступний запит:  
\begin{lstlisting}[language=SQL]
SELECT MAX(age) AS max_age
FROM employee
WHERE position = 'Admin';  
\end{lstlisting}

Максимальний вік адмінів дорівнює 32 роки (співробітник Zahar Vovk).

\textbf{Питання 10.} Використовуємо наступний запит:
\begin{lstlisting}[language=SQL]
SELECT *
FROM employee
WHERE salary > (SELECT AVG(salary) FROM employee);
\end{lstlisting}

На вихід буде виведено усі колонки співробітників Mashko Oleg, Galenda Anna, 
Osokor Galina.

\newpage

\section{Домашня Робота 3: Перетворення типів}

\begin{problem}
    Таблиця та сама, що і в попередньому завданні. Потрібно вивести результат
    наступних запитів:
    \begin{enumerate}
        \item \texttt{SELECT l\_name, f\_name, sex FROM employee WHERE dpt=0}.
        \item \texttt{SELECT l\_name, f\_name, sex FROM employee WHERE 'dpt'=0}.
        \item \texttt{SELECT f\_name, sex, age FROM employee WHERE salary>'10000'}.
        \item \texttt{SELECT CAST ('20160911' AS date)}.
        \item \texttt{SELECT l\_name, position, dpt FROM employee WHERE salary>CONCAT(age, '7000')}.
        \item \texttt{SELECT l\_name, f\_name, position FROM employee WHERE 'salary' >= 0}.
        \item Написати запит, який виведе прізвища та імена кожного працівника (одним полем), а
        також вік, стать, статус, посаду у порядку зростання віку.
    \end{enumerate}
\end{problem}

\textbf{Відповідь.}

\textbf{Питання 1.} Оскільки ми задали \texttt{dpt} як \texttt{VARCHAR}, то в більшості 
SQL-систем порівняння з нулем видасть помилку. Зокрема, мій компілятор видає наступну помилку:
\begin{center}
    \texttt{Conversion failed when converting the varchar value 'IT' to data type int.}
\end{center}

\textbf{Питання 2.} У цьому запиті, \texttt{dpt} не є посиланням на стовпець
таблиці, а просто текстовим рядком зі значенням \texttt{'dpt'}. SQL-системи
спробують порівняти текстовий рядок \texttt{'dpt'} із числом 0, що є несумісною
операцією, тому, зокрема, мій компілятор видає наступну помилку:
\begin{center}
    \texttt{Conversion failed when converting the varchar value 'dpt' to data type int.}
\end{center}

\textbf{Питання 3.} У цьому запиті, помилки не буде, і SQL системи переведуть 
текстовий рядок \texttt{'10000'} у число, щоб порівняти його з числом \texttt{salary}.
Таким чином, результатом буде:

\begin{table}[H]
    \centering
    \begin{tabular}{|c|c|c|}
        \hline
        \texttt{f\_name} & \texttt{sex} & \texttt{age} \\
        \hline
        Oleg & m & 22 \\
        Anna & m & 26 \\
        Galina & f & 18 \\
        \hline
    \end{tabular}
\end{table}

\textbf{Питання 4.} Команда \texttt{CAST} переведе текстовий рядок \texttt{'20160911'} у
тип \texttt{DATE}. Результатом буде дата \texttt{2016-09-11}, яка і виведеться на екран.

\textbf{Питання 5.} Команда спробує конкатенувати значення \texttt{age} та \texttt{'7000'} (наприклад, для 
співробітника Mashko Oleg, це буде \texttt{227000}), а потім порівняти результат з \texttt{salary}. В цьому 
випадку, конкатенація дасть у вихід строку з числом, тому кастінг спрацює --- програма не видасть помилку. 
Проте, через таку конкатенацію, число стає дуже великим і тому результатом буде пуста таблиця.

\textbf{Питання 6.} Тут все залежить від SQL системи. В двох різних компіляторах
(MYSQL), в моєму випадку, видається два різних результати:
\begin{itemize}
    \item \texttt{'salary'} інтерпретується як текстовий рядок, тому порівняння
    з нулем дасть помилку: \texttt{Conversion failed when converting the varchar
    value 'salary' to data type int.}
    \item \texttt{'salary'} інтерпретується як назва стовпця, тому порівняння
    з нулем буде виконано, але результатом буде уся таблиця.
\end{itemize}

Перший варіант виглядає більш логічно, а другий можливо спрацьовує через
особливості реалізації SQL системи (котра, наприклад, прибирає цю помилку і 
виводить warning).

\textbf{Питання 7.} Маємо наступний запит:
\begin{lstlisting}[language=SQL]
SELECT CONCAT(l_name, ' ', f_name) AS full_name, age, sex, status, position
FROM employee
ORDER BY age ASC;
\end{lstlisting}

\newpage

\section{Домашня Робота 4: Предикати}

\begin{problem}
    Таблиця та сама, що і в попередніх завданнях. За допомогою відомих предикатів:
    \begin{enumerate}
        \item Написати запит, щоб знайти всі зарплати між середньою та максимальною
        \item Написати запит, щоб знайти всі зарплати між мінімальною та середньою за
        департаментами
        \item Написати запит пошуку співробітника з ім'ям Oleg або Zahar
        \item Написати запит, що виводить список співробітників віком від 22 до 26 років.
        \item Порахувати кількість співробітників, які мають зарплату між 10000 і 20000
        \item Порахувати кількість неодружених працівників від мінімального до середнього
        віку
        \item Підрахувати кількість одружених співробітників на прізвище Petrenko
    \end{enumerate}
\end{problem}

\textbf{Відповідь.}

\textbf{Питання 1.} Маємо наступний запит:
\begin{lstlisting}[language=SQL]
SELECT salary
FROM employee
WHERE salary BETWEEN (SELECT AVG(salary) FROM employee) AND (SELECT MAX(salary) FROM employee);    
\end{lstlisting}

Тут, ми використали предикат \texttt{BETWEEN}, щоб вибрати всі зарплати, які
знаходяться між середньою та максимальною зарплатою. Результатом будуть значення 
\texttt{20000}, \texttt{25000}, \texttt{18000}. 

\textbf{Питання 2.} Маємо наступний запит:
\begin{lstlisting}[language=SQL]
SELECT dpt, salary
FROM employee e
WHERE salary BETWEEN 
        (SELECT MIN(salary) FROM employee WHERE dpt = e.dpt) AND
        (SELECT AVG(salary) FROM employee WHERE dpt = e.dpt);
\end{lstlisting}

Результатом буде наступна таблиця:

\begin{table}[H]
    \centering
    \begin{tabular}{|c|c|}
        \hline
        \texttt{dpt} & \texttt{salary} \\
        \hline
        IT & 10000 \\
        IT & 20000 \\
        Sales & 8000 \\
        \hline
    \end{tabular}
\end{table}

\textbf{Питання 3.} Маємо наступний запит:
\begin{lstlisting}[language=Python]
SELECT *
FROM employee
WHERE f_name = 'Oleg' OR f_name = 'Zahar';
\end{lstlisting}

\textbf{Питання 4.} Маємо запит:
\begin{lstlisting}[language=SQL]
SELECT *
FROM employee
WHERE age BETWEEN 22 AND 26;
\end{lstlisting}

\textbf{Питання 5.} Використовуємо запит:
\begin{lstlisting}[language=SQL]
SELECT COUNT(*) AS employee_count
FROM employee
WHERE salary BETWEEN 10000 AND 20000;
\end{lstlisting}

\textbf{Питання 6.} Тут команда дещо складніша:
\begin{lstlisting}[language=SQL]
SELECT COUNT(*) AS single_employee_count
FROM employee
WHERE status = 'single'
    AND age BETWEEN 
        (SELECT MIN(age) FROM employee) 
        AND (SELECT AVG(age) FROM employee);
\end{lstlisting}

Результатом буде просто \texttt{1}. Якщо замінити \texttt{COUNT(*)} на \texttt{*}, то
дізнаємось, що ця одна людина це Mashko Oleg.

\textbf{Питання 7.} Тут команда дещо простіша:
\begin{lstlisting}[language=SQL]
SELECT COUNT(*) AS married_petrenko_count
FROM employee
WHERE l_name = 'Petrenko' AND status = 'married';
\end{lstlisting}

\newpage

\section{Домашня Робота 5: Поєднання таблиць}

\begin{problem}
    Окрім таблиці \texttt{employee}, є ще таблиця \texttt{departments}:
    \begin{table}[H]
        \centering
        \begin{tabular}{|c|c|c|c|}
            \hline
            \texttt{dpt} & \texttt{nmb} & \texttt{rum\_nmb} & \texttt{tel} \\
            \hline
            IT & 3 & 401 & 3444191 \\
            QA & 2 & 402 & 3222311 \\
            Sales & 2 & 403 & 3222233 \\
            PM & 0 & 404 & 3224212 \\
            Design & 0 & 405 & 3224444 \\
            Accounting & 0 & 406 & 3221124 \\
            \hline
        \end{tabular}
    \end{table}
    Завдання --- написати запити (як з \texttt{JOIN}, так і без):
    \begin{enumerate}
        \item Вивести прізвища, імена та номери кімнат усіх співробітників.
        \item Вивести списки працівників усіх непустих кімнат по номерам кімнат.
        \item Вивести прізвища, імена, посади та номери кімнат всіх, хто працює у відділі Sales.
        \item Вивести список кімнат (у тому числі і порожніх) з працюючими в них співробітниками, їх
        посадами та телефонами
        \item Вивести прізвища, імена, вік, стать та зарплати співробітників, яких можна знайти за
        телефоном 3444191
        \item Вивести найбільш повну інформацію про співробітників на ім'я Ivan.
    \end{enumerate}
\end{problem}

\textbf{Відповідь.} Спочатку створимо таблицю \texttt{departments}:
\begin{lstlisting}[language=SQL]
-- Create the departments table
CREATE TABLE departments (
    dpt VARCHAR(50),
    nmb INT,
    rum_nmb INT,
    tel BIGINT
);

-- Insert data into the departments table
INSERT INTO departments (dpt, nmb, rum_nmb, tel) VALUES
('IT', 3, 401, 3444191),
('QA', 2, 402, 3222311),
('Sales', 2, 403, 3222233),
('PM', 0, 404, 3224212),
('Design', 0, 405, 3224444),
('Accounting', 0, 406, 3221124);
\end{lstlisting}

\textbf{Питання 1.} Маємо наступний запит:
\begin{lstlisting}[language=SQL]
SELECT e.l_name, e.f_name, d.rum_nmb
FROM employee e
JOIN departments d ON e.dpt = d.dpt;
\end{lstlisting}

Не використовуючи \texttt{JOIN}:
\begin{lstlisting}[language=SQL]
    SELECT l_name, f_name, (SELECT rum_nmb FROM departments WHERE dpt = employee.dpt) AS rum_nmb
    FROM employee;    
\end{lstlisting}

\textbf{Питання 2.} Перший варіант наводимо без команди \texttt{JOIN}:
\begin{lstlisting}[language=SQL]
SELECT e.l_name, e.f_name, 
(SELECT rum_nmb FROM departments WHERE dpt = e.dpt) AS rum_nmb
FROM employee e
WHERE (SELECT nmb FROM departments WHERE dpt = e.dpt) > 0
ORDER BY rum_nmb;
\end{lstlisting}

Далі, використовуючи \texttt{JOIN}:
\begin{lstlisting}[language=SQL]
SELECT d.rum_nmb, e.l_name, e.f_name
FROM employee e
JOIN departments d ON e.dpt = d.dpt
WHERE d.nmb > 0
ORDER BY d.rum_nmb;
\end{lstlisting}

\textbf{Питання 3.} Використовуємо \texttt{JOIN}:
\begin{lstlisting}[language=SQL]
SELECT e.l_name, e.f_name, e.position, d.rum_nmb
FROM employee e
JOIN departments d ON e.dpt = d.dpt
WHERE d.dpt = 'Sales';
\end{lstlisting}

Команда, де ми не використовуємо \texttt{JOIN}:
\begin{lstlisting}[language=SQL]
SELECT e.l_name, e.f_name, e.position,
(SELECT rum_nmb FROM departments WHERE dpt = e.dpt) AS rum_nmb
FROM employee e
WHERE e.dpt = 'Sales';
\end{lstlisting}

\textbf{Питання 4.} З командою \texttt{JOIN}:
\begin{lstlisting}[language=SQL]
SELECT d.rum_nmb, e.position, e.tel
FROM departments d
LEFT JOIN employee e ON d.dpt = e.dpt
ORDER BY d.rum_nmb;
\end{lstlisting}

Без команди \texttt{JOIN}:
\begin{lstlisting}[language=SQL]
SELECT d.rum_nmb,
    (SELECT position FROM employee WHERE d.dpt = employee.dpt LIMIT 1) AS position,
    (SELECT tel FROM employee WHERE d.dpt = employee.dpt LIMIT 1) AS tel
FROM departments d
ORDER BY d.rum_nmb;
\end{lstlisting}

Зауваження: тут ми використовуємо \texttt{LIMIT 1}, щоб вибрати лише одного 
співробітника з відділу, якщо їх кілька. 

\textbf{Питання 5.} Якщо використовувати \texttt{JOIN}:
\begin{lstlisting}[language=SQL]
SELECT e.l_name, e.f_name, e.age, e.sex, e.salary
FROM employee e
JOIN departments d ON e.dpt = d.dpt
WHERE d.tel = 3444191;
\end{lstlisting}

Якщо не використовувати, то маємо
\begin{lstlisting}[language=SQL]
SELECT e.l_name, e.f_name, e.age, e.sex, e.salary
FROM employee e
WHERE (SELECT tel FROM departments WHERE dpt = e.dpt) = 3444191;
\end{lstlisting}

\textbf{Питання 6.} Використовуючи \texttt{JOIN}:
\begin{lstlisting}[language=SQL]
SELECT e.l_name, e.f_name, e.age, e.sex, e.status, e.position, e.salary, d.rum_nmb, d.tel
FROM employee e
JOIN departments d ON e.dpt = d.dpt
WHERE e.f_name = 'Ivan';
\end{lstlisting}

Якщо не використовувати, то маємо
\begin{lstlisting}[language=SQL]
SELECT e.l_name, e.f_name, e.age, e.sex, e.status, e.position, e.salary,
    (SELECT rum_nmb FROM departments WHERE dpt = e.dpt) AS rum_nmb,
    (SELECT tel FROM departments WHERE dpt = e.dpt) AS tel
FROM employee e
WHERE e.f_name = 'Ivan';
\end{lstlisting}

\newpage

\section{Домашня Робота 6: Підзапити. Псевдоніми}

\begin{problem}
    Маємо дві таблиці з попереднього завдання --- \texttt{employee} та \texttt{departments}.
    За допомогою псевдонімів навести приклади для:
    \begin{enumerate}
        \item Використання псевдоніму для однієї таблиці.
        \item Використання псевдоніму для однієї колонки таблиці.
        \item Використання псевдоніму для результату агрегатної функції.
        \item Використання псевдоніму для результату підзапиту.
        \item Використання псевдонімів для таблиць та колонок при операції \texttt{JOIN} для 2х різних таблиць.
        \item Використання псевдонімів для поєднання таблиці з самою собою
    \end{enumerate}
\end{problem}

\textbf{Відповідь.}

\textbf{Питання 1.} Наприклад, можна скоротити назву таблиці \texttt{employees} до \texttt{e}:
\begin{lstlisting}[language=SQL]
SELECT e.l_name, e.f_name, e.age
FROM employee AS e
WHERE e.age > 25;
\end{lstlisting}

Результатом буде таблиця
\begin{table}[H]
    \centering
    \begin{tabular}{|c|c|c|}
        \hline
        \texttt{l\_name} & \texttt{f\_name} & \texttt{age} \\
        \hline
        Vovk & Zahar & 32 \\
        Kononov & Oleg & 26 \\
        Galenda	& Anna & 26 \\
        \hline
    \end{tabular}
\end{table}

\textbf{Питання 2.} Можна, наприклад, змінити назву колонок \texttt{l\_name}, \texttt{f\_name} та \texttt{salary}:
\begin{lstlisting}[language=SQL]
SELECT l_name AS LastName, f_name AS FirstName, salary AS MonthlySalary
FROM employee
WHERE salary > 15000;
\end{lstlisting}

Результатом буде таблиця
\begin{table}[H]
    \centering
    \begin{tabular}{|c|c|c|}
        \hline
        \texttt{LastName} & \texttt{FirstName} & \texttt{MonthlySalary} \\
        \hline
        Mashko & Oleg & 20000 \\
        Galenda & Anna & 25000 \\
        Osokor & Galina & 18000 \\
        \hline
    \end{tabular}
\end{table}

\textbf{Питання 3.} Наприклад, можна вивести середню, максимальну та мінімальну зарплату:
\begin{lstlisting}[language=SQL]
SELECT AVG(salary) AS AverageSalary, MAX(salary) AS MaxSalary, MIN(salary) AS MinSalary
FROM employee;
\end{lstlisting}

Результатом буде проста таблиця, де \texttt{AverageSalary=14916.66667}, \texttt{MaxSalary=25000}, \texttt{MinSalary=8000}.

\textbf{Питання 4.} Обчислимо середню зарплатню і виведемо список співробіників із зарплатою вище середньої:
\begin{lstlisting}[language=SQL]
SELECT l_name AS LastName, f_name AS FirstName, salary AS Salary, 
    (SELECT AVG(salary) FROM employee) AS AverageSalary
FROM employee
WHERE salary > (SELECT AVG(salary) FROM employee);
\end{lstlisting}

Результатом буде таблиця

\begin{table}[H]
    \centering
    \begin{tabular}{|c|c|c|c|}
        \hline
        \texttt{LastName} & \texttt{FirstName} & \texttt{Salary} & \texttt{AverageSalary} \\
        \hline
        Mashko & Oleg & 20000 & 14916 \\
        Galenda & Anna & 25000 & 14916 \\
        Osokor & Galina & 18000 & 14916 \\
        \hline
    \end{tabular}
\end{table}

\textbf{Питання 5.} Виведемо прізвища, імена, зарплати співробітників, номер їх 
кімнат та телефони, використовуючи псевдоніми для таблиць та колонок:
\begin{lstlisting}[language=SQL]
SELECT 
    e.l_name AS LastName, 
    e.f_name AS FirstName, 
    e.salary AS Salary, 
    d.rum_nmb AS RoomNumber, 
    d.tel AS Phone
FROM employee AS e
JOIN departments AS d ON e.dpt = d.dpt;
\end{lstlisting}

Результатом буде таблиця
\begin{table}[H]
    \centering
    \begin{tabular}{|c|c|c|c|c|}
        \hline
        \texttt{LastName} & \texttt{FirstName} & \texttt{Salary} & \texttt{RoomNumber} & \texttt{Phone} \\
        \hline
        Osokor & Galina & 18000 & 401 & 3444191 \\
        Galenda & Anna & 25000 & 401 & 3444191 \\
        Petrenko & Zoya & 10000 & 401 & 3444191 \\
        Mashko & Oleg & 20000 & 401 & 3444191 \\
        Kononov & Oleg & 8500 & 403 &3222233 \\
        Vovk & Zahar & 8000 & 403 & 3222233 \\
        \hline
    \end{tabular}
\end{table}

\textbf{Питання 6.} Візьмемо такий приклад: знайти пари співробітників, де один
заробляє більше іншого, і вивести лише їхні повні імена.
\begin{lstlisting}[language=SQL]
SELECT 
    CONCAT(e1.f_name, ' ', e1.l_name) AS Employee1_FullName, 
    CONCAT(e2.f_name, ' ', e2.l_name) AS Employee2_FullName
FROM employee AS e1
JOIN employee AS e2 ON e1.salary > e2.salary;
\end{lstlisting}

На виході маємо наступну таблицю:

\begin{table}[H]
    \centering
    \begin{tabular}{|c|c|}
        \hline
        \texttt{Employee1\_FullName} & \texttt{Employee2\_FullName} \\
        \hline
        Anna Galenda & Oleg Mashko \\
        Galina Osokor & Zoya Petrenko \\
        Anna Galenda & Zoya Petrenko \\
        Oleg Mashko & Zoya Petrenko\\
        Galina Osokor & Zahar Vovk\\
        Anna Galenda & Zahar Vovk\\
        Oleg Kononov & Zahar Vovk\\
        Zoya Petrenko & Zahar Vovk\\
        Oleg Mashko & Zahar Vovk\\
        Galina Osokor & Oleg Kononov\\
        Anna Galenda & Oleg Kononov\\
        Zoya Petrenko & Oleg Kononov\\
        Oleg Mashko & Oleg Kononov\\
        Anna Galenda & Galina Osokor\\
        Oleg Mashko & Galina Osokor\\
        \hline
    \end{tabular}
\end{table}

\newpage

\section{Домашня Робота 7: Регулярні вирази}

\begin{problem}
    Є таблиця \texttt{employee} з попередніх завдань. За допомогою регулярних виразів:
    \begin{enumerate}
        \item Вибрати прізвища та посади всіх, у кого у прізвищі, або в імені є nko.
        \item Вибрати імена, прізвища, посади співробітників, які мають у віці цифру 4 або 6.
        \item Вибрати прізвища та посади всіх співробітників із зарплатами рівно з трьома 0.
        \item Вивести список усіх співробітників з ім'ям або прізвищем на літеру O
        \item Вивести всі дані людини, у якої в даних статусу є літери g, e, але немає літери l
        \item Що виведе наступний запит
        \begin{lstlisting}[language=SQL]
SELECT fname, lname, age FROM employee WHERE fname LIKE `%k%`
        \end{lstlisting}
        \item Написати регулярний вираз для пошуку подвоєних приголосних у будь-якій кількості
        повторень (без прив'язки до таблиці)
        \item Написати регулярний вираз для пошуку \% в полі position, \% - це звичайний символ
        (використати escape конструкцію).
    \end{enumerate}
\end{problem}

\textbf{Відповідь.}

\textbf{Питання 1.} Маємо наступний запит:
\begin{lstlisting}[language=SQL]
SELECT l_name AS LastName, position AS Position
FROM employee
WHERE l_name LIKE '%nko%' OR f_name LIKE '%nko%';
\end{lstlisting}

Виведе лише один рядок: \texttt{Petrenko, Programmer}.

\textbf{Питання 2.} Маємо наступний запит:
\begin{lstlisting}[language=SQL]
SELECT f_name AS FirstName, l_name AS LastName, position AS Position
FROM employee
WHERE CAST(age AS CHAR) LIKE '%4%' OR CAST(age AS CHAR) LIKE '%6%';
\end{lstlisting}

На виході отримаємо три рядки:

\begin{table}[H]
    \centering
    \begin{tabular}{|c|c|c|}
        \hline
        \texttt{FirstName} & \texttt{LastName} & \texttt{Position} \\
        \hline
        Zoya & Petrenko & Programmer \\
        Anna & Galenda & Programmer \\
        Oleg & Kononov & Sales \\
        \hline
    \end{tabular}
\end{table}

\textbf{Питання 3.} Маємо наступний запит:
\begin{lstlisting}[language=SQL]
SELECT l_name AS LastName, position AS Position
FROM employee
WHERE CAST(salary AS CHAR) LIKE '%000';
\end{lstlisting}

На виході отримаємо п'ять рядків:

\begin{table}[H]
    \centering
    \begin{tabular}{|c|c|}
        \hline
        \texttt{LastName} & \texttt{Position} \\
        \hline
        Mashko & Admin \\
Petrenko & Programmer \\
Vovk & Admin \\
Galenda & Programmer \\
Osokor & QA \\
        \hline
    \end{tabular}
\end{table}

\textbf{Питання 4.} Маємо такий запит:
\begin{lstlisting}[language=SQL]
SELECT f_name AS FirstName, l_name AS LastName, position AS Position
FROM employee
WHERE f_name LIKE 'O%' OR l_name LIKE 'O%';
\end{lstlisting}

На виході отримаємо три рядки:

\begin{table}[H]
    \centering
    \begin{tabular}{|c|c|c|}
        \hline
        \texttt{FirstName} & \texttt{LastName} & \texttt{Position} \\
        \hline
        Oleg & Mashko & Admin \\
        Oleg & Kononov & Sales \\
        Galina & Osokor & QA \\
        \hline
    \end{tabular}
\end{table}

\textbf{Питання 5.} Для цього пункту маємо наступний запит:
\begin{lstlisting}[language=SQL]
SELECT *
FROM employee
WHERE status LIKE '%g%' 
    AND status LIKE '%e%' 
    AND status NOT LIKE '%l%';
\end{lstlisting}

\textbf{Питання 6.} Потрібно проаналізувати запит
\begin{lstlisting}[language=SQL]
SELECT fname, lname, age 
FROM employee 
WHERE fname LIKE `%k%`
\end{lstlisting}

Ця умова шукає всі записи, у яких ім'я (\texttt{f\_name}) містить літеру
\texttt{k} у будь-якому місці (на початку, всередині або в кінці). Символи \%
означають, що перед і після літери \texttt{k} можуть бути будь-які символи.
На виході буде пуста таблиця, бо в таблиці немає жодного співробітника з ім'ям,
яке містить літеру \texttt{k}.

\textbf{Питання 7.} Маємо такий вираз: \texttt{([bcdfghjklmnpqrstvwxyz]\{2,\})+}. По порядку:
\begin{itemize}
    \item \texttt{[bcdfghjklmnpqrstvwxyz]} --- всі малі приголосні латинського алфавіту. Можна врахувати і великі, тоді буде \texttt{[bcdfghjklmnpqrstvwxyzBCDFGHJKLMNPQRSTVWXYZ]}.
    \item \texttt{\{2,\}}: Це вказує, що потрібно знайти дві або більше однакових приголосних, що йдуть підряд.
    \item \texttt{+}: Квантифікатор \texttt{+} дозволяє знаходити повторення таких послідовностей приголосних. Тобто він шукає кілька таких подвоєних послідовностей приголосних.
\end{itemize}

\textbf{Питання 8.} В довільній таблиці можна використати вираз:
\begin{lstlisting}[language=SQL]
SELECT * 
FROM your_table
WHERE position REGEXP '\\%';
\end{lstlisting}

В SQL для правильного використання зворотного слешу в регулярному виразі
потрібно подвоїти зворотний слеш, тому що один зворотний слеш буде
екрануватися.

\newpage

\section{Домашня Робота 8: Створення, редагування, видалення}

\begin{problem}
    \begin{enumerate}
        \item Створіть бази даних \texttt{test1} і \texttt{test2}.
        \item Видаліть базу даних \texttt{test2}.
        \item Створіть у базі даних \texttt{test1} таблицю \texttt{table\_1} (з
        полями \texttt{id} --- \texttt{int}, \texttt{firstname} --- \texttt{varchar},
        \texttt{lastname} --- \texttt{varchar}, \texttt{position} --- \texttt{varchar},
        \texttt{department} --- \texttt{varchar}, \texttt{salary} --- \texttt{decimal},
        \texttt{age} --- \texttt{int}, \texttt{birthdate} --- \texttt{date}) та
        таблицю і \texttt{table\_2} (з полями
        \texttt{department} --- \texttt{varchar},
        \texttt{qtyemployees} --- \texttt{int}, \texttt{head} --- \texttt{varchar}).
        \item Перейменуйте таблицю \texttt{table\_1} на \texttt{employees}.
        \item Видаліть таблицю \texttt{table\_2}.
        \item Додайте в таблицю \texttt{employees} поле \texttt{status} ---
        \texttt{varchar(255)}.
        \item Видаліть із таблиці \texttt{employees} поле \texttt{age}.
        \item Перейменуйте в \texttt{employees} поле \texttt{firstname} на \texttt{name}.
        \item Змініть тип поля \texttt{age} з \texttt{int} на \texttt{varchar(50)}.
        \item Додайте кілька записів, а потім очистіть таблицю \texttt{employee}.
        \item Очистіть усі таблиці бази даних \texttt{test\_1}, потім видаліть її.
    \end{enumerate}
\end{problem}

\textbf{Відповідь.}

\textbf{Питання 1.} Створюємо бази данних
\begin{lstlisting}[language=SQL]
CREATE DATABASE test1;
CREATE DATABASE test2;
\end{lstlisting}

\textbf{Питання 2.} Видаляємо базу даних \texttt{test2}
\begin{lstlisting}[language=SQL]
DROP DATABASE test2;
\end{lstlisting}

\textbf{Питання 3.} Створюємо таблиці \texttt{table\_1} та \texttt{table\_2} у 
базі даних \texttt{test1}:
\begin{lstlisting}[language=SQL]
USE test1;

-- Creating table_1
CREATE TABLE table_1 (
    id INT,
    firstname VARCHAR(255),
    lastname VARCHAR(255),
    position VARCHAR(255),
    department VARCHAR(255),
    salary DECIMAL(10, 2),
    age INT,
    birthdate DATE
);

-- Creating table_2
CREATE TABLE table_2 (
    department VARCHAR(255),
    qtyemployees INT,
    head VARCHAR(255)
);
\end{lstlisting}

\textbf{Питання 4.} Перейменовуємо таблицю \texttt{table\_1} на \texttt{employees}:
\begin{lstlisting}[language=SQL]
RENAME TABLE table_1 TO employees;
\end{lstlisting}

\textbf{Питання 5.} Видаляємо таблицю \texttt{table\_2}:
\begin{lstlisting}[language=SQL]
DROP TABLE table_2;
\end{lstlisting}

\textbf{Питання 6.} Додаємо поле \texttt{status} у таблицю \texttt{employees}:
\begin{lstlisting}[language=SQL]
ALTER TABLE employees
ADD COLUMN status VARCHAR(255);
\end{lstlisting}

\textbf{Питання 7.} Видаляємо поле \texttt{age} з таблиці \texttt{employees}:
\begin{lstlisting}[language=SQL]
ALTER TABLE employees
DROP COLUMN age;
\end{lstlisting}

\textbf{Питання 8.} Перейменовуємо поле \texttt{firstname} на \texttt{name}:
\begin{lstlisting}[language=SQL]
ALTER TABLE employees
CHANGE COLUMN firstname name VARCHAR(255);
\end{lstlisting}

\textbf{Питання 9.} Оскільки поле \texttt{age} вже видалено, то наведемо приклад 
для зміни типу поля \texttt{salary} з \texttt{int} на \texttt{VARCHAR(50)}:
\begin{lstlisting}[language=SQL]
ALTER TABLE employees
MODIFY COLUMN salary VARCHAR(50);
\end{lstlisting}

\textbf{Питання 10.} Додаємо кілька записів у таблицю \texttt{employees}:
\begin{lstlisting}[language=SQL]
-- Додавання кількох записів
INSERT INTO employees (id, name, lastname, position, department, salary, birthdate, status)
VALUES 
(1, 'Dmytro', 'Zakharov', 'Student', 'Sales', 5000.00, '1985-06-15', 'Active'),
(2, 'Petya', 'Ugulkin', 'Developer', 'IT', 4500.00, '1990-02-20', 'Active'),
(3, 'Vasya', 'Pestolkin', 'Designer', 'Design', 4000.00, '1988-11-10', 'Inactive');

-- Clearing table employees
TRUNCATE TABLE employees;
\end{lstlisting}

\textbf{Питання 11.} Очищення усіх таблиць бази даних \texttt{test1} та видалення її:
\begin{lstlisting}[language=SQL]
USE test1;

-- Clearing all tables
TRUNCATE TABLE employees;

-- Dropping the database test1
DROP DATABASE test1;
\end{lstlisting}

\end{document}