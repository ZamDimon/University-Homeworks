\documentclass{hw_template}

\title{\huge\sffamily\bfseries Контрольна Робота з Еволюційних Систем \#2}
\author{\Large\sffamily Захаров Дмитро}
\date{\sffamily 15 листопада, 2024}

\begin{document}

\pagestyle{fancy}

\maketitle

\begin{center}
    \textbf{Варіант 5}
\end{center}

\tableofcontents

\pagebreak

\section{Завдання}

\begin{problem}
    Дано диференціально-алгебраїчне рівняння
    \begin{equation*}
        \boldsymbol{A}\frac{d\mathbf{x}}{dt} + \boldsymbol{B}\mathbf{x}(t) = \boldsymbol{f}(t), \quad t \geq 0, \quad \mathbf{x} \in \mathbb{C}^3
    \end{equation*}
    де $\boldsymbol{f}(t) = e^t \cdot \mathbf{1}_3$. Потрібно:
    \begin{enumerate}[(A)]
        \item Показати, що характеристичний жмуток матриць $\lambda\boldsymbol{A}+\boldsymbol{B}$ є регулярним і знайти його індекс.
        \item Обчислити спектральні проектори
        $\boldsymbol{P}_1,\boldsymbol{P}_2,\boldsymbol{Q}_1,\boldsymbol{Q}_2$
        жмутка матриць $\lambda\boldsymbol{A}+\boldsymbol{B}$ та побудувати
        характеристичну матрицю $\boldsymbol{G}$, знайти матриці $\boldsymbol{F}
        = \boldsymbol{G}^{-1}\boldsymbol{Q}_2\boldsymbol{A}$, $\boldsymbol{S} =
        -\boldsymbol{G}^{-1}\boldsymbol{Q}_1\boldsymbol{B}$ та матрицю $e^{\boldsymbol{S}t}$.
        \item Описати множину припустимих початкових даних $\mathbf{x}_0$, за
        яких початкова задача $\mathbf{x}(0) = \mathbf{x}_0$ для
        диференцiально-алгебраїчного рiвняння має єдиний розв’язок. Навести
        конкретний приклад такого $\mathbf{x}_0$.
    \end{enumerate}

    Матриці $\boldsymbol{A}$ та $\boldsymbol{B}$ наступні:
    \begin{equation*}
        \boldsymbol{A} = \begin{bmatrix}
            2 & 1 & 0 \\
            1 & 0 & 1 \\
            1 & 1 & -1
        \end{bmatrix}, \quad 
        \boldsymbol{B} = \begin{bmatrix}
            -2 & -2 & -1 \\
            1 & -1 & 0 \\
            0 & -1 & 2
        \end{bmatrix}
    \end{equation*}
\end{problem}

\subsection{Пункт (А)}
Для перевірки регулярності, обчислимо $\det(\boldsymbol{A} + \mu\boldsymbol{B})$. Маємо:
\begin{equation*}
    \det(\boldsymbol{A} + \mu\boldsymbol{B}) = \begin{bmatrix}
        2-2\mu & 1-2\mu & -\mu \\
        1+\mu & -\mu & 1 \\
        1 & 1-\mu & -1+2\mu
    \end{bmatrix} = 9\mu^2(-1+\mu)
\end{equation*}

Отже, жмуток є регулярним та має єдине власне значення $\lambda_0=1$. Далі знаходимо
\begin{equation*}
    \boldsymbol{R}_0(\mu) = (\boldsymbol{A}+\mu\boldsymbol{B})^{-1} = \frac{1}{9\mu^2(\mu-1)}\begin{bmatrix}
        -1+2\mu-2\mu^2 & 1-5\mu+5\mu^2 & 1 - 2\mu - \mu^2 \\
        2-\mu-2\mu^2 & -2+7\mu-4\mu^2 & -2+\mu-\mu^2 \\
        1 + \mu - \mu^2 & -1+2\mu-2\mu^2 & -1-\mu+4\mu^2
    \end{bmatrix}
\end{equation*}

Індекс $\text{ind}(\boldsymbol{A},\boldsymbol{B})$ жмутка матриць
$\lambda\boldsymbol{A}+\boldsymbol{B}$ це порядок полюса за $\mu=0$ матричної
функції $\boldsymbol{R}_0(\mu)$. Отже, потрібно подивитися на кожен елемент
$\boldsymbol{R}_0(\mu)$ та знайти його індекс. Бачимо, що для знаменника
$\mu=0$ є коренем другого порядка, а отже максимальний можливий індекс --- це 2.
Отже, достатньо знайти будь-який елемент матриці $\boldsymbol{R}_0(\mu)$, де 
відповідний чисельник не дорівнює нулю. Наприклад, перший елемент підходить:
\begin{equation*}
    \text{ind}\left(\frac{-1+2\mu-2\mu^2}{9\mu^2(\mu-1)}\right) = 2
\end{equation*}

Таким чином, жмуток є регулярним і має індекс 2.

\subsection{Пункт (Б)}
Спектральні проектори
$\boldsymbol{P}_1,\boldsymbol{P}_2,\boldsymbol{Q}_1,\boldsymbol{Q}_2$ та
характеристична матриця $\boldsymbol{G}$ жмутка матриць
$\lambda\boldsymbol{A}+\boldsymbol{B}$ обчислюються як:
\begin{align*}
    \boldsymbol{P}_2 &= \text{Res}_{\mu=0}\boldsymbol{R}_0(\mu)\boldsymbol{B} \\
    \boldsymbol{P}_1 &= \boldsymbol{E}_{3 \times 3} - \boldsymbol{P}_2 \\
    \boldsymbol{Q}_2 &= \text{Res}_{\mu=0}\boldsymbol{BR}_0(\mu) \\
    \boldsymbol{Q}_1 &= \boldsymbol{E}_{3 \times 3} - \boldsymbol{Q}_2 \\
    \boldsymbol{G} &= \boldsymbol{AP}_1 + \boldsymbol{BP}_2
\end{align*}

Отже, починаємо обраховувати поступово:
\begin{equation*}
    \boldsymbol{R}_0(\mu)\boldsymbol{B} = \begin{bmatrix}
        \frac{1-3\mu+3\mu^2}{3\mu^2(\mu-1)} & \frac{1}{3\mu(\mu-1)} & \frac{1-2\mu}{3\mu^2(\mu-1)} \\
        \frac{-2+3\mu}{3\mu^2(\mu-1)} & \frac{2-3\mu}{3\mu(1-\mu)} & \frac{-2+\mu}{3\mu^2(\mu-1)} \\
        \frac{1}{3\mu^2(1-\mu)} & \frac{1}{3\mu(1-\mu)} & \frac{1+\mu-3\mu^2}{3\mu^2(1-\mu)}
    \end{bmatrix}
\end{equation*}

Тепер обраховуємо лишки:
\begin{equation*}
    \boldsymbol{P}_2 = \text{Res}_{\mu=0}\boldsymbol{R}_0(\mu)\boldsymbol{B} = \begin{bmatrix}
        \frac{2}{3} & -\frac{1}{3} & \frac{1}{3} \\
        -\frac{1}{3} & \frac{2}{3} & \frac{1}{3} \\
        \frac{1}{3} & \frac{1}{3} & \frac{2}{3}
    \end{bmatrix}
\end{equation*}

Не будемо детально розписувати, як обраховується лишок у кожного елемента, але 
покажемо це на прикладі елемента на позиції $(1,1)$. Отже, маємо обрахувати
\begin{equation*}
    \text{Res}_{\mu=0}\frac{1-3\mu+3\mu^2}{3\mu^2(\mu-1)} 
    = \lim_{\mu \to 0} \frac{d}{d\mu}\left(\mu^2 \cdot \frac{1-3\mu+3\mu^2}{3\mu^2(\mu-1)}\right)
    = \lim_{\mu \to 0} \frac{2-6\mu+3\mu^2}{3(\mu-1)^2} = \frac{2}{3}
\end{equation*}

Далі, рахуємо матрицю $\boldsymbol{P}_1$:
\begin{equation*}
    \boldsymbol{P}_1 = \boldsymbol{E}_{3 \times 3} - \boldsymbol{P}_2 = 
    \frac{1}{3}\begin{bmatrix}
        1 & 1 & -1 \\
        1 & 1 & -1 \\
        -1 & -1 & 1
    \end{bmatrix}
\end{equation*}

Наступним кроком рахуємо $\boldsymbol{BR}_0(\mu)$:
\begin{equation*}
    \boldsymbol{BR}_0(\mu) = \begin{bmatrix}
        \frac{1+\mu-3\mu^2}{3\mu^2(1-\mu)} & \frac{1-2\mu}{3\mu^2(\mu-1)} & \frac{1+\mu}{3\mu^2(\mu-1)} \\
        \frac{1}{3\mu^2} & \frac{-1+3\mu}{3\mu^2} & -\frac{1}{3\mu^2} \\
        \frac{1}{3\mu(\mu-1)} & \frac{1}{3\mu(1-\mu)} & \frac{1-3\mu}{3\mu(1-\mu)}
    \end{bmatrix}
\end{equation*}

Тепер обраховуємо лишки і матриці $\boldsymbol{Q}_1$, $\boldsymbol{Q}_2$:
\begin{equation*}
    \boldsymbol{Q}_2 = \text{Res}_{\mu=0}\boldsymbol{BR}_0(\mu) = 
    \begin{bmatrix}
        \frac{2}{3} & \frac{1}{3} & -\frac{2}{3} \\
        0 & 1 & 0 \\
        -\frac{1}{3} & \frac{1}{3} & \frac{1}{3}
    \end{bmatrix}, \quad \boldsymbol{Q}_1 = \boldsymbol{E}_{3 \times 3} - 
    \boldsymbol{Q}_2 = \begin{bmatrix}
        \frac{1}{3} & -\frac{1}{3} & \frac{2}{3} \\
        0 & 0 & 0 \\
        \frac{1}{3} & -\frac{1}{3} & \frac{2}{3}
    \end{bmatrix}
\end{equation*}

Нарешті, характеристична матриця:
\begin{equation*}
    \boldsymbol{G} = \boldsymbol{AP}_1 + \boldsymbol{BP}_2 = \begin{bmatrix}
        0 & 0 & -3 \\
        1 & -1 & 0 \\
        2 & 1 & 0
    \end{bmatrix}, \quad \boldsymbol{G}^{-1} = \begin{bmatrix}
        0 & \frac{1}{3} & \frac{1}{3} \\
        0 & -\frac{2}{3} & \frac{1}{3} \\
        -\frac{1}{3} & 0 & 0
    \end{bmatrix}
\end{equation*}

Згідно умові, знайдемо матриці $\boldsymbol{F} =
\boldsymbol{G}^{-1}\boldsymbol{Q}_2\boldsymbol{A}$, $\boldsymbol{S} =
-\boldsymbol{G}^{-1}\boldsymbol{Q}_1\boldsymbol{B}$:
\begin{align*}
    \boldsymbol{F} &= \begin{bmatrix}
        \frac{1}{3} & 0 & \frac{1}{3} \\
        -\frac{2}{3} & 0 & -\frac{2}{3} \\
        -\frac{1}{3} & 0 & -\frac{1}{3}
    \end{bmatrix} \\
    \boldsymbol{S} &= \begin{bmatrix}
        \frac{1}{3} & \frac{1}{3} & -\frac{1}{3} \\
        \frac{1}{3} & \frac{1}{3} & -\frac{1}{3} \\
        -\frac{1}{3} & -\frac{1}{3} & \frac{1}{3}
    \end{bmatrix}
\end{align*}

Нарешті, залишається лише знайти $e^{\boldsymbol{S}t}$. Для цього знайдемо власні числа
матриці $\boldsymbol{S}$, знайшовши харастеристичний поліном:
\begin{equation*}
    \chi_S(\lambda) = \det(\boldsymbol{S} - \lambda\boldsymbol{E}_{3 \times 3}) =
    \lambda^2 - \lambda^3 = \lambda^2(1-\lambda)
\end{equation*}

Отже, маємо власне число $\lambda_{1,2}=0$ кратності $2$ та $\lambda_3=1$
кратності $1$. Отже, далі знаходимо відповідні власні вектори з рівняння
$\boldsymbol{S}\mathbf{v} = \mathbf{0}$ маємо $v_1+v_2-v_3=0$. Це відповідає 
множині векторів:
\begin{equation*}
    \mathbf{v} = \begin{bmatrix}
        \alpha \\
        \beta \\ 
        \alpha+\beta
    \end{bmatrix} = \begin{bmatrix}
        1 \\ 0 \\ 1
    \end{bmatrix}\alpha + \begin{bmatrix}
        0 \\ 1 \\ 1
    \end{bmatrix}\beta
\end{equation*}

Тому, оберемо наступні власні вектори:
\begin{equation*}
    \mathbf{v}_1 = \begin{bmatrix}
        1 \\ 0 \\ 1
    \end{bmatrix}, \quad \mathbf{v}_2 = \begin{bmatrix}
        0 \\ 1 \\ 1
    \end{bmatrix}
\end{equation*}

Тепер те саме, але для власного значення $\lambda_3=1$:
\begin{equation*}
    (\boldsymbol{S} - \boldsymbol{E}_{3 \times 3})\mathbf{v} = \mathbf{0}
\end{equation*}

Відповідна система рівнянь:
\begin{equation*}
    \begin{cases}
        -2v_1 + v_2 - v_3 = 0 \\
        v_1 - 2v_2 - v_3 = 0 \\
        v_1 + v_2 + 2v_3 = 0
    \end{cases}
\end{equation*}

Якщо вибрати $v_1 := \gamma$, то отримаємо $v_2=\gamma$, $v_3=-\gamma$. Тому,
в якості третього власного вектора обираємо:
\begin{equation*}
    \mathbf{v}_3 = \begin{bmatrix}
        1 \\ 1 \\ -1
    \end{bmatrix}
\end{equation*}

Таким чином, матриця $\boldsymbol{S}$ є діагоналізованою, а саме:
\begin{equation*}
    \boldsymbol{S} = \boldsymbol{U}\boldsymbol{\Lambda}\boldsymbol{U}^{-1}, \quad 
    \boldsymbol{U} = \begin{bmatrix}
        1 & 0 & 1 \\
        0 & 1 & 1 \\
        1 & 1 & -1
    \end{bmatrix}, \quad \Lambda = \text{diag}\{0,0,1\}.
\end{equation*}

Таким чином, експоненту можна знайти наступним чином:
\begin{equation*}
    e^{\boldsymbol{S}t} = \sum_{k=0}^{\infty} \frac{\boldsymbol{S}^kt}{k!} = \boldsymbol{U}\left(\sum_{k=0}^{\infty} \frac{\boldsymbol{\Lambda}^kt^k}{k!}\right)\boldsymbol{U}^{-1} = \boldsymbol{U}\begin{bmatrix}
        1 & 0 & 0 \\
        0 & 1 & 0 \\
        0 & 0 & e^{t}
    \end{bmatrix}\boldsymbol{U}^{-1} = \begin{bmatrix}
        \frac{1}{3}(2+e^t) & \frac{1}{3}(-1+e^t) & \frac{1}{3}(1-e^t) \\
        \frac{1}{3}(-1+e^t) & \frac{1}{3}(2+e^t) & \frac{1}{3}(1-e^t) \\
        \frac{1}{3}(1-e^t) & \frac{1}{3}(1-e^t) & \frac{1}{3}(2+e^t)
    \end{bmatrix}
\end{equation*}

\subsection{Пункт (В)}

Бачимо, що $\boldsymbol{F}^2=\boldsymbol{O}_{3 \times 3}$, а тому її індекс 
нільпотентності дорівнює $r=2$. Згідно теоремі з лекції, умова узгодження 
на початковий вектор $\mathbf{x}_0$ має вигляд:
\begin{equation*}
    \boldsymbol{P}_2\mathbf{x}_0 = \sum_{j=0}^{r-1}(-1)^j \frac{d^j}{dt^j}(\boldsymbol{F}^j \boldsymbol{G}^{-1}\boldsymbol{Q}_2f(t))\Big|_{t = 0}
\end{equation*}

В нашому конкретному випадку, вона запишеться як:
\begin{equation*}
    \boldsymbol{P}_2\mathbf{x}_0 = \boldsymbol{G}^{-1}\boldsymbol{Q}_2f(0) - \frac{d}{dt}(\boldsymbol{F}\boldsymbol{G}^{-1}\boldsymbol{Q}_2f(t))\Big|_{t = 0}
\end{equation*}

Тому обчислюємо усі вирази:
\begin{align*}
    \boldsymbol{G}^{-1}\boldsymbol{Q}_2f(0) &= \begin{bmatrix}
        4/9 \\ -5/9 \\ -1/9
    \end{bmatrix} \\
    \frac{d}{dt}(\boldsymbol{F}\boldsymbol{G}^{-1}\boldsymbol{Q}_2f(t))\Big|_{t = 0} &= \begin{bmatrix}
        1/9 \\ -2/9 \\ -1/9
    \end{bmatrix}
\end{align*}

Таким чином,
\begin{equation*}
    \boldsymbol{P}_2\mathbf{x}_0 = \begin{bmatrix}
        1/3 \\ -1/3 \\ 0
    \end{bmatrix}
\end{equation*}

Нехай $\mathbf{x}_0 = (v,u,w)$. Тоді, умова на $\mathbf{x}_0$ має вигляд:
\begin{equation*}
    \begin{bmatrix}
        \frac{2}{3} & -\frac{1}{3} & \frac{1}{3} \\
        -\frac{1}{3} & \frac{2}{3} & \frac{1}{3} \\
        \frac{1}{3} & \frac{1}{3} & \frac{2}{3}
    \end{bmatrix}\begin{bmatrix}
        v \\ u \\ w
    \end{bmatrix} = \begin{bmatrix}
        1/3 \\ -1/3 \\ 0
    \end{bmatrix} \iff \begin{cases}
        2v-u+w = 1 \\
        -v+2u+w = -1 \\
        v + u + 2w = 0
    \end{cases}
\end{equation*}

Звідси отримуємо: $u=-\frac{2}{3}+v, w=\frac{1}{3}-v$. Отже, якщо 
ввести параметр $\alpha$, то множину $\mathbf{x}_0$ можна описати як:
\begin{equation*}
    \mathbf{x}_0 = \begin{bmatrix}
        0 \\
        -\frac{2}{3} \\
        \frac{1}{3}
    \end{bmatrix} + \begin{bmatrix}
        1 \\
        1 \\
        -1
    \end{bmatrix}\alpha
\end{equation*}

Прикладом може слугувати вектор $\mathbf{x}_0 = (0,-2/3,1/3)$. Більш конкретно, 
розв'язком задачі буде:
\begin{equation*}
    x(t)=\frac{1}{9}e^t(2t+9\alpha), \quad y(t) = \frac{1}{9}e^t(-6+2t+9\alpha), \quad 
    z(t) = -\frac{1}{9}e^t(-3+2t+9\alpha).
\end{equation*}

\end{document}