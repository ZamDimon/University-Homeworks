\documentclass{hw_template}

\title{\huge\sffamily\bfseries Контрольна Робота з Еволюційних Систем \#1}
\author{\Large\sffamily Захаров Дмитро}
\date{\sffamily 3 листопада, 2024}

\begin{document}

\pagestyle{fancy}

\maketitle

\begin{center}
    \textbf{Варіант 5}
\end{center}

\tableofcontents

\pagebreak

\section{Завдання 1}

\begin{problem}
    Знайти загальний розв'язок різницевого рівняння:
    \begin{equation*}
        x_{k+1} = \left(\frac{k+2}{k+1}\right)^3x_k - 3 \cdot \frac{(k+2)^2}{k+3}
    \end{equation*}
\end{problem}

\textbf{Розв'язання.} Маємо лінійне неоднорідне скалярне різницеве рівняння першого порядку
$x_{k+1} = \alpha_k x_k + f_k$, де в нашому випадку:
\begin{equation*}
    \alpha_k = \left(\frac{k+2}{k+1}\right)^3, \quad f_k = -3 \cdot \frac{(k+2)^2}{k+3}.
\end{equation*}

Спочатку розв'яжемо однорідне рівняння:
\begin{equation*}
    x_{k+1} = \left(\frac{k+2}{k+1}\right)^3x_k.
\end{equation*}

Нехай $x_0$ задане. Тоді, розв'язок однорідного рівняння має вигляд:
\begin{equation*}
    x_k = x_0 \cdot \prod_{j=0}^{k-1} \left(\frac{j+2}{j+1}\right)^3 = x_0\left(\frac{2^3}{1^3} \cdot \frac{3^3}{2^3} \cdot \dots \cdot \frac{(k+1)^3}{k^3}\right) = (k+1)^3x_0.
\end{equation*}

Скористаємося методом варіації сталих. Нехай $x_k = c_k\prod_{j=0}^{k-1}\alpha_j = (k+1)^3c_k$. Підставимо це у наше початкове рівняння:
\begin{equation*}
    (k+2)^3c_{k+1} = \left(\frac{k+2}{k+1}\right)^3 \cdot (k+1)^3c_k - 3 \cdot \frac{(k+2)^2}{k+3}.
\end{equation*}

Звідси маємо:
\begin{equation*}
    (k+2)^3c_{k+1} = (k+2)^3c_k - 3 \cdot \frac{(k+2)^2}{k+3} \Rightarrow c_{k+1} = c_k - \frac{3}{(k+3)(k+2)}. 
\end{equation*}

Отже, якщо $c_0$ задане, то розв'язок має вигляд:
\begin{equation*}
    c_k = c_0 - 3\sum_{j=0}^{k-1}\frac{1}{(j+3)(j+2)} = c_0 - 3\sum_{j=0}^{k-1}\left(\frac{1}{j+2} - \frac{1}{j+3}\right) = c_0-3\left(\frac{1}{2}-\frac{1}{k+2}\right) = c_0 - \frac{3k}{4+2k}
\end{equation*}

Отже, загальний розв'язок початкового рівняння:
\begin{equation*}
    \boxed{x_k = (k+1)^3\left(c_0 - \frac{3k}{4+2k}\right)}
\end{equation*}

\pagebreak

\section{Завдання 2}

\begin{problem}
    Знайти загальний розв'язок різницевого рівняння:
    \begin{equation*}
        x_{k+2} - x_{k+1} - 6x_k = (10k-3)3^k
    \end{equation*}
\end{problem}

\textbf{Розв'язання.} Маємо лінійне неоднорідне стаціонарне різницеве рівняння
другого порядку. Розглянемо однорідну частину:
\begin{equation*}
    x_{k+2} - x_{k+1} - 6x_k = 0.
\end{equation*}

Маємо характеристичне рівняння:
\begin{equation*}
    \lambda^2 - \lambda - 6 = 0 \Rightarrow \lambda_1 = 3, \lambda_2 = -2.
\end{equation*}

Отже, загальний розв'язок однорідного рівняння має вигляд:
\begin{equation*}
    x_k = c_1 \cdot (-2)^k + c_2 \cdot 3^k.
\end{equation*}

Тепер нам потрібно знайти частковий розв'язок. Оскільки права частина має вигляд $f_k = (10k-3)3^k$ та $\lambda=3$ є коренем характеристичного рівняння першого порядку, то частковий розв'язок має вигляд:
\begin{equation*}
    \widetilde{x}_k = k \cdot 3^k \cdot (\alpha k+\beta).
\end{equation*}

Підставляємо в рівняння:
\begin{equation*}
    (k+2) \cdot 3^{k+2} \cdot (\alpha(k+2)+\beta) - (k+1) \cdot 3^{k+1} \cdot (\alpha(k+1)+\beta) - 6k \cdot 3^k \cdot (\alpha k+\beta) = (10k-3)3^k \\
\end{equation*}
Після спрощень маємо:
\begin{align*}    
    &3^k \cdot 3\alpha(11+10k) + 15 \cdot 3^k \beta = 10k \cdot 3^k - 3 \cdot 3^k \\
    & 33\alpha \cdot 3^k + 30\alpha k 3^k + 15 \cdot 3^k \beta = 10k \cdot 3^k - 3 \cdot 3^k \\
    & \textcolor{red}{(33\alpha + 15\beta)}3^k + \textcolor{blue}{30\alpha} \cdot k \cdot 3^k = \textcolor{red}{-3} \cdot 3^k + \textcolor{blue}{10} \cdot k \cdot 3^k
\end{align*}
Звідси маємо $30\alpha = 10 \implies \alpha = \frac{1}{3}$ і $33 \cdot \frac{1}{3} + 15\beta = -3$, звідки $\beta = -\frac{14}{15}$. Отже:
\begin{equation*}
    \widetilde{x}_k = k \cdot 3^k \cdot \left(\frac{1}{3} k - \frac{14}{15}\right)
\end{equation*}

Таким чином, загальний розв'язок рівняння має вигляд:
\begin{equation*}
    \boxed{x_k = c_1 \cdot (-2)^k + c_2 \cdot 3^k + k \cdot 3^k \cdot \left(\frac{1}{3} k - \frac{14}{15}\right)}
\end{equation*}

\pagebreak

\section{Завдання 3}

\begin{problem}
    Знайти загальний розв'язок системи різницевих рівнянь:
    \begin{equation*}
        \begin{cases}
            x_{k+1} = -8x_k + 2y_k + 2k + 29 \\
            y_{k+1} = -15x_k + 3y_k + 10 \cdot 3^k
        \end{cases}
    \end{equation*}
\end{problem}

\textbf{Розв'язання.} Маємо стаціонарну нормальну систему двох лінійних
різницевих рівнянь. Нам потрібно звести її до лінійного стаціонарного рівняння
відносно $x_k$. Для цього зробимо наступні перетворення:
\begin{align*}
    x_{k+2} &= -8x_{k+1} + 2\textcolor{blue}{y_{k+1}} + 2k + 31 \\
    &= -8x_{k+1} - 30x_k + 6\textcolor{blue}{y_k} + 20 \cdot 3^k + 2k + 31 \\
    &= -8x_{k+1} - 30x_k + 3(x_{k+1}+8x_k - 2k - 29) + 20 \cdot 3^k + 2k + 31 \\
    &= -5x_{k+1} - 6x_k - 4k - 56 + 20 \cdot 3^k
\end{align*}

Отже, остаточно маємо рівняння:
\begin{equation*}
    x_{k+2} + 5x_{k+1} + 6x_k = -4k - 56 + 20 \cdot 3^k
\end{equation*}

Розв'яжемо це рівняння. Маємо характеристичне рівняння:
\begin{equation*}
    \lambda^2 + 5\lambda + 6 = 0 \Rightarrow \lambda_1 = -2, \lambda_2 = -3.
\end{equation*}

Отже загальний розв'язок однорідного рівняння має вигляд:
\begin{equation*}
    x_k = c_1 \cdot (-2)^k + c_2 \cdot (-3)^k.
\end{equation*}

Тепер знайдемо частковий розв'язок. Права частина має вигляд $f_k = -4k - 56 + 20 \cdot 3^k$. Нехай права частина це лише $20 \cdot 3^k$, тоді частковий розв'язок шукатимемо у вигляді $\alpha \cdot 3^k$ і тому:
\begin{equation*}
    9\alpha \cdot 3^k + 5 \cdot 3\alpha \cdot 3^k + 6\alpha \cdot 3^k = 20 \cdot 3^k \Rightarrow \alpha = \frac{2}{3}
\end{equation*}

Якщо ж права частина лише $-4k-56$, то розв'язок шукаємо у вигляді $\beta k + \gamma$ і тому:
\begin{equation*}
    (7+12k)\beta + 12\gamma = -4k - 56 \Rightarrow \textcolor{red}{12\beta}k + \textcolor{blue}{7\beta + 12\gamma} = -4k-56
\end{equation*}

Звідси $\beta=-\frac{1}{3}$ та $\gamma = -\frac{161}{36}$. Отже, частковий розв'язок має вигляд:
\begin{equation*}
    \widetilde{x}_k = \frac{2}{3} \cdot 3^k - \frac{1}{3}k - \frac{161}{36}
\end{equation*}

Отже, загальний розв'язок системи має вигляд:   
\begin{equation*}
    \boxed{x_k = c_1 \cdot (-2)^k + c_2 \cdot (-3)^k + \frac{2}{3} \cdot 3^k - \frac{1}{3}k - \frac{161}{36}}
\end{equation*}

Тепер підставляємо це у перше рівняння системи та знаходимо $y_k$:
\begin{align*}
    y_k &= \frac{1}{2}\left(x_{k+1}+8x_k-2k-29\right) \\
    &= \boxed{3(-2)^k c_1 + \frac{5}{2}(-3)^k c_2 - \frac{835}{24} + \frac{11}{3} \cdot 3^k - \frac{5}{2}k}
\end{align*}

\pagebreak

\section{Завдання 4}

\begin{problem}
    Знайти загальний розв'язок системи різницевих рівнянь:
    \begin{equation*}
        \begin{cases}
            x_{k+1} = 3x_k + 2y_k + 2z_k \\
            y_{k+1} = 2x_k + 3y_k + 2z_k \\
            z_{k+1} = 2x_k + 2y_k + 3z_k
        \end{cases}
    \end{equation*}
\end{problem}

\textbf{Розв'язання.} Маємо лінійну однорідну стаціонарну еволюційну систему
різницевих рівнянь:
\begin{equation*}
    \boldsymbol{s}_{k+1} = \boldsymbol{A}\boldsymbol{s}_k, \quad \boldsymbol{s}_k = \begin{bmatrix} x_k \\ y_k \\ z_k \end{bmatrix}, \quad \boldsymbol{A} = \begin{pmatrix} 3 & 2 & 2 \\ 2 & 3 & 2 \\ 2 & 2 & 3 \end{pmatrix}
\end{equation*}

Знайдемо власні числа матриці $\boldsymbol{A}$. Маємо характеристичне рівняння:
\begin{equation*}
    \det \begin{bmatrix}
        3-\lambda & 2 & 2 \\
        2 & 3-\lambda & 2 \\
        2 & 2 & 3-\lambda
    \end{bmatrix} = (7-\lambda)(\lambda-1)^2 = 0 \Rightarrow \lambda_1 = 7, \lambda_2 = 1,
\end{equation*}

де $\lambda_1$ має кратність 1, а $\lambda_2$ має кратність 2. Знайдемо власні вектори:
\begin{equation*}
    (\boldsymbol{A}-7\boldsymbol{E}_{3 \times 3})\boldsymbol{v} = 0 \Rightarrow \begin{cases}
        -4v_1 + 2v_2 + 2v_3 = 0 \\
        2v_1 - 4v_2 + 2v_3 = 0 \\
        2v_1 + 2v_2 - 4v_3 = 0
    \end{cases} \Rightarrow \begin{cases}
        2v_1 = v_2 + v_3 \\
        v_1 = 2v_2 - v_3 \\
        v_1 + v_2 = 2v_3
    \end{cases}
\end{equation*}

Розв'язок цього рівняння $v_1=v_2=v_3$, а отже відповідний власний вектор $(1,1,1)$. Далі, для $\lambda_2=1$ маємо:
\begin{equation*}
    (\boldsymbol{A}-\boldsymbol{E}_{3 \times 3})\boldsymbol{v} = 0 \Rightarrow \begin{cases}
        2v_1 + 2v_2 + 2v_3 = 0 \\
        2v_1 + 2v_2 + 2v_3 = 0 \\
        2v_1 + 2v_2 + 2v_3 = 0
    \end{cases} \Rightarrow v_1 + v_2 + v_3 = 0 \Rightarrow v_1 = -u-w, v_2 = u, v_3 = w
\end{equation*} 

Отже, маємо:
\begin{equation*}
    \begin{bmatrix}
        v_1 \\ v_2 \\ v_3
    \end{bmatrix} = u \begin{bmatrix}
        -1 \\ 1 \\ 0
    \end{bmatrix} + w \begin{bmatrix}
        -1 \\ 0 \\ 1
    \end{bmatrix}
\end{equation*}

Отже, ми отримали три власних вектори:
\begin{equation*}
    \begin{bmatrix}
        1 \\ 1 \\ 1
    \end{bmatrix}, \begin{bmatrix}
        -1 \\ 1 \\ 0
    \end{bmatrix}, \begin{bmatrix}
        -1 \\ 0 \\ 1
    \end{bmatrix}
\end{equation*}

Таким чином, матриця $\boldsymbol{A}$ діагоналізується наступний чином:
\begin{equation*}
    \boldsymbol{A} = \boldsymbol{S}\boldsymbol{J}\boldsymbol{S}^{-1}, \quad \boldsymbol{S} = \begin{bmatrix}
        -1 & -1 & 1 \\
        0 & 1 & 1 \\
        1 & 0 & 1
    \end{bmatrix}, \quad \boldsymbol{J} = \begin{bmatrix}
        1 & 0 & 0 \\
        0 & 1 & 0 \\
        0 & 0 & 7
    \end{bmatrix}, \quad \boldsymbol{S}^{-1} = \begin{bmatrix}
        -\frac{1}{3} & -\frac{1}{3} & \frac{2}{3} \\
        -\frac{1}{3} & \frac{2}{3} & -\frac{1}{3} \\
        \frac{1}{3} & \frac{1}{3} & \frac{1}{3}
    \end{bmatrix}
\end{equation*}

Отже, розв'язок можемо записати у вигляді:
\begin{equation*}
    \boldsymbol{s}_n = \boldsymbol{A}^n\boldsymbol{s}_0 = \boldsymbol{S}\boldsymbol{J}^n\boldsymbol{S}^{-1}\begin{bmatrix}
        x_0 \\ y_0 \\ z_0
    \end{bmatrix} = \begin{bmatrix}
        (7^n+2)x_0 + (7^n-1)(y_0+z_0) \\
        (7^n-1)x_0 + (7^n+2)y_0 + (7^n-1)z_0 \\
        (7^n-1)x_0 + (7^n-1)y_0 + (7^n+2)z_0
    \end{bmatrix}
\end{equation*}

Або, остаточно:
\begin{equation*}
    \begin{cases}
        x_n = (7^n+2)c_1 + (7^n-1)(c_2+c_3) \\
        y_n = (7^n-1)c_1 + (7^n+2)c_2 + (7^n-1)c_3 \\
        z_n = (7^n-1)c_1 + (7^n-1)c_2 + (7^n+2)c_3
    \end{cases}
\end{equation*}

\end{document}