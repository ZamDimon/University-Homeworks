\documentclass{hw_template}

\title{\huge\sffamily\bfseries Домашня Робота з Еволюційних Систем \#1}
\author{\Large\sffamily Захаров Дмитро}
\date{\sffamily 8 вересня, 2024}

\begin{document}

\pagestyle{fancy}

\maketitle

\tableofcontents

\pagebreak

\section{Різницеві рівняння}

\subsection{Вправа 1.}
\begin{problem}
    Розв'язати лінійне різницеве рівняння першого порядку (тобто знайти загальний розв'язок рівнянь):
    \begin{equation*}
        x_{k+1} - \frac{k+2}{k+1}\cdot x_k = \frac{2}{k+3}
    \end{equation*}
\end{problem}

\textbf{Розв'язання.} Маємо рівняння виду $x_{k+1} = a_k x_k + f_k$, де $a_k = \frac{k+2}{k+1}$ та $f_k = \frac{2}{k+3}$. Спочатку розглянемо однорідне рівняння:
\begin{equation*}
    x_{k+1} = \frac{k+2}{k+1}x_k,
\end{equation*}

розв'язок якого є, очевидно:
\begin{equation*}
    x_k = x_0 \cdot \prod_{j=0}^{k-1} \frac{j+2}{j+1} = x_0 \cdot \left(\frac{2}{1} \cdot \frac{3}{2} \cdot \ldots \cdot \frac{k+1}{k}\right) = (k+1)x_0
\end{equation*}

Тепер скористаємось методом варіації сталих. Нехай тепер $x_k = c_k\prod_{j=0}^{k-1}a_j = (k+1)c_k$. Підставимо це у наше початкове рівняння:
\begin{equation*}
    (k+2)c_{k+1} = \frac{k+2}{k+1} \cdot (k+1)c_k + \frac{2}{k+3}
\end{equation*}

Звідси маємо:
\begin{equation*}
    c_{k+1} = c_k + \frac{2}{(k+2)(k+3)}
\end{equation*}

Звідки залишається порахувати суму:
\begin{equation*}
    c_k = c_0 + \sum_{j=0}^{k-1} \frac{2}{(j+2)(j+3)} = c_0 + 2\sum_{j=0}^{k-1} \left(\frac{1}{j+2} - \frac{1}{j+3}\right) = c_0 + 2\left(\frac{1}{2} - \frac{1}{k+2}\right)
\end{equation*}

Якщо позначимо $\widetilde{c}_0 := c_0 + 1$, то остаточно $c_k = \widetilde{c}_0 - \frac{2}{k+2}$. Тоді:
\begin{equation*}
    x_k = (k+1)\left(\widetilde{c}_0 - \frac{2}{k+2}\right)
\end{equation*}

Вправи 2-3 розв'язуються аналогічно.

\pagebreak

\subsection{Вправа 4.}
\begin{problem}
    Розв'язати лінійне різницеве рівняння першого порядку (тобто знайти загальний розв'язок рівнянь):
    \begin{equation*}
        x_{k+1} = x_k + \frac{1}{(3k+4)(3k+1)}
    \end{equation*}
\end{problem}

\textbf{Розв'язання.} Достатньо одразу скористатися формулою:
\begin{equation*}
    x_k = x_0 + \sum_{j=0}^{k-1} \frac{1}{(3j+4)(3j+1)} = x_0 + \frac{1}{3}\sum_{j=0}^{k-1} \left(\frac{1}{3j+1} - \frac{1}{3j+4}\right) = x_0 + \frac{1}{3} - \frac{1}{3(3k+1)}
\end{equation*}

Якщо позначити $\widetilde{x}_0 := x_0 + \frac{1}{3}$, то отримаємо розв'язок:
\begin{equation*}
    x_k = \widetilde{x}_0 - \frac{1}{3(3k+1)}
\end{equation*}

\subsection{Вправа 5.}
\begin{problem}
    Розв'язати лінійне різницеве рівняння першого порядку (тобто знайти загальний розв'язок рівнянь):
    \begin{equation*}
        x_{k+1} = x_k + \frac{(k+1)^2}{(2k+3)(2k+1)}
    \end{equation*}
\end{problem}

\textbf{Розв'язання.} Достатньо одразу скористатися формулою:
\begin{equation*}
    x_k = x_0 + \sum_{j=0}^{k-1} \frac{(j+1)^2}{(2j+3)(2j+1)}
\end{equation*}

Тут вже суму обрахувати дещо складніше. Для цього спочатку поділимо чисельник на знаменник:
\begin{equation*}
    (j+1)^2 = \frac{1}{4}(2j+3)(2j+1) + \frac{1}{4}
\end{equation*}

Тому звідси маємо:
\begin{equation*}
    x_k = x_0 + \sum_{j=0}^{k-1} \frac{1}{4} + \frac{1}{4}\sum_{j=0}^{k-1} \frac{1}{4(2j+3)(2j+1)}
\end{equation*}

Очевидно, що $\sum_{j=0}^{k-1}\frac{1}{4} = \frac{k}{4}$, а другу суму рахуємо як зазвичай:
\begin{equation*}
    \sum_{j=0}^{k-1} \frac{1}{4(2j+3)(2j+1)} = \frac{1}{8}\sum_{j=0}^{k-1} \left(\frac{1}{2j+1} - \frac{1}{2j+3}\right) = \frac{1}{8}\left(1 - \frac{1}{2k+1}\right) = \frac{1}{8} - \frac{1}{8(2k+1)}
\end{equation*}

Таким чином $x_k = \widetilde{x}_0 + \frac{k}{4} - \frac{1}{8(2k+1)}$.

\section{Застосування різницевих рівнянь}

\subsection{Вправа 1.}
\begin{problem}
    Вкладник поклав деяку суму на депозит пiд 10\% вiдсоткiв рiчних. Через скiльки рокiв його дохiд збiльшиться удвiчi (без урахування жодних зовнiшнiх iнвестицiй та витрат)?
\end{problem}

\textbf{Розв'язання.} Нехай вкладник має $z_n$ грошей після $n$ років, де $z_0$ --- початкова сума. Тоді маємо рівняння $z_{n+1}=(1+r)z_n$ для $r=0.1$. Розв'язок цього рівняння $z_n = (1+r)^nz_0$. Нас цікавить таке мінімальне $n$, за яке $z_n \geq 2z_0$. Для цього достатньо розглянути рівняння $(1+r)^n=2$, звідки $n_{\min}=\lceil \log 2/\log (1+r) \rceil = 8$.

\subsection{Вправа 2.}\label{exercise:2}
\begin{problem}
    Вкладник поклав 100 у.о. на депозит пiд 10 вiдсоткiв рiчних. Через скiльки рокiв його дохiд збiльшиться вдвiчi, якщо кожен рiк вiн одержує додатково 10 у.о.?
\end{problem}

\textbf{Розв'язання.} Нехай вкладник має $z_n$ грошей після $n$ років, де $z_0=100$ --- початкова сума. Тоді маємо рівняння $z_{n+1}=(1+r)z_n + f$ для $r=0.1$, $f=10$. Його розв'язок:
\begin{equation*}
    z_n = (1+r)^nz_0 + f\sum_{j=0}^{n-1}(1+r)^j = (1+r)^nz_0 + \frac{(1+r)^{n}-1}{r}\cdot f
\end{equation*}

Нам потрібно знайти таке мінімальне $n$, за яке $(1+r)^nz_0 + \frac{f}{r}(1+r)^n - \frac{f}{r} \geq 2z_0$. І це рівняння ми навіть можемо достатньо явно розв'язати:
\begin{equation*}
    (1+r)^n\left(z_0 + \frac{f}{r}\right) \geq 2z_0 + \frac{f}{r} \implies n_{\min} = \left\lceil \log\left(\frac{2z_0r + f}{z_0r + f}\right) \big/ \log(1+r) \right\rceil = 5
\end{equation*}

\subsection{Вправа 3.}
\begin{problem}
    Вкладник поклав 100 у.о. на депозит пiд 5 вiдсоткiв рiчних. Яку суму вiн одержить через 5 рокiв, якщо зовнiшнi надходження складають 20 у.о. у першi три та 30 у.о. протягом останнiх двох рокiв?
\end{problem}

\textbf{Розв'язання.} Нехай вкладник має $z_n$ грошей після $n$ років, де $z_0=100$ --- початкова сума. Тоді маємо рівняння $z_{n+1}=(1+r)z_n + f_n$ для $r=0.05$, а надходження мають вигляд:
\begin{equation*}
    f_n = \begin{cases}
        20, & n=0,1,2, \\
        30, & n=3,4.
    \end{cases}
\end{equation*}

Далі залишається рахувати. Маємо $z_1=100 \cdot 1.05 + 20 = 125$. Далі $z_2 = 125 \cdot 1.05 + 20 = 151.25$. Далі $z_3 = 151.25 \cdot 1.05 + 20 \approx 178.81$. Далі $z_4 = 178.81 \cdot 1.05 + 30 \approx 217.75$. Нарешті, $z_5 = 217.75 \cdot 1.05 + 30 \approx 258.64$.

\subsection{Вправа 4.}
\begin{problem}
    Нехай чисельнiсть населення у 1970 роцi деякого мiста складала 50 тис. осiб, а у 1980 роцi --- 75 тис. осiб. Припускаючи, що чисельнiсть населення у кiнцi року пропорцiйна чисельностi населення на початку року зi сталим коефiцiєнтом пропорцiйностi, знайти, якою буде чисельнiсть населення мiста у 2000 роцi.
\end{problem}

\textbf{Розв'язання.} Нехай $z_n$ --- кількість населення у тисячах у рік, починаючи з 1970. Згідно умові маємо рівняння $z_{n+1}=\alpha z_n$ для деякого $\alpha \in \mathbb{R}$. Тоді $z_n = \alpha^n z_0$ --- розв'язок рівняння. За умовою $z_0=50$, а також ми знаємо, що $z_{10}=75$. Звідси $\alpha^{10} = \frac{75}{50}$. Нарешті, нас питають значення $z_{30}$. Згідно нашої формули $z_{30} = \alpha^{30}z_0 = \left(\frac{75}{50}\right)^3 \cdot 50 = 168.75$ (тисяч).

\subsection{Вправа 5.}
\begin{problem}
    Нехай чисельнiсть населення в теперешнiй рiк складає 600 тис. осiб. Припускаючи, що коефiцiєнт народжуваностi довiрнює 5 вiдсоткiв та смертностi 0.1 вiдсоток, з’ясувати, через скiльки рокiв чисельнiсть населення сягне 1 млн (мiграцiю не враховувати).
\end{problem}

\textbf{Розв'язання.} Нехай $z_n$ --- кількість населення у тисячах у рік, починаючи з теперішнього. Згідно умови маємо рівняння $z_{n+1}=(1+\beta-\delta)z_n$ де $\beta=0.05,\delta=0.001$. Тоді $z_n = (1.049)^n z_0$ --- розв'язок рівняння. Згідно умови $z_0=600$, а також ми хочемо знайти таке мінімальне $n$, за якого $z_{n} \geq 1000$. Звідси $(1.049)^{n} \geq \frac{1000}{600}$. Отже $n_{\min} = \left\lceil \log\left(\frac{1000}{600}\right) \big/ \log 1.049 \right\rceil = 11$.

\subsection{Вправа 6.}
\begin{problem}
    Нехай чисельнiсть населення в теперешнiй рiк складає 300 тис. осiб. Припускаючи, що коефiцiєнт народжуваностi довiрнює 5 вiдсоткiв та смертностi 1 вiдсоток, з’ясувати, через скiльки рокiв чисельнiсть населення сягне 1 млн, якщо кожен рiк населення за рахунок мiграцiї збiльшується на 1 тис.
\end{problem}

\textbf{Розв'язання.} Задача по суті така сама, як і Вправа \ref{exercise:2}, тільки тут параметри такі: $r=\beta-\delta=0.04$, $z_0=300$ (у тисячах людей), $f=1$ і замість $2z_0$ маємо $Z:=1000$. Тоді 
\begin{equation*}
    n_{\min} = \left\lceil \log\left(\frac{Zr + f}{z_0r + f}\right) \big/ \log(1+r) \right\rceil = 30
\end{equation*}

\subsection{Вправа 7.}
\begin{problem}
    Знайти $\lim_{t \to \infty}D(t)$, де $D(t)$ --- позначає кiлькiсть речовини препарату в органiзмi людини пiсля $t$-го застосування препарату зi сталою дозою $f(t) \equiv D_0$.
\end{problem}

\textbf{Розв'язання.} Маємо рівняння $D(t+1)=(1-p)D(t)+D_0$ для $p \in [0,1)$. Його розв'язок 
\begin{equation*}
    D(t) = (1-p)^tD_0 + \sum_{j=0}^{t-1}(1-p)^jD_0 = (1-p)^tD_0 + \frac{1-(1-p)^t}{p}D_0   
\end{equation*}

Звідси $\lim_{t \to \infty}D(t) = \frac{D_0}{p}$ оскільки $(1-p)^t \xrightarrow[t \to \infty]{} 0$.

\subsection{Вправа 8.}
\begin{problem}
    Нехай одноразово введено препарат в органiзм та кожної доби виводиться 0.5 вiдсоткiв речовини. Через скiдьки дiб органiзм буде позбавлений 50 вiдсоткiв речовини?
\end{problem}

\textbf{Розв'язання.} Нехай доза через $t$ діб є $D_t$. Тоді маємо рівняння $D_{t+1}=(1-p)D_t$ для $p=0.005$. Його розв'язок $D_t = (1-p)^tD_0$. Ми хочемо знайти мінімальне $t$ за яке $D_t \leq 0.5D_0$, отже розглядаємо рівняння $(1-p)^t = 0.5$. Звідси $t_{\min} = \left\lceil \log 0.5 \big/ \log (1-p) \right\rceil = 139$.

\subsection{Вправа 9.}
\begin{problem}
    Знайдiть розв’язок початкової задачi (1) (ханойськi вежi) в явному виглядi. Переконайтесь в тому, що цей розв’язок уявляє собою послiдовнiсть цiлих чисел.
\end{problem}

\textbf{Розв'язання.} Рівняння мало вигляд $r_t=1+2r_{t-1}$ для $r_1=1$. Його розв'язок:
\begin{equation*}
    r_t = 1 + 2r_{t-1} = 1 + 2(1 + 2r_{t-2}) = 1 + 2 + 2^2r_{t-2} = \ldots = 1 + 2 + 2^2 + \ldots + 2^{t-1} = 2^t - 1
\end{equation*}

Це очевидно цілі числа.

\end{document}