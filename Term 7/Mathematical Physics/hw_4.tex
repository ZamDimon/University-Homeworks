\documentclass{hw_template}

\title{\huge\sffamily\bfseries Домашня Робота з Рівнянь Математичної Фізики \#4}
\author{\Large\sffamily Захаров Дмитро}
\date{\sffamily 9 листопада, 2024}

\begin{document}

\pagestyle{fancy}

\maketitle

\tableofcontents

\pagebreak

\section{Домашня Робота}

\subsection{Номер 3.3 (1).}

\begin{problem}
    Розв'язати задачу для рівняння $\Delta u = 0, x_1 > 0, x_2 > 0$ з граничними
    умовами $u\Big|_{x_1=0} = a$, $u\Big|_{x_2 = 0} = b$.
\end{problem}

\textbf{Розв'язання.} Як було показано, функція Гріна для області $\Omega = \{(x_1,x_2) \in \mathbb{R}^2: x_1 > 0, x_2 > 0\}$ має вигляд:
\begin{align*}
    G(\mathbf{x},\boldsymbol{\xi}) &= \frac{1}{4\pi}\log ((x_1-\xi_1)^2+(x_2+\xi_2)^2) + \frac{1}{4\pi}\log((x_1+\xi_1)^2+(x_2-\xi_2)^2) \\
    & -\frac{1}{4\pi}\log ((x_1-\xi_1)^2 + (x_2-\xi_2)^2) - \frac{1}{4\pi}\log((x_1+\xi_1)^2+(x_2+\xi_2)^2)
\end{align*}

Таким чином, розв'язок нашої задачі можна записати у вигляді
\begin{equation*}
    u(\mathbf{x}) = \int_{\Omega} G(\mathbf{x},\boldsymbol{\xi})f(\boldsymbol{\xi})d\boldsymbol{\xi} - \int_{\partial\Omega} \frac{\partial G(\mathbf{x},\boldsymbol{\xi})}{\partial \boldsymbol{\nu}}g(\boldsymbol{\xi})dS_{\boldsymbol{\xi}}
\end{equation*}

В нашому випадку, $f(\boldsymbol{\xi}) = 0$, а межа $\partial\Omega = \partial\Omega_1 \cup \partial\Omega_2$ може бути поділена на дві
частини: $\partial \Omega_1 = \{ (x_1,x_2) \in \mathbb{R}^2: x_1 = 0, x_2 > 0\}$
та $\partial \Omega_2 = \{ (x_1,x_2) \in \mathbb{R}^2: x_2 = 0, x_1 > 0\}$. Тоді
розв'язок можна записати у вигляді:
\begin{align*}
    u(x_1,x_2) &= -\int_{\partial \Omega_1} \frac{\partial G(\mathbf{x},\boldsymbol{\xi})}{\partial \boldsymbol{\nu}}g(\boldsymbol{\xi})dS_{\boldsymbol{\xi}} -\int_{\partial \Omega_2} \frac{\partial G(\mathbf{x},\boldsymbol{\xi})}{\partial \boldsymbol{\nu}}g(\boldsymbol{\xi})dS_{\boldsymbol{\xi}} \\
    &= -a\int_{\partial \Omega_1} \frac{\partial G(\mathbf{x},\boldsymbol{\xi})}{\partial \boldsymbol{\nu}}dS_{\boldsymbol{\xi}} - b\int_{\partial \Omega_2} \frac{\partial G(\mathbf{x},\boldsymbol{\xi})}{\partial \boldsymbol{\nu}}dS_{\boldsymbol{\xi}}
\end{align*}

Для границі $\partial \Omega_1$ маємо вектор нормалі $\boldsymbol{\nu}_1 = (-1,0)$, тому нас цікавить
похідна функції Гріна за цим напрямком за умови $x_1 = 0$ (було виведено на лекції):
\begin{align*}
    \frac{\partial G(\mathbf{x}, \boldsymbol{\xi})}{\partial \boldsymbol{\nu}_1}\Big|_{x_1=0} &= -\frac{\partial G(\mathbf{x}, \boldsymbol{\xi})}{\partial \xi_1}\Big|_{x_1=0} = \frac{1}{\pi}\left(\frac{x_1}{x_1^2+(x_2+\xi_2)^2} - \frac{x_1}{x_1^2+(x_2-\xi_2)^2}\right) \\
    \frac{\partial G(\mathbf{x}, \boldsymbol{\xi})}{\partial \boldsymbol{\nu}_2}\Big|_{x_2=0} &= -\frac{\partial G(\mathbf{x}, \boldsymbol{\xi})}{\partial \xi_2}\Big|_{x_2=0} = \frac{1}{\pi}\left(\frac{x_2}{(x_1+\xi_1)^2+x_2^2} - \frac{x_2}{(x_1-\xi_1)^2+x_2^2}\right)
\end{align*}

Тепер можемо підставити ці вирази у нашу формулу для розв'язку:
\begin{align*}
    u(x_1,x_2) &= -a\int_{\partial \Omega_1} \frac{\partial G(\mathbf{x},\boldsymbol{\xi})}{\partial \boldsymbol{\nu}}dS_{\boldsymbol{\xi}} - b\int_{\partial \Omega_2} \frac{\partial G(\mathbf{x},\boldsymbol{\xi})}{\partial \boldsymbol{\nu}}dS_{\boldsymbol{\xi}} \\
    &= -a\int_{0}^{\infty} \frac{1}{\pi}\left(\frac{x_1}{x_1^2+(x_2+\xi_2)^2} - \frac{x_1}{x_1^2+(x_2-\xi_2)^2}\right)d\xi_2 \\
    & -b\int_{0}^{\infty} \frac{1}{\pi}\left(\frac{x_2}{(x_1+\xi_1)^2+x_2^2} - \frac{x_2}{(x_1-\xi_1)^2+x_2^2}\right)d\xi_1
\end{align*}

Трошки перетосуємо інтеграли:
\begin{align*}
    u(x_1,x_2) &= -\frac{ax_1}{\pi}\int_{0}^{\infty}\left(\frac{1}{x_1^2+(x_2+\xi_2)^2} - \frac{1}{x_1^2+(x_2-\xi_2)^2}\right)d\xi_2 \\
    &- \frac{bx_2}{\pi}\int_{0}^{\infty} \left(\frac{1}{(x_1+\xi_1)^2+x_2^2} - \frac{1}{(x_1-\xi_1)^2+x_2^2}\right)d\xi_1
\end{align*}

Далі достатньо знайти кожен інтеграл окремо. Проте, всі чотири є частковими випадками наступного інтегралу:
\begin{align*}
    \mathcal{J}(\alpha,\beta) = \int_0^{+\infty} \frac{d\zeta}{\alpha^2 + (\beta + \zeta)^2}
\end{align*}

Тоді, відповідь буде мати вигляд:
\begin{align*}
    u(x_1,x_2) &= -\frac{ax_1}{\pi}\left(\mathcal{J}(x_1,x_2) - \mathcal{J}(x_1,-x_2)\right) - \frac{bx_2}{\pi}\left(\mathcal{J}(x_2,x_1) - \mathcal{J}(x_2,-x_1)\right)
\end{align*}

Саму функцію $\mathcal{J}(\alpha,\beta)$ можна легко порахувати явно:
\begin{align*}
    \mathcal{J}(\alpha,\beta) &= \frac{1}{\alpha^2}\int_0^{+\infty} \frac{d\zeta}{1 + \left(\frac{\beta + \zeta}{\alpha}\right)^2} = \frac{1}{\alpha}\int_0^{+\infty} \frac{d\left(\frac{\beta + \zeta}{\alpha}\right)}{1 + \left(\frac{\beta + \zeta}{\alpha}\right)^2} \\
    &= \frac{1}{\alpha}\arctan \frac{\beta+\zeta}{\alpha}\Big|_{\zeta \to 0}^{\zeta \to +\infty} = \frac{\pi}{2\alpha} - \frac{\arctan (\beta/\alpha)}{\alpha}
\end{align*}

Таким чином:
\begin{align*}
    u(x_1,x_2) &= -\frac{ax_1}{\pi}\left(\frac{\pi}{2x_1} - \frac{\arctan(x_2/x_1)}{x_1} - \frac{\pi}{2x_1} + \frac{\arctan(-x_2/x_1)}{x_1}\right) \\
    &- \frac{bx_2}{\pi}\left(\frac{\pi}{2x_2} - \frac{\arctan(x_1/x_2)}{x_2} - \frac{\pi}{2x_2} + \frac{\arctan(-x_1/x_2)}{x_2}\right) \\
    &= -\frac{ax_1}{\pi} \cdot \left(-\frac{2\arctan (x_2/x_1)}{x_1}\right) - \frac{bx_2}{\pi} \cdot \left(-\frac{2\arctan (x_1/x_2)}{x_2}\right) \\
    &= \frac{2a}{\pi}\arctan \frac{x_2}{x_1} + \frac{2b}{\pi}\arctan \frac{x_1}{x_2}
\end{align*}

\textbf{Відповідь:} $u(x_1,x_2) = \frac{2a}{\pi}\arctan \frac{x_2}{x_1} + \frac{2b}{\pi}\arctan \frac{x_1}{x_2}$.

\newpage

\subsection{Номер 3.3 (3).}

\begin{problem}
    Розв'язати задачу для рівняння $\Delta u = 0, x_1 > 0, x_2 > 0$ з граничними
    умовами $u\Big|_{x_1=0} = 0$, $u\Big|_{x_2 = 0} = \theta(x_1-1)$.
\end{problem}

\textbf{Розв'язання.} Згідно попередньому прикладу, все зводиться до обчислення інтегралу:
\begin{align*}
    u(\mathbf{x}) &= -\int_{\partial\Omega} \frac{\partial G(\mathbf{x}, \boldsymbol{\xi})}{\partial \boldsymbol{\nu}}\phi(\boldsymbol{\xi})dS_{\xi} = -\int_{\partial\Omega_2} \frac{\partial G(\mathbf{x}, \boldsymbol{\xi})}{\partial \boldsymbol{\nu}_2}\theta(\xi_1-1)dS_{\xi} \\
    &= -\frac{x_2}{\pi}\int_{0}^{\infty} \left(\frac{1}{(x_1+\xi_1)^2+x_2^2} - \frac{1}{(x_1-\xi_1)^2+x_2^2}\right)\theta(\xi_1-1)d\xi_1 \\
    &= -\frac{x_2}{\pi}\int_{1}^{\infty} \left(\frac{1}{(x_1+\xi_1)^2+x_2^2} - \frac{1}{(x_1-\xi_1)^2+x_2^2}\right)d\xi_1 \\
    &= -\frac{x_2}{\pi}\left(\int_1^{+\infty} \frac{d\xi_1}{(x_1+\xi_1)^2+x_2^2} - \int_1^{+\infty} \frac{d\xi_1}{(x_1-\xi_1)^2+x_2^2}\right)
\end{align*}

Як і в минулому прикладі, позначимо
\begin{align*}
    \mathcal{J}(\alpha,\beta) = \int_1^{+\infty} \frac{d\zeta}{\alpha^2 + (\beta + \zeta)^2} = \frac{1}{\alpha}\arctan \frac{\beta+\zeta}{\alpha}\Big|_{\zeta \to 1}^{\zeta \to +\infty} = \frac{\pi}{2\alpha} - \frac{1}{\alpha}\arctan \frac{\beta+1}{\alpha}
\end{align*}

Тоді, 
\begin{align*}
    u(x_1,x_2) &= -\frac{x_2}{\pi}\left(\mathcal{J}(x_2,x_1) - \mathcal{J}(x_2,-x_1)\right) \\
    &= -\frac{x_2}{\pi}\left(-\frac{1}{x_2}\arctan \frac{x_1+1}{x_2} + \frac{1}{x_2}\arctan \frac{-x_1+1}{x_2} \right) \\
    &= \frac{1}{\pi}\left(\arctan \frac{x_1-1}{x_2} + \arctan \frac{x_1+1}{x_2}\right)
\end{align*}

\textbf{Відповідь.} $u(x_1,x_2) = \frac{1}{\pi}\left(\arctan \frac{x_1-1}{x_2} + \arctan \frac{x_1+1}{x_2}\right)$.

\end{document}