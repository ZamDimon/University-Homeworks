\documentclass{hw_template}

\title{\huge\sffamily\bfseries Домашня Робота з Рівнянь Математичної Фізики \#5}
\author{\Large\sffamily Захаров Дмитро}
\date{\sffamily 23 листопада, 2024}

\begin{document}

\pagestyle{fancy}

\maketitle

\tableofcontents

\pagebreak

\section{Домашня Робота}

\subsection{Номер 11.5.}

\begin{problem}
    Розв'язати задачу $-\Delta u = 0$, $r<3$ за $u\Big|_{r=3}=1+\cos\varphi$.
\end{problem}

\textbf{Розв'язання.} Шукаємо розв'язок у вигляді:
\begin{equation*}
    u(r,\varphi) = C + \sum_{n=1}^{\infty}r^n(A_n\cos n\varphi + B_n \sin n\varphi).
\end{equation*}

Маємо, що $u(3,\varphi) = 1+\cos\varphi$. Таким чином,
\begin{equation*}
    u(3,\varphi) = C+\sum_{n=1}^{\infty}3^n(A_n\cos n\varphi + B_n \sin n\varphi) = 1+\cos\varphi.
\end{equation*}

Звідси видно, що $B_n \equiv 0$, $C=1$, а $3A_1=1$ ($A_n=0$ для $n>1$). Тому:
\begin{equation*}
    u(r,\varphi) = 1 + \frac{r}{3}\cos\varphi.
\end{equation*}

\textbf{Відповідь.} $u(r,\varphi) = 1 + \frac{r}{3}\cos\varphi$.

\newpage

\subsection{Номер 11.9.}

\begin{problem}
    Розв'язати задачу $-\Delta u = 1+\frac{1}{r}$, $2<r<3$ за $u\Big|_{r=2}=u\Big|_{r=3}=0$.
\end{problem}

\textbf{Розв'язання.} Шукаємо розв'язок у вигляді:
\begin{equation*}
    u(r,\varphi) = \sum_{n=0}^{\infty}(A_n(r)\cos n\varphi + B_n(r)\sin n\varphi).
\end{equation*}

Знайдемо Лапласіан:
\begin{equation*}
    -\Delta u(r,\varphi) = \sum_{n=0}^{\infty}(F_n(r)\cos n\varphi + G_n(r)\sin n\varphi),
\end{equation*}

де 
\begin{equation*}
    F_n(r) = -A_n''(r)-\frac{1}{r}A_n'(r)+\frac{n^2}{r^2}A_n(r), \quad G_n(r) = -B_n''(r)-\frac{1}{r}B_n'(r)+\frac{n^2}{r^2}B_n(r).
\end{equation*}

За умовою, отриманий вираз дорівнює $1+\frac{1}{r}$. В такому разі, легко бачити, що $F_n(r)=G_n(r)=0$ для всіх $n \geq 1$. Таким чином, маємо:
\begin{equation*}
    -A_0''(r)-\frac{1}{r}A_0'(r) = 1 + \frac{1}{r}, \quad -B_0''(r) - \frac{1}{r}B_0'(r) = 0.
\end{equation*} 

Додатково, маємо наступні граничні умови:
\begin{align*}
    u(2,\varphi) &= \sum_{n=0}^{\infty}(A_n(2)\cos n\varphi + B_n(2)\sin n\varphi) = 0, \\
    u(3,\varphi) &= \sum_{n=0}^{\infty}(A_n(3)\cos n\varphi + B_n(3)\sin n\varphi) = 0.
\end{align*}

Звідси видно, що $A_n(2) = A_n(3) = B_n(2) = B_n(3) = 0$ для всіх $n \geq 0$. Таким чином, маємо:
\begin{align*}
    &-A_0''(r)-\frac{1}{r}A_0'(r) = 1 + \frac{1}{r}, && A_0(2) = A_0(3) = 0, \\
    &-A_n''(r) - \frac{1}{r}A_n'(r)+\frac{n^2}{r^2}A_n(r) = 0, && A_n(2) = A_n(3) = 0, \quad n \geq 1, \\
    &-B_n''(r) - \frac{1}{r}B_n'(r)+\frac{n^2}{r^2}B_n(r) = 0, && B_n(2) = B_n(3) = 0, \quad n \geq 0 \\
\end{align*}

Друге та третє рівняння дають розв'язок $A_n \equiv 0$ та $B_n \equiv 0$ для
всіх $n \geq 0$ (окрім $n=0$ для другого рівняння). Отже, нас цікавить лише 
перше рівняння. Маємо рівняння:
\begin{equation*}
    -A_0''(r)-\frac{1}{r}A_0'(r) = 1 + \frac{1}{r}
\end{equation*}

Позначимо $q(r) := -A_0'(r)$, тоді $q'(r)=-A_0''(r)$ і рівняння зведеться 
до звичайного диференціального рівняння першого порядку:
\begin{equation*}
    q'(r) + \frac{1}{r}q(r) = 1 + \frac{1}{r} 
\end{equation*}

Маємо неоднорідне лінійне диференціальне рівняння першого порядку. Розв'язуємо однорідну частину:
\begin{equation*}
    \widetilde{q}'(r) + \frac{1}{r}\widetilde{q}(r) = 0 \Rightarrow \frac{d\widetilde{q}}{dr} = -\frac{\widetilde{q}}{r} \Rightarrow \widetilde{q}(r) = \frac{\alpha}{r}, \; \alpha \in \mathbb{R}
\end{equation*}

Отже, загальний розв'язок шукаємо у вигляді $q(r) = \alpha(r)/r$. Підставляємо в неоднорідне рівняння:
\begin{equation*}
    -\frac{1}{r^2}\alpha(r) + \frac{1}{r}\frac{d\alpha}{dr} + \frac{1}{r^2}\alpha(r) = 1 + \frac{1}{r} \Rightarrow \frac{1}{r}\frac{d\alpha}{dr} = 1+\frac{1}{r}
\end{equation*}

Отримане рівняння розв'язати вже зовсім легко: $\alpha'(r) = r+1$, а отже $\alpha(r) = \gamma+r+\frac{1}{2}r^2$ для $\gamma \in \mathbb{R}$. Таким чином, загальний розв'язок:
\begin{equation*}
    q(r) = \frac{\gamma+r+\frac{1}{2}r^2}{r} = 1 + \frac{\gamma}{r} + \frac{r}{2}
\end{equation*}

Далі, можемо знайти $A_0(r)$ зі співвідношення $q(r) = -A_0'(r)$:
\begin{equation*}
    \frac{dA_0}{dr} = -1 - \frac{\gamma}{r} - \frac{r}{2} \Rightarrow A_0(r) = -r - \frac{r^2}{4} + \delta + \gamma\log r, \quad \gamma, \delta \in \mathbb{R}
\end{equation*}

Користаємось початковою умовою, що $A_0(2) = A_0(3) = 0$:
\begin{equation*}
    \begin{cases}
        -2 - 1 + \delta + \gamma \log 2 = 0, \\
        -3 - \frac{9}{4} + \delta + \gamma \log 3 = 0.
    \end{cases} \iff \quad \begin{cases}
        \delta + \gamma \log 2 = 3, \\
        \delta + \gamma \log 3 = \frac{21}{4}.
    \end{cases}
\end{equation*}

Віднявши друге рівняння від першого, маємо $\gamma \log \frac{3}{2} =
\frac{9}{4}$, звідки $\gamma = \frac{9}{4\log \frac{3}{2}}$. Тоді, скажімо, з
першого рівняння, маємо $\delta = 3 - \frac{9 \log 2}{4\log \frac{3}{2}}$. Отже,
\begin{equation*}
    u(r,\varphi) = u(r) = -r - \frac{r^2}{4} + \frac{9}{4\log \frac{3}{2}}\log r + 3 - \frac{9 \log 2}{4\log \frac{3}{2}}.
\end{equation*}

\end{document}