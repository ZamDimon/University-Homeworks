\documentclass{hw_template}

\title{\huge\sffamily\bfseries Домашня Робота з Рівнянь Математичної Фізики \#3}
\author{\Large\sffamily Захаров Дмитро}
\date{\sffamily 26 жовтня, 2024}

\begin{document}

\pagestyle{fancy}

\maketitle

\tableofcontents

\pagebreak

\section{Домашня Робота}

\subsection{Номер 3.1 (4).}

\begin{problem}
    Знайти функцію Гріна методом конформних відображень для області:
    \begin{equation*}
        \Omega = \{(x_1,x_2) \in \mathbb{R}^2: x_1^2 + x_2^2 < R^2, x_2 > 0\}
    \end{equation*}
\end{problem}

\textbf{Розв'язання.} Метод конформних відображень полягає у тому, що якщо конформне відображення $w: \Omega \to \mathbb{C}$ відображає область $\Omega$ у одиничне коло $|z|<1$, то функція Гріна для області $\Omega$ може бути знайдена у вигляді:
\begin{equation*}
    G(\mathbf{x},\boldsymbol{\xi}) = \frac{1}{2\pi}\log\left|\frac{1-w(z)\overline{w(\zeta)}}{w(z)-w(\zeta)}\right|, \; \text{де} \; z = x_1+ix_2, \; \zeta = \xi_1+i\xi_2.
\end{equation*}

Отже, наша ціль --- це побудувати $w(z)$, що переведе верхню половину півкулі на одиничне коло. Це можна зробити за допомогою чотирьох відображень:
\begin{enumerate}
    \item $w_1: z \mapsto z/R$ відображає верхнє півколо радіусу $R$ у одиничне півколо радіусу $R$;
    \item $w_2: z \mapsto (1+z)/(1-z)$ відобразить верхню півкулю у перший квадрант;
    \item $w_3: z \mapsto z^2$ відобразить перший квадрант у верхню півплощину;
    \item $w_4: z \mapsto (z-i)/(z+i)$ відобразить верхню півплощину у одиничне коло.
\end{enumerate}

Отже, наше відображення має вигляд $w(z) = w_4 \circ w_3 \circ w_2 \circ w_1(z)$.  Треба це розписати:
\begin{align*}
    w(z) &= \frac{\left(\frac{1+z/R}{1-z/R}\right)^2-i}{\left(\frac{1+z/R}{1-z/R}\right)^2+i}
    = \frac{(1+z/R)^2-i(1-z/R)^2}{(1+z/R)^2+i(1-z/R)^2} \\
    &= \frac{\frac{1-i}{R^2}(z^2+2iRz+R^2)}{\frac{1+i}{R^2}(z^2-2iRz+R^2)} = -i \cdot \frac{z^2+2iRz+R^2}{z^2-2iRz+R^2}
\end{align*}

Далі, нам потрібно знайти спряжений вираз до цього:
\begin{align*}
    \overline{w(\zeta)} = \frac{\overline{-i(\zeta^2+2iR\zeta+R^2)}}{\overline{\zeta^2-2iR\zeta+R^2}} = \frac{i(\overline{\zeta}^2+2R\overline{i\zeta}+R^2)}{\overline{\zeta}^2-2R\overline{i\zeta}+R^2} = i \cdot \frac{\overline{\zeta}^2-2iR\overline{\zeta}+R^2}{\overline{\zeta}^2+2iR\overline{\zeta}+R^2}
\end{align*}

Далі починаємо рахувати сам дріб. Почнемо з чисельнику:
\begin{equation*}
    1-w(z)\overline{w(\zeta)} = 1 - \frac{z^2+2iRz+R^2}{z^2-2iRz+R^2} \cdot \frac{\overline{\zeta}^2-2iR\overline{\zeta}+R^2}{\overline{\zeta}^2+2iR\overline{\zeta}+R^2} = \frac{4iR(\overline{\zeta}-z)(R^2-\overline{\zeta}z)}{(R^2+2iR\overline{\zeta}+\overline{\zeta}^2)(R^2-2iRz+z^2)}
\end{equation*}

Тепер знайдемо знаменник:
\begin{equation*}
    w(z)-w(\zeta) = -i \cdot \frac{z^2+2iRz+R^2}{z^2-2iRz+R^2} + i \cdot \frac{\overline{\zeta}^2-2iR\overline{\zeta}+R^2}{\overline{\zeta}^2+2iR\overline{\zeta}+R^2} = -\frac{4R(\zeta-z)(R^2-\zeta z)}{(R^2-2iR\zeta+\zeta^2)(R^2-2iRz+z^2)}
\end{equation*}

Отже, маємо:
\begin{equation*}
    T(z,\zeta) = \frac{1-w(z)\overline{w(\zeta)}}{w(z)-w(\zeta)} = -i \cdot \frac{(\overline{\zeta}-z)(R^2-\overline{\zeta}z)(R^2-2iR\zeta+\zeta^2)}{(\zeta-z)(R^2-\zeta z)(R^2+2iR\overline{\zeta}+\overline{\zeta}^2)}
\end{equation*}

Спочатку знайдемо модулі ось цих двох довгих виразів:
\begin{align*}
    |R^2-2iR \zeta + \zeta^2| &= |R^2-2iR(\xi_1+i\xi_2)+(\xi_1+i\xi_2)^2| \\
    &= |R^2+\xi_1^2-\xi_2^2+2R\xi_2 - 2iR\xi_1 + 2i\xi_1\xi_2| \\
    &= \sqrt{(R^2+\xi_1^2-\xi_2^2+2R\xi_2)^2 + 4\xi_1^2(\xi_2-R)^2} 
\end{align*}
\begin{align*}
    |R^2+2iR\overline{\zeta}+\overline{\zeta}^2| &= |R^2+2iR(\xi_1-i\xi_2)+(\xi_1-i\xi_2)^2| \\
    &= |R^2+2R\xi_2+\xi_1^2-\xi_2^2 + (2R\xi_1-2\xi_1\xi_2)i| \\
    &= \sqrt{(R^2+\xi_1^2-\xi_2^2+2R\xi_2)^2 + 4\xi_1^2(\xi_2-R)^2} 
\end{align*}

Модулі вийшли однаковими. Дійсно, бо ці два вирази є спряженими:
\begin{equation*}
    \overline{R^2-2iR \zeta + \zeta^2} = R^2-2R\overline{i\zeta} + \overline{\zeta}^2 = R^2+2iR\overline{\zeta} + \overline{\zeta}^2
\end{equation*}

Тому, маємо:
\begin{equation*}
    |T(z,\zeta)| = |-i| \cdot \frac{|\overline{\zeta}-z|\cdot |R^2-\overline{\zeta}z|}{|\zeta-z| \cdot |R^2-\zeta z|} =  \frac{|\overline{\zeta}-z|\cdot |R^2-\overline{\zeta}z|}{|\zeta-z| \cdot |R^2-\zeta z|}  
\end{equation*}

З цього моменту можемо рахувати модулі усіх виразів:
\begin{align*}
    |\overline{\zeta}-z| &= |(\xi_1-i\xi_2)-(x_1+ix_2)| = \sqrt{(\xi_1-x_1)^2 + (\xi_2+x_2)^2}, \\
    |\zeta - z| &= |(\xi_1+i\xi_2)-(x_1+ix_2)| = \sqrt{(\xi_1-x_1)^2 + (\xi_2-x_2)^2},
\end{align*}
І наступних двох:
\begin{align*}
    |R^2-\overline{\zeta}z| &= |R^2-(\xi_1-i\xi_2)(x_1+ix_2)| = |R^2-\xi_1 x_1-\xi_2 x_2 + (x_1\xi_2-x_2\xi_1)i| \\
    &=\sqrt{(R^2-\xi_1 x_1-\xi_2 x_2)^2 + (x_1\xi_2-x_2\xi_1)^2}, \\
    |R^2-\zeta z| &= |R^2-(\xi_1+i\xi_2)(x_1+ix_2)| = |R^2-\xi_1 x_1+\xi_2 x_2 - (x_1\xi_2+x_2\xi_1)i| \\
    &=\sqrt{(R^2-\xi_1 x_1+\xi_2 x_2)^2 + (x_1\xi_2+x_2\xi_1)^2}.
\end{align*}

Отже, остаточно наша функція Гріна має вигляд:
\begin{align*}
    G(\mathbf{x},\boldsymbol{\xi}) &= \frac{1}{2\pi} \log \frac{|\overline{\zeta}-z|\cdot |R^2-\overline{\zeta}z|}{|\zeta-z| \cdot |R^2-\zeta z|} \\ 
    &= \frac{1}{2\pi} \log \frac{\sqrt{(\xi_1-x_1)^2 + (\xi_2+x_2)^2} \cdot \sqrt{(R^2-\xi_1 x_1-\xi_2 x_2)^2 + (x_1\xi_2-x_2\xi_1)^2}}{\sqrt{(\xi_1-x_1)^2 + (\xi_2-x_2)^2} \cdot \sqrt{(R^2-\xi_1 x_1+\xi_2 x_2)^2 + (x_1\xi_2+x_2\xi_1)^2}} \\
    &= \frac{1}{4\pi} \log \frac{(\xi_1-x_1)^2 + (\xi_2+x_2)^2}{(\xi_1-x_1)^2 + (\xi_2-x_2)^2} + \frac{1}{4\pi} \log \frac{(R^2-\xi_1 x_1-\xi_2 x_2)^2 + (x_1\xi_2-x_2\xi_1)^2}{(R^2-\xi_1 x_1+\xi_2 x_2)^2 + (x_1\xi_2+x_2\xi_1)^2}
\end{align*}

\pagebreak

\subsection{Номер 3.2 (1).}

\begin{problem}
    Розв'язати задачу для рівняння $\Delta u = 0$, $x_2 > 0$ з граничними умовами:
    \begin{equation*}
        u\Big|_{x_2=0} = \theta(x_1-1), \quad \theta(x) = \begin{cases}
            1, x > 0 \\
            0, x < 0
        \end{cases}
    \end{equation*}
\end{problem}

\textbf{Розв'язання.} Для початку згадаємо, що на практиці ми вже виписували функцію Гріна для такої задачі:
\begin{equation*}
    G(\mathbf{x}, \boldsymbol{\xi}) = \frac{1}{4\pi}\left(\log\left((x_1-\xi_1)^2 + (x_2+\xi_2)^2\right) - \log((x_1-\xi_1)^2+(x_2-\xi_2)^2)\right)
\end{equation*}

Тепер згадаємо, що розв'язок задачі Діріхле для рівняння Лапласа можна записати у вигляді:
\begin{equation*}
    u(\mathbf{x}) = \int_{\Omega} G(\mathbf{x}, \boldsymbol{\xi})f(\boldsymbol{\xi})d\boldsymbol{\xi} - \int_{\partial\Omega}\frac{\partial G(\mathbf{x}, \boldsymbol{\xi})}{\partial \boldsymbol{\nu}}\phi(\boldsymbol{\xi})dS_{\xi}
\end{equation*}

В нашому випадку $f(\boldsymbol{\xi}) \equiv 0$, тому все зводиться до обчислення другого інтегралу:
\begin{equation*}
    u(\mathbf{x}) = -\int_{\partial\Omega}\frac{\partial G(\mathbf{x}, \boldsymbol{\xi})}{\partial \boldsymbol{\nu}}\phi(\boldsymbol{\xi})dS_{\xi}
\end{equation*}

Отже, треба знайти часткову похідну за нормаллю. В нашому випадку, нормаль дивиться вертикально ``вниз'' від верхньої півплощини, тому $\boldsymbol{\nu} = (0,-1)$. Таким чином, маємо:
\begin{equation*}
    \frac{\partial G(\mathbf{x}, \boldsymbol{\xi})}{\partial \boldsymbol{\nu}} = -\frac{\partial G(\mathbf{x}, \boldsymbol{\xi})}{\partial \xi_2} = -\frac{1}{2\pi} \cdot \frac{x_2+\xi_2}{(x_1-\xi_1)^2+(x_2+\xi_2)^2} - \frac{1}{2\pi} \cdot \frac{x_2-\xi_2}{(x_1-\xi_1)^2+(x_2-\xi_2)^2}
\end{equation*}

Далі, оскільки ми інтегруємо за границею, то $\xi_2 = 0$. Тому, спростимо вираз і далі:
\begin{equation*}
    \frac{\partial G(\mathbf{x}, \boldsymbol{\xi})}{\partial \boldsymbol{\nu}}\Big|_{\xi_2=0} = -\frac{1}{2\pi} \cdot \frac{x_2}{(x_1-\xi_1)^2+x_2^2} - \frac{1}{2\pi} \cdot \frac{x_2}{(x_1-\xi_1)^2+x_2^2} = -\frac{1}{\pi} \cdot \frac{x_2}{(x_1-\xi_1)^2+x_2^2}
\end{equation*}

Отже, наш інтеграл запишеться як:
\begin{equation*}
    u(\mathbf{x}) = \int_{-\infty}^{+\infty} \frac{1}{\pi} \cdot \frac{x_2}{(x_1-\xi_1)^2+x_2^2} \cdot \theta(\xi_1-1)d\xi_1 = \frac{x_2}{\pi}\int_{-\infty}^{+\infty} \frac{\theta(\xi_1-1)d\xi_1}{(x_1-\xi_1)^2+x_2^2}
\end{equation*}

Помітимо, що за $\xi_1 < 1$, вираз під інтегралом нульовий, тому можемо змінити межі інтегрування:
\begin{equation*}
    u(\mathbf{x}) = \frac{x_2}{\pi}\int_{1}^{+\infty} \frac{d\xi_1}{(x_1-\xi_1)^2+x_2^2} = \frac{1}{\pi x_2}\int_1^{+\infty} \frac{d\xi_1}{1+\left(\frac{x_1-\xi_1}{x_2}\right)^2}
\end{equation*}

Отже, зробимо заміну $\zeta := \frac{\xi_1-x_1}{x_2}$, тоді $x_2d\zeta = d\xi_1$ і тому:
\begin{equation*}
    u(\mathbf{x}) = \frac{1}{\pi}\int_{(1-x_1)/x_2}^{+\infty} \frac{d\zeta}{1+\zeta^2} = \frac{1}{\pi}\arctan \left(\frac{\xi_1-x_1}{x_2}\right)\Big|_{1}^{+\infty} = \boxed{\frac{1}{\pi}\left(\frac{\pi}{2} - \arctan \left(\frac{1-x_1}{x_2}\right)\right)}
\end{equation*}

\end{document}