\documentclass[12pt,a4paper,oneside]{book}

% --- Setting up a language ---

\usepackage[utf8]{inputenc}
\usepackage[english,main=ukrainian]{babel}

% --- Detault Packages ---

\usepackage[dvipsnames]{xcolor}

\usepackage[toc,page]{appendix}
\usepackage{amsthm,amsfonts,mathtools,amscd,amssymb,mathrsfs}
\usepackage{latexsym}
\usepackage{euscript}
\usepackage{multicol}
\usepackage{bbm}
\usepackage{pgffor}   % For loops in TikZ
\usepackage{dsfont}
\usepackage{pgfplots}
\usepackage{graphicx}
\usepackage{tikz}
\usetikzlibrary{backgrounds}
\usepackage{array}
\usepackage{caption}
\usepackage{subcaption}
\usepackage{listings}
\usepackage{setspace}
\usepackage[shortlabels]{enumitem}

% 
\usepackage[unicode]{hyperref}
\hypersetup{colorlinks=true}
\tolerance=9000 
\textwidth=165mm
\textheight=250mm 
\oddsidemargin=4.6mm
\voffset=-20mm
\binoppenalty=1 \relpenalty=1 
\allowdisplaybreaks
\everymath{\displaystyle}
\renewcommand{\baselinestretch}{1.44}

\usepackage{dipl}
\pagestyle{headings}
\graphicspath{ {figures/} }

% --- Theorem Styles ---
\theoremstyle{dplplain}
\newtheorem{theorem}{Теорема}[chapter]
\newtheorem{corollary}[theorem]{Наслідок}%[chapter]
\newtheorem{statement}[theorem]{Твердження}%[chapter]
\newtheorem{lemma}[theorem]{Лема}%[chapter]
\theoremstyle{dpldefinition}
\newtheorem{definition}[theorem]{Означення}%[chapter]
\theoremstyle{dplremark}
\newtheorem{problem}[theorem]{Задача}%[chapter]
\newtheorem{remark}[theorem]{Зауваження}%[chapter]
\newtheorem{example}[theorem]{Приклад}%[chapter]

% --- Math shortings ---
\DeclareMathOperator*{\esssup}{ess\,sup}
\DeclareMathOperator*{\essinf}{ess\,inf}
\DeclareMathOperator*{\bigtimes}{\mbox{\raisebox{-0.5ex}{\huge$\times$}}}
\DeclareMathOperator*{\smalltimes}{\mbox{\raisebox{-0.3ex}{\large$\times$}}}
\DeclareMathOperator*{\ootimes}{\overline{\otimes}}

\DeclareMathOperator*{\argmax}{arg\,max}
\DeclareMathOperator*{\argmin}{arg\,min}

% --- Listings ---
% --- Default settings ---
% \lstset{language=Python}
% \lstset{frame=lines}
% \lstset{basicstyle=\normalsize}
% \lstset{prebreak=\raisebox{-2ex}[0ex][0ex]{\ensuremath{\hookleftarrow}}}
% \lstset{breaklines=true}
% \captionsetup[lstlisting]{font=large}

% --- My own settings ---
\definecolor{dkgreen}{rgb}{0,0.5,0}
\definecolor{gray}{rgb}{0.5,0.5,0.5}
\definecolor{mauve}{rgb}{0.58,0,0.82}
\lstset{
    numbers=left,  
    frame=tb,
    aboveskip=3mm,
    belowskip=3mm,
    showstringspaces=false,
    columns=fixed,
    framerule=1pt,
    rulecolor=\color{gray!35},
    backgroundcolor=\color{gray!5},
    basicstyle={\ttfamily\small},
    numberstyle=\footnotesize\color{gray},
    keywordstyle=\bfseries\color{MidnightBlue!95!black},
    commentstyle=\color{dkgreen},
    stringstyle=\color{mauve},
    breaklines=true,
    breakatwhitespace=true,
    tabsize=2,
    extendedchars=false,
    postbreak=\mbox{\hspace{-1.4em}\textcolor{purple}{$\hookrightarrow$}\space}
}

\newcommand{\remind}[1]{\textcolor{red}{\textbf{#1}}} % To remind me of unfinished work to fix later
\newcommand{\hide}[1]{} % To hide large blocks of code without using % symbols

% --- Bibliography ---
\usepackage[
    backend=biber,
    style=alphabetic
]{biblatex}
\addbibresource{main.bib}

% --- Some tikz functions ---
\newbox\dumbox
\newcommand{\mymark}[2]{%
  \setbox\dumbox=\hbox{#2}%
  \hbox to \wd\dumbox{\hss%
    \tikz[overlay,remember picture,baseline=(#1.base)]{ \node (#1) {\box\dumbox}; }%
    \hss}%
}

%% -- Draw small coefficient
\newcommand{\mysmall}[1]{%
  \tikz[overlay,remember picture]{%
    \node[blue,scale=.5, shift={(0,-.1)}] {$\times$#1};%
  }%
}

\begin{document}

    \begin{titlepage}
	\setstretch{1}
	\begin{center}
		\Large Харківський національний університет імені В.Н.~Каразіна\\
		Факультет математики і інформатики\\
		Кафедра прикладної математики
	\end{center}

	\vfill
		
	\begin{center}	
		\LARGE \bfseries Курсова науково-дослідницька робота
		% \\
		% 	{\normalsize \mdseries освітньо-кваліфікаційний рівень: \bfseries \slshape бакалавр}
		\\[0.5\baselineskip]
		{\mdseries на тему} \bfseries\slshape <<Математичні основи штучних нейронних мереж>> \\[0.5\baselineskip]
	\end{center}
	
	\vfill
	

	\setlength{\tabcolsep}{3pt}
	\hbox to \textwidth{\hfill\begin{tabular}{>{\slshape}lp{0.43\textwidth}}
			Виконав: &студент групи МП41 IV курсу
			
			(перший бакалаврський рівень), 
			
			спеціальності 113 
			
			``Прикладна математика''
			
			освітньої програми 
			
			``Прикладна математика''
			
			\textbf{Захаров~Д.О.}
			\\[0.5\baselineskip]
			Керівник: & доктор фіз.-мат. наук, 
			
			професор кафедри 
			
			прикладної математики
			
			\textbf{Ігнатович С.Ю.}	
			% \\[0.5\baselineskip]
			% Рецензент: & кандидат фіз.-мат. наук, 
			
			% доцент кафедри 
			
			% прикладної математики
			
			% \textbf{Хтось.}
	\end{tabular}	}
	
\vspace{\baselineskip}
	
	\begin{center}
		Харків --- 2024 рік
	\end{center}
	
\end{titlepage}

    \refstepcounter{page}

    \chapter*{Анотації}
\markboth{Анотації}{Анотації}
\addcontentsline{toc}{chapter}{Анотації}

\subsubsection{Захаров Д.О. Архітектури нейронних мереж з ефективними активаціями.}

Через зростання попиту у розробці архітектур нейронних мереж, потреба в
конфіденційності їх обчислень стала першорядною. З цієї причини, галузь
доказів із нульовим знанням і повністю гомоморфного шифрування активно
досліджувалися для роботи над нейронними мережами. Однак функції активації, які
використовуються в нейронних мережах, не завжди сумісні з існуючими
арифметизаціями криптографічних примітивів. У цій статті ми аналізуємо, чи можна
оптимізувати функції активації, які використовуються в нейронних мережах, щоб
зробити їх більш «криптографічно-сумісними».

\subsubsection{Zakharov D.O. Activation-Efficient Neural Network Architectures.}

Due to the rise of neural network architectures, the need for their privacy and
privacy-preserving computations became paramount. For that reason, the
fields of zero-knowledge proofs and fully homomorphic encryption have been
actively researched to work on top of neural networks. However, the
activation functions used in neural networks are not always compatible with
the existing arithmetizations of cryptographic primitives. In this paper, we
analyze whether the activation functions used in neural networks can be
optimized for the sake of making them more ``cryptographically friendly''.


    \tableofcontents

    \chapter*{Вступ}
\markboth{Вступ}{Вступ}
\addcontentsline{toc}{chapter}{Вступ}

Без всяких сумнівів, нейронні мережі є одними з найбільш популярних інструментів
машинного навчання для пошуку складних залежностей. Вони використовуються в
безлічі різних областей, таких як комп'ютерний зір \cite{cv-survey}, обробка
природних мов \cite{nlp-survey} та біометричних данних \cite{biometrics-survey},
розробка рекомендаційних систем \cite{recommendation-systems-survey}, генерації
зображень \cite{gan-survey} тощо. Наші нещодавні дослідження
\cite{our-biometrics-1,our-biometrics-2,our-biometrics-3} додатково підтвердили
високу ефективність нейронних мереж у задачах кібербезпеки та систем захисту
біометричних даних. Що уж там, мабуть кожна людина чула або використовувала
новітні розробки OpenAI --- архітектуру \textit{GPT-3} (Generative Pre-trained
Transformer) \cite{chatgpt} або \textit{Open AI o1}, що вже навіть здатна
розв'язувати задачі з міжнародної олімпіади з математики або аналізувати складні
наукові тексти. 

Проте, незважаючи на таку кількість різноманітних досліджень, більшість з них 
зводиться до доволі типічного алгоритму (звичайно, з варіаціями в залежності від
конкретної задачі):
\begin{enumerate}
    \item Визначення типу задачі (класифікація, регресія, сегментація, тощо).
    \item Підбір набору даних (далі, скорочено --- датасет).
    \item Вибір архітектури моделі, функції втрати та метрик якості.
    \item Тренування та корегування параметрів моделі для максимізації метрики якості.
    \item Аналіз результатів.
\end{enumerate}

Проте, протягом цього процесу, ми зазвичай пропускаємо одне доволі
фундаментальне питання: а чому, взагалі кажучи, обрані архітектури нейронних
мереж здатні вирішувати такі задачі? Звичайно, що для практичних задач це
питання часто не принципове: якщо воно працює і працює добре, то цього більш,
ніж достатньо\footnote{Тим не менш, в сучасних роботах іноді трапляється спроба
пояснити, чому описана методологія може теоретично дати, скажімо, мінімум для
певної метрики, як це було зроблено в оригінальному описі генеративних
адверсальних мереж (Generative Adversarial Networks) \cite{gan}}.

Саме тому ми присвятили цю роботу опису фундаменту нейронних
мереж та, в певній мірі, формалізації процесу навчання: що саме 
розв'язують нейронні мережі, чому вони (теоретично) здатні
апроксимувати важливі для нас залежності і як на практиці
реалізувати процес підбору параметрів моделі.



    % Вступ
    \usetikzlibrary{positioning, arrows.meta}

\chapter{Задачі машинного навчання}

\section{Що таке модель?}

Насправді, чітко поставити задачу сучасної теорії машинного навчання одним
визначенням дуже складно. Це пов'язано з тим, що підхід до розв'язку задачі дуже
залежить від того, що ми очікуємо від так званої \textit{моделі машинного
навчання}. Що ж ми розуміємо під терміном ``модель''?

Найчастіше, на вхід подається певний набір данних $\mathcal{D}$. Це можуть бути
картинки разом з маркуванням, що зображено. Можуть бути текстові данні, чисельні
данні, аудіо- або відео-записи, результати вимірювань на сенсорних пристроях,
тощо. Маючи цей набір даних, ми часто хочемо зрозуміти певні закономірності в
цих даних. Функцію, що бере певний вхід, що містить інформацію про об'єкт, і
повертає вихід, що містить закономірність, часто називаємий
\textit{передбаченням}, як раз і називають \textbf{моделлю}. Далі наведемо
кілька нетривільних прикладів з задач машинного навчання.

\begin{example}[Класифікація цифр]
	Будь-яке сіро-біле зображення $\boldsymbol{X}$ розміру $W \times H$ пікселів
	можна розглядати як матрицю $\boldsymbol{X} \in \mathbb{R}^{W \times H}$, де
	кожен елемент матриці $X_{i,j}$ --- це значення яскравості відповідного
	пікселя на позиції $(i,j)$ (наприклад, значення $0$ може позначати чорний
	колір, $1$ --- білий, а значення проміж --- степінь сірості)\footnote{Іноді
	таку множину явно записують як $[0,1]^{W \times H}$, щоб підкреслити, що
	значення нормалізовані на відрізок $[0,1]$. Тим не менш, використовуємо
	позначення $\mathbb{R}^{W \times H}$, оскільки оптимальна нормалізація
	данних залежить від обраної методології}. 
	
	Нехай нам наданий набір $\mathcal{D} =
	\{(\boldsymbol{X}_n,y_n)\}_{1 \leq n \leq N}$ --- пари
	``зображення-цифра'', де $y_n \in \{0,\dots,9\}$. Наша ціль ---
	побудувати так звану \textit{класифікаційну модель} $f: \mathbb{R}^{W \times
	H} \to \{0,\dots,9\}$, яка буде приймати на вхід зображення
	$\boldsymbol{X}$ та видавати цифру $f(\boldsymbol{X})$, що зображена.
	Приклад зображено на Рисунку~\ref{fig:digits} на базі набору данних MNIST
	\cite{mnist}.

	\begin{figure}
		\centering
		\begin{tikzpicture}
			% Define grid and spacing size
			\def\gridsize{3}
			\def\xspacing{4} % Horizontal spacing
			\def\yspacing{3} % Vertical spacing
			
			% Loop to create 3x3 grid with predictions
			\foreach \row in {1,2,3} {
				\foreach \col in {1,2,3} {
			
					% Calculate image index for sample prediction (0-9, can be random or set)
					\pgfmathtruncatemacro{\digitindex}{(\row-1)*\gridsize + \col - 1}

					% Placeholder integer predictions (customize as needed)
					\pgfmathtruncatemacro{\prediction}{mod(\digitindex, 10)} % Dummy prediction for illustration
					
					% Position of each MNIST image
					\node[draw, minimum size=1cm, inner sep=0pt] (img\row\col) at (\col*\xspacing, -\row*\yspacing)
					{\includegraphics[width=1.75cm]{figures/mnist/digit\digitindex.png}};
			
					% Prediction label with actual prediction value next to each digit
					\node[right=1.0cm of img\row\col] (pred\row\col) {\textbf{\prediction}};
			
					% Arrow pointing from image to prediction
					\draw[->, thick] (img\row\col) -- node[above] {\( f \)} (pred\row\col);
				}
			}
		\end{tikzpicture}
		\caption{Приклад класифікації цифр. Маючи зображення $X \in \mathbb{R}^{W \times H}$, наша функція дає дискретне передбачення $f(X) \in \{0,\dots,9\}$ --- цифра, яка зображена на зображенні $X$.}
		\label{fig:digits}
	\end{figure}

\end{example}

\begin{example}[Розпізнавання дрону]
	Нехай наша задача: це розпізнати розташування дронів на кадрі відео. Нехай в
	нас кольорове зображення розміру $W \times H$. Тоді зображення
	$\boldsymbol{X}$ береться з множини $\mathbb{R}^{W \times H \times 3}$,
	де замість яскравості пікселя, маємо тривимірний вектор $(r,g,b) \in
	\mathbb{R}^3$ --- колір пікселя (інтенсивність червоного, зеленого та синіх
	каналів, відповідно).

	Поділимо наше зображення на сітку розміру $n_W \times n_H$. Тоді в якості
	моделі можна взяти функцію $f: \mathbb{R}^{W \times H \times 3} \to
	\mathbb{R}^{n_W \times n_H}$, що видає матрицю
	$\{p(S_{i,j}|\boldsymbol{X})\}_{1 \leq i \leq n_W, \, 1 \leq j \leq n_H}$,
	де $S_{i,j}$ --- подія ``в клітинці $(i,j)$ сітки знаходиться дрон''. Ця
	модель візуально проілюстрована на Рис.~\ref{fig:drone}. Відмітимо, що ідея
	описаної конструкції частково використовується в архітектурі YOLO (You Look
	Only Once) \cite{yolo} --- одній з найпопулярніших архітектур для
	розпізнавання об'єктів на зображеннях.

	\begin{figure}
	
	\centering
	\begin{tikzpicture}

		% Set up background image
		\node[inner sep=0pt, anchor=center] at (0,0) {\includegraphics[width=10cm]{drone.jpg}}; % Replace 'background_image.jpg' with the path to your image
		
		% Define grid parameters
		\def\gridsize{7}
		\def\cellsize{1.5} % Size of each cell
		
		% Define the location of the object in the background (center of the image)
		\def\objectX{0} % X position of the object (centered in the image)
		\def\objectY{0} % Y position of the object (centered in the image)
		
		% Draw the gradient grid overlay with dashed borders
		\foreach \x in {-3,-2,-1,0,1,2} {
			\foreach \y in {-2,-1,0,1} {
				\definecolor{cellcolor}{rgb}{0.6, 0.8, 1} % Cold color (light blue)
				\ifnum\x=0\ifnum\y=0
					\definecolor{cellcolor}{rgb}{1, 0.6, 0.6} % Warm color (red/pink)
				\fi\fi
				\ifnum\x=-1\ifnum\y=0
					\definecolor{cellcolor}{rgb}{1, 0.6, 0.6} % Warm color (red/pink)
				\fi\fi
				\ifnum\x=-2\ifnum\y=0
					% Light yellow
					\definecolor{cellcolor}{rgb}{1, 1, 0.6}
				\fi\fi
				\ifnum\x=0\ifnum\y=-1
				\definecolor{cellcolor}{rgb}{1, 0.8, 0.4} % Intermediate warm (orange/yellow)
				\fi\fi
				\ifnum\x=-1\ifnum\y=-1
				\definecolor{cellcolor}{rgb}{1, 0.8, 0.4} % Intermediate warm (orange/yellow)
				\fi\fi

				% % Calculate distance from the object's location
				% \pgfmathsetmacro{\distance}{sqrt((\x - \objectX)^2 + (\y - \objectY)^2)}
		
				% % Set color based on distance to simulate probability heatmap
				% \ifdim \distance pt < 1.5pt
				% 	\definecolor{cellcolor}{rgb}{1, 0.6, 0.6} % Warm color (red/pink)
				% \else\ifdim \distance pt < 2.5pt
				% 	\definecolor{cellcolor}{rgb}{1, 0.8, 0.4} % Intermediate warm (orange/yellow)
				% \else
				% 	\definecolor{cellcolor}{rgb}{0.6, 0.8, 1} % Cold color (light blue)
				% \fi\fi
		
				% Overlay gradient cell with transparency
				\fill[cellcolor, opacity=0.5] (\x*\cellsize, \y*\cellsize) rectangle ++(\cellsize, \cellsize);
		
				% Draw dashed border around each cell
				\draw[line width=1pt, dashed] (\x*\cellsize, \y*\cellsize) rectangle ++(\cellsize, \cellsize);
			}
		}

		\draw[<-, ultra thick, bend right=90] (2.5,2) -- ++(3.0,-1) node[right] {$f_{1,5}(X) \approxeq 0.02$}; % Adjust coordinates for proper alignment
		\draw[<-, ultra thick, bend right=90] (0.5,-0.5) -- ++(6.0,-1) node[right] {$f_{3,4}(X) \approxeq 0.40$}; % Adjust coordinates for proper alignment

		\end{tikzpicture}
		\caption{Приклад розпізнавання дронів. Маючи зображення $\boldsymbol{X} \in \mathbb{R}^{W \times H \times 3}$, наша функція дає ймовірність того, що на кожному сегменті зображення знаходиться дрон. Чим тепліше кольори, тим вища ймовірність.}
		\label{fig:drone}
	\end{figure}
\end{example}

\begin{remark}
	Зверніть увагу, що в обох прикладах вище, модель $f$ видає дискретне або
	неперервне передбачення. Це може бути класифікація, регресія, сегментація,
	тощо. Це залежить від задачі, що ми розв'язуємо.
\end{remark}

\section{Параметризація моделей}
Проте, як саме ми будуємо моделі? Іншими словами, як обрати функцію $f$?
Зазвичай, ми маємо задати певне сімейство функцій $\mathcal{F}$, поміж яких ми
шукаємо ``найкращу''\footnote{Що таке ``найкраща'' функція, ми обговоримо
пізніше.} функцію. Наприклад, це може бути сімейство лінійних/квадратичних
функцій або функцій вигляду $f(x) = (\theta_1x^2, \theta_2 x)$. Звичайно,
оскільки алгоритм пошуку $f$ має бути заданий програмно, то ми не можемо
покласти в якості $\mathcal{F}$, скажімо, просто $L^2(\mathbb{R})$, бо тоді не
зрозуміло, як саме задати алгоритм пошуку $f$. Саме тому, для практичних
застосувань, ми \textit{параметризуємо} функції набором параметрів
$\boldsymbol{\theta}\in \Theta \subset \mathbb{R}^m$. Таким чином, наша модель
має вигляд $f(\mathbf{x}|\boldsymbol{\theta})$, де параметри
$\boldsymbol{\theta}$ можна змінювати, щоб зробити модель точною.

\begin{example}[Лінійна Регресія]
	Одна з класичних та найбільш відомих моделей --- це лінійна регресія. Нехай
	маємо набір даних $\mathcal{D} = \{(\mathbf{x}_n, y_n)\}_{1 \leq n \leq N}
	\subset \mathbb{R}^m \times \mathbb{R}$ і ми вважаємо, що залежність між
	$\mathbf{x}_n$ та $y_n$ лінійна. Іншими словами, ми введемо модель
	$f(\mathbf{x}|\boldsymbol{\theta}) = \boldsymbol{w}^{\top} \mathbf{x} +
	\beta$, де $\boldsymbol{\theta}=(\boldsymbol{w},\beta) \in \mathbb{R}^{m+1}$
	--- вектор параметрів. 

	В цілому, саме так дуже часто вводиться модель лінійної регресії. Проте, часто
	в літературі можна зустріти у якості моделі \textit{розподіл} величини $y$:
	\begin{equation}
		p(y|\mathbf{x},\boldsymbol{\theta}) = \mathcal{N}(y|\boldsymbol{w}^{\top}\mathbf{x} + \beta, \sigma^2),
	\end{equation}
	де $\boldsymbol{\theta} = (\boldsymbol{w},\beta,\sigma^2)$ --- вектор
	параметрів, а $\mathcal{N}(y|\mu,\sigma^2)$ --- щільність нормального
	розподілу. Така альтернатива дозволяє ввести більш гнучку модель, яка може
	давати степінь впевненості у своїх передбаченнях.
\end{example}

\begin{remark}
	Приклад вище легко узагальнити для випадку, коли вихід $\mathbf{y} \in \mathbb{R}^r$ --- вектор. В такому випадку, 
	моделлю є передбачення наступного розподілу:
	\begin{equation}
		p(\mathbf{y}|\mathbf{x},\boldsymbol{\theta}) = \prod_{i=1}^r \mathcal{N}(y_i|\boldsymbol{w}_i^{\top}\mathbf{x} + \beta_i, \sigma_i^2), \; \boldsymbol{w}_i \in \mathbb{R}^m, \; \beta_i,\sigma_i \in \mathbb{R}
	\end{equation}
\end{remark}

Отже, нехай ми вибрали параметризацію моделі. Як тепер обрати найкращі параметри?
Тут, наскільки б це не звучало банально, але знову все залежить від того, що ми
очікуємо від моделі:
\begin{itemize}
	\item Якщо наша модель має апроксимувати певну функцію $\varphi(\mathbf{x})$ на 
	обмеженій множині $\mathcal{X} \subset \mathbb{R}^r$, то ми можливо хочемо 
	мінімізувати $L^2(\mathcal{X},\mu)$ норму різниці:
	\begin{equation*}
		\hat{\boldsymbol{\theta}} = \argmin_{\boldsymbol{\theta} \in \Theta} \int_{\mathcal{X}} d\mu(\mathbf{x}) \left\| f(\mathbf{x}|\boldsymbol{\theta}) - \varphi(\mathbf{x}) \right\|_2^2
	\end{equation*}
	\item Можливо, ми хочемо максимізувати функцію правдоподібності:
	\begin{equation*}
		\hat{\boldsymbol{\theta}} = \argmax_{\boldsymbol{\theta} \in \Theta} \prod_{n=1}^N p(y_n|\mathbf{x}_n,\boldsymbol{\theta})
	\end{equation*}
	\item Якщо модель видає ймовірністний розподіл над простором $\Omega \subset \mathbb{R}^r$, то можливо ми хочемо мінімізувати відстань Кульбака-Лейблера до заданого розподілу $\pi(\mathbf{x})$:
	\begin{equation*}
		\hat{\boldsymbol{\theta}} = \argmin_{\boldsymbol{\theta} \in \Theta} D_{\mathbb{KL}}(f(\mathbf{x}|\boldsymbol{\theta})||\pi(\mathbf{x})) = \argmin_{\boldsymbol{\theta} \in \Theta} \int_{\Omega} f(\mathbf{x} | \boldsymbol{\theta}) \log \frac{f(\mathbf{x} | \boldsymbol{\theta})}{\pi(\mathbf{x})}d\mathbf{x}
	\end{equation*}
\end{itemize}

\begin{example}[Розв'язок лінійної регресії]
	Наприклад, нехай ми вирішуємо задачу лінійної регресії для набору даних
	$\mathcal{D} = \{(\mathbf{x}_n, y_n)\}_{1 \leq n \leq N}$. Нехай ми хочемо 
	мінімізовувати функцію правдоподібності:
	\begin{equation}
		\hat{\boldsymbol{\theta}} = \argmax_{\boldsymbol{\theta} \in \Theta} \prod_{n=1}^N p(y_n|\mathbf{x}_n,\boldsymbol{\theta}) = \argmax_{(\boldsymbol{w},\beta,\sigma^2)}\prod_{n=1}^N \mathcal{N}(y_n|\boldsymbol{w}^{\top}\mathbf{x}_n+\beta,\sigma^2)
	\end{equation}

	Можна довести, що якщо позначити $\mathbf{X} = [\mathbf{x}_1,\dots,\mathbf{x}_N] \in \mathbb{R}^{m \times N}$ --- матриця даних, а $\mathbf{y} = [y_1,\dots,y_N] \in \mathbb{R}^N$ --- вектор маркерів, то розв'язок має вигляд:
	\begin{equation}
		\hat{\boldsymbol{\theta}} = (\mathbf{X}\mathbf{X}^{\top})^{-1}\mathbf{X}\mathbf{y}
	\end{equation}
\end{example}

Проте, яку б ми теорію не використовували і які б параметризації моделей не
використовували, в кінці кінців, перед нами постає наступна задача оптимізації, що
розв'язується чисельно:
\begin{equation}
	\hat{\boldsymbol{\theta}} = \argmin_{\boldsymbol{\theta} \in \Theta} \mathcal{L}(\mathcal{D}|\boldsymbol{\theta}),
\end{equation}

де $\mathcal{L}(\mathcal{D}|\boldsymbol{\theta})$ --- функція втрат, яка
відображає як добре модель з параметрами $\boldsymbol{\theta}$ апроксимує дані
$\mathcal{D}$. Зауважимо, що хоч, в ідеалі, ми хочемо отримати вихід як
найближчий до ``істинного'', але в більшості випадків, ми не знаємо
``істинного'' вихідного розподілу. Тому, функція втрати, на практиці, 
залежить від набору данних $\mathcal{D}$, що подається на вхід, та від
параметризації $\boldsymbol{\theta}$.

Отже, стає питання: а як обрати параметризацію? Насправді, саме в цьому питанні
лежить більшість сучасних досліджень у глибокому навчанні: наприклад, у 1995
році, конволюційні нейронні мережі (Convolutional Neural Networks --- CNN)
прийшли на заміну повнозв'язним нейронним мережам \cite{lecun}, а механізм уваги
(Attention) у 2017 став основою багатьох NLP нейронних мереж \cite{attention}.
Щоб підкреслити важливість цього питання, наведемо приклад.

\begin{example}
	Нехай ми хочемо апроксимувати залежність $y(x)$ для набору $\mathcal{D} =
	\{(x_n,y_n)\}_{1 \leq n \leq N}$, причому навіть знаючи вибірку, ми не маємо
	уяви, яка має бути залежність $y(x)$. За невідомими причинами, нехай ми
	вирішили використати модель $f(x|\boldsymbol{\theta}) =
	\left(\sum_{i=1}^{1000}\theta_i\right)x$ для вектору параметрів
	$\boldsymbol{\theta} \in \mathbb{R}^{1000}$. Хоча ця модель містить досить
	велику кількість параметрів, вона не може апроксимувати жодні залежності
	окрім лінійних. Отже навіть якщо залежність $y$ від $x$ квадратична, що
	є відносно простою залежністю, модель не зможе її апроксимувати, незважаючи 
	на велику кількість параметрів.
\end{example}

\section{Основні задачі машинного навчання}

Отже, на практиці, ми хочемо мати як можна меньше параметрів, але при цьому 
модель повинна бути достатньо гнучкою, щоб апроксимувати будь-яку залежність. 
Інакшими слвоами, модель має апроксимувати як можна більш широкий клас 
функцій. 

Таким чином, підсумуємо, перед якими проблемами стоїть дослідник у глибокому
навчанні. Ми виділили три основні проблеми, які можна сформулювати наступним
чином: 
\begin{enumerate}
	\item \textbf{Проблема статистики/ймовірності/узагальнення:} маючи лише
	набір данних $\mathcal{D}$, не знаючи істинного розподілу чи функції, чи
	достатньо добре функція втрати $\mathcal{L}$ відображає степінь наближення
	моделі до істинної функції або розподілу?
	\item \textbf{Проблема оптимізації:} маючи функцію втрати $\mathcal{L}$,
	наскільки точно і чи взагалі можливо знайти оптимальні параметри
	$\boldsymbol{\theta}$ для мінімізації функції втрати?
	\item \textbf{Проблема апроксимації:} яка найкраща і чи взагалі існує така
	параметризація моделі, щоб вона була достатньо гнучкою, але при цьому мала
	якнайменшу кількість параметрів?\footnote{Мала кількість параметрів сприяє
	як і, очевидно, швидкості навчання та діставання передбачень, так і робить
	модель менш схильною до перенавчання та проблем градієнтів (стосується
	проблеми оптимізації).}
\end{enumerate}

В наступному підрозділі, ми розглянемо декілька теорем, що 
дозволяють відповісти на третє запитання. Після цього, ми перейдемо 
до методології другої проблеми, а саме, до методів оптимізації.


    % Опис MLP vs KAN
    \chapter{Теорія Апроксимації}

Приблизно на цьому етапі, більшість літератури з машинного та, зокрема,
глибокого навчання починається з фрази:

\begin{quote}
    \begin{center}
    ``\textit{Задамо багатошарову нейронну мережу з $\star$ шарами, де зв'язок
    активацій $\mathbf{x}^{\langle j \rangle}$ та $\mathbf{x}^{\langle j+1
    \rangle}$ задається рівнянням $\mathbf{x}^{\langle j+1 \rangle} =
    \phi^{\langle j \rangle}(\boldsymbol{W}^{\langle j
    \rangle}\mathbf{x}^{\langle j \rangle}+\boldsymbol{\beta}^{\langle j \rangle})$}''
    \end{center}
\end{quote}

При цьому, зазвичай, не обгрунтовується (окрім як базової інтуїції) вибір
саме такої формули для зв'язку між шарами. Ще рідше, чому така архітектура
може апроксимувати широкий клас функцій. Саме тому в цьому підрозділі
ми підійдемо до цього питання більш системно.

\section{Апроксимація сігмоїдальними функціями: Теорема Цибенко}

\subsection{Постановка задачі}
Один з перших результатів, що дозволяє відповісти на питання про апроксимацію
функцій, був отриманий Цибенко в 1989 році у роботі \cite{cybenko}. Результати
саме цієї роботи лежать в основі побудови перших щільних шарів у багатошарових
нейронних мережах (Dense Layers): в певному вигляді, ця робота містить одну з
перших архітектур, що дозволяють побудувати модель класифікації. Тому, в
багатьох джерелах, теорема Цибенка отримала назву універсальної апроксимаційної
теореми (Universal Approximation Theorem). Спочатку, введемо основний клас
функцій, що буде в серці нашої теореми: сігмоїдальні функції.

\begin{definition}
	\textbf{Сігмоїдальною функцією} $\sigma: \mathbb{R} \to \mathbb{R}$ називається
	функція, що задовольняє двом умовам:
	\begin{equation}
		\lim_{x \to +\infty} \sigma(x) = 1, \quad \lim_{x \to -\infty}\sigma(x) = 0.
	\end{equation}
\end{definition}

\begin{example}
	Найбільш відомою сігмоїдальною функцією є функція Логістичної регресії:
	\begin{equation}
		\sigma(x|\alpha) = \frac{1}{1+e^{-\alpha x}}, \quad \alpha > 0.
	\end{equation}

	Її зручність полягає у неперервності, диференційовності та зручності
	обчислення похідної, оскільки $\sigma' = \alpha\sigma(1-\sigma)$.
	Графіки цієї функції для різних параметрів $\alpha$ наведені на
	Рис.~\ref{fig:sigmoids}.
\end{example}

\begin{figure}
\centering
\begin{tikzpicture}
    \begin{axis}[
        axis lines=middle,
        xlabel={$x$},
        ylabel={$y$},
        ymin=0, ymax=1.2,
        xmin=-3.9, xmax=3.9,
        domain=-4:4,
        samples=100,
        grid=both,
        width=14cm, % Adjusted for wider aspect ratio
        height=8cm,
        legend style={at={(1.05,1)}, anchor=north west}
    ]
    
    % Plot different sigmoidal functions with bolder lines
    \addplot[ultra thick, blue] {1 / (1 + exp(-1 * x))};
    \addlegendentry{$\alpha=1$}
    
    \addplot[ultra thick, red] {1 / (1 + exp(-2 * x))};
    \addlegendentry{$\alpha=2$}
    
    \addplot[ultra thick, green] {1 / (1 + exp(-3 * x))};
    \addlegendentry{$\alpha=3$}
    
    \addplot[ultra thick, orange] {1 / (1 + exp(-4 * x))};
    \addlegendentry{$\alpha=4$}

    \end{axis}
\end{tikzpicture}
\caption{Графіки сігмоїдальних функцій $\sigma(x|\alpha)=1/(1+e^{-\alpha x})$ для різних параметрів $\alpha$.}
\label{fig:sigmoids}
\end{figure}

Робота Цибенко присвячена на той час широкозастосованій апроксимації функції $f:
\mathbb{R}^m \to \mathbb{R}$ за допомогою наступної суми (дивись
\cite{old-nets}):
\begin{equation}\label{eq:cybenko-g}
	\widehat{f}(\mathbf{x}) = \sum_{j=1}^n \alpha_j \sigma(\boldsymbol{w}_j^{\top}\mathbf{x} + \beta_j), \quad \boldsymbol{w}_j \in \mathbb{R}^m, \quad \alpha_j,\beta_j \in \mathbb{R}.
\end{equation}

Таким чином, ми маємо відносно просту параметризацію, що складається з
$\mathcal{O}(mn)$ параметрів. 

\subsection{Узгодження з сучасною термінологією}
Більш того, цю архітектуру достатньо легко описати
на сучасній термінології нейронних мереж: розглянемо модель з
Рис.~\ref{cybenko-net}. Діаграму читаємо наступним чином: кожен нейрон (коло)
відповідає певному дійсному значенню з $\mathbb{R}$. Вхідний шар має $m$
нейронів, що відповідають вхідному вектору $\mathbf{x} \in \mathbb{R}^m$.
Наступний крок --- це обрахунок $n$ виразів $\Sigma_j \gets
\boldsymbol{w}_j^{\top}\mathbf{x}+\beta_j$ для $1 \leq j \leq n$. Ці вирази
подаються на вхід сігмоїдальній функції $\sigma$, що називають
\textit{активаційною функцією}, що видає значення \textit{скритого шару} $z_j =
\sigma(\Sigma_j)$. Нарешті, вихідний шар це просто лінійна комбінація значень
скритого шару з вагами $\alpha_j$\footnote{Зараз такий б перехід на вихідний шар
би назвали звичайним шаром без активаційної функції}:
\begin{equation*}
	\widehat{f}(\mathbf{x}) = \sum_{j=1}^n \alpha_j z_j = \sum_{j=1}^n \alpha_j \sigma(\boldsymbol{w}_j^{\top}\mathbf{x} + \beta_j).
\end{equation*}

Альтернативно, ``на сучасний лад'' зараз цю формулу більшість дослідників записали б в наступному вигляді:
\begin{equation*}\label{eq:modern_cybenko}
	\widehat{f}(\mathbf{x}) = \boldsymbol{\alpha}^{\top}\sigma(\boldsymbol{W}\mathbf{x} + \boldsymbol{\beta}),
\end{equation*}

де $\boldsymbol{\alpha} \in \mathbb{R}^n$ --- вектор ваг скритого шару,
$\boldsymbol{W} \in \mathbb{R}^{m \times n}$ --- матриця ваг, а
$\boldsymbol{\beta} \in \mathbb{R}^n$ --- вектор зсувів (biases). 
\begin{remark}
    Тут і далі запис $\sigma(\mathbf{z})$ для вектору $\mathbf{z} \in
    \mathbb{R}^n$ розуміємо як вектор $(\sigma(z_1),\dots,\sigma(z_n)) \in \mathbb{R}^n$.
\end{remark}

\begin{figure}
	\centering
	\begin{tikzpicture}
		% Define layers and node style
		\tikzset{input-neuron/.style={
			circle, 
			draw=green!80!black, 
			line width=0.5mm,
			fill=green!20!white,
			minimum size=0.5cm
		}}
		\tikzset{hidden-neuron/.style={
			circle, 
			draw=blue!80!black, 
			line width=0.5mm,
			fill=blue!20!white,
			minimum size=0.5cm
		}}
		\tikzset{output-neuron/.style={
			circle, 
			draw=orange!80!black, 
			line width=0.5mm,
			fill=orange!20!white,
			minimum size=0.5cm
		}}
		
		% Input layer
		\foreach \i in {1, 2, 3} {
			\node[input-neuron] (I\i) at (0, -1.25*\i) {$x_{\i}$};
		}
	
		% Hidden layer
		\foreach \j in {1, 2, 3, 4, 5} {
			\node[hidden-neuron] (h\j) at (4, -1.25*\j + 1.25) {$\Sigma$};
			\node[hidden-neuron] (H\j) at (6, -1.25*\j + 1.25) {$z_{\j}$};
			\draw[very thick,-{Stealth[length=3.5mm]}] (h\j) to [edge label=$\sigma$] (H\j);
		}
	
		% Output layer
		\node[output-neuron] (O) at (10, -2.5) {$\widehat{f}(\mathbf{x})$};
	
		% Draw connections from input to hidden layer
		\foreach \i in {1, 2, 3} {
			\foreach \j in {1, 2, 3, 4, 5} {
				\draw[very thick,-{Stealth[length=3.5mm]}] (I\i) -- (h\j);
			}
		}
	
		% Draw connections from hidden to output layer
		\foreach \j in {1, 2, 3, 4, 5} {
			\draw[very thick,-{Stealth[length=3.5mm]}] (H\j) -- (O);
		}
	
		% Labels
		\node[above, align=center] at (0, 0.5) {\textcolor{green!80!black}{\textbf{Вхідний шар}}\\$m$ нейронів};
		\node[above, align=center] at (5, 0.5) {\textcolor{blue!80!black}{\textbf{Скритий шар}}\\$n$ нейронів};
		\node[above, align=center] at (10, 0.5) {\textcolor{orange!80!black}{\textbf{Вихідний шар}}\\1 нейрон};
	
	\end{tikzpicture}
	\caption{Архітектура нейронної мережі з оригінальної роботи Цибенко \cite{cybenko} для випадку $m=3$, $n=5$. Стрілочки позначають передачу значення з відповідною вагою.}
	\label{cybenko-net}
\end{figure}

\subsection{Теореми Цибенко}

Нехай $\mathcal{Q}_m = [0,1]^m$ є $m$-вимірним одиничним гіперкубом. Простір
неперервних функцій $f: \mathcal{Q}_m \to \mathbb{R}$ на $\mathcal{Q}_m$
позначимо як $\mathcal{C}(\mathcal{Q}_m)$ і введемо норму функції $f \in
\mathcal{C}(\mathcal{Q}_m)$ як:
\begin{equation*}
    \|f\|_{\mathcal{Q}_m} = \sup_{\mathbf{x} \in \mathcal{Q}_m} |f(\mathbf{x})|.
\end{equation*}

Один з головних результатів, отриманих Цибенко, наступний:
\begin{theorem}\label{theorem:cybenko_1}
    Нехай $\sigma$ будь-яка неперервна сігмоїдальна функція. Суми вигляду
    $\widehat{f}(\mathbf{x}) = \sum_{j=1}^n
    \alpha_j\sigma(\boldsymbol{w}_j^{\top}\mathbf{x} + \beta_j)$ є щільними у
    $\mathcal{C}(\mathcal{Q}_m)$ та $L^1(\mathcal{Q}_m)$. Інакшими словами, для
    будь-якої функції $f \in \mathcal{C}(\mathcal{Q}_m)$ та $\varepsilon > 0$,
    існує сума $\widehat{f}(\mathbf{x})$ така, що:
    \begin{enumerate}[(A)]
        \item $|\widehat{f}(\mathbf{x})-f(\mathbf{x})|<\varepsilon$ для всіх $\mathbf{x} \in
        \mathcal{Q}_m$.
        \item $\int_{\mathcal{Q}_m}|\widehat{f}(\mathbf{x})-f(\mathbf{x})|d\mathbf{x} <
        \varepsilon$.
    \end{enumerate}
\end{theorem}

Дуже просто цю теорему можна пояснити наступним чином: для будь-якої неперервної
на $\mathcal{Q}_m$ функції $f$ знайдеться параметризації нейронної мережі, що
дозволить апроксимувати за допомогою $\widehat{f}$ цю функцію з довільною точністю.
Зауважимо, що це \textit{теорема про існування} і вона не є конструктивною: вона
не дає алгоритму, який знаходить параметри
$\{\alpha_j,\boldsymbol{w}_j,\beta_j\}_{1 \leq j \leq n}$ для довільної функції
$f$ і навіть не показує, чи можна їх знайти за допомогою алгоритмів оптимізації.

Окрім доведення теореми про апроксимацію, Цибенко також показав, що задана сума
$\widehat{f}$ може апроксимувати класифікатор на $\mathcal{Q}_m$ з довільною
точністю. Більш конкретно, нехай $\mathcal{P}_0,\dots,\mathcal{P}_{C-1}$ ---
розбиття $\mathcal{Q}_m$ на $C$ підмножин (що називають \textit{класами}). Нехай
маємо функцію $f: \mathcal{Q}_m \to \{0,\dots,C-1\}$, що задана за наступним
правилом:
\begin{equation*}
    f(\mathbf{x}) = j \iff \mathbf{x} \in \mathcal{P}_j.
\end{equation*}

Ця функція, вочевидь, не є неперервною на $\mathcal{Q}_m$, тому Теорему
\ref{theorem:cybenko_1} застосувати не можна. Проте, можна показати, що і цю
функцію ми можемо апроксимувати за допомогою суми $\widehat{f}$ з довільною
точністю. Це дозволяє використовувати нейронні мережі для класифікації даних.
Розглянемо наступну теорему.
\begin{theorem}\label{theorem:cybenko_2}
    Нехай $\sigma$ будь-яка неперервна сігмоїдальна функція і функція
    $f$ задана як вище. Тоді для будь-якої такої функції існує сума
    \begin{equation*}
        \widehat{f}(\mathbf{x}) = \sum_{j=1}^n \alpha_j \sigma(\boldsymbol{w}_j^{\top}\mathbf{x} + \beta_j)
    \end{equation*}
    та множина $\mathcal{D} \subseteq \mathcal{Q}_m$ така, що міра
    $\mu(\mathcal{D}) \geq 1-\varepsilon$ та $|\widehat{f}(\mathbf{x}) -
    f(\mathbf{x})| < \varepsilon$ для всіх $\mathbf{x} \in \mathcal{D}$.
\end{theorem}

На відміну від Теореми \ref{theorem:cybenko_1}, Теорема \ref{theorem:cybenko_2}
не гарантує апроксимацію на усьому гіперкубі $\mathcal{Q}_m$. Проте, зі
збільшенням точності (тобто, зменьшенням $\varepsilon$) ми можемо збільшувати
міру тої області $\mathcal{D}$, на якій апроксимація ``гарна'' (себто в тій
області, на якій відхилення $\widehat{f}(\mathbf{x})$ від $f(\mathbf{x})$ меньше за
$\varepsilon$).

\subsection{Практичний Приклад}

\begin{example}
	Розглянемо більш конкретний приклад. Нехай нам потрібно побудувати 
	класифікатор для двох класів на квадраті $\mathcal{Q}_2$ (класифікацію з двох 
	класів називають \textit{бінарною}). Задамо дві області:
	\begin{equation*}
		\mathcal{P}_1 := \left\{(x_1,x_2) \in \mathcal{Q}_2: a^2\left(x_2-0.5\right)^2 - b^2\left(x_1-0.5\right)^2 < 1\right\}, \; \mathcal{P}_0 := \mathcal{Q}_2 \setminus \mathcal{P}_1,
	\end{equation*}

	де обрано $a:=5,b:=2\sqrt{5}$. Іншими словами, наша задача --- це апроксимувати індикатор функцію
	$f(\mathbf{x}) = \mathds{1}[\mathbf{x} \in \mathcal{P}_1]$. Для наглядності,
	обидві області зображені на Рис.~\ref{fig:classification_example}. В якості
	сігмоїдальної функції оберемо функцію логістичної регресії: $\sigma(x) :=
	1/(1+e^{-x})$ та візьмемо $n=6$ нейронів у скритому шарі. Таким чином,
	функція $\widehat{f}$ матиме вигляд:
	\begin{equation*}
		\widehat{f}(\mathbf{x}) = \sum_{j=1}^6 \frac{\alpha_j}{1+e^{-\boldsymbol{w}_j^{\top}\mathbf{x} + \beta_j}}, \quad \boldsymbol{w}_j \in \mathbb{R}^2, \quad \alpha_j,\beta_j \in \mathbb{R}.
	\end{equation*}

	\begin{figure}
		\centering
		\includegraphics[width=0.75\textwidth]{code/cybenko/classification-example.pdf}
		\caption{Класи $\mathcal{P}_0$ та $\mathcal{P}_1$ на квадраті
		$\mathcal{Q}_2$. Разом з класами, зображено набір данних
		$\mathcal{D}=\{(\mathbf{x}_n,\mathds{1}(\mathbf{x}_n \in
		\mathcal{P}_1))\}_{1 \leq n \leq N} \subset \mathcal{Q}_2 \times
		\{0,1\}$.}
		\label{fig:classification_example}
	\end{figure}

	Питання: якими мають бути параметри для того, щоб функція $\widehat{f}$
	апроксимувала функцію $f$ з гарною точністю? Виявляється, що достатньо
	непоганий результат можна отримати використовуючи наступні параметри:
	\begin{align*}
		\boldsymbol{\alpha} &\approxeq (-8.18, 3.81, 3.91, -3.41, 5.07, 1.16), \\
		\boldsymbol{\beta} &\approxeq (1.13, -2.20, 1.72, 12.47, 8.46, 5.02), \\
		\boldsymbol{W} &\approxeq \begin{bmatrix}
			0.06 & 4.85 & -4.39 & -11.81 & -7.59 & 14.19 \\
			-6.44 & -7.67 & -6.76 & -15.67 & -10.26 & -19.17
		\end{bmatrix}^{\top}.
	\end{align*}
	Помітимо, що ми трошки скоротили запис, сформувавши з параметрів вектори та
	матриці, як це було зроблено у Формулі~\ref{eq:modern_cybenko}.
\end{example}

Зверніть, що на Рис.~\ref{fig:classification_example} ми також зображуємо набір
даних $\mathcal{D}$, що складається з $N=1000$ точок з відповідним маркуванням
(бітом), що відповідає класу, до якого належить точка. Головна причина цього ---
мати спосіб знайти параметри моделі $\widehat{f}$: ми можемо використати, наприклад,
алгоритм градієнтного спуску для мінімізації функції втрат. 

\begin{remark}[Про тренування моделі]
	Забігаючи вперед, для підбору оптимальних параметрів ми використовували
	середньоквадратичну функцію втрати:
	\begin{equation*}
		\mathcal{L}(\mathcal{D}|\boldsymbol{\theta}) = \frac{1}{N}\sum_{n=1}^N \left(\widehat{f}(\mathbf{x}_n|\boldsymbol{\theta}) - y_n\right)^2,
	\end{equation*}
	і далі використовували алгоритм градієнтного спуску (Adam Optimizer \cite{adam}) для мінімізації цього виразу відносно параметрів $\boldsymbol{\theta}$. Більше деталей наведено у Додатку~\ref{appendix:cybenko-code}.
\end{remark}

Після тренування, результати зображені на
Рис.~\ref{fig:classification_result}(а). Помітимо, що вихід
$\widehat{f}(\mathbf{x})$ не є бінарним, але можна ввести поріг $\tau \in
\mathbb{R}$ такий, що передбачення $\widehat{y} := \mathds{1}(\widehat{f}(\mathbf{x}) >
\tau)$ відповідає класу 1 за умови $\widehat{f}(\mathbf{x}) > \tau$, а інакше ---
класу 0. На Рис.~\ref{fig:classification_result}(б) зображено результати
класифікації для $\tau=0.62$. Як бачимо, класифікатор працює досить добре.

\begin{figure}
	\centering
	\begin{subfigure}{0.49\textwidth}
		\centering
		\includegraphics[width=0.99\textwidth]{code/cybenko/classification-cont-prediction.pdf}
		\caption{Результат класифікації $\widehat{f}(\mathbf{x})$ на квадраті $\mathcal{Q}_2$.}
	\end{subfigure}
	\begin{subfigure}{0.49\textwidth}
		\centering
		\includegraphics[width=0.99\textwidth]{code/cybenko/classification-discr-prediction.pdf}
		\caption{Бінарний результат класифікації $\mathds{1}(\widehat{f}(\mathbf{x})>\tau)$ з порогом $\tau \approxeq 0.62$.}
	\end{subfigure}
	\caption{Результати класифікації для класів $\mathcal{P}_0$ та $\mathcal{P}_1$ на квадраті $\mathcal{Q}_2$.}
	\label{fig:classification_result}
\end{figure}

Також для цікавості, можна побудувати подібне зображення передбачень, але для 
кожного нейрону. На Рис.~\ref{fig:classification_neuron} зображено результати
передбачень для кожного нейрону у скритому шарі.

\begin{figure}
\begin{tikzpicture}[
    node distance=1.5cm and 1.5cm,
    every node/.style={inner sep=0pt, anchor=center},
    ->, >=Stealth
]
    % Initial image node
    \node (input) {
        \includegraphics[trim={1cm 0.5cm 3.75cm 1cm},clip,width=6cm]{code/cybenko/dataset.pdf}
    };

    % Hidden layer nodes arranged horizontally below the input
    \foreach \i [count=\j from 1] in {1, 2, 3, 4, 5, 6} {
        \node[below=2.25cm of input, xshift=(\j - 3.5) * 2.75cm] (hidden\i) {
            \includegraphics[trim={1cm 0.5cm 3.75cm 1cm},clip,width=2.75cm]{code/cybenko/layer-\i-prediction.pdf}
        };
        % Connect input to each hidden layer node
        \draw[ultra thick,-{Stealth[length=3.5mm]}] (input.south) -- (hidden\i.north);
    }

    % Final prediction node centered below the hidden layer
    \node[yshift=-12.5cm, xshift=-4.0cm] (output) {
        \includegraphics[trim={1cm 0.5cm 3.75cm 1cm},clip,width=6cm]{code/cybenko/classification-cont-prediction.pdf}
    };

	% Discrete prediction node right to the final prediction
	\node[right=2.5cm of output] (discrete) {
		\includegraphics[trim={1cm 0.5cm 3.75cm 1cm},clip,width=6cm]{code/cybenko/classification-discr-prediction.pdf}
	};

    % Connect hidden layer nodes to final output
    \foreach \i in {1, 2, 3, 4, 5, 6} {
        \draw[ultra thick,-{Stealth[length=3.5mm]}] (hidden\i.south) -- (output.north);
    }

	% Draw a transparent green rectangle line on the level of the input layer and 
	% make a transparent green background for the input layer
	\scoped[on background layer]\draw[green!50!black, fill=green, line width=0.5mm,
	fill opacity=0.2,dashed] ([xshift=-2.5cm,yshift=0.5cm]input.north west) rectangle ([xshift=2.5cm,yshift=-0.5cm]input.south east);

	% Same for hidden layer with a blue color and orange for output
	\scoped[on background layer]\draw[blue!50!black, fill=blue, line width=0.5mm,
	fill opacity=0.2,dashed] ([xshift=-8.5cm,yshift=0.5cm]hidden3.north east) rectangle ([xshift=8.5cm,yshift=-0.5cm]hidden3.south east);
	\scoped[on background layer]\draw[orange!50!black, fill=orange, line width=0.5mm,
	fill opacity=0.2,dashed] ([xshift=-6.5cm,yshift=0.10cm]output.north east) rectangle ([xshift=9.0cm,yshift=-0.10cm]output.south east);

	% Labels written verfically left to the boxes
	\node[rotate=90] at ([xshift=-6.0cm]input) {\textcolor{green!80!black}{\textbf{Вхідний шар}}};
	\node[rotate=90] at ([xshift=-7.50cm]hidden3) {\textcolor{blue!80!black}{\textbf{Скритий шар}}};
	\node[rotate=90] at ([xshift=-4.0cm]output) {\textcolor{orange!80!black}{\textbf{Вихідний шар}}};

	% Between input layer and hidden layer, write down x -> w^T x + b in a box with fill
	\node[below=0.60cm of input, fill=blue!20!white, dashed, minimum size=1cm, text width=5cm, align=center, rounded corners=.55cm] {\normalsize$\mathbf{x} \xrightarrow{\boldsymbol{w}^{\top}_j\mathbf{x} + \beta_j} \Sigma_j \xrightarrow{\sigma} z_j$};

	% Between hidden layer and output layer, write down sum of z_j * alpha_j
	\node[above=0.40cm of output, fill=orange!20!white, dashed, minimum size=1cm, text width=5cm, align=center, rounded corners=.55cm] {\normalsize$\widehat{f}(\mathbf{x}) \gets \langle \boldsymbol{\alpha}, \mathbf{z} \rangle$};

	% And arrow as well with a text
	\draw[ultra thick,-{Stealth[length=3.5mm]}] (output.east) -- (discrete.west) node[midway, above=0.25cm, align=center] {Поріг $\tau$};
\end{tikzpicture}

\caption{Результати передбачень для кожного нейрону у скритому шарі.}
\label{fig:classification_neuron}

\end{figure}

\subsection{Подальший розвиток архітектури Цибенко}

Звичайно, що науковці не зупинились на результатах Цибенко. В подальших 
архітектурах, дослідники ставили багато питань, таких як:
\begin{itemize}
	\item Що, якщо зробити кілька скритих шарів у мережі?
	\item Чи можна використовувати інші активаційні функції окрім сігмоїдів?
	\item А чи можна поєднувати два скритих шара іншим способом?
\end{itemize}

Саме тому, була введена багатошарова модель персептронів (Multi-Layer
Perceptrons --- MLP), яку ми сформулюємо нижче.
\begin{definition}[Багатошарова модель персептронів]\label{def:mlp}
	Нехай $\ell \in \mathbb{N}$ --- кількість шарів у мережі, а
	$n_0,\dots,n_{\ell} \in \mathbb{N}$ --- кількість нейронів у кожному шарі,
	де $\mathbf{x}^{\langle 0 \rangle} \in \mathbb{R}^{n_0}$ відповідає вхідному
	шару. Тоді, для знаходження виходу $\mathbf{x}^{\langle \ell \rangle} \in
	\mathbb{R}^{n_{\ell}}$, багатошарова модель персептронів використовує
	наступне рекурентне правило:
	\begin{equation*}
		\mathbf{x}^{\langle j+1 \rangle} = \phi^{\langle j \rangle}(\mathbf{z}^{\langle j \rangle}), \quad \mathbf{z}^{\langle j \rangle} = \boldsymbol{W}^{\langle j \rangle}\mathbf{x}^{\langle j \rangle} + \boldsymbol{\beta}^{\langle j \rangle}, \quad j = 0,\dots,\ell-1,
	\end{equation*}
	де $\phi^{\langle j \rangle}$ --- активаційна функція у шарі $j$, а
	$\boldsymbol{W}^{\langle j \rangle} \in \mathbb{R}^{n_{j+1} \times n_j}$ та
	$\boldsymbol{\beta}^{\langle j \rangle} \in \mathbb{R}^{n_{j+1}}$ ---
	матриця ваг та вектор зсуву у шарі $j$ відповідно. Таким чином,
	параметризація моделі є $\boldsymbol{\theta} =
	\left\{\boldsymbol{W}^{\langle j \rangle},\boldsymbol{\beta}^{\langle j
	\rangle}\right\}_{0 \leq j \leq \ell-1}$.
\end{definition}

\subsubsection{Навіщо більше шарів?}

Здавалося б, якщо ми можемо апроксимувати будь-яку функцію за допомогою одного
скритого шару, то навіщо нам багатошарові моделі? Виявляється, що більше шарів
дозволяють нам апроксимувати складніші функції за допомогою меншої кількості
нейронів і чисельно знаходити їх стає простіше. Емпірично, набір правил, що
описують ефективність моделі від кількості тренувальних параметрів, розміру
набору даних та інших факторів, називають законами масштабування
(\textit{Scaling Laws}). Зокрема, емпірично, середнє значення функції втрати $L$
залежить від кількості параметрів $N$ як $L \propto N^{-\alpha}$ для певної
константи $\alpha>0$. Більш детальне дослідження від інших параметрів таких як
гіперпараметри архітектури, розміру набору даних можна подивитися у джерелі
\cite{scaling}. Зокрема, джерело \cite{params-generalization} строго показує
асимптотичну залежність між кількістю параметрів та загальнізацією моделі для
задачі класифікації, а джерело \cite{params-generalization-2} доводить, що якщо
розмір вибірки $D$ і вхід складається з $m$ нейронів, то при кількості
параметрів $N=(D/(m\log D))^{1/2}$, то, асимтотично, $L_2$ норма різниці між
апроксимацією та реальною функцією обмежена як $\mathcal{O}((m/D)\log D)^{1/2}$.

\subsubsection{Навіщо інші активаційні функції?}
Окрім того, на практиці, логістична функція виявляється дуже незручною. Зокрема,
вона відноситься до класу функцій, що називаються \textit{ненасиченими} (або
\textit{non-saturating activation function}). Основна проблема полягає у тому,
що під час навчання моделі, ми використовуємо градієнтні методи, які полагаються
на значення матриці Якобіана функції втрати відносно параметрів моделі.
Спрощено, моделі змінюються на мале значення $\delta\boldsymbol{\theta}
\approxeq \eta
(\partial\mathcal{L}/\partial\boldsymbol{\theta})\Big|_{\boldsymbol{\theta}=\boldsymbol{\theta}_{\text{current}}}$.
У ході обчислення цього градієнту, ми використовуємо правило ланцюга, що
призводить до того, що кожен доданок у виразі містить вираз $\sigma'(z)$ у
добутку. Для логістичної функції, похідна $\sigma'(z) = \sigma(z)(1-\sigma(z))$ і
легко бачити, що як для малих, так і для великих значень $z$, ця похідна
дуже стрімко наближається до нуля, що призводить до проблеми \textit{вицідження
градієнту} (\textit{vanishing gradient problem}). Це означає, що градієнт
функції втрати може бути дуже малим, що призводить до того, що модель не
навчається. Зокрема, нижче ми наводимо популярні функції, що використовують 
для розв'язку цієї проблеми:
\begin{enumerate}
	\item \textbf{ReLU} (Rectified Linear Unit): $\phi(z) = \max\{0,z\}$. Ця 
	функція має похідну $\phi'(z) = \mathds{1}(z>0)$, тобто за додатніх
	значень $z$ градієнт не виціджується. Проте, для від'ємних значень $z$,
	градієнт нульовий, через що модель може ``тормозити''. Для виріження цієї 
	проблеми, було запропоновано модифікації ReLU, такі як Leaky ReLU.
	\item \textbf{Leaky ReLU}: $\phi(z) = \max\{\alpha z,z\}$ де $\alpha \in
	[0,1)$ --- мале значення (на практиці, порядку $10^{-3}$). Ця функція має
	похідну $\phi'(z)\Big|_{z<0} = \alpha$, що дозволяє градієнту не
	виціджуватись для від'ємних значень $z$.
	\item \textbf{ELU} (Exponential Linear Unit): $\phi(z) =
	\max\{0,z\}+\min\{\alpha(e^z-1),0\}$, де $\alpha$ --- додатне значення. Ця
	функція має похідну $\phi'(z)\Big|_{z<0} = \alpha e^z$. За великих від'ємних
	значень $z$, ця функція збігається з ReLU, але має неперервну та нестрого
	нульову похідну для від'ємних значень $z$.
\end{enumerate}

\subsubsection{Інша архітектура}

Ще одним питанням, яке виникає, це як поєднувати шари у мережі. Поки що, логіка
така: кожен нейрон $i$ у шарі $\ell$ з'єднаний з кожним нейроном $j$ у шарі
$\ell+1$ з певною вагою $w_{i,j}^{\langle \ell \rangle}$ (що і утворюють матрицю
ваг $\boldsymbol{W}^{\langle \ell \rangle}$). Проте, чи дійсно нам потрібно 
стільки зв'язків? І чи дійсно така репрезентація здатна відобразити достатньо
складні стосунки за малу кількість параметрів? Виявляється, що для певних задач
можна використовувати інші архітектури. Нижче наведемо два приклади, що 
показують проблематику зв'язків у мережі.

\begin{example}
	Уявіть, що вам потрібно побудувати бінарну класифікацію для кольорового
	зображення відносно невеликого розміру --- скажімо, $200 \times 200$
	пікселів. Якщо у якості входу взяти кожен окремий піксель як нейрон, то
	кількість параметрів у першому шарі буде $200 \times 200 \times 3 = 120000$.
	Уявімо, що у скритий шар ми поставимо буквально 10 нейронів (на практиці,
	така кількість мала для досягнення хорошої точності). Таким чином, кількість
	параметрів у моделі буде як мінімум $120000 \times 10 = 1.2$ млн. 
\end{example}

\begin{example}
	Що робити, якщо розмір входу та виходу, взагалі кажучи, не фіксовані 
	(наприклад, обробка тексту)? Звичайно, можна задати ``зазделегідь''
	максимальний розмір входу та виходу, але окрім аномальної кількості
	параметрів, точність такої моделі може бути дуже низькою.
\end{example}

У рамках цієї курсової роботи, ми не беремось за повний і строгий опис всіх
сучасних архітектур, проте навели основний фундамент для подальшого вивчення
глибокого навчання. Для більш детального огляду, рекомендуємо звернутися до
джерел \cite{book-1,book-2}.

\section{Теорема Колмогорова-Арнольда}

\subsection{Історія та Мотивація}

Ще один дуже цікавий спосіб підходу до апроксимації функцій --- це використання
теореми Колмогорова-Арнольда. Історично, ця теорема походить з 13 задачі
Гільберта, яка була запропонована на Паризькому конгресі математиків у 1900 році.
Вона задає доволі провокативне питання: а чи існують справді неперервні 
дійснозначні багатовимірні функції? Здавалося б, доволі дивне запитання,
але воно по суті і описує 13 задачу і, відповідно, її розв'язок --- теорему
Колмогорова-Арнольда. 

Більш конкретно, питання полягає у тому, чи можна будь-яку, скажімо, неперервну
функцію $f: \mathcal{Q}_m \to \mathbb{R}$ апроксимувати за допомогою суми та
композицій певного набору одновимірних неперервних функцій
$\phi_1,\dots,\phi_N$? Розглянемо декілька прикладів на основі
\cite{ka-explained}, щоб показати суть цієї задачі.

\begin{example}
	Нехай $f: \mathcal{Q}_2 \to \mathbb{R}$ задана як $f(x,y) = 3x+5y$. Якщо
	позначити $\phi_1(x)=3x$, $\phi_2=5y$, то $f(x,y) = \phi_1(x)+\phi_2(y)$.
	Отже, маємо функцію двох змінних, проте вона може бути записана як сума двох
	функцій однієї змінної.
\end{example}

\begin{example}
	Попередній приклад здається зовсім тривіальним. А що, якщо $f(x,y)=xy$? Помітимо 
	наступне\footnote{Тут і далі під записом $\log$ розуміємо натуральний логарифм}:
	\begin{equation*}
		f(x,y) = xy = e^{\log (x+1) + \log(y+1)} + \left(-x-0.5\right) + \left(-y-0.5\right).
	\end{equation*}
	Нехай $\phi_1(x) := e^x$, $\phi_2(x) := \log(x+1)$,
	$\phi_3 := -x-0.5$. Тоді:
	\begin{equation*}
		f(x,y) = \phi_1(\phi_2(x)+\phi_2(y)) + \phi_3(x) + \phi_3(y).
	\end{equation*}

	Отже, функція $f(x,y)$ може бути записана як сума та композиція функцій
	однієї змінної $\phi_1(x),\phi_2(x),\phi_3(x)$.
\end{example}

\begin{example}
	Нехай $f(x,y) = \sin (10e^x + y^{100})$, а також $\phi_1(x) := 10e^x,
	\phi_2(x) := y^{100}$ та $\phi_3(x) := \sin x$. Тоді $f(x,y) =
	\phi_3(\phi_1(x)+\phi_2(y))$, тобто так само маємо $f(x,y)$ як композицію та
	суму одновимірних функцій.
\end{example}

Отже, на основі цих прикладів можна зрозуміти суть 13 проблеми Гільберта.
\begin{statement}[Основна гіпотеза 13 проблеми Гільберта]
	Існує неперервна функція $f: \mathcal{Q}_3 \to \mathbb{R}$, що не може бути
	виражена як композиція та сума неперервних функцій $\phi_1,\dots,\phi_N \in
	\mathcal{C}(\mathbb{R}^2)$.
\end{statement}

Знадобилося більше 50 років для того, щоб довести, що це твердження
\textit{хибне}. У 1956 році Колмогоров довів, що функція будь-якої кількості
змінних (себто, $\mathcal{Q}_m$ може бути довільним гіперкубом) може бути
записана як сума та композиція трьохвимірних функцій. У 1957, у 19 років,
Арнольд показав, що три змінні можна замінити на дві, що власне і розв'язує в
більш загальному вигляді 13 проблему Гільберта. Нарешті, згодом Колмогоров 
показав, що дві змінні можна замінити на одну, що врешті-решт і дає 
відому теорему Колмогорова-Арнольда.

\begin{theorem}[Згідно джерелу \cite{ka-explained}]
	Для будь-якого натурального $m \geq 2$, існують неперервні функції
	$\phi_1,\dots,\phi_{2m+1} \in \mathcal{C}([0,1])$ та дійсні числа
	$\lambda_1,\dots,\lambda_m \in \mathbb{R}$ з такою властивістю, що для
	будь-якої функції $f \in \mathcal{C}(\mathcal{Q}_m)$ знайдеться неперервна
	функція $\Phi: \mathbb{R} \to \mathbb{R}$ така, що для будь-якого
	$\mathbf{x} = (x_1,\dots,x_m) \in \mathcal{Q}_m$ справедливо:
	\begin{equation*}
		f(\mathbf{x}) = \sum_{q=1}^{2m+1}\Phi\left(\sum_{p=1}^{m}\lambda_p\phi_q(x_p)\right)
	\end{equation*}
\end{theorem}

\begin{remark}\label{remark:ka-remark}
	В цій формулі дуже багато чого цікавого! По-перше, дивує сам факт не наближеної
	апроксимації, а точної рівності. По-друге, важливо помітити, що функції
	$\phi_1,\dots,\phi_{2m+1}$ не залежать від $f$, але залежать від $m$. Це
	означає, що ми можемо знайти ці функції один раз, і вони будуть працювати для
	будь-якої функції $f$! Після чого залишиться лише знайти $\Phi$.
\end{remark}

Здавалося б, враховуючи Зауваження~\ref{remark:ka-remark}, чому ми не можемо
спочатку знайти ці функції, а потім використовувати їх у прикладних задачах?
Річ у тому, що ніхто не гарантує, що ці функції взагалі мають бути диференційованими
і тим паче неперервно диференційованими. Саме тому, подальший пошук функції $\Phi$
автоматично стає майже неможливим завданням. Тому, як ми можемо використовувати
теорему Колмогорова-Арнольда для побудови нейронних мереж?

\subsection{Мережі Колмогорова-Арнольда}

Довгий час ідею теореми Колмогорова-Арнольда не пробували застосовувати до
глибокого навчання через ``поганість'' функцій $\Phi,\phi_1,\dots,\phi_{2m+1}$.
Проте, буквально у цьому році найпоширенішою темою дискусії у суспільстві
розробників глибокого навчання стала робота ``KAN: Kolmogorov-Arnold Networks''
\cite{kan}. По суті, до моменту публікації, єдина парадигма апроксимації функцій
полягала у побудові MLP мереж (та їх подальших варіацій у вигляді конволюційних,
рекурентних мереж тощо), що грунтується на вище описаній універсальній теоремі
апроксимації \ref{theorem:cybenko_1}. Однак, автори роботи \cite{kan} показали,
що і на основі репрезентації Колмогорова-Арнольда можна побудувати нейронні мережі. 
Яким чином?

Робота \cite{kan} використовує оригінальну теорему Колмогорова-Арнольда \cite{kolmogorov-original}.
\begin{definition}[Оригінальна теорема Колмогорова \cite{kolmogorov-original}]
	Для будь-якого натурального $m \geq 2$, існують неперервні функції
	$\phi_{p,q} \in \mathcal{C}([0,1])$ такі, що для будь-якої функції $f \in
	\mathcal{C}(\mathcal{Q}_m)$ знайдуться неперервні функції
	$\Phi_1,\dots,\Phi_{2m+1} \in \mathcal{C}(\mathbb{R})$ такі, що
	\begin{equation*}
		f(x_1,\dots,x_m) = \sum_{q=1}^{2m+1}\Phi_q\left(\sum_{p=1}^n \phi_{p,q}(x_p)\right)
	\end{equation*}
\end{definition}

Проте, як і для Означення~\ref{def:mlp} MLP мереж, нам потрібно вміти узагальнювати
означення для довільної кількості шарів та нейронів в кожному шарі. Робота 
\cite{kan} стала першою, що запропонувала таку узагальнену модель. Спочатку,
наведемо що є \textit{з'єднанням в KAN мережі}.

\begin{definition}
	\textbf{З'єднання KAN Мережі} між шаром з $n_{\text{in}} \in \mathbb{N}$
	нейронами (активаціями) та шаром з $n_{\text{out}} \in \mathbb{N}$ нейронами
	складається з матриці функцій $\boldsymbol{\Phi} = \{\phi_{q,p}\}_{1\leq
	p\leq n_{\text{in}}, 1 \leq q \leq n_{\text{out}}}$, де кожна функція
	$\phi_{q,p}: \mathbb{R} \to \mathbb{R}$ параметризується параметрами
	$\boldsymbol{\theta}_{q,p}$. Значення нейронів (активацій)
	$\mathbf{x}_{\text{out}} \in \mathbb{R}^{n_{\text{out}}}$ через попередні
	нейрони $\mathbf{x}_{\text{in}} \in \mathbb{R}^{n_{\text{in}}}$ визначається
	згідно рівнянню $\mathbf{x}_{\text{out}} = \boldsymbol{\Phi} \circ
	\mathbf{x}_{\text{in}}$, де під виразом $\boldsymbol{\Phi} \circ
	\mathbf{x}_{\text{in}} \in \mathbb{R}^{n_{\text{out}}}$, по аналогії з
	матричним добутком, мається на увазі:
	\begin{equation*}
		(\boldsymbol{\Phi} \circ \mathbf{x}_{\text{in}})_j = \sum_{i=1}^{n_{\text{in}}}\phi_{j,i}(x_{\text{in},i}), \quad j \in \{1,\dots,n_{\text{out}}\}.
	\end{equation*}
\end{definition}

Отже, ми можемо дати означення безпосередньо мережі.
\begin{definition}[Архітектура KAN]
	Нехай $\ell \in \mathbb{N}$ --- кількість шарів у мережі, а
	$n_0,\dots,n_{\ell} \in \mathbb{N}$ --- кількість нейронів у кожному шарі,
	де $\mathbf{x}^{\langle 0 \rangle} \in \mathbb{R}^{n_0}$ відповідає вхідному
	шару. Тоді, для знаходження виходу $\mathbf{x}^{\langle \ell \rangle} \in
	\mathbb{R}^{n_{\ell}}$, архітектура KAN використовує рекурентне правило
	$\mathbf{x}^{\langle j+1 \rangle} = \boldsymbol{\Phi}^{\langle j \rangle}
	\circ \mathbf{x}^{\langle j \rangle}$, $j \in \{0,\dots,\ell-1\}$, де
	$\boldsymbol{\Phi}^{\langle j \rangle} =
	\{\phi^{\langle j \rangle}_{q,p}\}_{1\leq p\leq n_j, 1 \leq q \leq
	n_{j+1}}$ --- матриця функцій-ваг $j$. В розгорнутому вигляді, нейронна мережа $\widehat{f}_{\text{KAN}}$ записується як:
	\begin{equation*}
		\widehat{f}_{\text{KAN}}(\mathbf{x}) = \left(\bigcirc_{j=1}^{\ell}\boldsymbol{\Phi}^{\langle \ell - j \rangle}\right)\circ \mathbf{x}
	\end{equation*}
\end{definition}

Таке визначення може здатися доволі заплутаним, тому давайте розглянемо приклад.

\begin{example}[Теорема Арнольда як частковий випадок KAN]
	Нагадаємо, що теорема Колмогорова-Арнольда стверджує, що будь-яка функція 
	$f \in \mathcal{C}(\mathcal{Q}_m)$ може бути записана як
	\begin{equation*}
		f(x_1,\dots,x_m) = \sum_{q=1}^{2m+1}\Phi_q\left(\sum_{p=1}^m \phi_{p,q}(x_p)\right).
	\end{equation*}

	Тоді, якщо ми визначимо дві матриці-ваги $\boldsymbol{\Phi}^{\langle 0 \rangle} = \{\phi_{p,q}\}_{1 \leq p \leq m, 1 \leq q \leq 2m+1}$ та
	$\boldsymbol{\Phi}^{\langle 1 \rangle} = \{\Phi_q\}_{1 \leq q \leq 2m+1}$ (матриця-рядок), то ми можемо записати
	\begin{equation*}
		f(x_1,\dots,x_m) = \widehat{f}_{\text{KAN}}(x_1,\dots,x_m) = \boldsymbol{\Phi}^{\langle 1 \rangle} \circ \boldsymbol{\Phi}^{\langle 0 \rangle} \circ \mathbf{x}
	\end{equation*}
\end{example}

Отже, залишається лише обрати параметризацію для кожної функції
$\phi_{p,q}^{\langle j \rangle}$. Оригінальна робота \cite{kan} пропонує 
використовувати лінійну комбінацію певної фіксованої базисної функції $\beta(x)$ та $B$-сплайн порядку $d$:
\begin{equation*}
	\phi_{p,q}^{\langle j \rangle}(x) = \omega_{\beta} \beta(x) + \omega_S\sum_{k=1}^{d} \theta_{p,q,k}^{\langle j \rangle} B_{k,d}(x),
\end{equation*}
де $B_{k,d}(x)$ --- $k$-тий $B$-сплайн порядку $d$, $\theta_{p,q,k}^{\langle j
\rangle}$ --- параметри моделі і $\omega_{\beta},\omega_S$ --- фіксовані ваги
для базисної функції та $B$-сплайнів відповідно (що є гіперпараметрами). В
роботі пропонується обрати $\beta(x) := x\sigma(x)$. 

Архітектура KAN для двох шарів зображена на Рис.~\ref{fig:kan-architecture}.

\begin{figure}
\centering
\begin{tikzpicture}[node distance=1.5cm, every node/.style={align=center}]
    % Layer 0 (Input Layer)
    \node[circle, draw, ultra thick, green!50!black, fill=green!20, minimum size=0.5cm] (x01) {$x_{1}^{\langle 0 \rangle}$};
    \node[circle, draw, ultra thick, green!50!black, fill=green!20, minimum size=0.5cm, right=1cm of x01] (x02) {$x_{2}^{\langle 0 \rangle}$};

    % Applying phi functions
    \node[rectangle, draw, ultra thick, blue!50!black, fill=blue!20, minimum width=1cm, minimum height=1cm, above left=1.5cm and 3.0cm of x01, rounded corners=.20cm] (phi11) {$\phi_{1,1}^{\langle 0 \rangle}$};
    \node[rectangle, draw, ultra thick, blue!50!black, fill=blue!20, minimum width=1cm, minimum height=1cm, above left=1.5cm and 1.5cm of x01, rounded corners=.20cm] (phi21) {$\phi_{2,1}^{\langle 0 \rangle}$};
    \node[rectangle, draw, ultra thick, blue!50!black, fill=blue!20, minimum width=1cm, minimum height=1cm, above left=1.5cm and 0.0cm of x01, rounded corners=.20cm] (phi31) {$\phi_{3,1}^{\langle 0 \rangle}$};
    
	\node[rectangle, draw, ultra thick, blue!50!black, fill=blue!20, minimum width=1cm, minimum height=1cm, above right=1.5cm and 0.0cm of x02, rounded corners=.20cm] (phi12) {$\phi_{1,2}^{\langle 0 \rangle}$};
    \node[rectangle, draw, ultra thick, blue!50!black, fill=blue!20, minimum width=1cm, minimum height=1cm, above right=1.5cm and 1.5cm of x02, rounded corners=.20cm] (phi22) {$\phi_{2,2}^{\langle 0 \rangle}$};
    \node[rectangle, draw, ultra thick, blue!50!black, fill=blue!20, minimum width=1cm, minimum height=1cm, above right=1.5cm and 3.0cm of x02, rounded corners=.20cm] (phi32) {$\phi_{3,2}^{\langle 0 \rangle}$};
    
    % Finding new activations (Layer 1)
    \node[circle, draw, ultra thick, gray!80!black, fill=gray!20, minimum size=1cm, above right=5.0cm and 0.5cm of x01] (x12) {$\boldsymbol{\Sigma}$};
    \node[circle, draw, ultra thick, gray!80!black, fill=gray!20, minimum size=1cm, left=1.5cm of x12] (x11) {$\boldsymbol{\Sigma}$};
    \node[circle, draw, ultra thick, gray!80!black, fill=gray!20, minimum size=1cm, right=1.5cm of x12] (x13) {$\boldsymbol{\Sigma}$};

	% Draw labels left to each sum node
	\node[left=-0.15cm of x11, font=\small, gray!80!black] {$x_1^{\langle 1 \rangle}$};
	\node[left=-0.15cm of x12, font=\small, gray!80!black] {$x_2^{\langle 1 \rangle}$};
	\node[left=-0.15cm of x13, font=\small, gray!80!black] {$x_3^{\langle 1 \rangle}$};

	% Drawing phi functions above Layer 1
	\node[rectangle, draw, ultra thick, blue!50!black, fill=blue!20, minimum width=1cm, minimum height=1cm, above=1cm of x11, rounded corners=.20cm] (phi111) {$\phi_{1,1}^{\langle 1 \rangle}$};
	\node[rectangle, draw, ultra thick, blue!50!black, fill=blue!20, minimum width=1cm, minimum height=1cm, above=1cm of x12, rounded corners=.20cm] (phi121) {$\phi_{1,2}^{\langle 1 \rangle}$};
	\node[rectangle, draw, ultra thick, blue!50!black, fill=blue!20, minimum width=1cm, minimum height=1cm, above=1cm of x13, rounded corners=.20cm] (phi131) {$\phi_{1,3}^{\langle 1 \rangle}$};

	% Layer 3: Draw orange output above x12 at distance 1cm
	\node[circle, draw, ultra thick, orange!80!black, fill=orange!20, minimum size=1cm, above=1.0cm of phi121] (x21) {$x_{1}^{\langle 2 \rangle}$};

    % Connections between layers
    \draw[ultra thick,-{Stealth[length=3.5mm]}] (x01) -- (phi11);
	\draw[ultra thick,-{Stealth[length=3.5mm]}] (x01) -- (phi21);
	\draw[ultra thick,-{Stealth[length=3.5mm]}] (x01) -- (phi31);

	\draw[ultra thick,-{Stealth[length=3.5mm]}] (x02) -- (phi12);
	\draw[ultra thick,-{Stealth[length=3.5mm]}] (x02) -- (phi22);
	\draw[ultra thick,-{Stealth[length=3.5mm]}] (x02) -- (phi32);
	
	% Connect the phi functions to the sum nodes
	\draw[ultra thick,-{Stealth[length=3.5mm]}] (phi11) -- (x11);
	\draw[ultra thick,-{Stealth[length=3.5mm]}] (phi12) -- (x11);
	\draw[ultra thick,-{Stealth[length=3.5mm]}] (phi21) -- (x12);
	\draw[ultra thick,-{Stealth[length=3.5mm]}] (phi22) -- (x12);
	\draw[ultra thick,-{Stealth[length=3.5mm]}] (phi31) -- (x13);
	\draw[ultra thick,-{Stealth[length=3.5mm]}] (phi32) -- (x13);

	% Connect the sum nodes to the phi functions in Layer 1
	\draw[ultra thick,-{Stealth[length=3.5mm]}] (x11) -- (phi111);
	\draw[ultra thick,-{Stealth[length=3.5mm]}] (x12) -- (phi121);
	\draw[ultra thick,-{Stealth[length=3.5mm]}] (x13) -- (phi131);

	% Connect the phi functions in Layer 1 to the output node
	\draw[ultra thick,-{Stealth[length=3.5mm]}] (phi111) -- (x21);
	\draw[ultra thick,-{Stealth[length=3.5mm]}] (phi121) -- (x21);
	\draw[ultra thick,-{Stealth[length=3.5mm]}] (phi131) -- (x21);

    % Annotations
    \node[right=0.5cm of x02, green!50!black] {\textbf{Input Layer}};
    \node[right=0.5cm of x13, gray!80!black] {\textbf{Hidden Layer}};
    \node[right=0.5cm of x21, orange!80!black] {\textbf{Output Layer}};
\end{tikzpicture}
\caption{Приклад KAN архітектури з трьома шарами з кількістю нейронів $n_0=2$, $n_1=3$, $n_2=1$.}
\label{fig:kan-architecture}
\end{figure}

Єдине, що ми ще не врахували --- а чому така репрезентація взагалі дає універсальну 
апроксимацію на компакті $\mathcal{Q}_m$? Дійсно, хоч теорема Колмогорова-Арнольда
дає точну рівність, але ми поки ніяк не гарантуємо, що якщо замінити функції 
$\{\phi_{p,q}\}$ на $B$-сплайни, то ми все одно можемо апроксимувати будь-яку
функцію. Це питання досліджується в роботі \cite{kan}, де показана наступна теорема.

\begin{theorem}[Теорема апроксимації KAN]
	Нехай функція $f$ має вигляд $f = \boldsymbol{\Phi}^{\langle\ell\rangle}
	\circ \dots \circ \boldsymbol{\Phi}^{\langle 0\rangle} \circ \mathbf{x}$, де
	усі функції $\phi_{p,q}^{\langle j \rangle}$ є $d+1$ разів неперервно
	диференційовані. Тоді, існує така константа $\gamma$, що залежить від $f$ і
	функцій $\{\phi_{p,q}\}$, та існують матриці
	$\widetilde{\boldsymbol{\Phi}}^{\langle\ell\rangle},\dots,\widetilde{\boldsymbol{\Phi}}^{\langle
	0\rangle}$, що складаються з $B$-сплайнів порядку $d$ і розміром сітки $n_G$, що
	для всіх $0 \leq r \leq d$ маємо
	\begin{equation*}
		\left\| f - \widetilde{\boldsymbol{\Phi}}^{\langle\ell\rangle}
		\circ \dots \circ \widetilde{\boldsymbol{\Phi}}^{\langle 0\rangle} \circ \mathbf{x} \right\|_{C^r} \leq \gamma n_G^{r-(d+1)},
	\end{equation*}
	де $\|g\|_{C^r} = \sup_{|\beta|\leq r}\sup_{\mathbf{x} \in \mathcal{Q}_m}|g^{(\beta)}(\mathbf{x})|$.
\end{theorem}

\subsection{Порівняння з MLP архітектурою}

Отже, що краще: MLP чи KAN? Наведемо деякі переваги та недоліки кожної з архітектур.

\begin{enumerate}
	\item \textbf{Кількість інструментів.} Оскільки останні 10 років було
	присвячено розвитку MLP архітектур, то для них існує набагато більше
	інструментів, бібліотек та фреймворків. Навіть якщо KAN має потенціал, то
	простий налаштунок та тренування може стати викликом.
	\item \textbf{Інтерпретованість.} Оскільки KAN використовує $B$-сплайни, то
	вони є більш інтерпретованими, ніж MLP. Це може бути корисним для задач, де
	важливо зрозуміти, як саме мережа приймає рішення.
	\item \textbf{Швидкість тренування.} Згідно з роботою \cite{kan}, KAN мережі
	поки приблизно в $10\times$ повільніші за MLP під час тренування. Проте,
	на практиці, цей показник часто не є критичним: значно більш важливою є
	точність та швидкість обчислення під час обрахунку передбачення.
	\item \textbf{Швидкість передбачення.} Для простоти аналізу, нехай маємо дві
	мережі, написані як на MLP, так і на KAN, у якої $\ell$ шарів, в кожному з
	яких $n$ нейронів і активаційна функція має степінь $d$. Тоді, кількість
	операцій для MLP мережі буде $\mathcal{O}(\ell n(n+d))$: на кожному переході
	між шарами, маємо $n^2$ операцій для обрахунку добутку матриця-вектор, потім
	$nd$ операцій для обрахунку активаційної функції над отриманим вектором. Для
	KAN, кількість операцій буде $\mathcal{O}(\ell n^2 d)$: на кожному шарі,
	маємо $n^2$ функцій-ваг, кожна з яких вимагає $d$ операцій для обрахунку.
	\item \textbf{Кількість параметрів.} Асимтотично, KAN мережі мають більше
	параметрів, ніж MLP. Дійсно, для MLP маємо $\mathcal{O}(n^2\ell)$ параметрів,
	у той час як KAN ще мають зберігати параметри для кожного $B$-сплайну, тобто 
	складність стає $\mathcal{O}(n^2\ell d)$. Проте, на практиці, можливо для KAN 
	потрібно менше нейронів та шарів для досягнення аналогічної точності.
\end{enumerate}

    % Опис конволюційних нейронних мереж
    \chapter{KAN у контексті комп'ютерного зору}

\section{Класичні конволюційні нейронні мережі}

Нагадаємо, що під час нашої попередньої дискусії, ми розглядали лише
мультишарові перцептрони (MLP), що описуються рекурентним співвідношенням
$\mathbf{x}^{\langle \ell+1 \rangle} = \phi(\boldsymbol{W}^{\langle \ell
\rangle}\mathbf{x}^{\langle \ell \rangle} + \boldsymbol{\beta}^{\langle \ell
\rangle})$. Проте, як видно, в цьому рівнянні кожне проміжне значення, включно з
початковим --- це вектор. Уявімо, що нам потрібно побудувати нейронну мережу на
основі зображення, яке є двовимірним масивом пікселів. Природньо розв'язати цю
задачу наступним чином: просто розгорнемо зображення в одновимірний вектор, а
потім застосуємо до нього MLP. І саме таким чином були побудовані одні з перших
моделей машинного навчання на зображеннях. Зокрема, навіть до розвитку глибокого
навчання, такий спосіб був використаний у біометричних системах для
розпізнавання облич Eigenface \cite{eigenface} та Fisherface \cite{fisherface},
де зображення обличчя перетворювалося в одновимірний вектор, а потім
застосовувався алгоритм PCA або LDA для зменшення розмірності \cite{pca}.

Проте, як показує сучасна практика, такий підхід не є оптимальним. Найбільша 
проблема наступна: нехай задача є бінарною класифікацією над зобарежннями 
розміру $100 \times 100$. У якості нейронної мережі, ми під'єднаємо 
вхідний шар до прихованого шару зі $100$ нейронами, що в свою чергу 
під'єднаний до виходу. Тоді, кількість параметрів у моделі буде 
$100 \times 100 \times 100 = 1 \, \text{млн}$, хоча модель містить 
буквально один малий прихований шар. 

Річ у тому, що зв'язків між нейронами у вхідному та прихованому шарах
дуже багато і, як показує практика, більшість з них є зайвими. Нам 
потрібно побудувати таку функцію переходу, яка б враховувала 
просторову структуру зображення. Для цього, ми можемо
використати концепцію \textit{згортки} (convolution).

Історично, згортки виникли як наступний оператор над простором 
функцій: нехай маємо функції $f(x)$ та $g(x)$, тоді згорткою 
називають вираз
\begin{equation}
    (f \star g)(x) = \int_{-\infty}^{\infty} f(\tau) g(x - \tau) d\tau.
\end{equation}

Проте, що в комп'ютерному зорі, ми будемо використовувати дискретну версію
згортки, причому в основному для трьохвимірних масивів. Почнемо, проте, з 
двовимірної версії. Визначимо згортку двохвимірного зображення наступним чином.
\begin{definition}
    Маючи зображення $\boldsymbol{X} \in \mathbb{R}^{W \times H}$ та так званий
    \textit{фільтр} або \textit{ядро} $\boldsymbol{K} \in \mathbb{R}^{f \times
    f}$, результатом будемо називати нове зображення
    $\boldsymbol{Y}=\boldsymbol{K} \star \boldsymbol{X}$ розміру $(W-f+1) \times
    (H-f+1)$, яке визначається як
    \begin{equation}
        Y_{i,j} = \sum_{u=0}^{f-1} \sum_{v=0}^{f-1} X_{i+u,j+v} \cdot K_{u,v}.
    \end{equation}
\end{definition}

Геометрично, фільтр накладається на зображення, починаючи з верхнього лівого
кута, і переміщається по зображенню, обчислюючи нове значення пікселя шляхом
обчислення скалярного добутку між фільтром та частиною зображення, на яку
він накладається. Таким чином, нове зображення є результатом застосування
фільтра до всіх можливих позицій у зображенні. Приклад застосування фільтра показано 
на Рисунку \ref{fig:conv_example}.

\begin{figure}
\[
\left[\begin{array}{ccccccc}
  0 & 1 & 1 & \mymark{TL1}{1}\mysmall{1} & 0\mysmall{0} & \mymark{TR1}{0}\mysmall{1} & 0\\
  0 & 0 & 1 &               1\mysmall{0} & 1\mysmall{1} &               0\mysmall{0} & 0\\
  0 & 0 & 0 & \mymark{BL1}{1}\mysmall{1} & 1\mysmall{0} & \mymark{BR1}{1}\mysmall{1} & 0\\
  0 & 0 & 0 & 1 & 1 & 0 & 0\\
  0 & 0 & 1 & 1 & 0 & 0 & 0\\
  0 & 1 & 1 & 0 & 0 & 0 & 0\\
  1 & 1 & 0 & 0 & 0 & 0 & 0\\
\end{array}\right]
\star
\left[\begin{array}{ccc}
  \mymark{TL2}{1} & 0 & \mymark{TR2}{1}\\
  0  & 1 &              0 \\
  \mymark{BL2}{1} & 0 & \mymark{BR2}{1}
\end{array}\right]
=
\left[\begin{array}{ccccccc}
  1 & 4 & 3 & \mymark{C}{4} & 1\\
  1 & 2 & 4 & 3 & 3\\
  1 & 2 & 3 & 4 & 1\\
  1 & 3 & 3 & 1 & 1\\
  3 & 3 & 1 & 1 & 0
\end{array}\right]
\]
\begin{tikzpicture}[overlay, remember picture,
    myedge1/.style={dashed, opacity=.3, blue},
    myedge2/.style={dashed, opacity=.3, green!40!black}]

  %% Draw boxes
  \draw[red, fill=red, fill opacity=.1, very thick]   (TL1.north west) rectangle (BR1.south east);
  \draw[blue, fill=blue, fill opacity=.1, very thick] (TL2.north west) rectangle (BR2.south east);
  \draw[green!60!black, fill=green, fill opacity=.1, very thick] (C.north west) rectangle (C.south east);

  %% Draw blue lines
  \draw[myedge1] (TL1.north west) -- (TL2.north west);
  \draw[myedge1] (BL1.south west) -- (BL2.south west);
  \draw[myedge1] (TR1.north east) -- (TR2.north east);
  \draw[myedge1] (BR1.south east) -- (BR2.south east);

  %% Draw green lines
  \draw[myedge2] (TL2.north west) -- (C.north west);
  \draw[myedge2] (BL2.south west) -- (C.south west);
  \draw[myedge2] (TR2.north east) -- (C.north east);
  \draw[myedge2] (BR2.south east) -- (C.south east);
\end{tikzpicture}
\caption{Приклад згортки зображення $7 \times 7$ з фільтром $3 \times 3$.}
\label{fig:conv_example}
\end{figure}

Коли ми тренуємо нейронну мережу, то сподіваємося, що вона навчиться 
правильно підбирати коефіцієнти у фільтрах. Наприклад, нейронна мережа 
може навчитися виділяти краї, текстури або інші важливі ознаки зображення.
Приклад показано на Рисунку \ref{fig:sobel}: якщо використати 
так звані фільтри Собеля:
\begin{equation}
    \boldsymbol{G}_X = \begin{bmatrix}
        -1 & 0 & +1 \\
        -2 & 0 & +2 \\
        +1 & +2 & +1 
    \end{bmatrix}, \quad \boldsymbol{G}_Y = \begin{bmatrix}
        -1 & -2 & -1 \\
        0 & 0 & 0 \\
        +1 & +2 & +1
    \end{bmatrix},
\end{equation}

що виділяють вертикальні та горизонтальні краї відповідно, а далі обрахувати
зображення $\boldsymbol{Y} := \sqrt{(\boldsymbol{G}_X \star \boldsymbol{X})^2 +
(\boldsymbol{G}_Y \star \boldsymbol{X})^2}$ (де зазначені операції робляться
по-елементно), то отримаємо виділені контури на зображенні.

\begin{figure}
    \centering
    \includegraphics[width=1.0\linewidth]{figures/convolutions/sobel.pdf}
    \caption{Приклад виділення країв на зображенні за допомогою накладання
    конволюцій (фільтри Собеля). Ліворуч --- оригінальне зображення, праворуч
    --- виділені краї.}
    \label{fig:sobel}
\end{figure}

\section{Згортки у нейронних мережах}

Проте, зазначена вище процедура була визначена лише для двовимірних
зображень та наявності одного фільтра. У нейронних мережах ми
зазвичай маємо справу з тривимірними масивами: наприклад, кольорове 
зображення є тривимірним масивом розміру $W \times H \times 3$, де
$3$ --- це кількість кольорових каналів (червоний, зелений, синій).
Тому, ми повинні розширити визначення згортки на тривимірні масиви.

\begin{definition}
    Нехай $\boldsymbol{X} \in \mathbb{R}^{W \times H \times C}$ --- набір з $C$
    зображень (кількість каналів) розміру $W \times H$, та $\boldsymbol{K} \in
    \mathbb{R}^{f \times f \times C \times C'}$ --- набір з $C'$ фільтрів (ядер)
    розміру $f \times f \times C$. Тоді, результатом згортки буде нове
    зображення (тензор) $\boldsymbol{Y} \in \mathbb{R}^{(W-f+1) \times (H-f+1)
    \times C'}$, яке визначається як
    \begin{equation}
        Y_{i,j,k} = \sum_{u=0}^{f-1} \sum_{v=0}^{f-1} \sum_{c=0}^{C-1} X_{i+u,j+v,c} \cdot K_{u,v,c,k}.
    \end{equation}
\end{definition}

Тут інтуїція дуже схожа: беремо фільтр $f \times f \times C$, накладаємо на
частину зображення $W \times H \times C$, обчислюємо скалярний добуток та
отримуємо нове значення одного пікселя. Проходимось по всьому об'єму, щоб
отримати зображення розміру $(W-f+1) \times (H-f+1)$. Далі ми це повторюємо для
всіх фільтрів $C'$, щоб отримати вже тензор $\mathbb{R}^{(W-f+1) \times (H-f+1)
\times C'}$. 


Проте, як можна помітити, задані операції конволюції ніяк не розв'язують нашу
початкову проблему: розмір входу та виходу майже ніяк не змінюються (оскільки на
практиці розмір фільтру $f \ll W,H$). Тому, нам потрібні інструменти, які
дозволять зменшувати розмір зображення. Для цього введемо декілька інструментів.
\begin{itemize}
    \item \textbf{Паддінг} (padding) --- це додавання нулів до країв зображення.
    Це дозволяє зберегти розмір зображення незмінним під час згортки (себто, 
    якщо у нас є зображення $W \times H$ та фільтр $f \times f$, то після
    паддінгу зображення до $(W+f-1) \times (H+f-1)$, ми отримаємо нове
    зображення початкового розміру $W \times H$). Ця операція не зменшує 
    розмір зображення, але працювати з нею стає зручніше надалі.
    \item \textbf{Страйд} (stride) --- це крок, на який фільтр переміщується по
    зображенню. Зазвичай, ми використовуємо $s=1$ або $s=2$ або в рідкісних
    випадках $s=3$. З використанням паддінгу та страйду $s$, ми можемо зменшити
    розмір зображення в кожному каналі з $W \times H$ до $\frac{W}{s} \times
    \frac{H}{s}$.
    \item \textbf{Пулінг} (pooling) --- це операція, яка зменшує розмір
    зображення шляхом обчислення статистики (максимуму або середнього) над
    невеликими ділянками зображення. Наприклад, якщо ми маємо $2 \times 2$
    ділянку зображення, то ми можемо взяти максимум або середнє значення
    пікселів у цій ділянці і замінити всю ділянку на це значення.
\end{itemize}

Таким чином, ідея побудови конволюційної нейронної мережі полягає в тому, що ми
будемо використовувати згортки для виділення ознак з зображення, а потім
зменшувати розмір зображення за допомогою пулінгу або страйду. Приклад
конволюційної нейронної мережі показано на Рисунку \ref{fig:convnet}. Як видно,
ми маємо декілька шарів конволюційних фільтрів, які виділяють ознаки з
зображення, а потім зменшують розмір зображення за допомогою пулінгу. Після
цього, ми використовуємо повнозв'язний шар, щоб отримати остаточну класифікацію
зображення. 

\begin{figure}
    \centering
    \includegraphics[width=1.0\linewidth]{figures/convolutions/convnet_fig.png}
    \caption{Приклад глибокої конволюційної нейронної мережі}
    \label{fig:convnet}
\end{figure}

Єдине ключове питання, що ми оминули --- це як саме накладати нелінійність 
у конволюційних нейронних мережах. Дійсно, якщо просто взяти композицію 
згорткових шарів, то ми отримаємо одне лише лінійне перетворення.
Тому, ми повинні накладати нелінійність після кожної згортки. Це можна 
зробити дуже просто: при кожному обрахунку суми $\sum_{u,v,c}X_{i+u,j+v,c}K_{u,v,c,k}$,
ми будемо накладати нелінійність $\phi$ на результат та додавати 
зсув $\beta_{i,j,k}$.
\begin{definition}
    Нехай $\boldsymbol{X} \in \mathbb{R}^{W \times H \times C}$ --- набір з $C$
    зображень (кількість каналів) розміру $W \times H$, та $\boldsymbol{K} \in
    \mathbb{R}^{f \times f \times C \times C'}$ --- набір з $C'$ фільтрів (ядер)
    розміру $f \times f \times C$. Тоді, результатом згортки з нелінійністю
    $\phi$ та страйдом $s$ буде нове зображення $\boldsymbol{Y} \in
    \mathbb{R}^{\frac{W}{s} \times \frac{H}{s} \times C'}$, яке визначається як
    \begin{equation}
        \textcolor{blue}{Y_{i,j,k} = \phi\left(\sum_{u=0}^{f-1} \sum_{v=0}^{f-1} \sum_{c=0}^{C-1} X_{i+u,j+v,c} K_{u,v,c,k} + \beta_{i,j,k}\right).}
    \end{equation}
\end{definition}

Ця формула є ключовою для побудови конволюційних нейронних мереж і ми 
будемо орієнтуватись на неї у подальшій дискусії. 

\section{Згортки Колмогорова-Арнольда}\label{section:kolmogorov}

Нарешті, ми дійшли до найцікавішого: побудова конволюцій Колмогорова-Арнольда.
Наша конструкція вмотивована конструкцією \cite{kan-cnn}, котру ми в цій роботі
проаналізуємо та дослідимо детальніше. Отже, як і в оригінальній повнозв'язаній 
нейронні мережі, ідея полягає в тому, що ми будемо використовувати 
параметризовані функції замість скалярних значень. Таким чином, 
кожен елемент в фільтрі буде окремою функцією, що ми будемо накладати.
Різницю між звичайною згорткою та згорткою Колмогорова-Арнольда можна
побачити на Рисунку \ref{fig:kan_conv}.
\begin{figure}
\noindent
    \begin{minipage}{0.48\linewidth}
    \textcolor{blue}{
    \begin{equation*}
    \boldsymbol{K}_{\text{MLP}} = \begin{bmatrix}
        k_{1,1} & \cdots & k_{1,f} \\
        \vdots & \ddots & \vdots \\
        k_{f,1} & \cdots & k_{f,f}
    \end{bmatrix} \in \mathbb{R}^{f \times f}
    \end{equation*}
    MLP згортка $f \times f$, складається з $f^2$ параметрів $k_{i,j} \in \mathbb{R}$.}
    \end{minipage}
    \hfill
    \begin{minipage}{0.48\linewidth}
    \textcolor{purple}{
    \begin{equation*}
    \boldsymbol{K}_{\text{KAN}} = \begin{bmatrix}
        \phi_{1,1}(x) & \cdots & \phi_{1,f}(x) \\
        \vdots & \ddots & \vdots \\
        \phi_{f,1}(x) & \cdots & \phi_{f,f}(x)
    \end{bmatrix} \in \mathcal{F}^{f \times f}
    \end{equation*}
    KAN згортка $f \times f$, складається з $f^2$ функцій
    $\phi_{\boldsymbol{i}}(x)=\omega_{\boldsymbol{i},\beta}\beta(x)+\omega_{\boldsymbol{i},S} S(x)$.}
    \end{minipage}
    \caption{Порівняння між звичайною згорткою та згорткою Колмогорова-Арнольда.}
    \label{fig:kan_conv}
\end{figure}

Таким чином, дамо наступне визначення згортки Колмогорова-Арнольда.
\begin{definition}
    Нехай $\boldsymbol{X} \in \mathbb{R}^{W \times H \times C}$ --- набір з $C$
    зображень (кількість каналів) розміру $W \times H$, та $\boldsymbol{K} \in
    \mathcal{F}^{f \times f \times C \times C'}$ --- набір з $C'$ фільтрів
    (ядер) розміру $f \times f \times C$, що складаються з параметризованих
    сплайнів. Тоді, результатом згортки буде нове зображення $\boldsymbol{Y} \in
    \mathbb{R}^{W \times H \times C}$:
    \begin{equation}
        \textcolor{purple}{Y_{i,j,k} = \sum_{u=0}^{f-1} \sum_{v=0}^{f-1} \sum_{c=0}^{C'-1} \phi_{u,v,c,k}(X_{i+u,j+v,k})},
    \end{equation}
    причому кожен $\phi_{\boldsymbol{i}}(x) = \omega_{\boldsymbol{i},\beta}\beta(x)+\beta_{\boldsymbol{i},S} \sum_j c_{\boldsymbol{i},j}B_j(x)$,
    $\beta(x)=x\sigma(x)$.
\end{definition}

Реалізацію згортки Колмогорова-Арнольда можна знайти в Додатку \ref{appendix:c-ckan-code}.

\subsection{Мотивація конволюцій Колмогорова-Арнольда}

В MLP згортках ми вже бачили геометричний сенс: наприклад, фільтр при правильній
побудові може виділяти краї або локальну геометрію зображення, проте складніші
висновки можуть бути отримані лише після кількох конволюцій поспіль.

Головна користь KAN згорток, що ми вбачаємо, наступна: KAN згортки дозволяють
будувати такі фільтри, що мають значно складнішу поведінку, ніж MLP згортки.
Наприклад, у роботі \cite{kan-cnn} наводять наступний приклад: нехай ми задали
поріг $\tau$ і усі пікселі, що мають більшу яскравість ніж $\tau$, стають ще
світлішими, а усі інші стають темнішими. Таким чином, наша функція $\phi_{i,j}$
виглядає як:
\begin{equation*}
    \phi_{i,j}(x) = \begin{cases}
        \alpha_{\text{bright}}\cdot x, & \text{якщо } x \geq \tau_{i,j}, \\
        \alpha_{\text{dark}}\cdot x, & \text{інакше.}
    \end{cases}
\end{equation*}

Помітимо, що таке перетворення зображення ми б не змогли реалізувати за
допомогою звичайної MLP згортки.

\section{$B$-сплайни}

Як ми вже зазначали, в якості активаційної функції автори \cite{kan} 
використовують $B$-сплайни, тому варто розглянути їх дещо детальніше.

$B$-сплайни --- це функції, які використовуються для апроксимації
функцій та побудови кривих. Вони є частиною більш загальної категорії
\textit{сплайнів}, які є функціями, що складаються з декількох поліномів,
які з'єднані разом у певних точках, що називаються \textit{вузлами} (knots).
$B$-сплайни є особливим випадком сплайнів, які мають певні властивості,
такі як неперервність та гладкість, що робить їх дуже корисними для
апроксимації функцій та побудови кривих.

В нашому конкретному випадку, ми будемо використовувати $B$-сплайни порядку $d$
для апроксимації функій у вигляді суми $S(x) = \sum_{i}c_iB_{i,d}(x)$ (з межами
суми ми визначимось дещо пізніше), де функції $B_{i,d}(x)$ природньо називають
\textit{базисними функціями} $B$-сплайнів порядку $d$. Вони визначаються
рекурсивно за допомогою формули Кокс-де Бьорна (Cox-de Boor recursion formula)
\cite{b-splines} над вузлами $x_0, x_1, \ldots, x_n$:
\begin{gather}
    B_{i,0}(x) = \begin{cases}
        1, & \text{якщо } x_i \leq x < x_{i+1}, \\
        0, & \text{інакше.}
    \end{cases} \\
    B_{i,d}(x) = \frac{x-x_i}{x_{i+d}-x_i}B_{i,d-1}(x) + \frac{x_{i+d+1}-x}{x_{i+d+1}-x_{i+1}}B_{i+1,d-1}(x), \quad d > 0
\end{gather}

Проілюструємо ці базисні функції на прикладі $B$-сплайнів порядку $0 \leq d \leq
3$ на множині вузлів $\{0,\dots,6\}$: дивись Рисунок \ref{fig:b-splines}.
\begin{figure}
    \begin{tabular}{cc}
      \includegraphics[width=0.5\textwidth]{code/splines/bspline_degree_0.pdf} &   \includegraphics[width=0.5\textwidth]{code/splines/bspline_degree_1.pdf} \\
      Степінь $d=0$ & Степінь $d=1$ \\
      \includegraphics[width=0.5\textwidth]{code/splines/bspline_degree_2.pdf} &   \includegraphics[width=0.5\textwidth]{code/splines/bspline_degree_3.pdf} \\
      Степінь $d=2$ & Степінь $d=3$
    \end{tabular}
    \caption{Три базисних $B$-сплайн полінома $\{B_{i,d}(x)\}_{i \in \{0,1,2\}}$
    для степенів поліномів $d \in \{0,1,2,3\}$ на множині вузлів $\{0,\dots,6\}$.}
    \label{fig:b-splines}
\end{figure}

Бачимо, що базисні функції $B_{i,d}(x)$ є неперервними та гладкими функціями,
які мають значення $0$ в усіх точках, окрім відрізку $(x_i,x_{i+d+1})$. Більш
того, можна показати, що $B_{i,d}(x) \in \mathcal{C}^{d-1}(\mathbb{R})$. Як бачимо, чим
більший порядок $d$, тим більший відрізок ``покриває'' базисний поліном
$B_{i,d}(x)$. Саме тому при $n$ вузлах, нам доступні лише 
$n-d-1$ базисних функцій $B_{i,d}(x)$. 

В оригінальній статті пропонується обирати $d=3$ та побудувати сітку над
областтю $[-1,1]$, розбивши її на $n$ рівних частин. Таким чином, маємо вузли
$x_i = -1 + 2i/n$ для $i=0,\dots,n$. Щоб отримати $n+d$ базисних функцій, стаття
пропонує розширити сітку в обидва боки, додавши по $d$ вузлів з обох сторін.
Таким чином, отримуємо наступний вигляд апроксимації:
\begin{equation}
    \widehat{f}(x|\boldsymbol{c}) = \sum_{i=1}^{n+d} c_i B_{i,d}(x), \quad x \in [-1,1].
\end{equation}

Продемонструємо, як відбувається пошук коефіцієнтів $\{c_i\}_{0 < i \leq n+d}$
на практиці. Нехай нам задана певна функція $f(x)$ і ми хочемо її апроксимувати
за допомогою $B$-сплайнів. Таку задачу можна розв'язати за допомогою методу
найменших квадратів
\begin{align}
    \widehat{\boldsymbol{c}} &= \argmin_{\boldsymbol{c} \in \mathbb{R}^{n+d}} \sum_{i=1}^{n+d} \left(f(x_i) - \widehat{f}(x_i|\boldsymbol{c})\right)^2 \\
    &= \argmin_{\boldsymbol{c} \in \mathbb{R}^{n+d}} \sum_{i=1}^{n+d} \left(f(x_i) - \sum_{j=1}^{n+d} c_j B_{j,d}(x_i)\right)^2
\end{align}

Як і у випадку з лінійною регресією, це зводиться до розв'язання системи
лінійних рівнянь. В якості більш цікавого прикладу, візьмемо функцію $f(x) = x^2
e^{x/8} \sin(2\pi x)$. Далі, згенеруємо $100$ точок на відрізку $[-1,1]$, де $y$
координата кожної точки $x_i$ буде задана як $f(x_i)+\varepsilon$, де
$\varepsilon \sim \mathcal{N}(0,0.1)$ --- випадковий шум з відносно малою
дисперсією. На Рисунку \ref{fig:bspline_approx} показано результат апроксимації
за допомогою $B$-сплайнів порядку $d=3$ з $n=15$ вузлами, де ми шукали 
коефіцієнти за допомогою градієнтного спуску (метод Adam \cite{adam}) 
з метрикою середньоквадратичної помилки (MSE). Як видно, отримана
апроксимація дуже близька до оригінальної функції, що свідчить про
досить хорошу якість апроксимації.

Реалізацію модуля $B$-сплайнів можна знайти в додатку \ref{appendix:b-spline-code}.

\begin{figure}
    \centering
    \includegraphics[width=0.8\linewidth]{figures/trained_spline.pdf}
    \caption{Приклад апроксимації функції $f(x) = x^2e^{x/8} \sin 2\pi x$ за
    допомогою $B$-сплайнів порядку $d=3$ з $n=10$ вузлами.}
    \label{fig:bspline_approx}
\end{figure}



    % Експерименти
    \chapter{Експеримент}

\section{Набір даних MNIST}

В цьому розділі ми проведемо експерименти з використанням набору даних MNIST.
Цей набір даних містить 70000 зображень рукописних цифр від 0 до 9, розміром $28
\times 28$ пікселів. Він широко використовується для навчання та тестування
алгоритмів комп'ютерного зору та машинного навчання. В даному експерименті ми
будемо використовувати 60000 зображень для навчання та 10000 зображень для
тестування. Кожне зображення є чорно-білим, і його пікселі мають значення від 0
до 255, де 0 --- це чорний колір, а 255 --- білий. Проте, ці значення пікселів
при нормалізуємо на відрізок $[0,1]$ для стабілізації процесу тренування. Набір
даних MNIST є стандартним набором даних для оцінки алгоритмів класифікації
зображень, тому він ідеально підходить для нашого експерименту. Приклади
зображень з набору даних MNIST були наведені на початку роботи, на
Рисунку~\ref{fig:digits}.

\section{Архітектура нейронної мережі та результати}

В нашому експерименті ми будемо використовувати нейронну мережу, що складається
з трьох конволюційних шарів KAN та одного лінійного шару на виході. 
Архітектура проілюстрована на Рисунку~\ref{fig:kan-arch}.
\begin{figure}
    \centering
    \tikzset{
        base/.style={
            draw,
            rectangle,
            minimum width=5cm,
            minimum height=1cm,
            ultra thick,
            rounded corners=15pt
        },
        maxpool/.style={
            base,
            fill=orange!10,
            draw=orange!90!black
        },
        kanconv/.style={
            base,
            fill=blue!10,
            draw=blue!80!black
        },
        fc/.style={
            base,
            fill=green!10,
            draw=green!70!black
        }
    }
    \begin{tikzpicture}
        % Input
        \node[base] (input) {\textbf{Вхід}};
        \node[left=1cm of input, thick] (image) {\includegraphics[width=1.75cm]{figures/mnist/digit5.png}};
        \draw[thick, line width=1.5pt, -{Stealth[length=7pt, width=7pt]}] (image) -- (input);
        \node[above=0.1cm of image] {\textcolor{gray}{$\boldsymbol{X} \in \mathbb{R}^{28 \times 28 \times 1}$}};

        \node[kanconv, below=0.1cm of input] (conv1) {KANConv, $3 \times 3$, $8$};
        \node[right=0.3cm of conv1] {\textcolor{gray}{720 параметрів}};
        
        \node[maxpool, below=0.1cm of conv1] (maxpool1) {MaxPool2D, $2 \times 2$};
        
        \node[kanconv, below=0.1cm of maxpool1] (conv2) {KANConv, $3 \times 3$, $16$};
        \node[right=0.3cm of conv2] {\textcolor{gray}{13.8k параметрів}};
        
        \node[maxpool, below=0.1cm of conv2] (maxpool2) {MaxPool2D, $2 \times 2$};
        
        \node[kanconv, below=0.1cm of maxpool2] (conv3) {KANConv, $3 \times 3$, $32$};
        \node[right=0.3cm of conv3] {\textcolor{gray}{41.5k параметрів}};
        
        \node[maxpool, below=0.1cm of conv3] (maxpool3) {MaxPool2D, $2 \times 2$};
        
        \node[fc, below=0.9cm of maxpool3] (fc) {$7 \cdot 7 \cdot 32$ нейронів};
        
        \node[fc, below=0.1cm of fc] (fc2) {Лінійний шар, $1568 \times 10$};
        \node[left=0.3cm of fc2] {\textcolor{gray}{15.4k параметрів}};
        \node[right=2.25cm of fc2] (output-label) {\textcolor{gray}{$\boldsymbol{y} \in \mathbb{R}^{10}$}};
        \draw[thick, line width=1.5pt, -{Stealth[length=7pt, width=7pt]}] (fc2) -- (output-label) node[midway, above] {Softmax};

        % Draw stealth arrows 
        % \draw[thick, line width=1.5pt, -{Stealth[length=7pt, width=7pt]}] (input) -- (conv1);
        % \draw[thick, line width=1.5pt, -{Stealth[length=7pt, width=7pt]}] (conv1) -- (maxpool1);
        % \draw[thick, line width=1.5pt, -{Stealth[length=7pt, width=7pt]}] (maxpool1) -- (conv2);
        % \draw[thick, line width=1.5pt, -{Stealth[length=7pt, width=7pt]}] (conv2) -- (maxpool2);
        % \draw[thick, line width=1.5pt, -{Stealth[length=7pt, width=7pt]}] (maxpool2) -- (conv3);
        % \draw[thick, line width=1.5pt, -{Stealth[length=7pt, width=7pt]}] (conv3) -- (maxpool3);
        % \draw[thick, line width=1.5pt, -{Stealth[length=7pt, width=7pt]}] (maxpool3) -- (conv4);
        \draw[thick, line width=1.5pt, -{Stealth[length=7pt, width=7pt]}] (maxpool3) -- (fc) node[midway, right] {Flatten};
    \end{tikzpicture}
    \caption{Архітектура нейронної мережі на основі конволюційних шарів 
    Колмогорова-Арнольда (CKAN).}
    \label{fig:kan-arch}
\end{figure}

Помітимо, що більшість параметрів в цій архітектурі зосереджені в перших трьох
конволюційних шарах KAN, які мають $720$, $13.8$k та $41.5$k параметрів
відповідно. Зауважимо, що замість останнього лінійного шару ми хотіли поставити
плоский шар KAN, проте через нього час тренування дуже помітно зростав. Окрім
того, процес навчання дуже складний: точність дуже часто не перевищувала 30\% і
пояснити це доволі важко. Тому ми вирішили залишити лінійний шар на виході, який
в даному випадку є Softmax класифікатором.

З лінійном шаром, ми отримали \textbf{87.8\% точності} на тестовому наборі 
даних MNIST. При цьому, $F_1$ міра дорівнює \textbf{87.2\%}. Крива 
тренування зображена на Рисунку~\ref{fig:train-curve}. Увесь код для 
запуску тренування лежить в репозиторії за наступним посиланням:
\begin{center}
    \url{https://github.com/ZamDimon/convolutional-kan}
\end{center}
\begin{figure}
    \centering
    \includegraphics[width=\textwidth]{figures/batch_loss_curve.pdf}
    \caption{Крива тренування нейронної мережі на основі конволюційних шарів 
    Колмогорова-Арнольда (CKAN).}
    \label{fig:train-curve}
\end{figure}


    % Висновки
    \chapter*{Висновки}
\markboth{Висновки}{Висновки}
\addcontentsline{toc}{chapter}{Висновки}

\hspace{\parindent} В цій роботі були розглянуті фундаментальні та формалізовані
принципи роботи нейронних мереж. Ми конкретизували основну проблематику 
машинного навчання та, зокрема, яка роль нейронних мереж у вирішенні цих
проблем. Ми навели визначення та теореми, що показують дві парадигми 
побудови архітектур нейронних мереж: мультишарові перцептрони (MLP) та 
мережі Колмогорова-Арнольда (KAN). Для обох парадигм ми навели теореми,
що доводять універсальність цих архітектур для апроксимації довільних 
функцій на гіперкубі $[0, 1]^m$. Ми також розглянули проблематику
вибору кількості шарів та нейронів у кожному з них, а також вибору
функції активації. Ми навели приклади використання нейронних мереж для
розв'язання задач класифікації і показали на практиці, що теорема 
Цибенко дійсно працює для складної функції. 

Нарешті, коли ми окреслили основну відмінність між MLP та KAN, ми розширили цю
ідею на конволюційні нейронні мережі (CNN) та провели експерименти на наборі
даних MNIST за допомогою нейронних мереж, що повністю складалися з шарів KAN.
Незважаючи на складнощі в процесі тренування, отримана точність у 87.8\%
точності показує перспективність використання KAN для задач комп'ютерного зору.

    \printbibliography

    \appendix

\chapter{Програмний код для тренування мережі Цибенко}\label{appendix:cybenko-code}

В цьому додатку, ми наведемо програмний код для тренування мережі Цибенко на
прикладі задачі класифікації. Для цього ми використаємо бібліотеку
\texttt{TensorFlow} на мові програмування \texttt{Python}. Для початку, ми 
імпортуємо необхідні бібліотеки:

\begin{lstlisting}[language=Python]
# Standard imports
from __future__ import annotations
from pathlib import Path

# Tensorflow and numpy imports
import tensorflow as tf
import numpy as np

# Matplotlib imports
import matplotlib.pyplot as plt
from matplotlib import ticker, cm 
from matplotlib.colors import LinearSegmentedColormap
\end{lstlisting}

Далі, створюємо функції для побудови графіків та генерації набору даних:

\begin{lstlisting}[language=Python]
# Selecting primary colors
PRIMARY_COLOR_POSITIVE = 'mediumspringgreen'
PRIMARY_COLOR_NEGATIVE = 'violet'
CMAP = LinearSegmentedColormap.from_list("Custom", [PRIMARY_COLOR_NEGATIVE, PRIMARY_COLOR_POSITIVE], N=20)

# Picking a number of points to draw
DATASET_SIZE = 1000
RADIUS = 0.8

def get_labels(x: np.ndarray) -> np.ndarray:
    """
    Based on array of R^2 coordinates, returns an array of bits indicating
    whether the point is included in the region
    """
    
    return 0.5*(x[:,1]-0.5)**2 - 0.4*(x[:,0]-0.5)**2 < 0.02

def generate_dataset() -> np.ndarray:
    """
    Generates a random dataset based on the curve provided
    """
    
    x = tf.random.uniform(shape=(DATASET_SIZE, 2))
    return x, get_labels(x)

def display_dataset(x: np.ndarray, y: np.ndarray, save_path: Path = None) -> None:
    """
    Displays the dataset in a form of a scatterplot
    
    Args:
        x - an array of points in R^2
        y - an array of bits, marking the class of each point
    """
    
    # Split the dataset into two parts
    x_positive = np.array([x for x, y in zip(x, y) if y])
    x_negative = np.array([x for x, y in zip(x, y) if not y])
    
    # Display two scatterplots
    fig, ax = plt.subplots()
    ax.set_xlim([0.0, 1.0])
    ax.set_ylim([0.0, 1.0])
    plt.scatter(x_positive[:,0], x_positive[:,1], color=PRIMARY_COLOR_POSITIVE, edgecolors='black')
    plt.scatter(x_negative[:,0], x_negative[:,1], color=PRIMARY_COLOR_NEGATIVE, edgecolors='black')
    plt.grid()
    
    # Showing the plot and saving if needed
    if save_path is not None:
        plt.savefig(save_path, transparent=True)
    
    plt.show()

def display_dataset_with_heatmap(x: np.ndarray, 
                                    y: np.ndarray, 
                                    fn: callable, 
                                    save_path: Path = None) -> None:
    """
    Displays the dataset in a form of a scatterplot together
    with the heatmap plotted using fn function
    
    Args:
        x - an array of points in R^2
        y - an array of bits, marking the class of each point
        fn - function from R^2 to R, according to which the heatmap is built
        save_path - path where image is saved. Select None to omit saving
    """
    
    # Preparing the plot
    fig, ax = plt.subplots()

    # Show the heatmap
    x_nodes = np.linspace(0.0, 1.0, 3000)
    y_nodes = np.linspace(0.0, 1.0, 3000)
    xx, yy = np.meshgrid(x_nodes, y_nodes)
    r1, r2 = xx.flatten(), yy.flatten()
    r1, r2 = r1.reshape((len(r1), 1)), r2.reshape((len(r2), 1))
    grid = np.hstack((r1, r2))
    prediction = np.array(fn(grid))
    zz = prediction.reshape(xx.shape)
    c = plt.contourf(xx, yy, zz, cmap=CMAP)
    fig.colorbar(c)

    # Split the dataset into two parts
    x_positive = np.array([x for x, y in zip(x, y) if y])
    x_negative = np.array([x for x, y in zip(x, y) if not y])
    
    # Display two scatterplots
    ax.set_xlim([0.0, 1.0])
    ax.set_ylim([0.0, 1.0])
    plt.scatter(x_positive[:,0], x_positive[:,1], color=PRIMARY_COLOR_POSITIVE, edgecolors='black')
    plt.scatter(x_negative[:,0], x_negative[:,1], color=PRIMARY_COLOR_NEGATIVE, edgecolors='black')
    plt.grid()
    
    # Showing the plot and saving if needed
    if save_path is not None:
        plt.savefig(save_path, transparent=True)
    
    plt.show()
\end{lstlisting}

Генеруємо датасет та зберігаємо його візуалізацію:
\begin{lstlisting}[language=Python]
x, y = generate_dataset()
display_dataset(x, y, save_path='dataset.pdf')
display_dataset_with_heatmap(x, y, get_labels, save_path='./classification-example.pdf')
# Convert array of bits to array of 0.0's and 1.0's
y = tf.cast(y, tf.float32)
\end{lstlisting}

Далі, головна частина: клас для специфікації архітектури мережі Цибенко та її тренування:
\begin{lstlisting}[language=Python]
class CybenkoNetwork:
"""
Class representing the Cybenko network
"""

ACTIVATION = 'sigmoid'
INITIALIZER = 'GlorotNormal'
LEARNING_RATE = 0.05

def __init__(self, hidden_layer_size: int = 6, learning_rate: float = 0.05) -> None:
    """
    Initializes the CybenkoNetwork instance.
    
    Args:
        - hidden_layer_size: number of hidden neurons (n from paper)
        - learning_rate: how fast to train the network
    """
    
    self._hidden_layer = tf.keras.layers.Dense(HIDDEN_LAYER_SIZE, 
                        activation=CybenkoNetwork.ACTIVATION, 
                        bias_initializer=CybenkoNetwork.INITIALIZER,
                        kernel_initializer=CybenkoNetwork.INITIALIZER)
    self._alpha = tf.Variable(tf.random.normal((1, hidden_layer_size)), name='alpha')
    self._optimizer = tf.keras.optimizers.legacy.Adam(learning_rate=learning_rate)
    self._mean = 0.5 # Value needed for final classification

def predict(self, x: np.ndarray) -> np.ndarray:
    """
    Based on the batch of inputs, gives a batch of predictions
    
    Args:
        x - batch of inputs
    """
    
    z = self._hidden_layer(x)
    return tf.matmul(self._alpha, tf.transpose(z))

def predict_binary(self, x: np.ndarray) -> np.ndarray:
    """
    Predicts the class of each x value in a form of bit
    
    Args:
        x - batch of inputs
    """
    
    prediction = self.predict(x)
    return prediction > self._mean

def train(self, x: np.ndarray, y: np.ndarray, epochs: int = 5000, batch_size: int = 1024) -> None:
    """
    Trains the model on given dataset (x, y) with the specified number of epochs 
    and batch size.
    
    Args:
        x, y - array of R^2 coordinates and corresponding label
        epochs - number of epochs to train with
        batch_size - number of pairs for each gradient iteration step
    """
    
    for epoch in range(epochs):
        for offset in range(0, len(x), batch_size):
            # Getting the batch
            xs, ys = x[offset: offset + batch_size], y[offset: offset + batch_size]

            with tf.GradientTape() as tape:
                # Forward pass: calculating the MSE loss
                loss_value = tf.reduce_mean((self.predict(xs) - np.array([ys]))**2)

            # Use the gradient tape to automatically retrieve
            # the gradients of the trainable variables with respect to the loss.
            grads = tape.gradient(loss_value, [self._alpha, *self._hidden_layer.trainable_variables])
            # Run one step of gradient descent by updating
            # the value of the variables to minimize the loss.
            self._optimizer.apply_gradients(zip(grads, [self._alpha, *self._hidden_layer.trainable_variables]))
        
        if (epoch + 1) % 100 == 0:
            print(f'Finished epoch {epoch+1}, loss value: {loss_value}...')
    
    print('Training finished!')
    # Calculating the mean score for the whole dataset 
    # (needed further to predict the class in the binary form)
    self._mean = np.mean(self.predict(x))
\end{lstlisting}

Далі, починаємо тренування:
\begin{lstlisting}[language=Python]
# Initializing and training the model
HIDDEN_LAYER_SIZE = 6
model = CybenkoNetwork(hidden_layer_size=HIDDEN_LAYER_SIZE, learning_rate=0.05)
model.train(x, y, epochs=5000, batch_size=1024)
\end{lstlisting}

Далі, залишається лише відобразити неперервне передбачення мережі,
дискретне передбачення та передбачення проміжних значень:
\begin{lstlisting}[language=Python]
display_dataset_with_heatmap(x, y, model.predict, save_path='classification-cont-prediction.pdf')
display_dataset_with_heatmap(x, y, model.predict_binary, save_path='classification-discr-prediction.pdf')
for i in range(HIDDEN_LAYER_SIZE):
    display_dataset_with_heatmap(x, y, 
        lambda x: np.array([model._hidden_layer(x)[:,i]]), 
        save_path=f'layer-{i+1}-prediction.pdf')
\end{lstlisting}

Нарешті, виводимо параметри моделі:
\begin{lstlisting}[language=Python]
print('Weights and biases:', model._hidden_layer.trainable_variables)
print('alpha weights:', model._alpha)
\end{lstlisting}

\chapter{Програмний код модуля $B$-сплайнів}\label{appendix:b-spline-code}
В цьому додатку, ми наведемо програмний код для реалізації модуля
$B$-сплайнів на фреймворку \texttt{PyTorch} на мові програмування
\texttt{Python}. 
\begin{lstlisting}[language=Python]
from __future__ import annotations

import torch
import torch.nn as nn

import numpy as np

class BSpline(nn.Module):
    """
    Class for B-spline interpolation using PyTorch.
    This class allows for the creation of B-spline curves with a specified number of control points and degree.
    The B-spline curve is defined by a set of control points and a knot vector.
    The B-spline basis functions are computed using the Cox-de Boor recursion formula.
    """
    
    def __init__(
        self, 
        knots: int, 
        degree: int = 3
    ) -> None:
        """
        Initializes the B-spline object.
        Args:
            knots (int): Number of control points.
            degree (int): Degree of the B-spline. Default is 3 (cubic B-spline).
        """
        
        super(BSpline, self).__init__()
        
        self.knots = knots
        self.degree = degree
        
        # Setting up the grid
        grid = torch.linspace(-1, 1, knots) # Knot vector of length knots + degree + 1
        grid = BSpline.extend_grid(grid, k=degree) # Extend the grid on either side by degree steps
        self.register_buffer('grid', grid)
        
        # Prepare trainable parameters
        coeffs_num = knots + 2 * degree - 1
        self.coeffs_num = coeffs_num
        self.coeffs = nn.Parameter(torch.randn(coeffs_num)) # Learnable coefficients
        

    @staticmethod
    def extend_grid(grid: np.ndarray, k: int) -> np.ndarray:
        """
        Extends the grid on either size by k steps

        Args:
            grid: number of splines x number of control points
            k: spline order

        Returns:
            new_grid: number of splines x (number of control points + 2 * k)
        """
        
        n_intervals = len(grid) - 1
        bucket_size = (grid[-1] - grid[0]) / n_intervals
        
        for i in range(k):
            grid = torch.cat([grid[:1] - bucket_size, grid])
            grid = torch.cat([grid, grid[-1:] + bucket_size])

        return grid
    

    def b_spline_basis(self, x: torch.Tensor) -> torch.Tensor:
        """
        Evaluates the B-spline basis functions at position x.
        """
        
        tensor_shape = x.shape
        basis = torch.zeros((self.knots+2*self.degree-1, self.degree+1, *tensor_shape), dtype=x.dtype, device=x.device)
        
        for i in range(self.knots+2*self.degree-1):
            basis[i, 0, :] = torch.where(
                (x >= self.grid[i]) & (x < self.grid[i+1]), 
                torch.ones_like(x), 
                torch.zeros_like(x)
            )
        
        for d in range(1, self.degree+1):
            for i in range(self.knots+2*self.degree-1-d):
                b1 = (x - self.grid[i]) * basis[i, d-1] / (self.grid[i+d] - self.grid[i])
                b2 = (self.grid[i+d+1] - x) * basis[i+1, d-1] / (self.grid[i+d+1] - self.grid[i+1])
                basis[i, d, :] = b1 + b2
        
        return basis[:, self.degree, ...] # The last column corresponds to the degree of the spline


    def forward(self, x: torch.Tensor) -> torch.Tensor:
        """
        Computes the forward pass of the B-spline.
        """
        
        basis_values = self.b_spline_basis(x)
        return basis_values.T @ self.coeffs
\end{lstlisting}

\chapter{Програмний код конволюційного шару Колмогорова-Арнольда}\label{appendix:c-ckan-code}

В цьому додатку, ми наведемо програмний код для реалізації конволюційного шару
Колмогорова-Арнольда на фреймворку \texttt{PyTorch} на мові програмування
\texttt{Python}.
\begin{lstlisting}[language=Python]
"""
Implementation of the KANConv2d layer as described in my
Bachelor's thesis. This layer is a convolutional layer that
uses a spline-based approach to learn the convolutional
kernel weights. The layer is designed to work with 2D input
data, such as images.
"""
from __future__ import annotations

from typing import Tuple, Union

import torch
import torch.nn as nn
import torch.nn.functional as F
import numpy as np


class KANConv2d(nn.Module):
    """
    Implementation of the KANConv2d layer as described in my 
    Bachelor's thesis. This layer is a convolutional layer that
    uses a spline-based approach to learn the convolutional
    kernel weights. The layer is designed to work with 2D input
    data, such as images.
    """
    
    def __init__(
        self,
        in_channels: int, 
        out_channels: int, 
        kernel_size: Union[Tuple[int, int], int], 
        stride: int = 1,
        padding: int = 1,
        spline_points: int = 5, 
        spline_degree: int = 3,
        spline_coefficients_variance: float = 0.1,
    ) -> None:
        """
        - `in_channels`: Number of input channels
        - `out_channels`: Number of output channels
        - `kernel_size`: Size of the convolutional kernel (assumes to be square if of type `int`)
        - `stride`: Stride of the convolution
        - `padding`: Padding added to all sides of the input
        - `spline_points`: Number of points for the spline basis functions
        - `spline_coefficients_variance`: Variance of the spline coefficients initialization
        - `spline_degree`: Degree of the spline basis functions
        """
        
        super(KANConv2d, self).__init__()
        
        if isinstance(kernel_size, int):
            kernel_size = (kernel_size, kernel_size)
        if isinstance(stride, int):
            stride = (stride, stride)
        if isinstance(padding, int):
            padding = (padding, padding)
        
        # Set the internal parameters
        self.in_channels = in_channels
        self.out_channels = out_channels
        self.kernel_size = kernel_size
        self.stride = stride
        self.padding = padding
        self.knots = spline_points
        self.degree = spline_degree
        
        # Learnable weights for beta(x) and spine basis functions
        self.beta_weights = nn.Parameter(torch.randn(out_channels, in_channels, kernel_size[0], kernel_size[1]))
        self.spline_weights = nn.Parameter(torch.randn(out_channels, in_channels, kernel_size[0], kernel_size[1]))

        # Setting up the grid
        grid = torch.linspace(-1, 1, self.knots) # Knot vector of length knots + degree + 1
        grid = KANConv2d.extend_grid(grid, d=self.degree) # Extend the grid on either side by degree steps
        self.register_buffer('grid', grid)
        
        # Prepare trainable parameters
        spline_coefficients = self.knots + 2 * self.degree - 1 # Number of coefficients for the spline representation
        self.spline_coefficients = spline_coefficients
        # Initialize the spline coefficients according to the normal distribution
        # with the specified variance
        std = np.sqrt(spline_coefficients_variance)
        self.coeffs = nn.Parameter(std * torch.randn(spline_coefficients, out_channels, in_channels, kernel_size[0], kernel_size[1])) # Learnable coefficients


    @staticmethod
    def extend_grid(grid: np.ndarray, d: int) -> np.ndarray:
        """
        Extends the grid on either size by d steps

        Args:
            grid: number of splines x number of control points
            d: spline order

        Returns:
            new_grid: number of splines x (number of control points + 2*d)
        """
        
        n_intervals = len(grid) - 1
        bucket_size = (grid[-1] - grid[0]) / n_intervals
        
        for _ in range(d):
            grid = torch.cat([grid[:1] - bucket_size, grid])
            grid = torch.cat([grid, grid[-1:] + bucket_size])

        return grid

    
    def b_spline_basis(self, x: torch.Tensor) -> torch.Tensor:
        """
        Evaluates the B-spline basis functions at position x.
        """
        
        tensor_shape = x.shape

        # Using x.device for basis, assuming grid will be used accordingly or is on the same device
        basis = torch.zeros((self.knots+2*self.degree-1, self.degree+1, *tensor_shape), dtype=x.dtype, device=x.device)
        
        # Ensure grid is on the same device as x for operations involving both
        grid_local = self.grid.to(x.device)

        for i in range(self.spline_coefficients):
            # Ensure operations are between tensors on the same device
            condition = (x >= grid_local[i]) & (x < grid_local[i+1])
            basis[i, 0, ...] = torch.where(
                condition,
                torch.ones_like(x), 
                torch.zeros_like(x)
            )
        
        for d in range(1, self.degree+1):
            for i in range(self.spline_coefficients-d):
                # Clone the slices of `basis` from the previous degree (d-1) before using them.
                # This prevents the inplace modification of `basis` (via .copy_ below)
                # from affecting the versions of these tensors needed by autograd.
                basis_i_prev_d = basis[i, d-1, ...].clone()
                basis_i_plus_1_prev_d = basis[i+1, d-1, ...].clone()

                # Denominators
                den1 = grid_local[i+d] - grid_local[i]
                den2 = grid_local[i+d+1] - grid_local[i+1]

                # Calculate b1
                # Note: Original code divides directly. If den1 can be zero, this can lead to inf/nan.
                # This fix focuses on the autograd error, not numerical stability of division by zero.
                if den1 == 0: # Avoid division by zero; result of this term is effectively 0 if numerator is finite.
                              # Or handle as per specific B-spline convention for coincident knots.
                              # For now, if den1 is 0, b1 will be 0 assuming basis_i_prev_d is not inf/nan.
                    b1 = torch.zeros_like(x)
                else:
                    b1 = (x - grid_local[i]) * basis_i_prev_d / den1
                
                # Calculate b2
                if den2 == 0: # Similar handling for den2
                    b2 = torch.zeros_like(x)
                else:
                    b2 = (grid_local[i+d+1] - x) * basis_i_plus_1_prev_d / den2
                
                # The inplace copy operation was the source of the autograd error.
                # By using cloned inputs for b1 and b2, the original `basis` tensor's
                # versions are not an issue for *their* gradient computation.
                basis[i, d, ...].copy_(b1 + b2)
        
        return basis[:, self.degree, ...]

    def forward(self, x: torch.Tensor) -> torch.Tensor:
        """
        Computes the forward pass of the KANConv2d layer.
        """
        
        # First, figure out the input/output dimensions
        batch_size, _, height, width = x.shape
        x = F.pad(x, 
                (self.padding[1], 
                self.padding[1], 
                self.padding[0], 
                self.padding[0]))
        out_height = (height + 2 * self.padding[0] - self.kernel_size[0]) // self.stride[0] + 1
        out_width = (width + 2 * self.padding[1] - self.kernel_size[1]) // self.stride[1] + 1
        output = torch.zeros(
            batch_size, 
            self.out_channels, 
            out_height, 
            out_width, 
            device=x.device
        )
        
        for i in range(out_height):
            for j in range(out_width):
                patch = x[:, :, 
                        i*self.stride[0]:i*self.stride[0]+self.kernel_size[0],
                        j*self.stride[1]:j*self.stride[1]+self.kernel_size[1]]
                
                # Evaluate the beta function with weights
                beta = F.silu(patch)
                beta = beta.unsqueeze(1).repeat(1, self.out_channels, 1, 1, 1)
                beta_weights = self.beta_weights.unsqueeze(0).repeat(batch_size, 1, 1, 1, 1)
                beta = beta_weights * beta
                beta = beta.permute(1, 0, 2, 3, 4)
                
                # Evaluate the B-spline basis functions
                basis_values = self.b_spline_basis(patch)
                
                # Extend all the shapes to be compatible
                coeffs = self.coeffs.unsqueeze(2).repeat(1, 1, batch_size, 1, 1, 1)
                basis_values = basis_values.unsqueeze(1).repeat(1, self.out_channels, 1, 1, 1, 1)
                
                # Compute the linear combination of the basis functions
                splines = torch.sum(coeffs*basis_values, dim=0) # [batch_size, out_channels, in_channels, kernel_size[0], kernel_size[1]]
                spline_weights = self.spline_weights.unsqueeze(1).repeat(1, batch_size, 1, 1, 1)
                splines = spline_weights * splines
                
                # Find sum of the splines and beta values
                activations = splines + beta
                
                # Finally, compute the output part of the convolution
                result = torch.sum(activations, dim=(2, 3, 4)) # [batch_size, out_channels, in_channels, kernel_size[0], kernel_size[1]]
                result = result.permute(1, 0)
                output[:, :, i, j] = result

        return output
\end{lstlisting}

\end{document}




